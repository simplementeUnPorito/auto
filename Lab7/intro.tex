\section{Introducción}

En el laboratorio anterior se trabajó con un sistema en lazo abierto, donde no existía corrección automática ante perturbaciones ni derivas en los componentes. Esto hacía al sistema más sensible y dependiente de la exactitud del modelo.  

En esta práctica se implementa un esquema en \textbf{lazo cerrado} mediante realimentación de estados, incorporando además una \textbf{acción integral} para asegurar seguimiento de referencia con error estacionario nulo. El integrador acumula el error entre la salida y la referencia, mientras que la realimentación permite ajustar la respuesta dinámica de manera precisa mediante la ubicación arbitraria de polos.  

De esta forma se obtiene un controlador más robusto y con mejor comportamiento frente a perturbaciones, combinando la teoría de control en el espacio de estados con la implementación digital en PSoC.  

\section{Objetivos}

\begin{itemize}
	\item Diseñar un controlador digital en lazo cerrado con realimentación de estados y acción integral.
	\item Implementar el control en la planta analógica utilizando el PSoC.
	\item Comparar el desempeño simulado y experimental, analizando la influencia del integrador y la robustez frente a perturbaciones.
\end{itemize}
