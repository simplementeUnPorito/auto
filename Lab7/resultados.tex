% =========================================================
\clearpage
\section{Resultados}

En las figuras se observa que la incorporación de la acción integral elimina la deriva frente a perturbaciones y garantiza error estacionario nulo, manteniendo una dinámica global comparable a la simulada. Tanto el observador \textit{predictivo} como el \textit{actual} replican la tendencia teórica, aunque el sobreimpulso real resulta no se asemeja a lo calculado para todos los casos.
Las discrepancias pueden atribuirse a tolerancias de componentes pasivos, desajustes de ganancia y cuantización DAC/ADC, además de pequeños desvíos de polos respecto a los ubicados por diseño.

\begin{table}[!t]
	\centering
	\caption{Comparación experimental y simulada: \emph{sin integrador} / \emph{con integrador}.}
	\label{tab:desempeno_exp}
	\begin{tabular}{lcc}
		\toprule
		\textbf{Caso} & {$t_s$ [ms] (\emph{sin}/\emph{con})} & {OS [\%] (\emph{sin}/\emph{con})} \\
		\midrule
		Predictor (Sim)     & 7.00/8.00  & 4.6/4.0 \\
		Actual (Sim)        & 7.00/8.00 & 4.6/4.0 \\
		Predictor (Exp)  &10.0/4.0   & 6.520/8.150  \\
		Actual (Exp)        & 5.940/6.850 & 10.0/5.0 \\
		\bottomrule
	\end{tabular}
\end{table}

La tabla~\ref{tab:desempeno_exp} resume los valores simulados y experimentales para ambos estimadores, contrastando \emph{sin} y \emph{con} integrador. 
Se confirma que el integrador elimina el error estacionario y mejora la robustez, a costa de un leve aumento del esfuerzo de control. 

\balance

% ====== Figuras de comparación (Sim vs Exp) ======
% Predictor: Simulación (sin vs con integrador) y resultado experimental (con integrador)
\compararfigsC
{sim/predictor_comp_sin_y_con_int.png}{Estimador \textbf{predictivo} (sim.): \emph{sin} vs \emph{con} integrador.}{fig:sim_pred_int}
{exp/PRED.JPG}{Estimador \textbf{predictivo} (exp.): \emph{con} integrador.}{fig:exp_pred_int}
{Comparación del estimador predictivo: simulación \emph{sin/con} integrador vs resultado experimental \emph{con} integrador.}
{fig:cmp_pred}

% Actual: Simulación (sin vs con integrador) y resultado experimental (con integrador)
\compararfigsC
{sim/actual_comp_sin_y_con_int.png}{Estimador \textbf{actual} (sim.): \emph{sin} vs \emph{con} integrador.}{fig:sim_act_int}
{exp/CURR.JPG}{Estimador \textbf{actual} (exp.): \emph{con} integrador.}{fig:exp_act_int}
{Comparación del estimador actual: simulación \emph{sin/con} integrador vs resultado experimental \emph{con} integrador.}
{fig:cmp_act}

% Esfuerzos de control (todas las variantes en simulación)
\insertarfigura{sim/comp_todos_los_esfuerzos.png}
{Esfuerzos de control para las cuatro configuraciones: \emph{sin} y \emph{con} integrador, estimador \emph{predictivo} y \emph{actual} (simulación).}
{fig:esfuerzos_sim}
{0.95}
