\section{Implementación}

En este laboratorio se reutilizaron los proyectos del Lab~6 (\emph{predictivo} y \emph{actual}) manteniendo la misma estructura de hardware en el PSoC.  
La única modificación fue la incorporación del integrador en el código C, lo que permitió que ambos esquemas trabajaran ahora en lazo cerrado con acción integral.

\[
v_{k+1} = v_k + (r_k - y_k),
\qquad
u_k = K_1 v_{k+1} - K_2 \hat{x}_k.
\]

Los bloques físicos (planta, DAC, ADC, UART e ISR de adquisición) se mantuvieron idénticos al laboratorio anterior, por lo que la adaptación fue directa y sin cambios en el diagrama del sistema.

% --- Circuito de la planta (referencia física)
\insertarfigura{exp/circuito_de_la_planta.png}
{Diagrama del circuito de la planta utilizado en el montaje experimental.}
{fig:planta_exp}
{0.9}

% --- Implementación del controlador predictivo con integrador
\insertarfigura{exp/circuito_de_control_pred.png}
{Implementación del controlador \textbf{predictivo} con integrador en PSoC (diagrama de control).}
{fig:impl_pred}
{0.9}

% --- Implementación del controlador actual con integrador
\insertarfigura{exp/circuito_de_control_curr.png}
{Implementación del controlador \textbf{actual} con integrador en PSoC (diagrama de control).}
{fig:impl_curr}
{0.9}


% --- Resultados experimentales del osciloscopio
\compararfigsCwide
{exp/PRED.JPG}{Respuesta experimental con estimador \textbf{predictivo} (con integrador).}{fig:osc_pred}
{exp/CURR.JPG}{Respuesta experimental con estimador \textbf{actual} (con integrador).}{fig:osc_curr}
{Capturas del osciloscopio: ambos esquemas con acción integral muestran eliminación del error estacionario y seguimiento.}
{fig:osc_psoc}{0.8}
\balance
