
\section{Implementación}

En este laboratorio se reutilizaron los proyectos del Lab~6 (\emph{predictivo} y \emph{actual}) manteniendo la misma estructura de hardware en el PSoC.  
La única modificación fue la incorporación del integrador en el código C, lo que permitió que ambos esquemas trabajaran ahora en lazo cerrado con acción integral.

\[
v_{k+1} = v_k + (r_k - y_k),
\qquad
u_k = K_1 v_{k+1} - K_2 \hat{x}_k.
\]

Los bloques físicos (planta, DAC, ADC, UART e ISR de adquisición) se mantuvieron idénticos al laboratorio anterior, por lo que la adaptación fue directa y sin cambios en el diagrama del sistema.

\compararfigsC
{placeholder.png}{Implementación del controlador predictivo con integrador.}{fig:impl_pred}
{placeholder.png}{Implementación del controlador actual con integrador.}{fig:impl_curr}
{Implementaciones en PSoC de los esquemas con acción integral.}
{fig:impl_psoc}

Ambos códigos —\texttt{main\_pred.c} y \texttt{main\_curr.c}— se ejecutaron correctamente en la placa, mostrando un seguimiento más robusto y una reducción del error estacionario frente a perturbaciones.

\balance
