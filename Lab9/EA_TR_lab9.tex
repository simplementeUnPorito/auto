% !TeX program = pdflatex
\documentclass[conference]{IEEEtran}

% ======= Idioma y codificación =======
\usepackage[utf8]{inputenc}
\usepackage[T1]{fontenc}
\usepackage[spanish, es-nodecimaldot]{babel}
\usepackage{microtype}

% ======= Matemática =======
\usepackage{amsmath, amssymb, amsfonts}

% ======= Colores y gráficos =======
\usepackage{xcolor}
\usepackage{graphicx}
% \usepackage{subcaption}
\usepackage[caption=false,font=footnotesize]{subfig}
\usepackage{float}
\usepackage{booktabs}

% ======= Números y unidades =======
\usepackage{siunitx}
\sisetup{
	output-decimal-marker = {,},
	group-separator = {.},
	group-minimum-digits = 4,
	per-mode = symbol,
	round-mode = places,
	round-precision = 3,
	table-number-alignment = center
}

% ======= Maquetación =======
\usepackage{cuted}
\usepackage{balance}

% ======= Listas =======
\usepackage{enumitem}
\setlist{leftmargin=*, itemsep=2pt, topsep=4pt}

% ======= Bibliografía =======
\usepackage[numbers]{natbib}
\bibliographystyle{IEEEtranN}

% ======= Listados =======
\usepackage{listings}
\usepackage{listingsutf8}

% Config listings
\lstset{
	inputencoding=utf8/latin1,
	extendedchars=true,
	upquote=true,
	breaklines=true,
	columns=fullflexible,
	keepspaces=true,
	frame=single,
	numbers=left, numberstyle=\tiny, numbersep=6pt,
	xleftmargin=1em,
	captionpos=b,
	literate=
	{á}{{\'a}}1 {é}{{\'e}}1 {í}{{\'\i}}1 {ó}{{\'o}}1 {ú}{{\'u}}1
	{Á}{{\'A}}1 {É}{{\'E}}1 {Í}{{\'I}}1 {Ó}{{\'O}}1 {Ú}{{\'U}}1
	{ñ}{{\~n}}1 {Ñ}{{\~N}}1
	{ü}{{\"u}}1 {Ü}{{\"U}}1
	{¿}{{\textquestiondown}}1 {¡}{{\textexclamdown}}1
	{º}{{$^\circ$}}1
	{“}{{``}}1 {”}{{''}}1 {‘}{{`}}1 {’}{{'}}1
	{—}{{---}}1 {–}{{--}}1
	{→}{{\textrightarrow}}1
	{≈}{{$\approx$}}1
	{µ}{{\textmu}}1
}

% Lenguaje Matlab
\lstdefinelanguage{Matlab}{
	morekeywords={break,case,catch,continue,else,elseif,end,for,function,global,if,otherwise,persistent,return,switch,try,while,classdef,properties,methods},
	sensitive=true,
	morecomment=[l]\%,
	morecomment=[s]{\%\{} {\%},
	morestring=[m]'
}

\lstdefinestyle{matlabstyle}{
	language=Matlab,
	basicstyle=\ttfamily\small,
	keywordstyle=\bfseries\color{blue!70!black},
	commentstyle=\itshape\color{green!40!black},
	stringstyle=\color{red!60!black},
	numbers=left, numberstyle=\tiny, numbersep=6pt,
	frame=single,
	breaklines=true,
	columns=fullflexible,
	keepspaces=true,
	captionpos=b,
	linewidth=\columnwidth,
	xleftmargin=1em
}

% ======= Macros =======
\newcommand{\zetaD}{\zeta}
\newcommand{\wn}{\omega_n}
\newcommand{\Ts}{T_s}
\newcommand{\ESS}{\mathrm{ESS}}
\newcommand{\Gz}{G(z)}
\newcommand{\Gs}{G(s)}
\newcommand{\Nbar}{N_{\mathrm{bar}}}
\newcommand{\Ki}{K_i}
\newcommand{\Aaug}{A_{\mathrm{aug}}}
\newcommand{\Baug}{B_{\mathrm{aug}}}
\newcommand{\Caug}{C_{\mathrm{aug}}}

% ======= Macros figuras =======
\newcommand{\insertarfigura}[4]{%
	\begin{figure}[!t]
		\centering
		\includegraphics[width=#4\linewidth]{#1}
		\caption{#2}
		\label{#3}
	\end{figure}
}

% ======= Hyperref =======
\usepackage[unicode,hidelinks]{hyperref}

% ======= Metadatos LAB 9 =======
\title{Práctica de Laboratorio 9: Filtro de Kalman y Regulador LQG}

\author{
	\IEEEauthorblockN{Elías Álvarez}
	\IEEEauthorblockA{Carrera de Ing. Electrónica\\Universidad Católica Nuestra Señora de la Asunción\\Asunción, Paraguay\\Email: elias.alvarez@universidadcatolica.edu.py}
	\and
	\IEEEauthorblockN{Tania Romero}
	\IEEEauthorblockA{Carrera de Ing. Electrónica\\Universidad Católica Nuestra Señora de la Asunción\\Asunción, Paraguay\\Email: tania.romero@universidadcatolica.edu.py}
}

\begin{document}
	\maketitle
	
	\section{Objetivos}
	Los objetivos de esta práctica consisten en aplicar las técnicas de control óptimo y estimación estocástica para el diseño de un controlador LQG implementado sobre la planta analógica utilizada en laboratorios anteriores. En particular, se busca:
	
	\begin{itemize}
		\item Comprender el funcionamiento y la formulación del regulador óptimo LQR, así como su relación con el criterio cuadrático.
		\item Diseñar un controlador por realimentación de estados que permita obtener respuestas rápidas y estables, considerando diferentes elecciones de matrices de peso.
		\item Comprender el funcionamiento del filtro de Kalman y su rol como estimador óptimo en presencia de ruido de proceso y de medición.
		\item Implementar dos estimadores: uno obtenido mediante el cálculo del valor óptimo de la ganancia de Kalman a partir de relaciones S/R conocidas, y otro cuya dinámica sea cinco veces más rápida que la del sistema en lazo cerrado.
		\item Integrar el controlador LQR y el filtro de Kalman para conformar un esquema LQG completo, evaluando su desempeño.
		\item Comparar los resultados obtenidos mediante simulación con los resultados experimentales obtenidos a través del PSoC.
		\item Analizar ventajas, limitaciones y sensibilidad del controlador LQG frente al ruido y variaciones de la planta.
	\end{itemize}
	
	\section{Materiales}
	Para el desarrollo de esta práctica se utilizaron los siguientes materiales y herramientas:
	
	\begin{itemize}
		\item Computadora personal con \texttt{MATLAB}.
		\item Planta analógica utilizada en los laboratorios anteriores (sistema de dos etapas RC con amplificadores operacionales).
		\item Sistema de adquisición e implementación digital basado en \texttt{PSoC}.
		\item Osciloscopio y generador de señales para la inyección de señales y verificación de comportamiento real.
	\end{itemize}
	
	\section{Modelo de la Planta}
	En este laboratorio se continúa utilizando el modelo dinámico obtenido y validado experimentalmente en los Laboratorios 7 y 8. Dicho modelo corresponde a la planta analógica compuesta por dos etapas RC en cascada, implementadas con amplificadores operacionales en configuración no inversora. 
	
	El sistema se modeló previamente mediante las ecuaciones en espacio de estados, considerando como variables de estado los voltajes en los capacitores de cada etapa. El análisis mediante leyes de Kirchhoff permitió obtener las ecuaciones diferenciales del sistema continuo, cuya forma general es:
	\begin{equation}
		\dot{x}(t) = F x(t) + G u(t),
	\end{equation}
	\begin{equation}
		y(t) = H x(t) + J u(t),
	\end{equation}
	donde las matrices \textit{F}, \textit{G}, \textit{H} y \textit{J} ya fueron derivadas, verificadas y utilizadas exitosamente en laboratorios anteriores.
	
	En este laboratorio no se repite el proceso completo de modelado, ya que la planta real no sufrió modificaciones y el modelo fue validado tanto analítica como experimentalmente. Por lo tanto, se toma como punto de partida el modelo continuo ya establecido, y su correspondiente versión discretizada, añadiendo únicamente el integrador que ya venía siendo utilizado en prácticas previas para garantizar error estacionario nulo ante entradas de referencia escalón.
	
	Este modelo, junto con su versión digital, será utilizado para el diseño del LQR, del filtro de Kalman y, en consecuencia, del regulador LQG.
	
	\section{Selección del Tiempo de Muestreo}
	El tiempo de muestreo utilizado en esta práctica es \boldmath$T_s = 1~\text{ms}$, correspondiente a una frecuencia de muestreo de 1 kHz. Esta elección no se realizó de manera arbitraria, sino que fue determinada y justificada en los laboratorios anteriores.
	
	Durante las prácticas previas, se evaluaron los polos del sistema continuo y se verificó que la dinámica de la planta posee constantes de tiempo suficientemente pequeñas como para que una frecuencia de muestreo del orden de 1 kHz permita capturar adecuadamente su comportamiento. En particular, se contrastó con el criterio general de muestreo:
	\begin{equation}
		T_s < \frac{\pi}{ \omega_n},
	\end{equation}
	donde \textit{$\omega_n$} representa la frecuencia natural dominante del sistema. 
	
	Los resultados obtenidos en los Laboratorios 7 y 8 demostraron que un muestreo de 1 kHz garantiza:
	\begin{itemize}
		\item Una discretización adecuada del sistema continuo.
		\item Ausencia de aliasing relevante en la señal medida.
		\item Suficiente tiempo disponible para el cálculo de control dentro del PSoC.
		\item Estabilidad y precisión en los reguladores diseñados (LQR e integral anteriormente).
	\end{itemize}
	
	Dado que el objetivo de este laboratorio no es reestudiar la discretización, sino diseñar el filtro de Kalman y el LQG sobre una base sólida, se mantiene el valor ya verificado de \boldmath$T_s = 1~\text{ms}$ como tiempo de muestreo oficial para el diseño y posterior implementación.
	
	\section{Diseño del Regulador LQR}

El diseño del regulador óptimo se realizó sobre el sistema aumentado, siguiendo exactamente la misma metodología empleada en los laboratorios anteriores. El objetivo es incluir un integrador que garantice error estacionario nulo ante entradas de referencia tipo escalón. Para ello, se construyó el sistema aumentado utilizado en MATLAB.

\subsection{Sistema aumentado utilizado}

El sistema aumentado empleado en el diseño del LQR se definió de acuerdo con el código MATLAB:

\begin{equation}
	\hat{A} =
	\begin{bmatrix}
		A & B \\
		0 & 0
	\end{bmatrix}, \qquad
	\hat{B} =
	\begin{bmatrix}
		0 \\ I
	\end{bmatrix}, \qquad
	\hat{C} =
	\begin{bmatrix}
		C & 0
	\end{bmatrix}.
\end{equation}

Esto se corresponde con:

\begin{lstlisting}[style=matlabstyle]
	Ahat = [A B; zeros(m,n+m)];
	Bhat = [zeros(n,m); eye(m)];
	Chat = [C zeros(1,m)];
\end{lstlisting}

Este aumento incorpora explícitamente el estado del integrador \(z\), pero la implementación final de la realimentación se realiza mediante la matriz de transformación \texttt{Aux}, tal como establece la metodología de la cátedra.

\subsection{Normalización de estados}

Debido a las ganancias distintas en cada etapa de la planta, se utilizó un vector de normalización para equilibrar la contribución de cada estado en la función de costo del LQR. Dicho vector se definió como:

\begin{equation}
	S_x = 
	\begin{bmatrix}
		\dfrac{1}{k_1},\;
		\dfrac{1}{k_1 k_2},\;
		1
	\end{bmatrix},
	\qquad
	k_1 = \left|\dfrac{R_2}{R_1}\right|,
	\qquad
	k_2 = \left|\dfrac{R_4}{R_3}\right|.
\end{equation}

Código MATLAB correspondiente:

\begin{lstlisting}[style=matlabstyle]
	k1 = abs(R2/R1);
	k2 = abs(R4/R3);
	Sx = [1/k1, 1/(k1*k2), 1];
\end{lstlisting}

\subsection{Definición de pesos \(Q_1\) y \(Q_2\)}

Siguiendo los requerimientos del laboratorio, se diseñaron dos controladores distintos:

\begin{itemize}
	\item \textbf{Diseño rápido}: penaliza fuertemente los estados para obtener la respuesta más veloz posible.
	\item \textbf{Diseño de bajo esfuerzo de control}: penaliza más la entrada para obtener una acción más suave.
\end{itemize}

Los pesos utilizados fueron:

\[
Q_{\text{fast}} = [100,\;100,\;1], \qquad
Q_{\text{soft}} = [2,\;4,\;1.5].
\]

Las matrices finales empleadas en el LQR fueron:

\begin{equation}
	Q_1^{(i)} = \mathrm{diag}(Q_i \odot S_x),
	\qquad
	Q_2 = \{20,\;4\}.
\end{equation}

Código MATLAB:

\begin{lstlisting}[style=matlabstyle]
	Q_fast = [100 100 1];
	Q_soft = [2 4 1.5];
	
	Q1{1} = diag(Q_fast .* Sx);
	Q1{2} = diag(Q_soft .* Sx);
	
	Q2 = {20, 4};
\end{lstlisting}

\subsection{Cálculo de la ganancia \(K\)}

Para cada par \((Q_1, Q_2)\), se calculó la ganancia LQR del sistema aumentado:

\begin{equation}
	\hat{K_i} = \mathrm{dlqr}(\hat{A}, \hat{B}, Q_1^{(i)}, Q_2^{(i)}).
\end{equation}

Luego, siguiendo la metodología del curso, se transformó la ganancia hacia su forma final:

\begin{equation}
	K_i = \left(\hat{K_i} + [0 \; 0 \; 1]\right)
	\begin{bmatrix}
		A-I_n & B \\
		CA & CB
	\end{bmatrix}^{-1}
\end{equation}

obteniéndose finalmente:

\[
K_2^{(i)} = K_i(1,1:n), \qquad
K_1^{(i)} = K_i(1,n+1).
\]

Código MATLAB exacto:

\begin{lstlisting}[style=matlabstyle]
	[Ki,~,Pi] = dlqr(Ahat, Bhat, Q1{i}, Q2{i});
	Ki = (Ki + [zeros(1,n) eye(m)]) * Aux;
	
	K2{i} = Ki(1,1:n);
	K1{i} = Ki(1,n+1:end);
\end{lstlisting}
	
	
	
	\section{Diseño del Filtro de Kalman}

El diseño del estimador óptimo se realizó utilizando el comando \texttt{dlqe}, que implementa el estimador lineal cuadrático discreto (LQE). A diferencia del diseño manual mediante la ecuación de Riccati, \texttt{dlqe} calcula la ganancia óptima del filtro de Kalman resolviendo internamente la ecuación de Riccati mediante el método de augmentación del sistema, garantizando estabilidad y convergencia del estimador.

\subsection{Modelo estocástico utilizado}

El sistema en tiempo discreto se modeló como:

\begin{equation}
	x(k+1) = A x(k) + B u(k) + w(k),
\end{equation}
\begin{equation}
	y(k) = C x(k) + v(k),
\end{equation}

donde:

\[
w(k) \sim \mathcal{N}(0, Q), \qquad
v(k) \sim \mathcal{N}(0, R).
\]

\subsection{Construcción de las covarianzas \(Q\) y \(R\)}

Los desvíos estándar del ruido de proceso \(\sigma_{x1}\), \(\sigma_{x2}\) se obtuvieron a partir de:

\begin{itemize}
	\item ruido térmico de resistores,
	\item ruido filtrado por los polos de la planta,
	\item ruido de cuantización del DAC,
	\item ruido externo inyectado a la planta.
\end{itemize}

El desvío estándar del ruido de medición \(\sigma_v\) proviene del ruido de cuantización del ADC del PSoC.

Así, las matrices de covarianza utilizadas fueron:

\[
Q = \mathrm{diag}(\sigma_{x1}^2,\, \sigma_{x2}^2),
\qquad
R = \sigma_v^2.
\]

Código correspondiente:

\begin{lstlisting}[style=matlabstyle]
	Q = diag((sigma_w.^2));   % [sigma_x1^2, sigma_x2^2]
	R = diag((sigma_v.^2));   % sigma_v^2
\end{lstlisting}

\subsection{Cálculo de la ganancia óptima de Kalman}

La función utilizada fue:

\begin{lstlisting}[style=matlabstyle]
	L_kal = disenar_kalman_simple(A, C, B1, sigma_w, sigma_v);
\end{lstlisting}

donde el diseño se implementa como:

\begin{lstlisting}[style=matlabstyle]
	function L = disenar_kalman_simple(A,C,G,sigma_w,sigma_v)
	Q = diag((sigma_w.^2));
	R = diag((sigma_v.^2));
	[L,~,~] = dlqe(A,G,C,Q,R);   % LQE discreto
	end
\end{lstlisting}

La ganancia obtenida:

\[
L_\text{kal} =
\mathrm{dlqe}(A,\,G,\,C,\,Q,\,R),
\]

corresponde al estimador óptimo que minimiza el error cuadrático medio:

\[
\hat{x}(k+1) = A\hat{x}(k) + B u(k)
+ L_\text{kal}\left( y(k) - C\hat{x}(k) \right).
\]

Este filtro representa la mejor estimación posible en presencia de ruido, ponderada según la relación \(S/R\) entre ruido de proceso y ruido de medición.

\section{Diseño del Observador Determinístico Rápido}

Para comparar con el estimador óptimo, también se implementó un observador determinístico mediante Ackermann, con los polos ubicados cinco veces más rápidos que los polos dominantes del lazo cerrado con LQR.

Primero se obtuvieron los polos del sistema aumentado en lazo cerrado:

\[
p_\mathrm{cl} = \mathrm{eig}(A_\mathrm{cl}).
\]

En MATLAB:

\begin{lstlisting}[style=matlabstyle]
	Pcl = eig(Acl);
	[~, idx] = sort(abs(Pcl), 'descend');
	Pdom = Pcl(idx(1:n));
	p_obs = Pdom.^5;   % polos 5x más rápidos
\end{lstlisting}

Luego se diseñó la ganancia del observador:

\begin{equation}
	L_{\mathrm{rap}} =
	\mathrm{acker}(A^\top,\,(CA)^\top,\,p_{\mathrm{obs}}^\top)^\top.
\end{equation}

Código correspondiente:

\begin{lstlisting}[style=matlabstyle]
	L_rapido = acker(A', (C*A)', p_obs').';
\end{lstlisting}

	
	\section{Simulaciones}
	
	En esta sección se presentan las simulaciones obtenidas a partir del código desarrollado en \texttt{MATLAB}.  
	Se comparan dos estimadores:
	
	\begin{itemize}
		\item El \textbf{filtro de Kalman}, diseñado mediante \texttt{dlqe} a partir de las covarianzas reales S/R.
		\item Un \textbf{observador de Luenberger rápido}, cuyos polos se fijan como los polos dominantes del lazo cerrado elevados a la quinta potencia (\(p_{\text{obs}} = p_{\text{cl}}^{5}\)).
	\end{itemize}
	
	A continuación se insertan todas las figuras provistas en la carpeta \texttt{Sim/}.
	
	% ================= FIGURA 1 =================
	\begin{figure}[H]
		\centering
		\includegraphics[width=0.95\linewidth]{Sim/Kalman_vs__Obs_rapido_conRuido_1}
		\caption{Comparación entre Kalman y observador rápido — Caso 1, con ruido.}
		\label{fig:sim_conruido1}
	\end{figure}
	
	% ================= FIGURA 2 =================
	\begin{figure}[H]
		\centering
		\includegraphics[width=0.95\linewidth]{Sim/Kalman_vs__Obs_rapido_sinRuido_1}
		\caption{Comparación entre Kalman y observador rápido — Caso 1, sin ruido.}
		\label{fig:sim_sinruido1}
	\end{figure}
	
	% ================= FIGURA 3 =================
	\begin{figure}[H]
		\centering
		\includegraphics[width=0.95\linewidth]{Sim/Kalman_vs__Obs_rapido_sinRuido_2}
		\caption{Comparación entre Kalman y observador rápido — Caso 2, sin ruido.}
		\label{fig:sim_sinruido2}
	\end{figure}
	
	% ================= FIGURA 4 =================
	\begin{figure}[H]
		\centering
		\includegraphics[width=0.95\linewidth]{Sim/Kalman_vs__Obs_rapido_conRuido_2}
		\caption{Comparación entre Kalman y observador rápido — Caso 2, con ruido.}
		\label{fig:sim_conruido2}
	\end{figure}
	
	

	\section{Implementación}
	
	En esta sección se presentan las capturas obtenidas con el osciloscopio Tektronix TDS1012B durante la implementación real del controlador.  
	Se muestran tanto los casos con filtro de Kalman como los casos sin Kalman (observador rápido).  
	Las señales corresponden al al primer y seguno estado de la planta.
	
	% =============== FIG 1 ===============
	\begin{figure}[H]
		\centering
		\includegraphics[width=0.95\linewidth]{Exp/Kalman_1}
		\caption{Implementación — caso con Kalman y primer LQR.}
		\label{fig:exp_kalman1}
	\end{figure}
	
	% =============== FIG 2 ===============
	\begin{figure}[H]
		\centering
		\includegraphics[width=0.95\linewidth]{Exp/Kalman_2}
		\caption{Implementación — caso con Kalman y segundo LQR.}
		\label{fig:exp_kalman2}
	\end{figure}
	
	% =============== FIG 3 ===============
	\begin{figure}[H]
		\centering
		\includegraphics[width=0.95\linewidth]{Exp/No_Kalman_1}
		\caption{Implementación — caso sin Kalman y primer LQR.}
		\label{fig:exp_nokalman1}
	\end{figure}
	
	% =============== FIG 4 ===============
	\begin{figure}[H]
		\centering
		\includegraphics[width=0.95\linewidth]{Exp/No_Kalman_2}
		\caption{Implementación — caso sin Kalman y segundo LQR.}
		\label{fig:exp_nokalman2}
	\end{figure}
	
	
	% =========================================================
\section{Resultados y Conclusiones}

En esta sección se comparan los resultados simulados y experimentales para los dos diseños LQR evaluados: un caso rápido
($t_r \approx 8$ ms, $t_s \approx 15$ ms, $OS = 0\%$ esperado) y un caso lento
($t_r \approx 18$ ms, $t_s \approx 35$ ms, $OS = 0\%$ esperado).
Se incluye la actuación del estimador de Kalman y el observador rápido ($\times 5$) en cada situación.

\begin{table}[!t]
	\centering
	\caption{Desempeño observado: tiempos de respuesta y presencia de ruido.}
	\label{tab:cmp_lqg}
	\begin{tabular}{lccc}
		\toprule
		\textbf{Caso} & $t_r$ [ms] & $t_s$ [ms] & OS [\%] \\
		\midrule
		% ----- Caso 1 -----
		\multicolumn{4}{c}{\textbf{Caso 1 (rápido): esperado } $t_r=8$ ms, $t_s=15$ ms, $OS=0\%$} \\
		\midrule
		Kalman (Sim)      & $\approx 8$ & $\approx 15$ & 0 \\
		Obs. rápido (Sim) & $\approx 8$ & $\approx 15$ & 0 \\
		Kalman (Exp)      & $\approx 8$--10 & $\approx 15$--20 & $\approx 3$--5 \\
		Obs. rápido (Exp) & $\approx 8$--10 & $\approx 15$--20 & $\approx 3$--6 \\
		\midrule
		% ----- Caso 2 -----
		\multicolumn{4}{c}{\textbf{Caso 2 (lento): esperado } $t_r=18$ ms, $t_s=35$ ms, $OS=0\%$} \\
		\midrule
		Kalman (Sim)      & $\approx 18$ & $\approx 35$ & 0 \\
		Obs. rápido (Sim) & $\approx 18$ & $\approx 35$ & 0 \\
		Kalman (Exp)      & $\approx 18$--20 & $\approx 35$--40 & $\approx 0$ \\
		Obs. rápido (Exp) & $\approx 18$--20 & $\approx 35$--40 & $\approx 0$--2 \\
		\bottomrule
	\end{tabular}
\end{table}

\subsection*{Conclusiones}

Para el \textbf{primer caso LQR} (dinámica rápida), tanto el estimador de Kalman como el observador rápido lograron tiempos de subida y asentamiento muy próximos a los valores esperados. Sin embargo, en las mediciones reales ambos presentan un leve sobreimpulso, atribuible a tolerancias de componentes, errores de cuantización y pequeñas variaciones de ganancia en la planta.

En este régimen rápido se observa una diferencia importante entre ambos estimadores:  
\begin{itemize}
	\item El \textbf{observador rápido} muestra variaciones altas luego del transitorio, especialmente en el estado $x_1$, donde se inyecta el ruido.  
	\item El \textbf{Kalman} suaviza significativamente la estimación y mantiene el estado más estable.  
\end{itemize}

En la salida $y$ (que corresponde al estado $x_2$), estas diferencias se atenúan porque el bloque físico entre $x_1$ y $x_2$ incluye un filtro RC que mitiga el ruido, aunque las oscilaciones siguen siendo apreciables en el observador rápido.

\vspace{1em}

Para el \textbf{segundo caso LQR} (dinámica más lenta), el desempeño del filtro de Kalman fue notablemente superior.  
\begin{itemize}
	\item El \textbf{estado $x_1$} casi no muestra ruido con Kalman, pero sí presenta oscilaciones importantes con el observador rápido.  
	\item En el \textbf{estado $x_2$ (la salida)}, Kalman elimina prácticamente todo el ruido, mientras que el observador rápido deja ver pequeñas ondulaciones residuales.  
\end{itemize}

\vspace{1em}

Algo interesante es que \textbf{en ninguno de los dos casos los observadores impusieron sus polos}.  
Esto ocurre porque:
\begin{itemize}
	\item En el caso del \textbf{Kalman}, la varianza del ruido real ($V_{\text{pp}} = 10$ mV) no fue lo suficientemente alta como para forzar que el filtro privilegie al modelo antes que a la medición; por eso la planta dominó la dinámica del estimador.
	\item En el \textbf{observador rápido}, sus polos siempre fueron mucho más rápidos que los de la planta, por lo que la dinámica dominante siguió siendo la de la planta física.
\end{itemize}

\vspace{1em}

Finalmente, los resultados demuestran que un \textbf{buen modelo de la planta} permite obtener un estimador de alto desempeño mediante el filtro de Kalman.  
Kalman se adapta a las condiciones reales sin necesidad de imponer polos artificialmente rápidos, y logra filtrar de manera efectiva el ruido sobre los estados, especialmente cuando la dinámica del sistema no es extremadamente rápida.



	
	%\onecolumn
\appendices
\section{Códigos de Matlab}

% --- Plantilla: pegá el contenido de cada archivo entre lstlisting ---
% Sugerencia: si te queda DENSO, podés intercalar \clearpage entre archivos.

\subsection{\texttt{comparar\_intersample.m}}
\begin{lstlisting}[style=matlabstyle,caption={Comparación inter-muestra.}]
	function comparar_intersample(S_all, T)
	% COMPARAR_INTERSAMPLE - Compara esfuerzos y salidas continuas
	% Uso:
	%   comparar_intersample(S_all, T)
	% Entradas:
	%   S_all : cell array de structs devueltos por ver_intersample
	%   T     : vector de Ts correspondientes
	
	estilos = {'-','--',':','-.'}; % distintos estilos
	colores = lines(numel(T));     % paleta distinta para cada curva
	leg = arrayfun(@(Ts) sprintf('T_s = %.4g ms', Ts*1e3), T, 'UniformOutput', false);
	
	figure('Name','Comparativa esfuerzos y respuestas continuas');
	tiledlayout(2,1);
	
	% --- Subplot 1: esfuerzos ---
	ax1 = nexttile; hold on; grid on; grid minor;
	for k = 1:numel(T)
	est = estilos{mod(k-1,numel(estilos))+1};
	stairs(S_all{k}.tk*1e3, S_all{k}.uk, est, ...
	'Color', colores(k,:), 'LineWidth', 1);
	end
	xlabel('milisegundos');
	ylabel('Voltios');
	title('Esfuerzos del controlador para distintos T_s');
	legend(leg, 'Location', 'best');
	
	% --- Subplot 2: salidas continuas ---
	ax2 = nexttile; hold on; grid on; grid minor;
	for k = 1:numel(T)
	est = estilos{mod(k-1,numel(estilos))+1};
	plot(S_all{k}.tc*1e3, S_all{k}.yc, est, ...
	'Color', colores(k,:), 'LineWidth', 1);
	end
	xlabel('milisegundos');
	ylabel('Voltio');
	title('Respuesta continua de G(s) con u_{ZOH}(t)');
	legend(leg, 'Location', 'best');
	
	linkaxes([ax1 ax2],'x'); % sincroniza zoom en X
	end
\end{lstlisting}

\subsection{\texttt{calculos_auxiliares.m}
\begin{lstlisting}[style=matlabstyle,caption={Utilidades auxiliares.}]
	% Ejemplo simple de simulación con saturación
	clear all
	close all
	% Planta continua: G(s) = 1/(s+1)
	s = tf('s');
	G = 1/((s+1));
	
	% Controlador discreto PI: C(z) = Kp + Ki*T/(z-1)
	Kp = 1; Ki = 0.001; T = 0.01;
	C = c2d(pid(Kp,Ki), T, 'tustin'); % discreto
	
	% Parámetros simulación
	N = 500;
	refd = ones(N,1);      % escalón
	umin = -2; umax = 2;   % saturación
	
	% Discretizar planta
	Gd = c2d(G,T,'zoh');
	zpk( feedback(C*Gd,1))
	% Obtener coeficientes
	[Nc,Dc] = tfdata(C,'v');  Nc = Nc(:)'; Dc = Dc(:)';
	[Nb,Db] = tfdata(Gd,'v'); Nb = Nb(:)'; Db = Db(:)';
	
	% Inicializar
	yc = zeros(N,1);   % salida de planta
	u  = zeros(N,1);   % señal aplicada
	uc = zeros(N,1);   % señal de controlador (sin saturación)
	e  = zeros(N,1);
	
	for k=1:N
	e(k) = refd(k) - yc(k);
	
	% --- controlador discreto (uc) ---
	uc_k = 0;
	for i=0:length(Nc)-1
	if k-i>=1, uc_k = uc_k + Nc(i+1)*e(k-i); end
	end
	for i=1:length(Dc)-1
	if k-i>=1, uc_k = uc_k - Dc(i+1)*uc(k-i); end
	end
	uc(k) = uc_k;          % señal antes de saturación
	u(k)  = min(max(uc_k,umin), umax);  % saturada
	
	% --- planta discreta ---
	yk = 0;
	for i=0:length(Nb)-1
	if k-i>=1, yk = yk + Nb(i+1)*u(k-i); end
	end
	for i=1:length(Db)-1
	if k-i>=1, yk = yk - Db(i+1)*yc(k-i); end
	end
	if k<N, yc(k+1)=yk; end
	end
	
	% Gráficos
	t = (0:N-1)*T;
	subplot(3,1,1); plot(t,yc); ylabel('y[k]')
	subplot(3,1,2); plot(t,u); ylabel('u[k]')
	subplot(3,1,3); plot(t,e); ylabel('e[k]'); xlabel('t [s]')
\end{lstlisting}

\subsection{\texttt{gen\_ref\_pulso\_blocks.m}}
\begin{lstlisting}[style=matlabstyle,caption={Generador de referencia por pulsos (bloques).}]
	function [t, refd] = gen_ref_pulso_blocks(Ts_ref, T_total, ref_amp, pulse_amp, f_hz)
	% Alterna bloques de alto/bajo alrededor de ref_amp con frecuencia f_hz.
	% Alto = ref_amp + pulse_amp
	% Bajo = ref_amp - pulse_amp
	
	N      = max(1, round(T_total / Ts_ref));     % total de muestras
	t      = (0:N-1)' * Ts_ref;
	Nblk   = max(1, round((1/(2*f_hz)) / Ts_ref));% muestras por medio periodo
	
	hi = ref_amp + pulse_amp;
	lo = ref_amp - pulse_amp;
	
	pattern = [repmat(hi, Nblk, 1); repmat(lo, Nblk, 1)];
	refd = repmat(pattern, ceil(N/(2*Nblk)), 1);
	refd = refd(1:N);
	end
\end{lstlisting}

\subsection{\texttt{gen\_ref\_rampa.m}}
\begin{lstlisting}[style=matlabstyle,caption={Generador de referencia tipo rampa.}]
	% === Pegar aquí el contenido de gen_ref_rampa.m ===
\end{lstlisting}

\subsection{\texttt{guardar\_resultados.m}}
\begin{lstlisting}[style=matlabstyle,caption={Guardado de resultados y figuras.}]
	%% ================== Guardar figuras e informe ==================
	% Carpeta de salida
	outDir = fullfile(pwd, 'resultados');
	figDir = fullfile(outDir, 'figuras');
	if ~exist(outDir, 'dir'), mkdir(outDir); end
	if ~exist(figDir, 'dir'), mkdir(figDir); end
	
	% --- 1) Guardar TODAS las figuras abiertas (las que genera comparar_intersample y otras)
	figs = findall(0, 'Type', 'figure');
	tsStamp = datestr(now, 'yyyymmdd_HHMMSS');  % timestamp único
	for i = 1:numel(figs)
	try
	set(figs(i), 'PaperPositionMode', 'auto');
	fname = sprintf('fig_%02d_%s.png', i, tsStamp);
	print(figs(i), fullfile(figDir, fname), '-dpng', '-r300');
	catch ME
	warning('No pude exportar la figura %d: %s', i, ME.message);
	end
	end
	%%
	
	% --- 2) Armar archivo .txt con: Ts | Planta(G) | C_m1 | C_m2 (solo FT)
	txtPath = fullfile(outDir, sprintf('resumen_control_%s.txt', tsStamp));
	fid = fopen(txtPath, 'w');
	if fid < 0
	error('No pude crear el archivo de resumen en %s', txtPath);
	end
	
	fprintf(fid, 'Resumen de control por periodo de muestreo\n');
	fprintf(fid, 'Formato: Ts [s] | Planta G | C_m1 | C_m2\n');
	fprintf(fid, '---------------------------------------------------------------\n');
	
	% String "limpio" de G continua (única para todas las filas)
	
	Nrows = min([numel(T), numel(C_all_m1), numel(C_all_m2)]);
	for k = 1:Nrows
	Ts_k   = T(k);
	G_show = tf2str_z(Gd_all_m2{k});
	C1_show = tf2str_z(C_all_m1{k});
	C2_show = tf2str_z(C_all_m2{k});
	fprintf(fid, '%.12g | %s | %s | %s\n', Ts_k, G_show, C1_show, C2_show);
	end
	fclose(fid);
	
	% Guarda también el workspace por si querés reusar
	save(fullfile(outDir, sprintf('workspace_%s.mat', tsStamp)), ...
	'T','G','C_all_m1','C_all_m2','Gd_all_m1','Gd_all_m2');
	fprintf('Listo.\nResumen: %s\n', txtPath);
	
	function s = tf2str_simple(sys)
	% Devuelve "(num)/(den)" en una línea.
	% - Si es continuo: variable 's'
	% - Si es discreto: variable 'z^-1' + sample time
	[num, den, Ts] = tfdata(sys, 'v');
	if isct(sys)
	s = sprintf('(%s)/(%s)', poly2str(num,'s'), poly2str(den,'s'));
	else
	s = sprintf('(%s)/(%s); Ts=%.10g', poly2str(num,'z^-1'), poly2str(den,'z^-1'), Ts);
	end
	s = regexprep(s, '\s+', ' '); % compactar espacios
	end
	function s = tf2str_z(sys)
	% Escribe la FT discreta en variable 'z' (no z^-1)
	assert(~isct(sys),'Solo discreto');
	[num, den, ~] = tfdata(sys,'v');  % coef. en z^-1 descendentes
	% Convertimos a polinomios en z multiplicando por z^-N y revirtiendo:
	num_z = fliplr(num); den_z = fliplr(den);
	s = sprintf('(%s)/(%s);', poly2str(num_z,'z'), poly2str(den_z,'z'));
	s = regexprep(s, '\s+', ' ');
	end
\end{lstlisting}

\subsection{\texttt{hojaDeCalculos.m}}
\begin{lstlisting}[style=matlabstyle,caption={Hoja de cálculos para el diseño de los compensadores.}]
	close all
	clear all
	
	addpath('..\Lab1\');
	addpath('..\Lab2\');
	addpath('..\Lab3\');
	
	%% Definicion de parametros
	R_1 = 15e3;
	R_3 = 15e3;
	C_2 = 100e-9;
	
	R_2 = 82e3;
	R_4 = 82e3;
	C_1 = 0.22e-6;
	
	%% Generar funcion de transferencia d
	numStage = [-R_3/R_1 -R_4/R_2];
	denStage = { [C_2*R_3 1], [C_1*R_4 1] };
	
	% Usamos celdas para guardar los tf de cada stage
	Gstage = cell(1,2);
	G = 1;
	for i = 1:2
	Gstage{i} = tf(numStage(i), denStage{i});
	G = G*Gstage{i};
	end
	
	%% Analizamos en el tiempo
	[tr, ts, wn] = plot_step_info(G);
	disp('Planta continua G(s):')
	G
	zpk(G)
	
	
	%% ================== Barrido sobre un vector de T ==================
	% Vector de periodos de muestreo a probar (ejemplo: usar T1, T2 y más)
	Tn = 2*pi/wn;
	T = [1.25e-3 Tn/32 Tn/16 Tn/8 Tn/4 Tn/2];   % <-- ajustá a gusto
	
	
	% Parámetros de la simulación para ver_intersample
	Tsim = Tn; 
	Nups = 40;   % sobremuestreo del ZOH (submuestras por periodo)
	
	% Resultados de cada Ts
	S_all_m1 = cell(1, numel(T));
	C_all_m1 = cell(1, numel(T));
	Gd_all_m1 = cell(1, numel(T));
	for k = 1:numel(T)
	Ts = T(k);
	
	% Discretizar la planta con ZOH a Ts
	Gd_k = c2d(G, Ts, 'zoh');
	
	% Método 1 (oscilaciones entre muestras): F = z^-1 ; C = F/(Gd*(1-F))
	z_k = tf([1 0], 1, Ts);
	C_k = 1/(Gd_k)*1/(z_k-1);
	
	% Simulación intersample (usa tu función ver_intersample)
	S_k = ver_intersample(G, C_k, round(Tsim/T(k)), Nups);  %#ok<NASGU>
	
	% Guardar
	S_all_m1{k}  = S_k;
	C_all_m1{k}  = C_k;
	Gd_all_m1{k} = Gd_k;
	end
	
	comparar_intersample(S_all_m1, T);
	
	
	% %% no funciona
	% % parámetros
	% C = C_all_m1{1};
	% Gd = Gd_all_m1{1};
	% Ts_ref  = C.Ts;     % 0.5 ms entre muestras de referencia
	% 
	% % 1) pulso montado
	% f = 1/(Ts*300);
	% Tfin = 3/f;
	% [t, refd] = gen_ref_pulso_blocks(Ts,Tfin,2.05,0.5,f); umin=0; umax=5;
	% figure;
	% plot(t,refd);
	% [yd,ud,ucd,ed,t] = sim_lazo_discreto_sat(Gd,C,refd,umin,umax);
	% 
	% figure;
	% subplot(3,1,1); stairs(t,yd); grid on; ylabel('y[k]');hold on;stairs(t,refd)
	% subplot(3,1,2); stairs(t,ud,'-');hold on; stairs( t,ucd,'--'); grid on; ylabel('u[k]'); legend('u','uc')
	% subplot(3,1,3); stairs(t,ed); grid on; ylabel('e[k]'); xlabel('t [s]')
	% 
	% 
	% Nups = 40;           % submuestras por periodo (p.ej. 40)
	% Muse = 0;            % 0 => usar todo u[k]
	% S = ver_intersample_desde_u(G, Ts, ud, Muse, Nups);
	
	
	%% ================== Metodo 2 (orden 1 del numerador A) ==================
	S_all_m2 = cell(1, numel(T));
	C_all_m2 = cell(1, numel(T));
	Gd_all_m2 = cell(1, numel(T));
	for k = 1:numel(T)
	Ts = T(k);
	
	% Planta discreta
	Gd_k = c2d(G, Ts, 'zoh');
	
	% === A(z): orden 1 (A = a1 z + a0). En z^{-1} MATLAB da [a0 a1].
	Azinv = Gd_k.Numerator{1};
	if numel(Azinv) < 2
	error('El numerador de Gd_k no es de orden 1.');
	end
	a0 = Azinv(3); 
	a1 = Azinv(2);
	
	% === K y b0 (¡correctos!)
	denom = a1 + a0;
	if abs(denom) < 1e-12
	error('a1 + a0 = 0 -> no hay solución con l=2 y (z-1)B.');
	end
	K  = 1/denom;
	b0 = a0/denom;
	
	% === B(z) = z + b0 (en variable z, mayor->menor)
	Bz = [1, b0];
	
	% === D(z): quita polos en z=1 del denominador de Gd
	p = zpk(Gd_k).P{1};
	has_pole_at_1 = any(abs(p - 1) < 1e-6);
	p_no1 = p(abs(p - 1) >= 1e-6);
	Dz = poly(p_no1);              % en z (mayor->menor)
	if max(abs(imag(Dz))) < 1e-12, Dz = real(Dz); end
	
	% === Denominador del controlador según tenga o no polo en z=1
	if has_pole_at_1
	% Caso Eq. (6.54): C(z) = K*D(z) / (z + b0)
	denC_z = Bz;
	else
	% Caso Eq. (6.51): C(z) = K*D(z) / [(z-1)(z + b0)]
	denC_z = conv([1 -1], Bz);
	end
	
	C_k = tf(K * Dz, denC_z, Ts);
	
	% Simulación inter-muestra
	S_k = ver_intersample(G, C_k, round(Tsim/Ts), Nups);
	
	% Guardar
	S_all_m2{k}  = S_k;
	C_all_m2{k}  = C_k;
	Gd_all_m2{k}  = Gd_k;
	end
	
	comparar_intersample(S_all_m2, T);
	
	
	
	% %% no funciona
	% % parámetros
	% C = C_all_m2{1};
	% Gd = Gd_all_m2{1};
	% Ts_ref  = C.Ts;     % 0.5 ms entre muestras de referencia
	% 
	% % 1) pulso montado
	% f = 1/(Ts*300);
	% Tfin = 3/f;
	% [t, refd] = gen_ref_pulso_blocks(Ts,Tfin,2.05,0.5,f); umin=-300; umax=500;
	% figure;
	% plot(t,refd);
	% [yd,ud,ucd,ed,t] = sim_lazo_discreto_sat(Gd,C,refd,umin,umax);
	% 
	% figure;
	% subplot(3,1,1); stairs(t,yd); grid on; ylabel('y[k]');hold on;stairs(t,refd)
	% subplot(3,1,2); stairs(t,ud,'-');hold on; stairs( t,ucd,'--'); grid on; ylabel('u[k]'); legend('u','uc')
	% subplot(3,1,3); stairs(t,ed); grid on; ylabel('e[k]'); xlabel('t [s]')
	% 
	% 
	% Nups = 40;           % submuestras por periodo (p.ej. 40)
	% Muse = 0;            % 0 => usar todo u[k]
	% S = ver_intersample_desde_u(G, Ts, ud, Muse, Nups);
\end{lstlisting}

\subsection{\texttt{pruebasMeQuieroMatar.m}}
\begin{lstlisting}[style=matlabstyle,caption={Banco de pruebas y exploración de parámetros.}]
	% === Pegar aquí el contenido de pruebasMeQuieroMatar.m ===
\end{lstlisting}

\subsection{\texttt{sim\_lazo\_discreto\_sat.m}}
\begin{lstlisting}[style=matlabstyle,caption={Simulación de lazo discreto con saturación.}]
	function [yd, ud, ucd, ed, t] = sim_lazo_discreto_sat(Gd, Cd, refd, umin, umax)
	% SIM_LAZO_DISCRETO_SAT
	% Simula un lazo discreto con controlador Cd y planta Gd (ambos tf discretos),
	% separando la señal de control antes de saturación (uc) y la aplicada (u).
	%
	% Entradas:
	%   Gd   : planta discreta (tf) en z, causal
	%   Cd   : controlador discreto (tf) en z, causal
	%   refd : referencia discreta r[k] (vector fila o columna)
	%   umin, umax : límites de saturación del esfuerzo
	%
	% Salidas:
	%   yd  : salida de la planta y[k] (mismo largo que refd)
	%   ud  : esfuerzo aplicado u[k] (saturado)
	%   ucd : esfuerzo del controlador sin saturar u_c[k]
	%   ed  : error e[k] = r[k] - y[k]
	%   t   : tiempo discreto (k*Ts)
	
	Ts = 0; try, Ts = Cd.Ts; catch, end
	if isempty(Ts) || Ts <= 0, error('Cd debe tener Ts > 0.'); end
	
	% ===== Coeficientes en z^{-1} y normalización a(0)=1 =====
	[Nc, Dc] = tfdata(Cd, 'v');  Nc = Nc(:).'; Dc = Dc(:).';
	[Nb, Db] = tfdata(Gd, 'v');  Nb = Nb(:).'; Db = Db(:).';
	
	if abs(Dc(1)) < 1e-12, error('Controlador impropio/no causal: Dc(1)=0'); end
	if abs(Db(1)) < 1e-12, error('Planta discreta impropia/no causal: Db(1)=0'); end
	
	% Normalizar para que los denominadores empiecen en 1
	Nc = Nc / Dc(1);   Dc = Dc / Dc(1);
	Nb = Nb / Db(1);   Db = Db / Db(1);
	
	% ===== Preparar señales =====
	refd = refd(:);           % columna
	N = numel(refd);
	t = (0:N-1).' * Ts;
	
	% y con un paso extra para causalidad explícita (y[k+1])
	y  = zeros(N+1,1);
	ed = zeros(N,1);
	ucd = zeros(N,1);   % señal del controlador (sin saturación)
	ud  = zeros(N,1);   % señal aplicada (saturada)
	
	% ===== Bucle causal =====
	% Ecuaciones:
	%   Dc(q^-1) * uc[k] = Nc(q^-1) * e[k]
	%   Db(q^-1) * y[k]  = Nb(q^-1) * u[k]
	%
	% Ojo: acá actualizamos y(k+1) para respetar el retardo de ZOH.
	
	for k = 1:N
	% error con la salida disponible del mismo índice (y[k])
	ed(k) = refd(k) - y(k);
	
	% ---- Controlador (sin saturación): uc[k] ----
	uc_k = 0.0;
	
	% término por e[k-i]
	for i = 0:length(Nc)-1
	if k-i >= 1
	uc_k = uc_k + Nc(i+1) * ed(k-i);
	end
	end
	
	% realimentación por uc[k-i], i >= 1
	for i = 1:length(Dc)-1
	if k-i >= 1
	uc_k = uc_k - Dc(i+1) * ucd(k-i);
	end
	end
	
	ucd(k) = uc_k;
	
	% ---- Saturación y esfuerzo aplicado ----
	ud(k) = min(max(uc_k, umin), umax);
	
	% ---- Planta: y[k+1] con u[k-i] e y[k-i] ----
	y_next = 0.0;
	
	% parte directa por u[k-i]
	for i = 0:length(Nb)-1
	if k-i >= 1
	y_next = y_next + Nb(i+1) * ud(k-i);
	end
	end
	
	% realimentación por y[k-i], i >= 1
	for i = 1:length(Db)-1
	if k-i >= 1
	y_next = y_next - Db(i+1) * y(k-i+1);  % nota: y es (k+1)
	end
	end
	
	% avanzar estado de la planta
	y(k+1) = y_next;
	end
	
	% Salida alineada con refd
	yd = y(1:N);
	end
\end{lstlisting}

\subsection{\texttt{ver\_intersample.m}}
\begin{lstlisting}[style=matlabstyle,caption={Visualización del comportamiento inter-muestra.}]
	function S = ver_intersample(G, C, M, N)
	% VER_INTERSAMPLE - Visualiza oscilaciones entre-muestras con ZOH explícito
	% Uso:
	%   S = ver_intersample(G, C, M, N)
	% Entradas:
	%   G : planta continua (tf/ss continuo)
	%   C : controlador digital (tf/ss discreto, con Ts > 0)
	%   M : 0 => simular hasta t = SettlingTime(G)
	%       >0 => simular M*Ts
	%   N : factor de sobremuestreo del ZOH (submuestras por periodo, ej. 30)
	
	Ts = getTsStrict(C);
	if Ts <= 0
	error('El controlador C debe ser discreto con Ts > 0.');
	end
	
	% Discretizar la planta
	Gd = c2d(G, Ts, 'zoh');
	
	% Lazos cerrados
	Gcl = feedback(C*Gd, 1);  % r->y[k]
	Ucl = feedback(C, Gd);    % r->u[k]
	
	Tfinal = M*Ts;
	
	K = max(1, floor(Tfinal/Ts));
	tk = (0:K).' * Ts;
	
	% Respuestas discretas
	[yk, ~] = step(Gcl, tk);
	rk = ones(size(yk));
	ek = rk - yk;
	uk = lsim(C, ek, tk);
	
	% ZOH explícito
	[tc, uc] = zoh_stretch(uk, Ts, N);
	yc = lsim(G, uc, tc);
	
	% ====== SUBPLOTS (x en ms) ======
	figure('Name','Intersample analysis');
	tiledlayout(2,1);
	
	% y_c(t) y y[k]
	nexttile;
	plot(tc*1e3, yc, 'LineWidth',1.2); grid on; hold on; grid minor;
	stairs(tk*1e3, yk, 'k.');
	xlabel('milisegundos'); ylabel('Voltios');
	title(sprintf('Respuesta de la planta controlada al escalón T_s = %.4g ms',Ts*1e3));
	legend('y_c(t)','y[k]','Location','best');
	
	% u[k]
	nexttile;
	stairs(tk*1e3, uk, 'LineWidth',1.5); grid on; grid minor;
	xlabel('milisegundos'); ylabel('Voltios');
	title('Esfuerzo del controlador para el escalón');
	legend('u[k]','Location','best');
	
	% Salida estructurada
	S = struct('Ts', Ts, 'Tfinal', tk(end), ...
	'tk', tk, 'yk', yk, 'uk', uk, 'ek', ek, ...
	'tc', tc, 'uc', uc, 'yc', yc, ...
	'Gd', Gd, 'Gcl', Gcl, 'Ucl', Ucl);
	end
	
	% Helpers
	function Ts = getTsStrict(sys)
	Ts = 0;
	try, Ts = sys.Ts; catch, end
	if isempty(Ts), Ts = 0; end
	end
	
	function [t_hi, u_hi] = zoh_stretch(u_k, T, M)
	if size(u_k,2) > 1, u_k = u_k(:); end
	u_hi = repelem(u_k, M);
	t_hi = (0:numel(u_hi)-1).' * (T/M);
	end
\end{lstlisting}

\subsection{\texttt{ver\_intersample\_desde\_u.m}}
\begin{lstlisting}[style=matlabstyle,caption={Reconstrucción inter-muestra desde $u[k]$.}]
	function S = ver_intersample_desde_u(G, Ts, ud, M, N)
	% VER_INTERSAMPLE_DESDE_U - Simula inter-muestra con G(s) y un esfuerzo u[k] dado.
	% Uso:
	%   S = ver_intersample_desde_u(G, Ts, ud, M, N)
	% Entradas:
	%   G  : planta continua (tf/ss continuo)
	%   Ts : periodo de muestreo del esfuerzo u[k]
	%   ud : vector de esfuerzo discreto u[k] (fila o columna)
	%   M  : (opcional) nº de muestras a usar
	%        0 o [] => usar todo ud
	%        >0     => usar las primeras M muestras
	%   N  : (opcional) sobremuestreo del ZOH (submuestras por periodo, ej. 30; default 30)
	%
	% Salida (struct):
	%   S.Ts, S.tk, S.uk, S.yk       : señales discretas (k*Ts)
	%   S.tc, S.uc, S.yc             : señales continuas (ZOH + lsim(G))
	%   (y figura con y_c(t) y y[k], y u[k])
	
	if nargin < 5 || isempty(N), N = 30; end
	if nargin < 4 || isempty(M), M = 0;  end
	if Ts <= 0, error('Ts debe ser > 0.'); end
	
	% --- Acomodar u[k] y recorte opcional ---
	uk = ud(:);                      % columna
	if M > 0
	M = min(M, numel(uk));
	uk = uk(1:M);
	end
	
	K  = numel(uk) - 1;              % última muestra = k=K
	tk = (0:K).' * Ts;
	
	% --- Reconstrucción inter-muestra por ZOH explícito ---
	[tc, uc] = zoh_stretch(uk, Ts, N);    % uc(t) pieza-constante
	yc       = lsim(G, uc, tc);           % salida continua
	
	% Muestras de la salida continua exactamente en k*Ts
	yk = yc(1:N:end);   % cada N submuestras corresponde a un instante k*Ts
	% (Por construcción, length(yk) == length(uk) == length(tk))
	
	% --- Plots (eje en ms) ---
	figure('Name','Intersample desde u[k]');
	tiledlayout(2,1);
	
	% y_c(t) y y[k]
	nexttile; hold on; grid on; grid minor;
	plot(tc*1e3, yc, 'LineWidth', 1.2);
	stairs(tk*1e3, yk, 'k.', 'LineWidth', 1.0);
	xlabel('t [ms]'); ylabel('y');
	title(sprintf('Salida continua y_c(t) y muestras y[k]  (T_s = %.4g ms)', Ts*1e3));
	legend('y_c(t)','y[k]','Location','best');
	
	% u[k]
	nexttile; hold on; grid on; grid minor;
	stairs(tk*1e3, uk, 'LineWidth', 1.2);
	xlabel('t [ms]'); ylabel('u[k]');
	title('Esfuerzo aplicado');
	
	% --- Salida estructurada ---
	S = struct('Ts', Ts, ...
	'tk', tk, 'uk', uk, 'yk', yk, ...
	'tc', tc, 'uc', uc, 'yc', yc);
	end
	
	% ===== Helper: ZOH explícito =====
	function [t_hi, u_hi] = zoh_stretch(u_k, T, M)
	% Repite cada muestra M veces (ZOH) y arma el vector de tiempo continuo
	if size(u_k,2) > 1, u_k = u_k(:); end
	u_hi = repelem(u_k, M);
	t_hi = (0:numel(u_hi)-1).' * (T/M);
	end
	
	
	
\end{lstlisting}


	\balance
\end{document}
