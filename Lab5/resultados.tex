\newpage
\section{Resultados}
\insertarfigurawide{matlab/comp1.png}
{Comparación simulación vs.\ experimento — Caso~1. Métricas detalladas (RMSE y errores máximos) indicadas en cada panel.}
{fig:comp1}{1}


[INSERTAR FIGURA]
%\insertarfigurawide{matlab/comp2.png}
%{Comparación simulación vs.\ experimento — Caso~2. Métricas detalladas (RMSE y errores máximos) indicadas en cada panel.}
%{fig:comp2}{1}
\subsection{Comparación simulación vs. experimento}

La comparación entre la simulación y las mediciones experimentales se presenta en las Figuras~\ref{fig:comp1} y~\ref{fig:comp2}. 
En ambos casos se grafican, para las señales $x_1$, $x_2(=y)$ y $u$, la traza experimental frente a la simulada y, en paneles adyacentes, el error correspondiente. 
Las figuras fueron generadas directamente desde el script de \textit{Matlab} que remuestrea (por ZOH) las señales simuladas a la grilla temporal del osciloscopio, asegurando una comparación punto a punto usando la misma referencia. 
De esta forma se garantiza que los resultados reflejen únicamente las diferencias reales entre el modelo teórico y la implementación física.



\subsection{Métricas cuantitativas}

A continuación se resumen las métricas principales para cada señal: \textbf{RMSE absoluto}, \textbf{RMSE porcentual}, \textbf{error máximo absoluto} y \textbf{error máximo porcentual}. 
La Tabla~\ref{tab:metrics_exp1} corresponde al \textit{Caso 1 (Exp1)} y la Tabla~\ref{tab:metrics_exp2} queda preparada para el \textit{Caso 2 (Exp2)}.

\begin{table}[H]
	\centering
	\caption{Métricas de comparación — Caso 1 (Exp1).}
	\label{tab:metrics_exp1}
	\begin{tabular}{l
			S[table-format=1.6]
			S[table-format=2.3]
			S[table-format=1.6]
			S[table-format=3.3]}
		\toprule
		\textbf{Señal} & \textbf{RMSE\_abs} & \textbf{RMSE\_\%} & \textbf{ErrMax\_abs} & \textbf{ErrMax\_\%} \\
		\midrule
		X1 & 0.054706 & 15.542 & 0.25854 & 73.451 \\
		X2 & 0.056899 & 15.466 & 0.30790 & 83.691 \\
		U  & 0.298060 & 28.241 & 1.36910 & 129.72 \\
		\bottomrule
	\end{tabular}
\end{table}

\begin{table}[H]
	\centering
	\caption{Métricas de comparación — Caso 2 (Exp2).}
	\label{tab:metrics_exp2}
	\begin{tabular}{l
			S[table-format=1.3,table-omit-exponent]
			S[table-format=1.3,table-omit-exponent]
			S[table-format=1.3,table-omit-exponent]
			S[table-format=1.3,table-omit-exponent]}
		\toprule
		\textbf{Señal} & \textbf{RMSE\_abs} & \textbf{RMSE\_\%} & \textbf{ErrMax\_abs} & \textbf{ErrMax\_\%} \\
		\midrule
		X1 & {-} & {-} & {-} & {-} \\
		X2 & {-} & {-} & {-} & {-} \\
		U  & {-} & {-} & {-} & {-} \\
		\bottomrule
	\end{tabular}
\end{table}

\subsection{Discusión y conclusiones}

Se observa que la respuesta de la planta sigue la forma esperada: el \textit{rising time} es muy similar al de la simulación y la señal permanece dentro de los valores previstos. 
El error medio se concentra alrededor de cero salvo en las zonas próximas a la saturación del actuador, donde aumentan las discrepancias, atribuibles a no linealidades de la planta y a las limitaciones del DAC. 
Además, el esfuerzo medido se aprecia como una \emph{versión achatada} del esfuerzo simulado, consistente con la compresión que introduce la saturación y el rango dinámico reducido del hardware.

Es importante notar que las simulaciones no consideran efectos de muestreo imperfecto, ruido en las mediciones ni perturbaciones eléctricas presentes en la práctica. 
Dado que las señales medidas tienen amplitudes muy pequeñas, incluso un nivel bajo de ruido produce diferencias visibles, lo cual explica parte de las discrepancias observadas entre el modelo y el comportamiento real. 
En conjunto, los resultados confirman que los controladores \emph{implementables} reproducen la dinámica cualitativa diseñada, con un nivel de error razonable para el contexto experimental y dentro de los márgenes esperados para este tipo de implementación física.
