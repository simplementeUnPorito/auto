% !TeX program = pdflatex
\documentclass[conference]{IEEEtran}
\pagestyle{plain}

% ======= Idioma y codificación =======
\usepackage[spanish, es-nodecimaldot]{babel}
\usepackage[utf8]{inputenc}
\usepackage[T1]{fontenc}

% ======= Bibliografía (IEEE + natbib numérico) =======
\usepackage[numbers]{natbib}
\bibliographystyle{IEEEtranN}

% ======= Matemática y tipografía =======
\usepackage{amsmath, amssymb, amsfonts}
\usepackage{microtype}
\usepackage{float}           % (déjalo una sola vez)

% ======= Números y unidades =======
\usepackage{siunitx}
\sisetup{
	output-decimal-marker = {,}, % coma decimal
	group-separator = {.},
	group-minimum-digits = 4,
	per-mode = symbol,
	round-mode = places,
	round-precision = 3,
	table-number-alignment = center
}

% ======= Gráficos y floats =======
\usepackage{graphicx}
% \usepackage{subcaption}   % <-- QUITAR esto en IEEEtran
\usepackage[caption=false,font=footnotesize]{subfig} % <-- USAR ESTO
\usepackage{booktabs}
\usepackage{cuted}
\usepackage{balance}

% ======= Listas =======
\usepackage{enumitem}
\setlist{leftmargin=*, itemsep=2pt, topsep=4pt}

% ======= Colores (para listings) =======
\usepackage{xcolor}

% ======= LISTINGS =======
\usepackage{listings}
\usepackage{listingsutf8}    % ok con pdfLaTeX
\lstset{
	inputencoding=utf8,
	extendedchars=true,
	basicstyle=\ttfamily\small,
	breaklines=true, columns=fullflexible, keepspaces=true,
	frame=single, numbers=left, numberstyle=\tiny, numbersep=6pt,
	xleftmargin=1em,
	captionpos=b,
	literate=
	{á}{{\'a}}1 {é}{{\'e}}1 {í}{{\'\i}}1 {ó}{{\'o}}1 {ú}{{\'u}}1
	{Á}{{\'A}}1 {É}{{\'E}}1 {Í}{{\'I}}1 {Ó}{{\'O}}1 {Ú}{{\'U}}1
	{ñ}{{\~n}}1 {Ñ}{{\~N}}1
	{ü}{{\"u}}1 {Ü}{{\"U}}1
	{¿}{{\textquestiondown}}1 {¡}{{\textexclamdown}}1
	{º}{{$^\circ$}}1
}

% Lenguajes
\lstdefinelanguage{Matlab}{
	morekeywords={break,case,catch,continue,else,elseif,end,for,function,global,if,otherwise,persistent,return,switch,try,while},
	sensitive=true,
	morecomment=[l]\%,
	morestring=[m]'
}
\lstdefinestyle{matlabstyle}{
	language=Matlab,
	basicstyle=\ttfamily\small,
	keywordstyle=\bfseries\color{blue!70!black},
	commentstyle=\itshape\color{green!40!black},
	stringstyle=\color{red!60!black},
	numbers=left, numberstyle=\tiny, numbersep=6pt,
	frame=single, framerule=0.4pt,
	columns=fullflexible, keepspaces=true,
	showstringspaces=false,
	breaklines=true,
	tabsize=4,
	captionpos=b,
	linewidth=\columnwidth
}

\lstdefinestyle{cstyle}{
	language=C,
	inputencoding=utf8,
	basicstyle=\ttfamily\small,
	keywordstyle=\color{blue}\bfseries,
	stringstyle=\color{red!60!black},
	commentstyle=\color{green!40!black}\itshape,
	numbers=left, numberstyle=\tiny, stepnumber=1, numbersep=8pt,
	showstringspaces=false, breaklines=true, columns=fullflexible, keepspaces=true,
	frame=single, rulecolor=\color{black!20},
	linewidth=\columnwidth, xleftmargin=1em
}

% ======= Macros rápidas =======
\newcommand{\zetaD}{\zeta}
\newcommand{\wn}{\omega_n}
\newcommand{\Ts}{T_s}
\newcommand{\ESS}{\mathrm{ESS}}
\newcommand{\Gz}{G(z)}
\newcommand{\Gs}{G(s)}
\lstset{
	mathescape=false,  % asegura que $ no abre modo matemático dentro del listing
	upquote=true       % comillas rectas en verbatim
}

% ======= COMPARAR FIGURAS =======
\newcommand{\compararfigsC}[8]{%
	\begin{figure}[H]
		\centering
		\subfloat[#2\label{#3}]{%
			\includegraphics[width=0.49\columnwidth]{#1}}
		\hspace{0.3em}
		\subfloat[#5\label{#6}]{%
			\includegraphics[width=0.49\columnwidth]{#4}}
		\caption{#7}
		\label{#8}
	\end{figure}%
}

% Macro para figura simple
\newcommand{\insertarfigura}[4]{%
	\begin{figure}[H]
		\centering
		\includegraphics[width=#4\linewidth]{#1}
		\caption{#2}
		\label{#3}
	\end{figure}
}

% Macro figura a dos columnas
\newcommand{\insertarfigurawide}[5][]{%
	\begin{figure*}[#1]
		\centering
		\includegraphics[width=#5\textwidth]{#2}
		\caption{#3}
		\label{#4}
	\end{figure*}%
}

% ======= Hyperref al final del preámbulo =======
\usepackage[unicode,hidelinks]{hyperref}


% ======= Metadatos =======
\title{Práctica de Laboratorio 5: Ubicación Arbitraria de Polos.}

\author{
	\IEEEauthorblockN{Elías Álvarez}
	\IEEEauthorblockA{Carrera de Ing. Electrónica \\ 
		Universidad Católica Nuestra Señora de la Asunción \\
		Asunción, Paraguay \\
		Email: elias.alvarez@universidadcatolica.edu.py}
	\and
	\IEEEauthorblockN{Tania Romero}
	\IEEEauthorblockA{Carrera de Ing. Electrónica \\
		Universidad Católica Nuestra Señora de la Asunción \\
		Asunción, Paraguay \\
		Email: tania.romero@universidadcatolica.edu.py}
	\and
	\IEEEauthorblockN{\hspace*{2.3em}Docente: Lic. Montserrat González}
	\IEEEauthorblockA{
		\hspace*{2.3em}Facultad de Ingeniería \\
		\hspace*{2.3em}Universidad Católica Nuestra Señora de la Asunción \\
		\hspace*{2.3em}Asunción, Paraguay}
	\and
	\IEEEauthorblockN{\hspace*{2.3em}Docente: PhD. Enrique Vargas}
	\IEEEauthorblockA{\hspace*{2.3em}Facultad de Ingeniería \\  % ← empuja a la derecha
		\hspace*{2.3em}Universidad Católica Nuestra Señora de la Asunción \\
		\hspace*{2.3em}Asunción, Paraguay}
}







\begin{document}
	\maketitle
	\section{Introducción}


\section{Objetivos}

\begin{itemize}
	\item Diseñar controladores digitales para una planta analógica utilizando estados estimados.
	\item Aplicar el diseño de realimentación de estados con estimadores de predicción y de actualización.
	\item Presentar y discutir los resultados experimentales en comparación con los obtenidos por simulación, dentro de un informe técnico razonado.
\end{itemize}

	\section{Implementación del Sistema}

Durante la etapa de implementación no se intentó replicar directamente los controladores diseñados en \textit{Matlab} con las especificaciones originales, 
ya que se determinó que dichos parámetros no eran implementables con el hardware disponible. 
Los valores de ganancia requeridos generaban esfuerzos imposibles de aplicar sin saturar el DAC, 
por lo que incluso antes de probar en el PSoC se optó por \textbf{relajar las especificaciones} 
y diseñar controladores implementables que preservaran el comportamiento cualitativo del sistema.

En otras palabras, se mantuvo el mismo enfoque de control por realimentación de estados, 
pero con polos menos agresivos y ganancias adaptadas al rango físico del actuador. 
Aun así, los controladores “relajados” solamente pudieron trabajar con referencias muy pequeñas 
($0{,}4$~V para el Caso~1 y $0{,}3$~V para el Caso~2); 
valores mayores producían saturación inmediata y lo único observable en el osciloscopio era ruido amplificado.

El sistema digital completo se muestra en la Figura~\ref{fig:circuito_digital}. 
Allí se incluyen los convertidores \textsc{ADC SAR}, 
el circuito de sincronización con dos flip-flops tipo~D para detectar el fin de conversión de ambos canales, 
y un módulo \textsc{UART} utilizado para enviar comandos al PSoC. 
A través de este canal se podían modificar en tiempo real la frecuencia del temporizador, 
abrir o cerrar el lazo, y cambiar la amplitud de referencia, 
lo cual facilitó enormemente las pruebas experimentales.

[AÑADIR FOTO CIRCUITO]
%\insertarfigura{exp/circuitoDigital.png}
%{Circuito digital implementado en el PSoC, con los módulos SAR, flip-flops de sincronización y UART para control.}
%{fig:circuito_digital}{0.9}

La planta analógica se muestra en la Figura~\ref{fig:circuito_analogico}. 
Junto a ella se encuentra el DAC principal, que actúa como actuador aplicando los esfuerzos de control calculados. 
Además, se incorporó un segundo DAC destinado a generar la referencia en voltios; 
si bien esta referencia se define digitalmente, resultó útil disponer de su equivalente analógico para visualizarla en el osciloscopio durante las pruebas.


[AÑADIR FOTO CIRCUITO]
%\insertarfigura{exp/circuitoAnalogico.png}
%{Circuito analógico correspondiente a la planta física utilizada en el experimento.}
%{fig:circuito_analogico}{0.9}

La referencia digital se generó mediante un temporizador configurable 
y un registro tipo~T, definiendo el valor de referencia como $\mathrm{refA}/2$ 
cuando el bit lógico leído era~1 y $-\mathrm{refA}/2$ cuando era~0. 
De esta forma se obtuvo una señal cuadrada de amplitud controlada, 
útil para evaluar las respuestas transitorias del sistema.

Es importante destacar que todos los cálculos del controlador se realizaron 
desacoplando la componente de \textsc{DC}, 
ya que al no hacerlo se amplifica la componente continua en la etapa de salida 
y se produce saturación mucho más rápida.

En el firmware, para facilitar el traslado del diseño desde \textit{Matlab} a lenguaje~C, 
se implementó la librería \texttt{mat.c}. 
En ella se definieron los tipos de datos y funciones necesarios para realizar las operaciones matriciales requeridas en el cálculo del esfuerzo de control, 
utilizando únicamente las ganancias $K$ y $N_{\mathrm{var}}$.

Una vez calculado el esfuerzo de control (en valor alterno), 
se le sumó $V_{\mathrm{DD}}/2$ para montarlo sobre el nivel medio del DAC 
y mantener compatibilidad con el circuito analógico. 
De esta manera se logró que las mediciones en el osciloscopio 
fuesen consistentes con las simulaciones en \textit{Matlab}.

\insertarfigura{exp/openLoop.png}
{Respuesta experimental del sistema en lazo abierto.}
{fig:openloop}{0.85}

Las frecuencias de muestreo efectivas fueron de aproximadamente 
$418{,}5$~Hz para el Caso~1 y $313{,}9$~Hz para el Caso~2. 
Las Figuras~\ref{fig:exp1} y~\ref{fig:exp2} muestran las respuestas observadas experimentalmente.

\insertarfigura{exp/Exp1.png}
{Respuesta experimental correspondiente al Caso~1 ($f_s \approx 418{,}5$~Hz).}
{fig:exp1}{0.85}

\insertarfigura{exp/Exp2.png}
{Respuesta experimental correspondiente al Caso~2 ($f_s \approx 313{,}9$~Hz).}
{fig:exp2}{0.85}

Aunque las señales obtenidas presentan cierto nivel de ruido —algo esperable dada su baja amplitud—, 
se lograron implementar controladores funcionales y estables dentro de las limitaciones del hardware disponible. 
Las respuestas observadas mantienen la forma y tendencia previstas en las simulaciones, 
aunque restringidas a un rango reducido de amplitudes.
\end{document}
