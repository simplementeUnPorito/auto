% !TeX program = pdflatex
\documentclass[conference]{IEEEtran}

% ======= Idioma y codificación =======
\usepackage[utf8]{inputenc}
\usepackage[T1]{fontenc}
\usepackage[spanish, es-nodecimaldot]{babel}
\usepackage{microtype}

% ======= Matemática =======
\usepackage{amsmath, amssymb, amsfonts}

% ======= Colores y gráficos =======
\usepackage{xcolor}
\usepackage{graphicx}
% \usepackage{subcaption} % <-- NO usar en IEEEtran
\usepackage[caption=false,font=footnotesize]{subfig} % correcto con IEEEtran
\usepackage{float}
\usepackage{booktabs}

% ======= Números y unidades =======
\usepackage{siunitx}
\sisetup{
	output-decimal-marker = {,}, % coma decimal
	group-separator = {.},
	group-minimum-digits = 4,
	per-mode = symbol,
	round-mode = places,
	round-precision = 3,
	table-number-alignment = center
}

% ======= Maquetación 2 col. especial / balance =======
\usepackage{cuted}
\usepackage{balance}

% ======= Listas =======
\usepackage{enumitem}
\setlist{leftmargin=*, itemsep=2pt, topsep=4pt}

% ======= Bibliografía (IEEE + natbib numérico) =======
\usepackage[numbers]{natbib}
\bibliographystyle{IEEEtranN}

% ======= Listados de código (cargar paquetes ANTES de \lstset) =======
\usepackage{listings}
\usepackage{listingsutf8}

% Config global de listings: aceptar UTF-8 y caer a Latin1 si el archivo no es UTF-8 puro
\lstset{
	inputencoding=utf8/latin1,
	extendedchars=true,
	upquote=true,
	breaklines=true,
	columns=fullflexible,
	keepspaces=true,
	frame=single,
	numbers=left, numberstyle=\tiny, numbersep=6pt,
	xleftmargin=1em,
	captionpos=b,
	% Mapeos para Unicode y símbolos frecuentes en tus .c/.m (pdfLaTeX)
	literate=
	{á}{{\'a}}1 {é}{{\'e}}1 {í}{{\'\i}}1 {ó}{{\'o}}1 {ú}{{\'u}}1
	{Á}{{\'A}}1 {É}{{\'E}}1 {Í}{{\'I}}1 {Ó}{{\'O}}1 {Ú}{{\'U}}1
	{ñ}{{\~n}}1 {Ñ}{{\~N}}1
	{ü}{{\"u}}1 {Ü}{{\"U}}1
	{¿}{{\textquestiondown}}1 {¡}{{\textexclamdown}}1
	{º}{{$^\circ$}}1
	{“}{{``}}1 {”}{{''}}1 {‘}{{`}}1 {’}{{'}}1
	{—}{{---}}1 {–}{{--}}1
	{→}{{\textrightarrow}}1
	{≈}{{$\approx$}}1
	{µ}{{\textmu}}1
}

% Lenguaje Matlab (una sola definición)
\lstdefinelanguage{Matlab}{
	morekeywords={break,case,catch,continue,else,elseif,end,for,function,global,if,otherwise,persistent,return,switch,try,while,classdef,properties,methods},
	sensitive=true,
	morecomment=[l]\%,
	morecomment=[s]{\%\{}{\%\}},
	morestring=[m]',
}

% Estilos por lenguaje
\lstdefinestyle{cstyle}{
	language=C,
	inputencoding=utf8,
	basicstyle=\ttfamily\small,
	keywordstyle=\color{blue}\bfseries,
	stringstyle=\color{red!60!black},
	commentstyle=\color{green!40!black}\itshape,
	numbers=left, numberstyle=\tiny, stepnumber=1, numbersep=8pt,
	showstringspaces=false, breaklines=true, columns=fullflexible, keepspaces=true,
	frame=single, rulecolor=\color{black!20},
	linewidth=\columnwidth, xleftmargin=1em
}

\lstdefinestyle{matlabstyle}{
	language=Matlab,
	inputencoding=utf8,
	basicstyle=\ttfamily\small,
	keywordstyle=\bfseries\color{blue!70!black},
	commentstyle=\itshape\color{green!40!black},
	stringstyle=\color{red!60!black},
	numbers=left, numberstyle=\tiny, numbersep=6pt,
	frame=single, framerule=0.4pt,
	columns=fullflexible, keepspaces=true,
	showstringspaces=false,
	breaklines=true,
	tabsize=4,
	captionpos=b,
	linewidth=\columnwidth,
	xleftmargin=1em
}

% ======= Macros rápidas =======
\newcommand{\zetaD}{\zeta}
\newcommand{\wn}{\omega_n}
\newcommand{\Ts}{T_s}
\newcommand{\ESS}{\mathrm{ESS}}
\newcommand{\Gz}{G(z)}
\newcommand{\Gs}{G(s)}

% ======= Macros de figuras =======
\newcommand{\compararfigsC}[8]{%
	\begin{figure}[!t]
		\centering
		\subfloat[#2\label{#3}]{%
			\includegraphics[width=0.49\columnwidth]{#1}}
		\hspace{0.3em}
		\subfloat[#5\label{#6}]{%
			\includegraphics[width=0.49\columnwidth]{#4}}
		\caption{#7}
		\label{#8}
	\end{figure}%
}

\newcommand{\insertarfigura}[4]{%
	\begin{figure}[!t]
		\centering
		\includegraphics[width=#4\linewidth]{#1}
		\caption{#2}
		\label{#3}
	\end{figure}
}

\newcommand{\insertarfigurawide}[5][]{%
	\begin{figure*}[#1]
		\centering
		\includegraphics[width=#5\textwidth]{#2}
		\caption{#3}
		\label{#4}
	\end{figure*}%
}

% ======= Hyperref al final del preámbulo =======
\usepackage[unicode,hidelinks]{hyperref}

% ======= Metadatos =======
\title{Práctica de Laboratorio 5: Ubicación Arbitraria de Polos.}

\author{
	\IEEEauthorblockN{Elías Álvarez}
	\IEEEauthorblockA{Carrera de Ing. Electrónica\\
		Universidad Católica Nuestra Señora de la Asunción\\
		Asunción, Paraguay\\
		Email: elias.alvarez@universidadcatolica.edu.py}
	\and
	\IEEEauthorblockN{Tania Romero}
	\IEEEauthorblockA{Carrera de Ing. Electrónica\\
		Universidad Católica Nuestra Señora de la Asunción\\
		Asunción, Paraguay\\
		Email: tania.romero@universidadcatolica.edu.py}
	\and
	\IEEEauthorblockN{\hspace*{2.3em}Docente: Lic. Montserrat González}
	\IEEEauthorblockA{\hspace*{2.3em}Facultad de Ingeniería\\
		\hspace*{2.3em}Universidad Católica Nuestra Señora de la Asunción\\
		\hspace*{2.3em}Asunción, Paraguay}
	\and
	\IEEEauthorblockN{\hspace*{2.3em}Docente: PhD. Enrique Vargas}
	\IEEEauthorblockA{\hspace*{2.3em}Facultad de Ingeniería\\
		\hspace*{2.3em}Universidad Católica Nuestra Señora de la Asunción\\
		\hspace*{2.3em}Asunción, Paraguay}
}

\begin{document}
	\maketitle
	
	\section{Introducción}


\section{Objetivos}

\begin{itemize}
	\item Diseñar controladores digitales para una planta analógica utilizando estados estimados.
	\item Aplicar el diseño de realimentación de estados con estimadores de predicción y de actualización.
	\item Presentar y discutir los resultados experimentales en comparación con los obtenidos por simulación, dentro de un informe técnico razonado.
\end{itemize}

	\section{Desarrollo}

\subsection{Modelado del Sistema}

\subsubsection{Obtención de las matrices del sistema $F$, $G$, $H$ y $J$}

Para el modelado del sistema se parte del circuito mostrado en la Figura~\ref{fig:circuit_planta}, a partir del cual se determinan las ecuaciones de estado mediante el análisis de los lazos de realimentación y las relaciones de tensión en los componentes.

\insertarfigura{Otros/circuit.png}{Circuito de la planta.}{fig:circuit_planta}{1}

El sistema se describe mediante las siguientes ecuaciones en espacio de estados:

\begin{equation}
	\dot{x}(t) = F\,x(t) + G\,V_i(t)
	\label{eq:1}
\end{equation}

\begin{equation}
	y(t) = H\,x(t) + J\,V_i(t)
	\label{eq:2}
\end{equation}

Para obtener las expresiones de las variables de estado, se parte del equivalente del paralelo entre un resistor y un capacitor:
\[
R \parallel \frac{1}{sC} = \frac{R}{1 + sRC}
\]

Considerando que ambos amplificadores operacionales se encuentran en configuración no inversora, se obtienen las siguientes relaciones:

\[
V_a = \frac{-R_2}{1 + sR_2C_1}\frac{V_i}{R_1}
\quad \Rightarrow \quad
sV_a = -\frac{1}{R_1C_1}V_i - \frac{1}{R_2C_1}V_a
\]

\[
V_o = -\frac{R_4}{1 + sR_4C_2}\frac{V_a}{R_3}
\quad \Rightarrow \quad
sV_o = -\frac{1}{R_3C_2}V_a - \frac{1}{R_4C_2}V_o
\]

Definiendo como variables de estado $x_1(t) = V_a$ y $x_2(t) = V_o$, las ecuaciones anteriores se expresan en forma matricial como:

\[
\begin{bmatrix}
	\dot{x}_1(t) \\[4pt]
	\dot{x}_2(t)
\end{bmatrix}
=
\begin{bmatrix}
	-\dfrac{1}{R_2C_1} & 0 \\[4pt]
	-\dfrac{1}{R_3C_2} & -\dfrac{1}{R_4C_2}
\end{bmatrix}
\begin{bmatrix}
	x_1(t) \\[4pt]
	x_2(t)
\end{bmatrix}
+
\begin{bmatrix}
	-\dfrac{1}{R_1C_1} \\[4pt]
	0
\end{bmatrix}
V_i(t)
\]

y la ecuación de salida queda definida como:

\[
y(t) =
\begin{bmatrix}
	0 & 1
\end{bmatrix}
\begin{bmatrix}
	x_1(t) \\[4pt]
	x_2(t)
\end{bmatrix}
+ 0\cdot V_i(t)
\]

Sustituyendo los valores de los componentes 
$C_1 = 211.1\times10^{-9}\,\text{F}$, 
$R_1 = 80.55\times10^{3}\,\Omega$, 
$R_2 = 81.09\times10^{3}\,\Omega$, 
$C_2 = 103.07\times10^{-9}\,\text{F}$, 
$R_3 = 14.878\times10^{3}\,\Omega$ y 
$R_4 = 14.76\times10^{3}\,\Omega$, 
se obtienen las siguientes matrices numéricas:

\[
F =
\begin{bmatrix}
	-58.42 & 0 \\[4pt]
	-652.11 & -657.37
\end{bmatrix}, \quad
G =
\begin{bmatrix}
	-58.81 \\[4pt]
	0
\end{bmatrix}, \quad
H =
\begin{bmatrix}
	0 & 1
\end{bmatrix}
\]
\begin{equation}
	y \quad
	J = 0
	\label{eq:J}
\end{equation}
\subsubsection{Mostrar el diagrama de bloques del sistema}


\subsection{Discretización del Sistema}
\insertarfigura{Otros/Diagramas1.png}{Diagrama de bloques del sistema continuo.}{fig:diag_continuo}{1}

\subsubsection{Elección del tiempo de muestreo $T_s = 1~\text{ms}$}

Para la discretización del sistema continuo descrito por las ecuaciones~(\ref{eq:1}) y~(\ref{eq:2}), se busca obtener un modelo equivalente en tiempo discreto que relacione las variables de estado y la señal de entrada en instantes de muestreo definidos.  
Las ecuaciones del sistema discreto se expresan como:

\begin{equation}
	X(k+1) = A\,X(k) + B\,u(k)
	\label{eq:3}
\end{equation}

\begin{equation}
	Y(k) = C\,X(k) + D\,u(k)
	\label{eq:4}
\end{equation}

Usando las matrices continuas $F$, $G$, $H$ y $J$ obtenidas previamente, las matrices discretas se determinan mediante las siguientes expresiones:

\[
A = \mathrm{e}^{F T_s}, \qquad 
B = F^{-1}\!\left(\mathrm{e}^{F T_s} - I\right)G, \qquad 
C = H \qquad 
\]

\[
 \& \quad D = J
\]
\subsubsection{Obtención de las matrices discretas $A$, $B$, $C$ y $D$}

Con un tiempo de muestreo $T_s = 1~\text{ms}$, se obtienen las siguientes matrices discretizadas:

\[
A =
\begin{bmatrix}
	0.943 & 0 \\[4pt]
	-0.462 & 0.518
\end{bmatrix}, \qquad
B =
\begin{bmatrix}
	-0.0571 \\[4pt]
	0.0152
\end{bmatrix}, 
\]
\begin{equation}
	C =
	\begin{bmatrix}
		0 & 1
	\end{bmatrix}, \qquad \& \qquad
	D = 0
	\label{eq:d}
\end{equation}

Estas matrices representan el modelo digital equivalente del sistema continuo, y serán utilizadas posteriormente para el diseño del controlador e implementación en el \texttt{PSoC}.

\subsubsection{Diagrama de bloques del sistema discretizado}

En la Figura~\ref{fig:diag_discreto} se presenta el diagrama de bloques correspondiente al sistema discretizado, donde se observa la relación entre las variables de estado, la entrada $u(k)$ y la salida $Y(k)$.

\insertarfigura{Otros/Diagramas2.png}{Diagrama de bloques del sistema discretizado.}{fig:diag_discreto}{1}


\subsubsection{Verificación de la controlabilidad y observabilidad del sistema}

La matriz de controlabilidad se obtiene a partir de la siguiente relación general:

\[
x(n) - A^{n}x(0) = \sum_{i=0}^{n-1} A^{n-i-1}B\,u(i)
\]

lo que lleva a la siguiente forma matricial:

\[
x(n) - A^{n}x(0) =
\begin{bmatrix}
	A^{n-1}B & A^{n-2}B & \cdots & AB & B
\end{bmatrix}
\begin{bmatrix}
	u(0) \\[2pt]
	u(1) \\[2pt]
	\vdots \\[2pt]
	u(n-1)
\end{bmatrix}
\]

De esta expresión, se define la matriz de controlabilidad como:

\[
\mathcal{C} =
\begin{bmatrix}
	B & AB
\end{bmatrix}
\]

Para el sistema analizado, la matriz resultante es:

\[
\mathcal{C} =
\begin{bmatrix}
	-0.05344 & -0.05713 \\[4pt]
	0.03435 & 0.01527
\end{bmatrix}
\]

El rango de esta matriz es $n = 2$, lo que indica que el sistema es completamente controlable.

\bigskip
La matriz de observabilidad se obtiene a partir de la expresión general:

\[
Y(n-1) = 
\begin{bmatrix}
	C \\[4pt]
	CA \\[4pt]
	\vdots \\[4pt]
	CA^{n-1}
\end{bmatrix}
X(0)
\]

Por lo tanto, la matriz de observabilidad queda definida como:

\[
\mathcal{O} =
\begin{bmatrix}
	C \\[4pt]
	CA
\end{bmatrix}
\]

Sustituyendo los valores del sistema:

\[
\mathcal{O} =
\begin{bmatrix}
	0 & 1 \\[4pt]
	-0.462 & 0.518
\end{bmatrix}
\]

El rango de la matriz de observabilidad también resulta ser $n = 2$.\\  
Por lo tanto, se concluye que el sistema es **completamente controlable y observable**, cumpliendo con las condiciones necesarias para el diseño de control mediante realimentación de estados.

\subsubsection{Comparar los resultados obtenidos con las simulaciones realizadas en \texttt{MATLAB}}
\textbf{Resultados de las matrices continuas:}
\[
F =
\begin{bmatrix}
	-58.42 & 0 \\[4pt]
	-652.10 & -657.30
\end{bmatrix}, \quad
G =
\begin{bmatrix}
	-58.81 \\[4pt]
	0
\end{bmatrix}, \quad
H =
\begin{bmatrix}
	0 & 1
\end{bmatrix}
\]

\[
	\& \quad
	J = [\,0\,]
\]

\textbf{Resultados de las matrices discretas:}
\[
A =
\begin{bmatrix}
	0.9433 & 0 \\[4pt]
	-0.4628 & 0.5182
\end{bmatrix}, \quad
B =
\begin{bmatrix}
	-0.05712 \\[4pt]
	0.01527
\end{bmatrix}, \quad
C =
\begin{bmatrix}
	0 & 1
\end{bmatrix}
\]

\[
	\& \quad
	D = [\,0\,]
\]
Se observa que los resultados de las matrices obtenidas son congruentes con los valores calculados en las ecuaciones~(\ref{eq:J}) y~(\ref{eq:d}), verificando la coherencia entre el modelo teórico y los resultados obtenidos mediante \texttt{MATLAB}.

\insertarfigurawide{matlab/Sim1_noImplementable.png}{Respuesta simulada en \texttt{MATLAB} correspondiente al Caso 2($f_s = 1000 Hz$).}{fig:simulacion_noImplementable}{1}

\insertarfigurawide{matlab/Sim2_noImplementable.png}{Respuesta simulada en \texttt{MATLAB} correspondiente al Caso 2($f_s = 1000 Hz$).}{fig:simulacion_noImplementable2}{1}

En las figuras \ref{fig:simulacion_noImplementable} y \ref{fig:simulacion_noImplementable2}, del \texttt{MATLAB} se pueden observar que no es implementable.
	\section{Implementación del Sistema}

Durante la etapa de implementación no se intentó replicar directamente los controladores diseñados en \textit{Matlab} con las especificaciones originales, 
ya que se determinó que dichos parámetros no eran implementables con el hardware disponible. 
Los valores de ganancia requeridos generaban esfuerzos imposibles de aplicar sin saturar el DAC, 
por lo que incluso antes de probar en el PSoC se optó por \textbf{relajar las especificaciones} 
y diseñar controladores implementables que preservaran el comportamiento cualitativo del sistema.

En otras palabras, se mantuvo el mismo enfoque de control por realimentación de estados, 
pero con polos menos agresivos y ganancias adaptadas al rango físico del actuador. 
Aun así, los controladores “relajados” solamente pudieron trabajar con referencias muy pequeñas 
($0{,}4$~V para el Caso~1 y $0{,}3$~V para el Caso~2); 
valores mayores producían saturación inmediata y lo único observable en el osciloscopio era ruido amplificado.

El sistema digital completo se muestra en la Figura~\ref{fig:circuito_digital}. 
Allí se incluyen los convertidores \textsc{ADC SAR}, 
el circuito de sincronización con dos flip-flops tipo~D para detectar el fin de conversión de ambos canales, 
y un módulo \textsc{UART} utilizado para enviar comandos al PSoC. 
A través de este canal se podían modificar en tiempo real la frecuencia del temporizador, 
abrir o cerrar el lazo, y cambiar la amplitud de referencia, 
lo cual facilitó enormemente las pruebas experimentales.

[AÑADIR FOTO CIRCUITO]
%\insertarfigura{exp/circuitoDigital.png}
%{Circuito digital implementado en el PSoC, con los módulos SAR, flip-flops de sincronización y UART para control.}
%{fig:circuito_digital}{0.9}

La planta analógica se muestra en la Figura~\ref{fig:circuito_analogico}. 
Junto a ella se encuentra el DAC principal, que actúa como actuador aplicando los esfuerzos de control calculados. 
Además, se incorporó un segundo DAC destinado a generar la referencia en voltios; 
si bien esta referencia se define digitalmente, resultó útil disponer de su equivalente analógico para visualizarla en el osciloscopio durante las pruebas.


[AÑADIR FOTO CIRCUITO]
%\insertarfigura{exp/circuitoAnalogico.png}
%{Circuito analógico correspondiente a la planta física utilizada en el experimento.}
%{fig:circuito_analogico}{0.9}

La referencia digital se generó mediante un temporizador configurable 
y un registro tipo~T, definiendo el valor de referencia como $\mathrm{refA}/2$ 
cuando el bit lógico leído era~1 y $-\mathrm{refA}/2$ cuando era~0. 
De esta forma se obtuvo una señal cuadrada de amplitud controlada, 
útil para evaluar las respuestas transitorias del sistema.

Es importante destacar que todos los cálculos del controlador se realizaron 
desacoplando la componente de \textsc{DC}, 
ya que al no hacerlo se amplifica la componente continua en la etapa de salida 
y se produce saturación mucho más rápida.

En el firmware, para facilitar el traslado del diseño desde \textit{Matlab} a lenguaje~C, 
se implementó la librería \texttt{mat.c}. 
En ella se definieron los tipos de datos y funciones necesarios para realizar las operaciones matriciales requeridas en el cálculo del esfuerzo de control, 
utilizando únicamente las ganancias $K$ y $N_{\mathrm{var}}$.

Una vez calculado el esfuerzo de control (en valor alterno), 
se le sumó $V_{\mathrm{DD}}/2$ para montarlo sobre el nivel medio del DAC 
y mantener compatibilidad con el circuito analógico. 
De esta manera se logró que las mediciones en el osciloscopio 
fuesen consistentes con las simulaciones en \textit{Matlab}.

\insertarfigura{exp/openLoop.png}
{Respuesta experimental del sistema en lazo abierto.}
{fig:openloop}{0.85}

Las frecuencias de muestreo efectivas fueron de aproximadamente 
$418{,}5$~Hz para el Caso~1 y $313{,}9$~Hz para el Caso~2. 
Las Figuras~\ref{fig:exp1} y~\ref{fig:exp2} muestran las respuestas observadas experimentalmente.

\insertarfigura{exp/Exp1.png}
{Respuesta experimental correspondiente al Caso~1 ($f_s \approx 418{,}5$~Hz).}
{fig:exp1}{0.85}

\insertarfigura{exp/Exp2.png}
{Respuesta experimental correspondiente al Caso~2 ($f_s \approx 313{,}9$~Hz).}
{fig:exp2}{0.85}

Aunque las señales obtenidas presentan cierto nivel de ruido —algo esperable dada su baja amplitud—, 
se lograron implementar controladores funcionales y estables dentro de las limitaciones del hardware disponible. 
Las respuestas observadas mantienen la forma y tendencia previstas en las simulaciones, 
aunque restringidas a un rango reducido de amplitudes.
	\section{Resultados}

\subsection{Respuestas temporales (simulado vs.\ experimental)}
En esta sección se presentan comparaciones \emph{en el dominio del tiempo} entre simulación y experimento para: planta sin compensar, C1 (proporcional), C2 (lead), C3 (integrador+lead). En cada figura se incluyen las métricas \emph{RMSE}, \emph{NRMSE} y \(e_{\max}\) (definidas en las ecuaciones \eqref{eq:rmse_temporal}–\eqref{eq:emax_temporal}). Se muestran versiones \emph{con} y \emph{sin} remoción de offset para evidenciar el descalce de referencia señalado en la implementación.

% ===== AJUSTAR RUTAS =====
\insertarfigura{img/OpenLoop/comparacionLazoAbierto.png}
{Planta sin compensación en lazo abierto: respuesta al escalón (simulado vs.\ experimental, con bandas de tolerancia para \(\sigma = \frac{tolerance}{3}\)).}
{fig:step_openLoop}{1}



\insertarfigurawide{img/C1_Lead/conOffset}
{Compensador de adelanto: respuesta al escalón y esfuerzo (simulado vs.\ experimental, \emph{con} offset).}
{fig:step_c1_con_offset}{1}

\insertarfigurawide{img/C1_Lead/sinOffset}
{Compensador de adelanto: respuesta al escalón y esfuerzo (simulado vs.\ experimental, \emph{sin} offset).}
{fig:step_c1_sin_offset}{1}

\insertarfigurawide{img/C1_K/compConOffset}
{Compensador Proporcional: respuesta al escalón y esfuerzo (simulado vs.\ experimental, \emph{con} offset).}
{fig:step_prop_con_offset}{1}

\insertarfigurawide{img/C1_K/compSinOffset}
{Compensador Proporcional: respuesta al escalón y esfuerzo (simulado vs.\ experimental, \emph{sin} offset).}
{fig:step_prop_sin_offset}{1}

\insertarfigurawide{img/C2/conOffset}
{Compensador de adelanto + integrador: respuesta al escalón y esfuerzo (simulado vs.\ experimental, \emph{con} offset).}
{fig:step_c2_con_offset}{1}

\insertarfigurawide{img/C2/sinOFFset}
{Compensador de adelanto + integrador: respuesta al escalón y esfuerzo (simulado vs.\ experimental, \emph{sin} offset).}
{fig:step_c2_sin_offset}{0.92}



\subsection{Métricas de desempeño temporal}
Las métricas usadas en cada gráfica (y en la tabla resumen) son:
\begin{equation}
	\label{eq:rmse_temporal}
	\mathrm{RMSE}=\sqrt{\frac{1}{N}\sum_{k=1}^{N}\big(y[k]-\hat{y}[k]\big)^2},
\end{equation}
\begin{equation}
	\label{eq:rmse_porct}
	\mathrm{NRMSE}=\frac{\mathrm{RMSE}}{y_{\max}-y_{\min}},
\end{equation}
\begin{equation}
	\label{eq:emax_temporal}
	e_{\max}=\max_k\,\lvert y[k]-\hat{y}[k]\rvert.
\end{equation}
\balance
% ===== Resultados (solo tiempo) =====
En la tabla \ref{tab:comparativa_temporal_min}, los índices de \emph{tiempo de subida} \(t_r\) y \emph{sobreimpulso} \(M_p\) se obtienen de las curvas de esta sección y anteriores. El error en estado estacionario \(e_{ss}\) se reporta cuando aplica (lazo cerrado), y el \emph{error cuadratico medio normalizado} entre las mediciones y simulaciones se especifica tanto \emph{con} como \emph{sin} el offset de las señales.

% EN EL TEXTO (sin \onecolumn ni \balance aquí)
\begin{table}[t]
	\centering
	\caption{Comparativa temporal (simulado vs.\ experimental).}
	\label{tab:comparativa_temporal_min}
	\small
	\setlength{\tabcolsep}{4pt}
	\renewcommand{\arraystretch}{1.1}
	\begin{adjustbox}{max width=\columnwidth}
		\begin{tabular}{lcccc}
			\toprule
			\textbf{Sistema} &
			\makecell{\(\mathbf{t_r}\) [ms]\\(sim/exp)} &
			\makecell{\(\mathbf{M_p}\) [\%]\\(sim/exp)} &
			\makecell{\(\mathbf{e_{ss}}\) [\%]\\(sim/exp)} &
			\makecell{\(\mathbf{NRMSE}\) [\%]\\(con/sin off)}\\
			\midrule
			Lazo abierto       & 40.0 / 34.2   & 0 / 0                 & 0 / 1.8182             & 19.14 / 18.43 \\
			C1 (lead)          & 21.17 / 19.8  & 0 / 0                 & 45.77 / 42--46.79      & 38.02 / 7.31 \\
			Proporcional       & 3.736 / 3.950 & 15.4 / 10--20         & 12.5 / \(\approx 15.825\) & 19.24 / 6.57 \\
			Lead + integrador  & 13.7 / 17.10  & 7.433 / \(\approx 5\) & 0 / 0                  & 13.54 / 5.61 \\
			\bottomrule
		\end{tabular}
	\end{adjustbox}
\end{table}

\subsection{Error de velocidad (seguimiento de rampa)}
Para el compensador C2 (integrador+lead) se ensayó seguimiento a rampa \(r[k]=kT\) (pendiente \(1\ \mathrm{u}/\mathrm{s}\)). El error en régimen para sistemas tipo~1 en discreto es
\begin{equation}
	\label{eq:ess_rampa_discreto}
	e_{ss}=\frac{T}{K_{v,z}},\qquad K_{v,z}=\lim_{z\to1}(z-1)\,L(z).
\end{equation}
Con el \(L(z)\) obtenido, se midió/estimó \(K_v\simeq 78.43\Rightarrow e_{ss}^{(\mathrm{teo})}\approx 1/78.43\approx 1.28\%\), en concordancia con el valor observado tras un transitorio breve.

% ===== AJUSTAR RUTA =====
\insertarfigura{img/C2/rampaC2linda}
{Seguimiento de rampa con Compensador integrador + lead.}
{fig:rampa_c2}{0.92}

\subsection{Discusión de discrepancias}
Las diferencias entre curvas simuladas y experimentales se explican principalmente por: (i) \textbf{limitación/saturación} en la señal de esfuerzo debido a la carga en el DAC (la excitación queda “achatada”), y (ii) \textbf{descalce de referencia} (offset). A ello se suman tolerancias de componentes pasivos, lo que desplaza levemente parámetros característicos de la planta. Aun así, la \emph{dinámica global} buscada (forma de la respuesta y tiempos) se mantuvo acorde a la simulación.




\section{Conclusiones}

El conjunto de ensayos y comparaciones en el \emph{dominio del tiempo} demuestra que, a pesar de las limitaciones de implementación y de la dispersión de la planta real, el desempeño experimental se mantiene coherente con el diseño y dentro de las bandas esperadas. A continuación se sintetizan los hallazgos principales.

\subsection*{Ajuste simulación–experimento}
Al simular con la \textbf{misma entrada} medida y remover el \textbf{offset estático} entre DAC y ADC, el ajuste mejora de forma notable (véase la tabla~\ref{tab:comparativa_temporal_min}). Lo que persiste se explica por \emph{efectos dinámicos} no ideales: como saturaciones suaves del actuador (no linealidad) y  atenuaciones por carga o discrepancias entre resistencias que definen la ganancia.

\subsection*{Limitaciones de hardware observadas}
Se verificó que la \textbf{corriente de salida del DAC} no es suficiente para excitar directamente la planta ---aunque no logramos conseguir los rangos exactos de corriente que puede suministrar en el datasheet---, lo que produce \emph{achatamiento} y \emph{recorte} del esfuerzo de control y, en consecuencia, discrepancias de forma en las respuestas. Además, la referencia $\mathrm{VDDA}/2$ y las líneas de alimentación presentan \emph{desacople insuficiente}, favoreciendo derivas de nivel que se manifiestan como offset. Estas condiciones explican parte del desajuste restante aun tras corregir el offset en posprocesado.

\subsection*{Planta real vs.\ modelo nominal}
La planta implementada difiere levemente del modelo continuo asumido por \emph{tolerancias} de pasivos, ordenamiento de etapas e impedancia de entrada vista por la fuente del esfuerzo. Aun así, las respuestas medidas se ubican mayormente \textbf{dentro de la banda de tolerancias} obtenida por análisis estadístico, lo que respalda la \emph{validez del modelo} para propósitos de diseño y la \emph{robustez} del procedimiento seguido.



\subsection*{Acciones recomendadas}
\begin{itemize}
	\item Añadir un \textbf{buffer} a la salida del DAC para eliminar el error de carga en el DAC y evitar recortes del esfuerzo.
	\item \textbf{Biaspassear} la referencia $\mathrm{VDDA}/2$ y las líneas de alimentación, mejorando estabilidad de nivel y rechazo de ruido.
\end{itemize}

\subsection*{Conclusión general}
En conjunto, los resultados muestran que el \textbf{método de diseño es robusto}: aun bajo carga del DAC, offsets y dispersión de componentes, las respuestas experimentales se mantienen \emph{razonablemente cercanas} a las simuladas, especialmente con el compensador  con integrador + adelanto, que satisface los objetivos de seguimiento y estabilidad con un compromiso adecuado entre rapidez y exactitud.

	
	\onecolumn
\section{Anexo.}

	
\end{document}
