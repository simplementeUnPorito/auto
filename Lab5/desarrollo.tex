\section{Desarrollo}

\subsection{Modelado del Sistema}

\subsubsection{Obtención de las matrices del sistema $F$, $G$, $H$ y $J$}

Para el modelado del sistema se parte del circuito mostrado en la Figura~\ref{fig:circuit_planta}, a partir del cual se determinan las ecuaciones de estado mediante el análisis de los lazos de realimentación y las relaciones de tensión en los componentes.

\insertarfigura{Otros/circuit.png}{Circuito de la planta.}{fig:circuit_planta}{1}

El sistema se describe mediante las siguientes ecuaciones en espacio de estados:

\begin{equation}
	\dot{x}(t) = F\,x(t) + G\,V_i(t)
	\label{eq:1}
\end{equation}

\begin{equation}
	y(t) = H\,x(t) + J\,V_i(t)
	\label{eq:2}
\end{equation}

Para obtener las expresiones de las variables de estado, se parte del equivalente del paralelo entre un resistor y un capacitor:
\[
R \parallel \frac{1}{sC} = \frac{R}{1 + sRC}
\]

Considerando que ambos amplificadores operacionales se encuentran en configuración no inversora, se obtienen las siguientes relaciones:

\[
V_a = \frac{-R_2}{1 + sR_2C_1}\frac{V_i}{R_1}
\quad \Rightarrow \quad
sV_a = -\frac{1}{R_1C_1}V_i - \frac{1}{R_2C_1}V_a
\]

\[
V_o = -\frac{R_4}{1 + sR_4C_2}\frac{V_a}{R_3}
\quad \Rightarrow \quad
sV_o = -\frac{1}{R_3C_2}V_a - \frac{1}{R_4C_2}V_o
\]

Definiendo como variables de estado $x_1(t) = V_a$ y $x_2(t) = V_o$, las ecuaciones anteriores se expresan en forma matricial como:

\[
\begin{bmatrix}
	\dot{x}_1(t) \\[4pt]
	\dot{x}_2(t)
\end{bmatrix}
=
\begin{bmatrix}
	-\dfrac{1}{R_2C_1} & 0 \\[4pt]
	-\dfrac{1}{R_3C_2} & -\dfrac{1}{R_4C_2}
\end{bmatrix}
\begin{bmatrix}
	x_1(t) \\[4pt]
	x_2(t)
\end{bmatrix}
+
\begin{bmatrix}
	-\dfrac{1}{R_1C_1} \\[4pt]
	0
\end{bmatrix}
V_i(t)
\]

y la ecuación de salida queda definida como:

\[
y(t) =
\begin{bmatrix}
	0 & 1
\end{bmatrix}
\begin{bmatrix}
	x_1(t) \\[4pt]
	x_2(t)
\end{bmatrix}
+ 0\cdot V_i(t)
\]

Sustituyendo los valores de los componentes 
$C_1 = 211.1\times10^{-9}\,\text{F}$, 
$R_1 = 80.55\times10^{3}\,\Omega$, 
$R_2 = 81.09\times10^{3}\,\Omega$, 
$C_2 = 103.07\times10^{-9}\,\text{F}$, 
$R_3 = 14.878\times10^{3}\,\Omega$ y 
$R_4 = 14.76\times10^{3}\,\Omega$, 
se obtienen las siguientes matrices numéricas:

\[
F =
\begin{bmatrix}
	-58.42 & 0 \\[4pt]
	-652.11 & -657.37
\end{bmatrix}, \quad
G =
\begin{bmatrix}
	-58.81 \\[4pt]
	0
\end{bmatrix}, \quad
H =
\begin{bmatrix}
	0 & 1
\end{bmatrix}
\]
\[
y \quad
J = 0
\]
\subsubsection{Mostrar el diagrama de bloques del sistema}


\subsection{Discretización del Sistema}

\subsubsection{Elegir el tiempo de muestreo igual a $T_s = 1~\text{ms}$}
Para la discretización del sistema se tienen que llegar a las siguientes expresiones:
\begin{equation}
	X(k+1) = A\,X(k) + B\,u(k)
	\label{eq:3}
\end{equation}

\begin{equation}
	Y(k) = C\,X(k) + D\,u(k)
	\label{eq:4}
\end{equation}

Usando las matrices, calculadas anteriormente se pueden obtener las matrices digitalizadas de la siguiente manera: 
\[
A=e^{FT_s}, \quad B=F^{-1}(e^{FT_s}-I)G, \quad C = H \quad y \quad D = J
\]
\subsubsection{Obtener las matrices discretas del sistema $A$, $B$, $C$ y $D$}
Las matrices discretizadas son las siguientes:
\[
A =
\begin{bmatrix}
	0.943 & 0 \\[4pt]
	-0.462 & 0.518
\end{bmatrix}, \quad
B =
\begin{bmatrix}
	-0.0571 \\[4pt]
	0.0152
\end{bmatrix}, \quad
C =
\begin{bmatrix}
	0 & 1
\end{bmatrix}
\]
\[
y \quad
D = 0
\]

\subsubsection{Mostrar el diagrama de bloques del sistema discretizado}

\subsubsection{Verificar la controlabilidad y la observabilidad del sistema}
La matriz de controlabilidad se obtiene de la siguiente manera:
\[
	x(n)-A^{n}x(0)=\sum_{n-1}^{i=0}A^{n-i-1}Bu(i)
\]
Llegando a la siguiente expresión:
\[
	x(n)-A^{n}x(0)=
	\begin{bmatrix}
		A^{n-1}B & A^{n-2} &... & AB & B
	\end{bmatrix}
	\begin{bmatrix}
		u(0) \\
		u(1) \\
		. \\
		u(n-1)
	\end{bmatrix}
\]

La matriz de controlabilidad es:
\[
	\begin{bmatrix}
		A^{n-1}B & A^{n-2} &... & AB & B
	\end{bmatrix}
\]
Matriz resultante:
\[
\begin{bmatrix}
	-0.05344 & 0 \\
	0.03435 & 0.01527
\end{bmatrix}
\]
El rango resultante de la matriz es $n = 2$
Para la matriz de observabilidad la expresión inicial es:
\[
	Y(n-1)-\sum_{n-2}^{i = 0}-\sum_{n-2}^{i = 0}Bu(i) = CA^{n-1}X(0)
\]
resultando en la matriz de observabilidad en :
\[
\begin{bmatrix}
	CA^{-1} \\
	: \\
	CA^{-n} 
\end{bmatrix}
\]
\subsubsection{Comparar los resultados obtenidos con las simulaciones realizadas en \texttt{MATLAB}}



