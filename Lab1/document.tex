\documentclass[conference]{IEEEtran}
\IEEEoverridecommandlockouts
\usepackage{siunitx}
\sisetup{
	round-mode          = figures,   % redondeo por cifras significativas
	round-precision     = 3,         % 3 cifras significativas
	table-number-alignment = center, % centra columnas
	scientific-notation = engineering,  % usa científica solo si hace falta
	retain-unity-mantissa = false,      % evita 1.0e0 para números "normales"
}




% --- Codificación y español ---
\usepackage[utf8]{inputenc}
\usepackage[T1]{fontenc}
\usepackage[spanish]{babel}
\usepackage[utf8]{inputenc}
% --- Tipografía y microajustes (reduce overfull hbox) ---
\usepackage{microtype}
\emergencystretch=2em

% --- Gráficos y floats ---
\usepackage{graphicx}
\usepackage{float} % para [H] si lo usas (art/report ok)

% --- Hipervínculos (con soporte unicode) ---
\usepackage[unicode,hidelinks]{hyperref}

% --- Tablas numéricas y redondeo ---
\usepackage{siunitx}
\sisetup{
	round-mode=places,
	round-precision=3,
	table-number-alignment = center,
}

% --- Tablas más flexibles ---
\usepackage{booktabs}
\usepackage{tabularx}

% --- Código: listings con UTF-8 real ---
\usepackage{listings}
\usepackage{listingsutf8}
\lstset{
	inputencoding=utf8,
	extendedchars=true,
	basicstyle=\ttfamily\small,
	breaklines=true, columns=fullflexible, keepspaces=true,
	frame=single, numbers=left, numberstyle=\tiny, xleftmargin=1em,
	% Mapeo por si alguna terminal te mete bytes raros:
	literate=
	{á}{{\'a}}1 {é}{{\'e}}1 {í}{{\'\i}}1 {ó}{{\'o}}1 {ú}{{\'u}}1
	{Á}{{\'A}}1 {É}{{\'E}}1 {Í}{{\'I}}1 {Ó}{{\'O}}1 {Ú}{{\'U}}1
	{ñ}{{\~n}}1 {Ñ}{{\~N}}1
	{¿}{{\textquestiondown}}1 {¡}{{\textexclamdown}}1
}

% The preceding line is only needed to identify funding in the first footnote. If that is unneeded, please comment it out.
\usepackage{cite}
\usepackage[utf8]{inputenc}     % <- Para UTF-8
\usepackage[T1]{fontenc}        % <- Para ñ y tildes correctas
\usepackage[spanish]{babel}     % <- Para idioma español
\usepackage{hyperref}
% ==== SOPORTE UTF-8 Y ESPAÑOL ====
\usepackage[utf8]{inputenc}   % codificación UTF-8
\usepackage[T1]{fontenc}      % mejor manejo de acentos/ñ en PDF
\usepackage[spanish]{babel}   % traducción de entornos automáticos

% ==== PAQUETES EXTRA RELEVANTES ====
\usepackage{graphicx}         % para figuras
\usepackage{float}            % para usar [H] en tablas/figuras
\usepackage{hyperref}         % enlaces y referencias en PDF
\usepackage{caption}          % mejor control de captions

% ==== LISTINGS CON UTF-8 ====
\usepackage{listings}
\usepackage{listingsutf8}
\lstset{
	inputencoding=utf8,         % clave: permite acentos directamente
	extendedchars=true,
	basicstyle=\ttfamily\small, % fuente monoespaciada
	breaklines=true,            % partir líneas largas
	columns=fullflexible,
	keepspaces=true,
	frame=single,               % recuadro alrededor del código
	numbers=left,               % numeración de líneas
	numberstyle=\tiny,
	xleftmargin=1em
}

\usepackage{amsmath,amssymb,amsfonts}
\usepackage{algorithmic}
\usepackage{graphicx}
\usepackage{textcomp}
\usepackage{xcolor}
\usepackage{fancyvrb}
\usepackage{listings}
\lstset{
	basicstyle=\ttfamily\small,
	breaklines=true,
	frame=none,
	columns=fullflexible
}
\usepackage{stfloats}

\def\BibTeX{{\rm B\kern-.05em{\sc i\kern-.025em b}\kern-.08em
		T\kern-.1667em\lower.7ex\hbox{E}\kern-.125emX}}

\usepackage{float}

% Macro para insertar figuras centradas con ancho ajustable
% Ejemplo de uso \insertarfigura{ruta}{leyenda}{etiqueta}{escala entre 0 y 1}
\newcommand{\insertarfigura}[4]{
	\begin{figure}[H]
		\centerline{\includegraphics[width=#4\linewidth]{#1}}
		\caption{#2}
		\label{#3}
	\end{figure}
	% Referencia cruzada si se desea en el texto: véase la Fig.~\ref{#4}
}
% Macro: \insertarfigurawide{ruta}{caption}{label}{scale}
\usepackage{cuted} % en el preámbulo

% 
\newcommand{\insertarfigurawide}[4]{%
	\begin{figure*}[ht]
		\centering
		\includegraphics[width=#4\textwidth]{#1}
		\caption{#2}
		\label{#3}
	\end{figure*}
}

\usepackage{subcaption} % en el preámbulo


% --- Configuración para código Matlab ---

% Para listings en 2 columnas sin desbordar
\lstset{
	linewidth=\columnwidth,     % fuerza el ancho a la columna
	xleftmargin=0pt,            % sin margen extra
	framexleftmargin=0pt,
	captionpos=b,               % leyenda abajo (recomendado en IEEE)
}

\lstdefinelanguage{Matlab}{
	morekeywords={break,case,catch,continue,else,elseif,end,for,function,
		global,if,otherwise,persistent,return,switch,try,while},
	sensitive=true,
	morecomment=[l]\%,
	morestring=[m]',
}
\lstdefinestyle{matlabstyle}{
	language=Matlab,
	basicstyle=\ttfamily\small,
	keywordstyle=\color{blue}\bfseries,
	commentstyle=\color{green!40!black}\itshape,
	stringstyle=\color{red!60!black},
	numbers=left,
	numberstyle=\tiny,
	stepnumber=1,
	numbersep=6pt,
	breaklines=true,
	showstringspaces=false,
	frame=single,
	rulecolor=\color{black!20}
}

% --- Código C con listings ---
\usepackage{listings}
\usepackage{listingsutf8}
\usepackage{xcolor}

\lstdefinestyle{cstyle}{
	language=C,
	inputencoding=utf8,
	basicstyle=\ttfamily\small,         % fuente monoespaciada
	keywordstyle=\color{blue}\bfseries, % palabras clave en azul
	stringstyle=\color{red!60!black},   % strings en rojo oscuro
	commentstyle=\color{green!40!black}\itshape, % comentarios en verde itálico
	numbers=left,                       % numerar líneas
	numberstyle=\tiny,
	stepnumber=1,
	numbersep=8pt,
	showstringspaces=false,
	breaklines=true,
	columns=fullflexible,
	keepspaces=true,
	frame=single,
	rulecolor=\color{black!20},
	linewidth=\columnwidth,             % para que no se salga de columna IEEE
	xleftmargin=1em
}

% (Opcional) palabras clave extra de PSoC:
\lstdefinelanguage{CwithPSOC}[]{C}{
	morekeywords={uint8,uint16,uint32,int8,int16,int32,
		bool,true,false,size_t,inline,volatile,
		CY_ISR,CY_ISR_PROTO,CyGlobalIntEnable,UART_Start,UART_PutString}
}
\lstdefinestyle{cpsoc}{style=cstyle,language=CwithPSOC}

% Mapeo de acentos en comentarios (para español):
\lstset{literate=
	{á}{{\'a}}1 {é}{{\'e}}1 {í}{{\'\i}}1 {ó}{{\'o}}1 {ú}{{\'u}}1
	{Á}{{\'A}}1 {É}{{\'E}}1 {Í}{{\'I}}1 {Ó}{{\'O}}1 {Ú}{{\'U}}1
	{ñ}{{\~n}}1 {Ñ}{{\~N}}1
}


\begin{document}
	% ------------------- CARÁTULA -------------------
\title{Práctica de Laboratorio 1: Diseño de un Controlador PID}
\author{
	\IEEEauthorblockN{Elías Álvarez}
	\IEEEauthorblockA{
		Carrera de Ingeniería Electrónica \\
		Universidad Católica Nuestra Señora de la Asunción \\
		Asunción, Paraguay \\
		Email: elias.alvarez@universidadcatolica.edu.py}
	\and
	\IEEEauthorblockN{Tania Romero}
	\IEEEauthorblockA{
		Carrera de Ingeniería Electrónica \\
		Universidad Católica Nuestra Señora de la Asunción \\
		Asunción, Paraguay \\
		Email: tania.romero@universidadcatolica.edu.py}
	\and
	\IEEEauthorblockN{\hspace*{3.5em}Docente: Lic. Montserrat González}
	\IEEEauthorblockA{
		\hspace*{3.5em}Facultad de Ingeniería \\
		\hspace*{3.5em}Universidad Católica Nuestra Señora de la Asunción \\
		\hspace*{3.5em}Asunción, Paraguay}
	\and
		\IEEEauthorblockN{\hspace*{3.5em}Docente: PhD. Enrique Vargas}
		\IEEEauthorblockA{%
			\hspace*{3.5em}Facultad de Ingeniería \\  % ← empuja a la derecha
			\hspace*{3.5em}Universidad Católica Nuestra Señora de la Asunción \\
			\hspace*{3.5em}Asunción, Paraguay}
		
}

\maketitle

\begin{abstract}
	En la presente práctica se desarrolló el diseño e implementación de un controlador PID orientado a mejorar la dinámica de una planta bajo control en lazo cerrado. A partir de los ensayos realizados, se evaluó la influencia de los parámetros proporcional, integral y derivativo sobre el desempeño del sistema, analizando su impacto en la respuesta transitoria. Asimismo, se aplicaron diferentes métodos de sintonía para ajustar el controlador y se verificaron experimentalmente los resultados, observando el compromiso entre desempeño y esfuerzo de control.
\end{abstract}



	\section{Introducción}
El control de sistemas dinámicos constituye un pilar fundamental en la ingeniería, ya que permite regular y optimizar el comportamiento de diversos procesos industriales. Entre las estrategias de control más utilizadas se encuentra el controlador PID (Proporcional–Integral–Derivativo), debido a su sencillez, robustez y eficacia en la mejora de la respuesta de sistemas de distinta naturaleza.  

En esta práctica de laboratorio se plantea el diseño e implementación de un controlador PID aplicado a una planta determinada, con el fin de analizar cómo cada uno de sus parámetros influye en la dinámica del sistema. La experimentación permitirá comprender los efectos sobre la rapidez de respuesta, la estabilidad y la reducción del error en estado estacionario, además de familiarizarse con distintos métodos de sintonía.

\section{Objetivos Específicos}
\begin{itemize}
	\item Diseñar e implementar un controlador PID que permita modificar la dinámica de una planta
	para mejorar la respuesta transitoria y garantizar la estabilidad del sistema en lazo cerrado,
	asegurando un error en estado estacionario (ESS) igual a cero.
\end{itemize}

	
\section{Objetivos Específicos}
\begin{itemize}
	\item Comprender el efecto de cada componente del controlador PID y su influencia en la rapidez de
	respuesta,el amortiguamiento del sistema, la reducción del error y la estabilidad.
	\item Capacidad para ajustar los parámetros del controlador utilizando diferentes métodos de sintonía
	y evaluar el desempeño de diferentes implementaciones de controladores PID.
	\item Verificar y analizar experimentalmente el comportamiento del sistema controlado, observando
	los efectos del controlador sobre la dinamica de la planta y los costos del control.
\end{itemize}
	\section{Desarrollo}

\subsection{Modelado del Sistema}

\subsubsection{Obtención de las matrices del sistema $F$, $G$, $H$ y $J$}

Para el modelado del sistema se parte del circuito mostrado en la Figura~\ref{fig:circuit_planta}, a partir del cual se determinan las ecuaciones de estado mediante el análisis de los lazos de realimentación y las relaciones de tensión en los componentes.

\insertarfigura{Otros/circuit.png}{Circuito de la planta.}{fig:circuit_planta}{1}

El sistema se describe mediante las siguientes ecuaciones en espacio de estados:

\begin{equation}
	\dot{x}(t) = F\,x(t) + G\,V_i(t)
	\label{eq:1}
\end{equation}

\begin{equation}
	y(t) = H\,x(t) + J\,V_i(t)
	\label{eq:2}
\end{equation}

Para obtener las expresiones de las variables de estado, se parte del equivalente del paralelo entre un resistor y un capacitor:
\[
R \parallel \frac{1}{sC} = \frac{R}{1 + sRC}
\]

Considerando que ambos amplificadores operacionales se encuentran en configuración no inversora, se obtienen las siguientes relaciones:

\[
V_a = \frac{-R_2}{1 + sR_2C_1}\frac{V_i}{R_1}
\quad \Rightarrow \quad
sV_a = -\frac{1}{R_1C_1}V_i - \frac{1}{R_2C_1}V_a
\]

\[
V_o = -\frac{R_4}{1 + sR_4C_2}\frac{V_a}{R_3}
\quad \Rightarrow \quad
sV_o = -\frac{1}{R_3C_2}V_a - \frac{1}{R_4C_2}V_o
\]

Definiendo como variables de estado $x_1(t) = V_a$ y $x_2(t) = V_o$, las ecuaciones anteriores se expresan en forma matricial como:

\[
\begin{bmatrix}
	\dot{x}_1(t) \\[4pt]
	\dot{x}_2(t)
\end{bmatrix}
=
\begin{bmatrix}
	-\dfrac{1}{R_2C_1} & 0 \\[4pt]
	-\dfrac{1}{R_3C_2} & -\dfrac{1}{R_4C_2}
\end{bmatrix}
\begin{bmatrix}
	x_1(t) \\[4pt]
	x_2(t)
\end{bmatrix}
+
\begin{bmatrix}
	-\dfrac{1}{R_1C_1} \\[4pt]
	0
\end{bmatrix}
V_i(t)
\]

y la ecuación de salida queda definida como:

\[
y(t) =
\begin{bmatrix}
	0 & 1
\end{bmatrix}
\begin{bmatrix}
	x_1(t) \\[4pt]
	x_2(t)
\end{bmatrix}
+ 0\cdot V_i(t)
\]

Sustituyendo los valores de los componentes 
$C_1 = 211.1\times10^{-9}\,\text{F}$, 
$R_1 = 80.55\times10^{3}\,\Omega$, 
$R_2 = 81.09\times10^{3}\,\Omega$, 
$C_2 = 103.07\times10^{-9}\,\text{F}$, 
$R_3 = 14.878\times10^{3}\,\Omega$ y 
$R_4 = 14.76\times10^{3}\,\Omega$, 
se obtienen las siguientes matrices numéricas:

\[
F =
\begin{bmatrix}
	-58.42 & 0 \\[4pt]
	-652.11 & -657.37
\end{bmatrix}, \quad
G =
\begin{bmatrix}
	-58.81 \\[4pt]
	0
\end{bmatrix}, \quad
H =
\begin{bmatrix}
	0 & 1
\end{bmatrix}
\]
\begin{equation}
	y \quad
	J = 0
	\label{eq:J}
\end{equation}
\subsubsection{Mostrar el diagrama de bloques del sistema}


\subsection{Discretización del Sistema}
\insertarfigura{Otros/Diagramas1.png}{Diagrama de bloques del sistema continuo.}{fig:diag_continuo}{1}

\subsubsection{Elección del tiempo de muestreo $T_s = 1~\text{ms}$}

Para la discretización del sistema continuo descrito por las ecuaciones~(\ref{eq:1}) y~(\ref{eq:2}), se busca obtener un modelo equivalente en tiempo discreto que relacione las variables de estado y la señal de entrada en instantes de muestreo definidos.  
Las ecuaciones del sistema discreto se expresan como:

\begin{equation}
	X(k+1) = A\,X(k) + B\,u(k)
	\label{eq:3}
\end{equation}

\begin{equation}
	Y(k) = C\,X(k) + D\,u(k)
	\label{eq:4}
\end{equation}

Usando las matrices continuas $F$, $G$, $H$ y $J$ obtenidas previamente, las matrices discretas se determinan mediante las siguientes expresiones:

\[
A = \mathrm{e}^{F T_s}, \qquad 
B = F^{-1}\!\left(\mathrm{e}^{F T_s} - I\right)G, \qquad 
C = H \qquad 
\]

\[
 \& \quad D = J
\]
\subsubsection{Obtención de las matrices discretas $A$, $B$, $C$ y $D$}

Con un tiempo de muestreo $T_s = 1~\text{ms}$, se obtienen las siguientes matrices discretizadas:

\[
A =
\begin{bmatrix}
	0.943 & 0 \\[4pt]
	-0.462 & 0.518
\end{bmatrix}, \qquad
B =
\begin{bmatrix}
	-0.0571 \\[4pt]
	0.0152
\end{bmatrix}, 
\]
\begin{equation}
	C =
	\begin{bmatrix}
		0 & 1
	\end{bmatrix}, \qquad \& \qquad
	D = 0
	\label{eq:d}
\end{equation}

Estas matrices representan el modelo digital equivalente del sistema continuo, y serán utilizadas posteriormente para el diseño del controlador e implementación en el \texttt{PSoC}.

\subsubsection{Diagrama de bloques del sistema discretizado}

En la Figura~\ref{fig:diag_discreto} se presenta el diagrama de bloques correspondiente al sistema discretizado, donde se observa la relación entre las variables de estado, la entrada $u(k)$ y la salida $Y(k)$.

\insertarfigura{Otros/Diagramas2.png}{Diagrama de bloques del sistema discretizado.}{fig:diag_discreto}{1}


\subsubsection{Verificación de la controlabilidad y observabilidad del sistema}

La matriz de controlabilidad se obtiene a partir de la siguiente relación general:

\[
x(n) - A^{n}x(0) = \sum_{i=0}^{n-1} A^{n-i-1}B\,u(i)
\]

lo que lleva a la siguiente forma matricial:

\[
x(n) - A^{n}x(0) =
\begin{bmatrix}
	A^{n-1}B & A^{n-2}B & \cdots & AB & B
\end{bmatrix}
\begin{bmatrix}
	u(0) \\[2pt]
	u(1) \\[2pt]
	\vdots \\[2pt]
	u(n-1)
\end{bmatrix}
\]

De esta expresión, se define la matriz de controlabilidad como:

\[
\mathcal{C} =
\begin{bmatrix}
	B & AB
\end{bmatrix}
\]

Para el sistema analizado, la matriz resultante es:

\[
\mathcal{C} =
\begin{bmatrix}
	-0.05344 & -0.05713 \\[4pt]
	0.03435 & 0.01527
\end{bmatrix}
\]

El rango de esta matriz es $n = 2$, lo que indica que el sistema es completamente controlable.

\bigskip
La matriz de observabilidad se obtiene a partir de la expresión general:

\[
Y(n-1) = 
\begin{bmatrix}
	C \\[4pt]
	CA \\[4pt]
	\vdots \\[4pt]
	CA^{n-1}
\end{bmatrix}
X(0)
\]

Por lo tanto, la matriz de observabilidad queda definida como:

\[
\mathcal{O} =
\begin{bmatrix}
	C \\[4pt]
	CA
\end{bmatrix}
\]

Sustituyendo los valores del sistema:

\[
\mathcal{O} =
\begin{bmatrix}
	0 & 1 \\[4pt]
	-0.462 & 0.518
\end{bmatrix}
\]

El rango de la matriz de observabilidad también resulta ser $n = 2$.\\  
Por lo tanto, se concluye que el sistema es **completamente controlable y observable**, cumpliendo con las condiciones necesarias para el diseño de control mediante realimentación de estados.

\subsubsection{Comparar los resultados obtenidos con las simulaciones realizadas en \texttt{MATLAB}}
\textbf{Resultados de las matrices continuas:}
\[
F =
\begin{bmatrix}
	-58.42 & 0 \\[4pt]
	-652.10 & -657.30
\end{bmatrix}, \quad
G =
\begin{bmatrix}
	-58.81 \\[4pt]
	0
\end{bmatrix}, \quad
H =
\begin{bmatrix}
	0 & 1
\end{bmatrix}
\]

\[
	\& \quad
	J = [\,0\,]
\]

\textbf{Resultados de las matrices discretas:}
\[
A =
\begin{bmatrix}
	0.9433 & 0 \\[4pt]
	-0.4628 & 0.5182
\end{bmatrix}, \quad
B =
\begin{bmatrix}
	-0.05712 \\[4pt]
	0.01527
\end{bmatrix}, \quad
C =
\begin{bmatrix}
	0 & 1
\end{bmatrix}
\]

\[
	\& \quad
	D = [\,0\,]
\]
Se observa que los resultados de las matrices obtenidas son congruentes con los valores calculados en las ecuaciones~(\ref{eq:J}) y~(\ref{eq:d}), verificando la coherencia entre el modelo teórico y los resultados obtenidos mediante \texttt{MATLAB}.

\insertarfigurawide{matlab/Sim1_noImplementable.png}{Respuesta simulada en \texttt{MATLAB} correspondiente al Caso 2($f_s = 1000 Hz$).}{fig:simulacion_noImplementable}{1}

\insertarfigurawide{matlab/Sim2_noImplementable.png}{Respuesta simulada en \texttt{MATLAB} correspondiente al Caso 2($f_s = 1000 Hz$).}{fig:simulacion_noImplementable2}{1}

En las figuras \ref{fig:simulacion_noImplementable} y \ref{fig:simulacion_noImplementable2}, del \texttt{MATLAB} se pueden observar que no es implementable.
	\section{Resultados}

\subsection{Respuestas temporales (simulado vs.\ experimental)}
En esta sección se presentan comparaciones \emph{en el dominio del tiempo} entre simulación y experimento para: planta sin compensar, C1 (proporcional), C2 (lead), C3 (integrador+lead). En cada figura se incluyen las métricas \emph{RMSE}, \emph{NRMSE} y \(e_{\max}\) (definidas en las ecuaciones \eqref{eq:rmse_temporal}–\eqref{eq:emax_temporal}). Se muestran versiones \emph{con} y \emph{sin} remoción de offset para evidenciar el descalce de referencia señalado en la implementación.

% ===== AJUSTAR RUTAS =====
\insertarfigura{img/OpenLoop/comparacionLazoAbierto.png}
{Planta sin compensación en lazo abierto: respuesta al escalón (simulado vs.\ experimental, con bandas de tolerancia para \(\sigma = \frac{tolerance}{3}\)).}
{fig:step_openLoop}{1}



\insertarfigurawide{img/C1_Lead/conOffset}
{Compensador de adelanto: respuesta al escalón y esfuerzo (simulado vs.\ experimental, \emph{con} offset).}
{fig:step_c1_con_offset}{1}

\insertarfigurawide{img/C1_Lead/sinOffset}
{Compensador de adelanto: respuesta al escalón y esfuerzo (simulado vs.\ experimental, \emph{sin} offset).}
{fig:step_c1_sin_offset}{1}

\insertarfigurawide{img/C1_K/compConOffset}
{Compensador Proporcional: respuesta al escalón y esfuerzo (simulado vs.\ experimental, \emph{con} offset).}
{fig:step_prop_con_offset}{1}

\insertarfigurawide{img/C1_K/compSinOffset}
{Compensador Proporcional: respuesta al escalón y esfuerzo (simulado vs.\ experimental, \emph{sin} offset).}
{fig:step_prop_sin_offset}{1}

\insertarfigurawide{img/C2/conOffset}
{Compensador de adelanto + integrador: respuesta al escalón y esfuerzo (simulado vs.\ experimental, \emph{con} offset).}
{fig:step_c2_con_offset}{1}

\insertarfigurawide{img/C2/sinOFFset}
{Compensador de adelanto + integrador: respuesta al escalón y esfuerzo (simulado vs.\ experimental, \emph{sin} offset).}
{fig:step_c2_sin_offset}{0.92}



\subsection{Métricas de desempeño temporal}
Las métricas usadas en cada gráfica (y en la tabla resumen) son:
\begin{equation}
	\label{eq:rmse_temporal}
	\mathrm{RMSE}=\sqrt{\frac{1}{N}\sum_{k=1}^{N}\big(y[k]-\hat{y}[k]\big)^2},
\end{equation}
\begin{equation}
	\label{eq:rmse_porct}
	\mathrm{NRMSE}=\frac{\mathrm{RMSE}}{y_{\max}-y_{\min}},
\end{equation}
\begin{equation}
	\label{eq:emax_temporal}
	e_{\max}=\max_k\,\lvert y[k]-\hat{y}[k]\rvert.
\end{equation}
\balance
% ===== Resultados (solo tiempo) =====
En la tabla \ref{tab:comparativa_temporal_min}, los índices de \emph{tiempo de subida} \(t_r\) y \emph{sobreimpulso} \(M_p\) se obtienen de las curvas de esta sección y anteriores. El error en estado estacionario \(e_{ss}\) se reporta cuando aplica (lazo cerrado), y el \emph{error cuadratico medio normalizado} entre las mediciones y simulaciones se especifica tanto \emph{con} como \emph{sin} el offset de las señales.

% EN EL TEXTO (sin \onecolumn ni \balance aquí)
\begin{table}[t]
	\centering
	\caption{Comparativa temporal (simulado vs.\ experimental).}
	\label{tab:comparativa_temporal_min}
	\small
	\setlength{\tabcolsep}{4pt}
	\renewcommand{\arraystretch}{1.1}
	\begin{adjustbox}{max width=\columnwidth}
		\begin{tabular}{lcccc}
			\toprule
			\textbf{Sistema} &
			\makecell{\(\mathbf{t_r}\) [ms]\\(sim/exp)} &
			\makecell{\(\mathbf{M_p}\) [\%]\\(sim/exp)} &
			\makecell{\(\mathbf{e_{ss}}\) [\%]\\(sim/exp)} &
			\makecell{\(\mathbf{NRMSE}\) [\%]\\(con/sin off)}\\
			\midrule
			Lazo abierto       & 40.0 / 34.2   & 0 / 0                 & 0 / 1.8182             & 19.14 / 18.43 \\
			C1 (lead)          & 21.17 / 19.8  & 0 / 0                 & 45.77 / 42--46.79      & 38.02 / 7.31 \\
			Proporcional       & 3.736 / 3.950 & 15.4 / 10--20         & 12.5 / \(\approx 15.825\) & 19.24 / 6.57 \\
			Lead + integrador  & 13.7 / 17.10  & 7.433 / \(\approx 5\) & 0 / 0                  & 13.54 / 5.61 \\
			\bottomrule
		\end{tabular}
	\end{adjustbox}
\end{table}

\subsection{Error de velocidad (seguimiento de rampa)}
Para el compensador C2 (integrador+lead) se ensayó seguimiento a rampa \(r[k]=kT\) (pendiente \(1\ \mathrm{u}/\mathrm{s}\)). El error en régimen para sistemas tipo~1 en discreto es
\begin{equation}
	\label{eq:ess_rampa_discreto}
	e_{ss}=\frac{T}{K_{v,z}},\qquad K_{v,z}=\lim_{z\to1}(z-1)\,L(z).
\end{equation}
Con el \(L(z)\) obtenido, se midió/estimó \(K_v\simeq 78.43\Rightarrow e_{ss}^{(\mathrm{teo})}\approx 1/78.43\approx 1.28\%\), en concordancia con el valor observado tras un transitorio breve.

% ===== AJUSTAR RUTA =====
\insertarfigura{img/C2/rampaC2linda}
{Seguimiento de rampa con Compensador integrador + lead.}
{fig:rampa_c2}{0.92}

\subsection{Discusión de discrepancias}
Las diferencias entre curvas simuladas y experimentales se explican principalmente por: (i) \textbf{limitación/saturación} en la señal de esfuerzo debido a la carga en el DAC (la excitación queda “achatada”), y (ii) \textbf{descalce de referencia} (offset). A ello se suman tolerancias de componentes pasivos, lo que desplaza levemente parámetros característicos de la planta. Aun así, la \emph{dinámica global} buscada (forma de la respuesta y tiempos) se mantuvo acorde a la simulación.




\section{Conclusiones}

El conjunto de ensayos y comparaciones en el \emph{dominio del tiempo} demuestra que, a pesar de las limitaciones de implementación y de la dispersión de la planta real, el desempeño experimental se mantiene coherente con el diseño y dentro de las bandas esperadas. A continuación se sintetizan los hallazgos principales.

\subsection*{Ajuste simulación–experimento}
Al simular con la \textbf{misma entrada} medida y remover el \textbf{offset estático} entre DAC y ADC, el ajuste mejora de forma notable (véase la tabla~\ref{tab:comparativa_temporal_min}). Lo que persiste se explica por \emph{efectos dinámicos} no ideales: como saturaciones suaves del actuador (no linealidad) y  atenuaciones por carga o discrepancias entre resistencias que definen la ganancia.

\subsection*{Limitaciones de hardware observadas}
Se verificó que la \textbf{corriente de salida del DAC} no es suficiente para excitar directamente la planta ---aunque no logramos conseguir los rangos exactos de corriente que puede suministrar en el datasheet---, lo que produce \emph{achatamiento} y \emph{recorte} del esfuerzo de control y, en consecuencia, discrepancias de forma en las respuestas. Además, la referencia $\mathrm{VDDA}/2$ y las líneas de alimentación presentan \emph{desacople insuficiente}, favoreciendo derivas de nivel que se manifiestan como offset. Estas condiciones explican parte del desajuste restante aun tras corregir el offset en posprocesado.

\subsection*{Planta real vs.\ modelo nominal}
La planta implementada difiere levemente del modelo continuo asumido por \emph{tolerancias} de pasivos, ordenamiento de etapas e impedancia de entrada vista por la fuente del esfuerzo. Aun así, las respuestas medidas se ubican mayormente \textbf{dentro de la banda de tolerancias} obtenida por análisis estadístico, lo que respalda la \emph{validez del modelo} para propósitos de diseño y la \emph{robustez} del procedimiento seguido.



\subsection*{Acciones recomendadas}
\begin{itemize}
	\item Añadir un \textbf{buffer} a la salida del DAC para eliminar el error de carga en el DAC y evitar recortes del esfuerzo.
	\item \textbf{Biaspassear} la referencia $\mathrm{VDDA}/2$ y las líneas de alimentación, mejorando estabilidad de nivel y rechazo de ruido.
\end{itemize}

\subsection*{Conclusión general}
En conjunto, los resultados muestran que el \textbf{método de diseño es robusto}: aun bajo carga del DAC, offsets y dispersión de componentes, las respuestas experimentales se mantienen \emph{razonablemente cercanas} a las simuladas, especialmente con el compensador  con integrador + adelanto, que satisface los objetivos de seguimiento y estabilidad con un compromiso adecuado entre rapidez y exactitud.

	\onecolumn
	\tableofcontents
	
\end{document}