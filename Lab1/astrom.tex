\subsubsection{Implementación del PID de Astr\"om}

\onecolumn
\begin{lstlisting}[style=matlabstyle,caption={Script en Matlab},label={lst:mat6}]
function sim_pid_euler_astrom(G, T, Kp, Ti, Td,N,stepSize,sigma,umin,umax)

% ===== Coeficientes PID incremental (Euler + filtro derivativo N) =====
alpha2 = Td*Ti/(T^2);
alpha1 = N*Ti/T;
beta2  = Kp*(N*Td*Ti + Td*Ti)/(T^2);
beta1  = Kp*(N*Ti + Td)/T;
beta0  = Kp*N;
U0 = alpha2 + alpha1;
U1 = -(2*alpha2 + alpha1);
U2 = alpha2;
E0 = beta2 + beta1 + beta0;
E1 = -(2*beta2 + beta1);
E2 = beta2;
% --- Control analógico ---
numC = Kp*[Ti*Td, Ti, 1];
denC = [Ti, 0];
controlador = tf(numC, denC);
cloop_c = feedback(controlador*G, 1);
info = stepinfo(cloop_c);
tend = info.SettlingTime*3;
wn = (1.8/ info.RiseTime);
Tstep = (2*pi/wn)/3000;
t = 0:Tstep:tend;
td = 0:T:tend;
%yc = step(cloop_c, t);
refc  = stepSize*ones(size(t)) + sigma*randn(size(t));   % ref continua
% Salida continua con la MISMA referencia (ZOH)
yc = lsim(cloop_c, refc, t);
% "Digitalización" por muestreo ideal en kT:
refd  = interp1(t, refc, td, 'linear', 'extrap');        % r[k] = r_c(kT)
% --- Discretización y lazo discreto ---
td = 0:T:tend;
Gd = c2d(G, T, 'zoh');
[numD, denD] = tfdata(Gd, 'v');
b0 = numD(2);
b1 = numD(3);
a1 = denD(2);
a2 = denD(3);
yd   = zeros(size(td));
ed   = zeros(size(td));
ud   = zeros(size(td));
%refd = ones(size(td));
for k = 3:numel(td)-1
	ed(k) = refd(k) - yd(k);
	ud_sinsaturar = (-U1*ud(k-1) + -U2*ud(k-2) + E0*ed(k) + E1*ed(k-1) + E2*ed(k-2))/U0;
	if ud_sinsaturar >umax
		ud(k)=umax;
	elseif ud_sinsaturar<umin
		ud(k) = umin;
	else
		ud(k) = ud_sinsaturar;
	end
	yd(k+1) = b0*ud(k) + b1*ud(k-1) - a1*yd(k) - a2*yd(k-1);
end
% === Graficar ===
yshift = circshift(yd, -2);
figure;
yyaxis left
plot(t*1000, refc, ':','Color','#B0B0B0'); hold on;
plot(t*1000, yc, '--','LineWidth',1.3);
stairs(td*1000, yshift,'-','LineWidth',1.2);
ylabel('Salida');
yyaxis right
plot(td*1000, ud, 'r--','LineWidth',1.2);
ylabel('u_d(k)');
xlabel('Tiempo [ms]');
title(sprintf('PID Astrom: Kp=%.3g, Ti=%.3g, Td=%.3g, N=%.3g', Kp, Ti, Td,N));
legend(sprintf('Escalon Ruidoso (A=%.3g, $\\sigma$=%.3g)', stepSize, sigma), ...
'Respuesta al Escalon Continua sin Wind-Up', ...
'Respuesta al Escalon Discreta con Wind-Up', ...
sprintf('Esfuerzo de Control (Rango:%.3g-%.3g)',umin,umax), ...
'Location', 'best', 'Interpreter','latex');
grid on;
end

	
\end{lstlisting}

\twocolumn