
\subsubsection{ Implementación con un tiempo de muestreo de 10 veces el ancho de banda del sistema
}


Si deseamos muestrear con una frecuencia 10 veces superior el ancho de banda estimado:
\[
f_s = 10 f_{\mathrm{bw}}, \qquad T = \frac{1}{f_s}.
\]

La simulación se realiza con:
\[
\texttt{sim\_pid\_euler\_ideal(G,\,T,\,Kp,\,Ti,\,Td,\,1);}
\]

En la Figura~\ref{fig:fs10} se observa que la respuesta discreta no reproduce bien a la continua: se introduce retardo debido a la frecuencia de muestreo insuficiente, oscila al inicio del seguimiento y en general, la respuesta al escalón no desempeña como se planteó en el diseño continuo.
Un muestreo de 10 veces el ancho de banda se hace incapaz de emular al compesandor analógico diseñado.




\insertarfigurawide{./img/fs10.png}{Respuesta al escalón con $f_s=10 f_{\mathrm{bw}}$ (curva azul), su versión continua (curva azul puntuada) y el esfuerzo generado por el compensador (curva naranja puntuada), utilizando el primer compensador diseñado.}{fig:fs10}{1}



\subsubsection{ Implementación con un tiempo de muestreo de 30 veces el ancho de banda del sistema
}

Al triplicar la frecuencia de muestreo:
\[
f_s = 30 f_{\mathrm{bw}}, \qquad T = \frac{1}{f_s},
\]
Podemos ver en la Figura~\ref{fig:fs30} muestra una mejora clara: la respuesta discreta se asemeja mucho más a la continua, con menor error con respecto a la misma.


\insertarfigurawide{./img/fs30.png}{Respuesta al escalón con $f_s=30 f_{\mathrm{bw}}$ (curva azul), su versión continua (curva azul puntuada) y el esfuerzo generado por el compensador (curva naranja puntuada), utilizando el primer compensador diseñado.}{fig:fs30}{1}
\paragraph{Limitación con Ziegler--Nichols.}  
Al usar parámetros obtenidos con el método de Ziegler-Nichols, el control logra buen seguimiento, pero el esfuerzo de control $u_d[k]$ supera en más de dos órdenes de magnitud la capacidad de los DAC del PSoC (Figura~\ref{fig:zn_effort}), lo que lo vuelve una solución impracticable.


\insertarfigurawide{./img/zn_effort.png}{Respuesta al escalón con $f_s=30 f_{\mathrm{bw}}$ (curva azul), su versión continua (curva azul puntuada) y el esfuerzo generado por el compensador (curva naranja puntuada), utilizando el segundo compensador diseñado.}{fig:zn_effort}{1}

Para evidenciarlo, añadimos ruido y saturación al esfuerzo en la simulación. El resultado (Figura~\ref{fig:sat_noise}) confirma que el sistema no puede seguir al escalón bajo esas condiciones, por lo que el compensador es descartado.


\insertarfigurawide{./img/sat_noise.png}{Respuesta al escalón con $f_s=30 f_{\mathrm{bw}}$ (curva azul), su versión continua (curva azul puntuada) y el esfuerzo generado por el compensador (curva naranja puntuada) saturable, utilizando el segundo compensador diseñado, añadiendo ruido a la entrada (curva gris clara).}{fig:sat_noise}{1}

\clearpage
\newpage
La función utilizada en para la simulación anterior es la siguiente:

\begin{strip}
	\vspace{-\baselineskip} % ajusta si ves demasiado aire arriba
	\noindent
	\begin{minipage}{0.98\textwidth}
		
\begin{lstlisting}[language=Matlab,style=matlabstyle, caption={función para la simulación del PID discreto con saturación y ruido.}, label={lst:pid_sat_noise}]
	function sim_pid_euler(G, T, Kp, Ti, Td, stepSize, sigma, umin, umax)
		% --- Control continuo (referencia) ---
		numC = Kp*[Ti*Td, Ti, 1]; denC = [Ti, 0];
		controlador = tf(numC, denC);
		cloop_c = feedback(controlador*G, 1);
		info  = stepinfo(cloop_c);
		tend  = 3*info.SettlingTime;
		wn    = 1.8/info.RiseTime;
		Tstep = (2*pi/wn)/1000;
		t  = 0:Tstep:tend;
		td = 0:T:tend;
		
		% Referencia con ruido
		refc = stepSize*ones(size(t)) + sigma*randn(size(t));
		yc   = lsim(cloop_c, refc, t);
		refd = interp1(t, refc, td, 'linear', 'extrap');
		
		% --- Planta discreta (ZOH) ---
		Gd = c2d(G, T, 'zoh');
		[numD, denD] = tfdata(Gd, 'v');
		b0 = numD(2); b1 = numD(3);
		a1 = denD(2); a2 = denD(3);
		
		% Inicialización
		yd = zeros(size(td));
		ed = zeros(size(td));
		ud = zeros(size(td));
		
		% --- PID incremental con saturación ---
		for k = 3:numel(td)-1
			ed(k) = refd(k) - yd(k);
			u_inc = ud(k-1) + Kp*((1+T/Ti+Td/T)*ed(k) ...
			-(1+2*Td/T)*ed(k-1) ...
			+(Td/T)*ed(k-2));
			% Saturación
			if u_inc > umax
				ud(k) = umax;
			elseif u_inc < umin
				ud(k) = umin;
			else
				ud(k) = u_inc;
			end
			% Planta discreta
			yd(k+1) = b0*ud(k) + b1*ud(k-1) - a1*yd(k) - a2*yd(k-1);
		end
		
		% --- Gráficos ---
		yshift = circshift(yd, -2);
		figure; hold on; grid on;
		yyaxis left
		plot(t*1000, refc, ':','Color',[0.7 0.7 0.7]);
		plot(t*1000, yc,  '--','LineWidth',1.3);
		stairs(td*1000, yshift,'-','LineWidth',1.2);
		ylabel('Salida');
		yyaxis right
		plot(td*1000, ud, 'r--','LineWidth',1.2);
		ylabel('u_d[k]');
		xlabel('Tiempo [ms]');
		title(sprintf('PID discreto: Kp=%.3g, Ti=%.3g, Td=%.3g', Kp, Ti, Td));
		legend('Referencia','Continuo ideal','Discreto sat+ruido','u_d');
	end
\end{lstlisting}
\end{minipage}
\vspace{-\baselineskip} % ajusta si ves aire abajo
\end{strip}
\clearpage
\newpage
\balance
\subsubsection{Implementación del modelo alternativo y análisis con ruido y saturación}

Dado que la sintonía inicial de Ziegler--Nichols exige un esfuerzo de control incompatible con el actuador disponible, nos centramos en implementar el \emph{modelo alternativo} (PID ajustado por prueba y error con discretización incremental).  

\paragraph{Efectos del ruido (escenario amplificado).}
Para visualizar con claridad la sensibilidad al ruido (especialmente por el término derivativo), forzamos un caso más severo: desviación estándar del \(10\%\) de la amplitud de referencia. En la simulación:


\begin{lstlisting}[language=Matlab,breaklines=true,basicstyle=\ttfamily\footnotesize]
	sim_pid_euler(G,T/30,Kp,Ti,Td,2.5,0.25,0,5);
\end{lstlisting}

La Figura~\ref{fig:ruido_10pct} evidencia el efecto del ruido sobre \(u_d[k]\) y la salida discreta, pero sigue dando un resultado aceptable en el seguimiendo de la referencia aun en este caso extremo.



\paragraph{Efectos de la saturación (windup) sin ruido.}
Para aislar el fenómeno de \emph{windup} quitamos el ruido y exigimos amplitud elevada, manteniendo límites del actuador:
\begin{lstlisting}[language=Matlab,breaklines=true,basicstyle=\ttfamily\footnotesize]
	sim_pid_euler(G,T/30,Kp,Ti,Td,5,0,0,5);
\end{lstlisting}

La Figura~\ref{fig:windup} muestra la saturación de \(u_d[k]\) y la consecuente degradación del seguimiento: lento despegue y respuesta sobreamortiguada.

\clearpage

\insertarfigurawide{./img/ruido_10pct.png}{Respuesta al escalón con $f_s=30 f_{\mathrm{bw}}$ (curva azul), su versión continua (curva azul puntuada) y el esfuerzo generado por el compensador (curva naranja puntuada) saturable, utilizando el primer compensador diseñado, añadiendo ruido a la entrada (curva gris clara) con \(\sigma = 10\%\).}{fig:ruido_10pct}{1}

\insertarfigurawide{./img/windup.png}{Respuesta al escalón con $f_s=30 f_{\mathrm{bw}}$ (curva azul), su versión continua (curva azul puntuada) y el esfuerzo generado por el compensador (curva naranja puntuada) saturable, utilizando el primer compensador diseñado, con un escalón de 5V.}{fig:windup}{1}

\clearpage