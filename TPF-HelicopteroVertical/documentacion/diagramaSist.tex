\section{Diagrama del Sistema}
La \textbf{Figura~\ref{fig:DiagramaControl}} muestra el diagrama de bloques del sistema de control de altura en lazo cerrado. La referencia de altura \(z_{\text{ref}}\) es comparada con la señal de salida medida \(z(t)\), generando el error \(e(t)\), el cual es procesado por el controlador.

El controlador representa el algoritmo de control implementado, pudiendo corresponder a un controlador estudiado en el semestre. A partir del error, el controlador genera la señal de control \(u(t)\), que constituye el esfuerzo aplicado al sistema.

La señal \(u(t)\) actúa sobre el actuador, conformado por el motor brushless y la hélice, el cual convierte la señal de control en empuje mecánico. Dicho empuje excita la planta física, cuya dinámica vertical determina la altura real del sistema \(y(t)\).

La altura es medida mediante un sensor TFMini Plus, que proporciona la señal \(z(t)\) utilizada para cerrar el lazo de control. En el diagrama se indican explícitamente las fuentes de ruido asociadas tanto al controlador como al sensor, reflejando las perturbaciones y no idealidades presentes en el sistema real.

\insertarfigura{img/Planta/Diagrama.jpg}{Diagrama de bloques de Control.}{fig:DiagramaControl}{1}

La \textbf{Figura~\ref{fig:Diagrama2}} muestra un diagrama de bloques que representa la implementación física y funcional completa del sistema de control de altura. En dicho diagrama se identifican claramente los siguientes subsistemas:
\begin{itemize}
	\item la cadena de energía (\texttt{batería} $\rightarrow$ \texttt{ESC} $\rightarrow$ \texttt{motor}),
	\item la cadena de control (\texttt{MATLAB} $\rightarrow$ \texttt{PSoC} $\rightarrow$ \texttt{PWM}),
	\item la planta física correspondiente al sistema bajo control,
	\item y la cadena de medición y realimentación (\texttt{planta} $\rightarrow$ \texttt{sensor} $\rightarrow$ \texttt{PSoC} $\rightarrow$ \texttt{MATLAB}).
\end{itemize}



\insertarfigura{img/Planta/Diagrama-Page-2.jpg}{Diagrama de bloques, implementación física.}{fig:Diagrama2}{1}

\subsection{Flujo de energía (parte electro--energética)}

La batería utilizada es una LiPo de tres celdas (3S), con tensión nominal de \(11{,}1\,\text{V}\). 
Cada celda presenta una tensión máxima de \(4{,}2\,\text{V}\), por lo que la batería completamente cargada alcanza:

\[
V_{\text{max}} = 12{,}6\,\text{V}
\]

Con el objetivo de preservar la integridad química de la batería y evitar degradación prematura, el rango operativo se restringe aproximadamente a:

\[
V \in [11{,}5,\; 12{,}5]\,\text{V}
\]

Evitar descargas por debajo de \(11{,}5\,\text{V}\) resulta fundamental, ya que tensiones inferiores pueden producir daño irreversible en las celdas LiPo.

La batería alimenta directamente al controlador electrónico de velocidad (ESC) de \(40\,\text{A}\), el cual convierte la tensión continua en señales trifásicas moduladas para accionar el motor brushless. 

El ESC incorpora disipadores térmicos integrados sobre los dispositivos de potencia (MOSFETs), cuya función es evacuar el calor generado durante la conmutación y conducción de corriente. Dado que el sistema opera en un régimen de potencia relativamente elevado para el tamaño del conjunto, la gestión térmica resulta crítica para evitar sobrecalentamientos que puedan producir fallas o reducción de eficiencia.

La presencia de disipadores mejora la transferencia térmica hacia el ambiente, aumentando la confiabilidad del sistema durante los intervalos de operación de aproximadamente \(5\,\text{min}\) continuos.

A medida que la batería se descarga, se observa una disminución progresiva de la tensión disponible, lo que impacta directamente en la capacidad de generación de empuje del motor. Este fenómeno es perceptible incluso en aplicaciones aeronáuticas (como drones), donde la pérdida de tensión se traduce en menor capacidad de sustentación.

En la práctica experimental, el tiempo de operación continua es del orden de \(5\,\text{min}\) como máximo. Esto se debe a que la batería utilizada posee una capacidad limitada para el nivel de potencia demandado por el sistema. 

El tiempo típico de recarga completa es aproximadamente \(1{,}5\,\text{h}\), lo que introduce una restricción significativa en la repetibilidad de los ensayos experimentales, constituyendo un factor limitante en el desarrollo de la práctica.

En esta etapa del sistema no se realiza acción de control propiamente dicha, sino únicamente conversión y transferencia de potencia hacia el actuador.



\subsection{Flujo de control (parte de mando)}

El PSoC actúa como el controlador embebido del sistema. A partir de los algoritmos de control implementados (PID, control en espacio de estados, LQG, entre otros), el PSoC genera una señal PWM de tipo servo que es enviada al ESC.

Esta señal PWM representa la variable de control \(u(t)\) del sistema y determina el nivel de empuje aplicado al motor. Para asegurar una correcta referencia eléctrica y el funcionamiento adecuado del sistema, el PSoC y el ESC comparten una conexión de tierra común (GND).

El sistema MATLAB se comunica con el PSoC mediante una interfaz de datos seriales, lo que permite:
\begin{itemize}
	\item enviar referencias de altura,
	\item modificar parámetros de control,
	\item recibir y visualizar datos del sistema en tiempo real.
\end{itemize}

En este esquema, MATLAB cumple una función de supervisión, ajuste y análisis experimental, mientras que el PSoC ejecuta el control en tiempo real.

\subsection{Medición y realimentación (cierre del lazo)}

La planta física, al desplazarse verticalmente, genera una altura real que es medida mediante el sensor de distancia láser TFMini Plus. Este sensor entrega la medición de altura al PSoC a través de su interfaz de recepción (RX).

La señal medida es utilizada por el PSoC para calcular el error entre la referencia y la salida real del sistema, cerrando de esta manera el lazo de control de altura. Adicionalmente, los datos medidos pueden ser enviados a MATLAB para su visualización, almacenamiento y análisis experimental.


