% ============================================================

Se diseñó un regulador óptimo discreto de tipo LQR a partir del modelo
en espacio de estados discretizado mediante retención de orden cero (ZOH)
con tiempo de muestreo:

\[
T_s = 0.01 \ \text{s}
\]

El objetivo del diseño es minimizar el funcional de costo cuadrático infinito:

\[
J=\sum_{k=0}^{\infty}(x_k^TQx_k+u_k^TRu_k)
\]

donde \(Q\succeq 0\) penaliza la energía de los estados y \(R\succ 0\) penaliza
el esfuerzo de control.

% ============================================================
\paragraph{Matrices de ponderación}
% ============================================================

Se adoptó una sintonización práctica basada en escalas típicas del experimento:
un cambio de referencia del orden de \(\Delta y \approx 20\) y una restricción del
mando aproximadamente \(|u|\lesssim 300\) (en unidades del actuador).

El diseño penaliza principalmente la salida mediante \(C^TC\), agregando un término
pequeño para asegurar buena condición numérica:

\[
Q = w_y(C^TC) + 10^{-8}I_n,
\qquad
R = \frac{w_u}{u_{\max}}
\]

con \(w_y=20\), \(w_u=500\) y \(u_{\max}=300\).

Para el modelo discretizado utilizado, las matrices resultantes fueron:

\[
Q=
\begin{bmatrix}
	6.8188\times 10^{-3} & -6.9883\times 10^{-3} &  7.0544\times 10^{-3}\\
	-6.9883\times 10^{-3} &  7.1620\times 10^{-3} & -7.2298\times 10^{-3}\\
	7.0544\times 10^{-3} & -7.2298\times 10^{-3} &  7.2982\times 10^{-3}
\end{bmatrix}
\]

\[
R = 1.6667
\]

% ============================================================
\paragraph{Ecuación de Riccati discreta y ganancia óptima}
% ============================================================

La solución del problema se obtiene resolviendo la ecuación de Riccati discreta:

\[
P=A^TPA - A^TPB(R+B^TPB)^{-1}B^TPA + Q
\]

La ganancia óptima resulta:

\[
K=(R+B^TPB)^{-1}B^TPA
\]

Para el modelo discretizado se obtuvo:

\[
K=
\begin{bmatrix}
	0.5119095 & -0.4937470 & 0.4765571
\end{bmatrix}
\]

% ============================================================
\paragraph{Polos del sistema en lazo cerrado}
% ============================================================

La estabilidad se verifica mediante los polos del sistema:

\[
A_{\text{cl}} = A - BK
\]

Los polos obtenidos fueron:

\[
\lambda(A-BK)=
\begin{aligned}
	&0.9728458 + 0.0327300\,j\\
	&0.9728458 - 0.0327300\,j\\
	&0.9663930
\end{aligned}
\]

Se observa que todos los polos se encuentran dentro del círculo unitario,
garantizando estabilidad discreta.

% ============================================================
\subsubsection{Simulación del LQR con observador, saturación y ruido}
% ============================================================

Con el fin de aproximar el comportamiento del sistema real y evitar conclusiones
optimistas, se implementó un entorno de simulación que contempla:

\begin{itemize}
	\item Observador de estados en dos variantes:
	\begin{itemize}
		\item Predictor: corrige usando \(y_k\).
		\item Actual: corrige usando \(y_{k+1}\).
	\end{itemize}
	\item Prefiltro de referencia \(N_{\mathrm{bar}}\).
	\item Saturación del actuador con límite \(\pm u_{\max}\).
	\item Ruido de medición, ruido de actuador y ruido de proceso configurables.
	\item Comparación doble precisión vs \texttt{float32}.
\end{itemize}

Los observadores se diseñaron mediante ubicación de polos:

\[
p_{\mathrm{obs}}=\{0.8\pm 0.25j,\ 0.9\}
\]

\[
K_{e,\mathrm{pred}}=\mathrm{place}(A^T,C^T,p_{\mathrm{obs}})^T
\qquad
K_{e,\mathrm{act}}=\mathrm{place}(A^T,(CA)^T,p_{\mathrm{obs}})^T
\]

La estructura implementada fue:

\[
u_k = N_{\mathrm{bar}}\,r_k - K\,\hat{x}_k
\]

\insertarfigura{img/LQR/LQR_step.png}
{Respuesta temporal simulada del sistema con regulador LQR y observador.}
{fig:lqr_step_sim}{1}

\insertarfigura{img/LQR/LQR_esfuerzo.png}
{Esfuerzo de control simulado $u(k)$.}
{fig:lqr_u_sim}{1}

% ============================================================
\subsubsection{Implementación experimental}
% ============================================================

\insertarfigura{img/LQR/LQR_pred_practico.png}
{Respuesta experimental con observador predictor.}
{fig:lqr_pred_prac}{1}

\insertarfigura{img/LQR/LQR_act_practico.png}
{Respuesta experimental con observador actual.}
{fig:lqr_act_prac}{1}

En simulación el tiempo de subida fue aproximadamente:

\[
t_r^{\text{sim}} \approx 0.75\,\text{s}
\]

sin sobreimpulso apreciable.

Experimentalmente se observaron tiempos de subida del orden de
\(0.55\,\text{s}\) a \(0.69\,\text{s}\) dependiendo del punto de operación,
con error estacionario inferior a 1 cm.

Se evidenció nuevamente la presencia de ruido significativo en el esfuerzo,
más marcado en el observador predictor que en el actual.

% ============================================================
\subsubsection{Conclusión}
% ============================================================

A diferencia del método de ubicación arbitraria de polos, el regulador LQR
no fija directamente las raíces del sistema en lazo cerrado, sino que
determina la ganancia óptima a partir de la minimización explícita de un
funcional de costo que pondera la energía de los estados y el esfuerzo
de control.

En este enfoque, los polos resultan consecuencia de los pesos definidos
en \(Q\) y \(R\). Esto introduce un criterio sistemático basado en la
relación esfuerzo–desempeño.

En este trabajo, al utilizar un modelo identificado sin interpretación
física directa de los estados, no fue posible penalizar estados
individuales (como velocidad o inercia del motor). Por ello se penalizó la
salida mediante \(C^TC\).

En modelos físicos estructurados, el LQR permitiría penalizar estados
específicos según objetivos de desempeño, lo que constituye una ventaja
conceptual importante frente a la ubicación arbitraria de polos.

La selección de \(Q\) y \(R\) requiere criterio ingenieril y ajustes
iterativos. Sin embargo, una vez definidos adecuadamente, el método
proporciona resultados consistentes y estructurados.

Persiste la sensibilidad al ruido de medición introducida por el
observador. Polos de estimación agresivos amplifican ruido; polos
más lentos reducen este efecto pero enlentecen la convergencia.

En conjunto, el regulador LQR mostró un desempeño experimental sólido,
con seguimiento estable y transitorios coherentes con la simulación,
constituyendo un enfoque más fundamentado que la simple ubicación
arbitraria de polos.
