% ============================================================

% ============================================================

Se diseñó un regulador óptimo discreto de tipo LQR a partir del modelo
en espacio de estados discretizado mediante retención de orden cero (ZOH)
con tiempo de muestreo:

\[
\Ts = 0.01 \ \text{s}
\]

El objetivo del diseño es minimizar el funcional de costo cuadrático infinito:

\[
J=\sum_{k=0}^{\infty}(x_k^TQx_k+u_k^TRu_k)
\]

donde \(Q\succeq 0\) penaliza la energía de los estados y \(R\succ 0\) penaliza
el esfuerzo de control.

\paragraph{Matrices de ponderación}

Se adoptó una sintonización práctica basada en escalas típicas del experimento:
un cambio de referencia del orden de \(\Delta y \approx 20\) y una restricción del
mando aproximadamente \(|u|\lesssim 300\) (en unidades del actuador).
El diseño penaliza principalmente la salida mediante \(C^TC\), agregando un término
pequeño para asegurar buena condición numérica:

\[
Q = w_y(C^TC) + 10^{-8}I_n,
\qquad
R = \frac{w_u}{u_{\max}}
\]

con \(w_y=20\), \(w_u=500\) y \(u_{\max}=300\). Para el modelo discretizado utilizado,
las matrices resultantes fueron:

\[
Q=
\begin{bmatrix}
	6.8188\times 10^{-3} & -6.9883\times 10^{-3} &  7.0544\times 10^{-3}\\
	-6.9883\times 10^{-3} &  7.1620\times 10^{-3} & -7.2298\times 10^{-3}\\
	7.0544\times 10^{-3} & -7.2298\times 10^{-3} &  7.2982\times 10^{-3}
\end{bmatrix}
\]

\[
R = 1.6667
\]

\paragraph{Ecuación de Riccati discreta y ganancia óptima}

La solución del problema se obtiene resolviendo la ecuación de Riccati discreta:

\[
P=A^TPA - A^TPB(R+B^TPB)^{-1}B^TPA + Q
\]

La ganancia óptima resulta:

\[
K=(R+B^TPB)^{-1}B^TPA
\]

Para el modelo discretizado se obtuvo:

\[
K=
\begin{bmatrix}
	0.5119095 & -0.4937470 & 0.4765571
\end{bmatrix}
\]

\paragraph{Polos del sistema en lazo cerrado}

La estabilidad se verifica mediante los polos del sistema:

\[
A_{\text{cl}} = A - BK
\]

Los polos obtenidos fueron:

\[
\lambda(A-BK)=
\begin{aligned}
	&0.9728458 + 0.0327300\,j\\
	&0.9728458 - 0.0327300\,j\\
	&0.9663930
\end{aligned}
\]

Se observa que todos los polos se encuentran dentro del círculo unitario,
garantizando estabilidad discreta.

\paragraph{Referencia al apéndice de implementación}

El código completo utilizado para:
\begin{itemize}
	\item cargar y discretizar la planta,
	\item definir \(Q\) y \(R\) mediante parámetros \textit{knobs} (\(w_y\), \(w_u\)),
	\item calcular \(K\) y verificar controlabilidad,
	\item diseñar observadores (predictor y actual) por ubicación de polos,
	\item calcular el prefiltro \(N_{\mathrm{bar}}\) para seguimiento,
	\item y simular el lazo con/sin saturación, con/sin ruido, y en doble precisión y \texttt{float32},
\end{itemize}
se incluye en el Apéndice~\ref{ap:lqr_obs_sim}.

% ============================================================
\subsubsection{Simulación del LQR con observador, saturación y ruido}
% ============================================================

Con el fin de aproximar el comportamiento del sistema real y evitar conclusiones
optimistas, se implementó un entorno de simulación que contempla:

\begin{itemize}
	\item \textbf{Observador de estados} en dos variantes:
	\begin{itemize}
		\item \textit{Predictor}: corrige usando \(y_k\).
		\item \textit{Actual}: corrige usando \(y_{k+1}\).
	\end{itemize}
	\item \textbf{Prefiltro de referencia} \(N_{\mathrm{bar}}\) para mejorar el seguimiento
	de referencia en lazo cerrado (estructura \(u_k = N_{\mathrm{bar}}r_k - K\hat{x}_k\)).
	\item \textbf{Saturación} del actuador con límite \(\pm u_{\max}\) para representar
	las restricciones del PWM/ESC.
	\item \textbf{Ruido} configurable para emular incertidumbres del experimento:
	\begin{itemize}
		\item Ruido de medición: \(y_{\mathrm{meas}} = y_{\mathrm{true}} + v_y\), con \(v_y\sim\mathcal{N}(0,\sigma_y^2)\).
		\item Ruido del actuador (jitter) y cuantización: \(u_{\mathrm{app}} = \mathrm{sat}\left(\mathrm{quant}(u_{\mathrm{cmd}}+v_u)\right)\).
		\item Ruido de proceso: \(x_{k+1} = Ax_k + Bu_k + w_k\) (configurable como prueba de estrés).
	\end{itemize}
	\item \textbf{Comparación numérica} entre simulación en doble precisión y \texttt{float32}
	para anticipar efectos de implementación embebida.
\end{itemize}

En particular, para evitar comparaciones sesgadas, las secuencias de ruido se
pre-generaron y se reutilizaron de forma idéntica tanto en doble precisión como en
\texttt{float32}.

\paragraph{Observadores y ubicación de polos}

Los observadores se diseñaron mediante ubicación arbitraria de polos en el plano-\(z\),
definiendo polos deseados:

\[
p_{\mathrm{obs}}=\{0.8\pm 0.25j,\ 0.9\}
\]

y obteniendo:

\[
K_{e,\mathrm{pred}}=\mathrm{place}(A^T,C^T,p_{\mathrm{obs}})^T
\qquad
K_{e,\mathrm{act}}=\mathrm{place}(A^T,(CA)^T,p_{\mathrm{obs}})^T
\]

\paragraph{Prefiltro \(N_{\mathrm{bar}}\)}

Para mejorar el seguimiento de referencia (SISO), se calculó un prefiltro
\(N_{\mathrm{bar}}\) tal que la ganancia estática equivalente sea adecuada, utilizando
la rutina auxiliar \texttt{refi}:

\[
u_k = N_{\mathrm{bar}}\,r_k - K\,\hat{x}_k
\]

\paragraph{Diagnósticos para evitar autoengaño}

Se incluyeron verificaciones directas sobre el ruido inyectado:

\[
e_y(k)=y_{\mathrm{meas}}(k)-y_{\mathrm{true}}(k)
\]

y se reportaron métricas como RMS\((e_y)\) y conteo de saturaciones en \(u\), para
confirmar que las perturbaciones y límites realmente están actuando sobre el lazo.

\paragraph{Resultados de simulación}

La figura~\ref{fig:lqr_step_sim} muestra la respuesta temporal simulada del lazo
cerrado con LQR, observador y saturación/ruido configurables. La figura~\ref{fig:lqr_u_sim}
presenta el esfuerzo de control asociado, evidenciando los instantes de saturación
y la actividad del regulador ante perturbaciones.

\insertarfigura{img/LQR/LQR_step.png}
{Respuesta temporal simulada del sistema con regulador LQR y observador (con saturación y ruido según configuración).}
{fig:lqr_step_sim}{1}

\insertarfigura{img/LQR/LQR_esfuerzo.png}
{Esfuerzo de control simulado $u(k)$ para el regulador LQR (incluyendo saturación y ruido según configuración).}
{fig:lqr_u_sim}{1}

% ============================================================
\subsubsection{Implementación experimental del LQR}
% ============================================================

En la implementación real se evaluaron ambas variantes del observador (predictor y actual)
manteniendo el mismo regulador LQR. Las figuras~\ref{fig:lqr_pred_prac} y~\ref{fig:lqr_act_prac}
muestran la respuesta experimental, evidenciando diferencias en el esfuerzo de control y el
comportamiento transitorio debido a no linealidades, saturación y variaciones del punto de operación.

\insertarfigura{img/LQR/LQR_pred_practico.png}
{Respuesta experimental en lazo cerrado con regulador LQR y observador \textit{predictor}.}
{fig:lqr_pred_prac}{1}

\insertarfigura{img/LQR/LQR_act_practico.png}
{Respuesta experimental en lazo cerrado con regulador LQR y observador \textit{actual}.}
{fig:lqr_act_prac}{1}

\subsubsection{Discusión}

\paragraph{Observador predictor}

En la simulación (figura~\ref{fig:lqr_step_sim}) la respuesta presenta
un tiempo de subida aproximado de $t_r^{\text{sim}} \approx 0.75\,\text{s}$
sin sobreimpulso apreciable. Este comportamiento es consistente con los
polos dominantes del sistema en lazo cerrado, ubicados próximos al
círculo unitario y con amortiguamiento elevado.

En la implementación experimental se observó una diferencia significativa
durante la primera subida (10\,cm a 30\,cm), donde el tiempo de subida
fue aproximadamente $15.17\,\text{s}$. Esta discrepancia se atribuye a
condiciones iniciales alejadas del punto de operación lineal, presencia
de fricción estática, retardo del conjunto ESC--motor y penalización del
esfuerzo impuesta por el funcional LQR, que favorece una respuesta de
menor energía antes que máxima rapidez.

En una segunda subida (30\,cm a 45\,cm), ya en un entorno más próximo al
equilibrio dinámico, el tiempo de subida fue de $0.55\,\text{s}$
(figura~\ref{fig:lqr_pred_prac}), mostrando una concordancia mucho mayor
con la predicción del modelo lineal. En ambas transiciones el
sobreimpulso fue prácticamente nulo.

Respecto al esfuerzo de control, se observa una evolución progresiva
hasta alcanzar el equilibrio, seguida de picos moderados asociados a la
corrección fina de la altura bajo saturación y no linealidades del
actuador.

\paragraph{Observador actual}

En simulación, el comportamiento temporal coincide con el del predictor
en términos de tiempo de subida y ausencia de sobreimpulso
(figura~\ref{fig:lqr_u_sim}), en concordancia con el principio de
separación.

Experimentalmente, para una transición de 75\,cm a 95\,cm, se obtuvo un
tiempo de subida aproximado de $0.69\,\text{s}$ con error estacionario
inferior a 1\,cm (figura~\ref{fig:lqr_act_prac}). Se observan picos de
esfuerzo más continuos que en el predictor, lo cual puede atribuirse a
una corrección más inmediata basada en la medición disponible y a la
interacción con ruido y saturación del actuador.

En alturas elevadas, el esfuerzo tiende a estabilizarse alrededor del
valor de equilibrio requerido para compensar el peso, reduciendo la
actividad transitoria.


\paragraph{Referencia al apéndice}

El script completo de simulación (incluyendo funciones locales \texttt{sim\_obs\_loop},
\texttt{plot\_zplane}, \texttt{refi} y el \textit{fallback} \texttt{dlqr\_iter\_nolic})
se incluye en el Apéndice~\ref{ap:lqr_obs_sim}.
