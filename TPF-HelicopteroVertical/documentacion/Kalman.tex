% ============================================================
\subsubsection{Modelo con Ruido}
% ============================================================

Para incorporar incertidumbre y modelar explícitamente la presencia de
perturbaciones no modeladas y ruido del sensor, se adopta la siguiente
representación estocástica discreta:

\[
x_{k+1} = A x_k + B u_k + w_k
\]

\[
y_k = C x_k + v_k
\]

donde:

\begin{itemize}
	\item $w_k \sim \mathcal{N}(0,Q)$ representa el ruido de proceso,
	asociado a dinámica no modelada, perturbaciones aerodinámicas
	y simplificaciones del modelo identificado.
	\item $v_k \sim \mathcal{N}(0,R)$ representa el ruido de medición,
	proveniente del sensor láser de distancia.
\end{itemize}

\paragraph{Estimación de $R$ (ruido de medición)}

La varianza del ruido de medición se obtuvo mediante ensayos empíricos,
midiendo la dispersión de la señal del sensor con la planta en reposo.
Siendo $\sigma_v$ la desviación estándar medida (en cm), se adopta:

\[
R = \sigma_v^2
\]

En los ensayos realizados se obtuvo:

\[
\sigma_v = 2.043\ \text{cm}
\qquad\Longrightarrow\qquad
R = 4.174\ \text{cm}^2
\]

\paragraph{Parametrización y sintonización de $Q$ (ruido de proceso)}

La matriz de covarianza del ruido de proceso se parametrizó como:

\[
Q = q\,I_n
\]

donde $q$ es un escalar positivo ajustable e $I_n$ es la matriz identidad
de dimensión $n$. El valor de $q$ se determinó mediante consistencia
estadística de la innovación normalizada (ver Apéndice~\ref{ap:kalman_tuning}).

El valor óptimo obtenido fue:

\[
q = 50.8022\times 10^{-3}
\qquad\Longrightarrow\qquad
Q = q\,I_n
\]

% ============================================================
\subsubsection{Filtro de Kalman en régimen permanente}
% ============================================================

Se utilizó un estimador de Kalman discreto en su variante \textit{current estimator}
(\texttt{kalman(...,'current')}). La ganancia en régimen permanente \(L\) se obtiene
a partir de la solución estacionaria \(P\) de la ecuación de Riccati discreta:

\[
P = A P A^T - A P C^T \left(C P C^T + R\right)^{-1} C P A^T + Q
\]

y la ganancia queda:

\[
L = A P C^T \left(C P C^T + R\right)^{-1}
\]

La ganancia obtenida en MATLAB para el modelo discretizado fue:

\[
L =
\begin{bmatrix}
	97.4968\\
	192.3014\\
	94.7820
\end{bmatrix}
\]

La estabilidad del observador se verifica con los polos del sistema estimador:

\[
\lambda(A - L C A) = \mathbf{973.0274e-003 + 30.8455e-003i,\\ 973.0274e-003 - 30.8455e-003i,
	\\969.4430e-003}
\]

% ============================================================
\subsubsection{Control integral y realimentación de estados (LQI)}
% ============================================================

Con el objetivo de eliminar error en régimen permanente ante referencias tipo escalón,
se incorporó un integrador de error. Definiendo el estado integral \(\xi_k\):

\[
\xi_{k+1} = \xi_k + \left(r_k - y_k\right)
\]

se construyó el sistema aumentado:

\[
\begin{bmatrix}
	x_{k+1}\\
	\xi_{k+1}
\end{bmatrix}
=
\begin{bmatrix}
	A & B\\
	-C & 1
\end{bmatrix}
\begin{bmatrix}
	x_k\\
	\xi_k
\end{bmatrix}
+
\begin{bmatrix}
	B\\
	0
\end{bmatrix}
u_k
+
\begin{bmatrix}
	0\\
	1
\end{bmatrix}
r_k
\]

y se diseñó una ley de control tipo LQI:

\[
u_k = -K_x\,\hat{x}_k + K_i\,\xi_k
\]

donde \(\hat{x}_k\) proviene del estimador de Kalman.

Los valores obtenidos mediante \texttt{dlqr} (con pesos heurísticos) fueron:

\[
K_x = \begin{bmatrix}
	16.5343   -15.8046    15.1271
\end{bmatrix}
\]

\[
K_i = \mathbf{3.2067}
\]

La dinámica del lazo cerrado puede analizarse mediante los polos del sistema
aumentado (planta + integrador + control), y los polos de la planta discretizada
se presentan como referencia:

\[
\lambda(A) = \mathbf{ 999.9972e-003 +  0.0000e+000i
	972.0409e-003 + 24.1158e-003i
	972.0409e-003 - 24.1158e-003i
}
\]

\[
\lambda(A_{\text{cl}}) = \mathbf{973.5939e-003 + 31.8308e-003i
	973.5939e-003 - 31.8308e-003i
	967.1966e-003 +  0.0000e+000i
	-671.1275e-018 +  0.0000e+000i}
\]

\paragraph{Nota práctica sobre saturación}

En la implementación práctica se aplica saturación al mando \(u_k\) para respetar
los límites del actuador (PWM). La presencia de saturación puede degradar el
cumplimiento exacto de la dinámica diseñada y producir integrador acumulado, por lo
que en firmware se complementa con estrategias de anti-windup cuando corresponde.

% ============================================================
\subsubsection{Resultados con ruido y validación}
% ============================================================

Para evaluar el desempeño se simuló el sistema incluyendo tanto ruido de medición
(como dispersión del sensor) como ruido de proceso (dinámica no modelada). Se comparó:

\begin{itemize}
	\item Respuesta de salida con y sin ruido, y seguimiento de referencia.
	\item Esfuerzo de control requerido en presencia de ruido.
	\item Estabilidad del sistema mediante mapas de polos y ceros.
\end{itemize}

Las siguientes figuras muestran los resultados obtenidos:

\insertarfigura{img/LQGi/LQGi_step_ruido.png}
{Respuesta temporal de la salida en presencia de ruido (medición y proceso), comparada con la referencia.}
{fig:lqgi_step_ruido}{1}

\insertarfigura{img/LQGi/LQGi_esfuerzo_ruido.png}
{Esfuerzo de control (señal de mando) en presencia de ruido.}
{fig:lqgi_esfuerzo_ruido}{1}

\insertarfigura{img/LQGi/LQGi_practico.png}
{Resultado práctico obtenido con la implementación sobre la planta real (ensayos experimentales).}
{fig:lqgi_practico}{1}

\insertarfigura{img/LQGi/LQGi_pzmap_NaranajaOBS_AzulPLANTA.png}
{Mapa de polos y ceros. En naranja se muestran los polos del observador  y en azul los polos asociados a la planta/sistema.}
{fig:lqgi_pzmap}{1}

% ============================================================
\subsubsection{Discusión}
% ============================================================

El uso combinado de realimentación de estados con integrador y estimación de estados
mediante Kalman (estructura LQG/LQI) presenta las siguientes ventajas:

\begin{itemize}
	\item El integrador permite eliminar el error estacionario frente a referencias constantes,
	compensando sesgos y pequeñas perturbaciones.
	\item El filtro de Kalman mejora la realimentación al reconstruir estados no medidos a partir
	de un modelo dinámico y mediciones ruidosas.
	\item El ajuste de \(Q\) y \(R\) controla el compromiso entre confianza en el modelo y confianza
	en la medición, lo cual resulta crítico ante ruido del sensor y variaciones no modeladas.
\end{itemize}

Limitaciones y consideraciones:

\begin{itemize}
	\item La elección \(Q = qI\) es una aproximación heurística; el sistema real puede presentar
	incertidumbres no homogéneas entre estados.
	\item El diseño LQI/LQG supone un modelo lineal y no contempla explícitamente saturaciones del actuador,
	por lo que el desempeño real puede diferir del esperado.
	\item El criterio \(\mathrm{var}(\eta)\approx 1\) es práctico para ajustar \(q\), pero no reemplaza un
	modelado físico detallado del ruido de proceso.
\end{itemize}

\paragraph{Referencia al apéndice de cómputo de ganancias}

El código completo utilizado para calcular \(L\), \(K_x\), \(K_i\), los polos de la planta,
los polos del observador y los polos del lazo cerrado se incluye en el
Apéndice~\ref{ap:lqg_ganancias}.
