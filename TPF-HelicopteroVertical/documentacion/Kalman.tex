
\subsubsection{Modelo con Ruido}

Para incorporar incertidumbre en el modelo,
se considera la siguiente representación estocástica:

\[
x_{k+1} = Ax_k + Bu_k + w_k
\]

\[
y_k = Cx_k + v_k
\]

donde:

\begin{itemize}
	\item $w_k \sim \mathcal{N}(0,Q)$ es el ruido de proceso.
	\item $v_k \sim \mathcal{N}(0,R)$ es el ruido de medición.
\end{itemize}

Las matrices de covarianza utilizadas fueron:

\[
Q = \textbf{[Completar]}
\]

\[
R = \textbf{[Completar]}
\]


% ============================================================
\subsubsection{Ecuaciones del Filtro de Kalman}
% ============================================================

El filtro consta de dos etapas:

\paragraph{Predicción}

\[
\hat{x}_{k|k-1} = A\hat{x}_{k-1} + Bu_{k-1}
\]

\[
P_{k|k-1} = A P_{k-1} A^T + Q
\]

\paragraph{Actualización}

\[
L_k = P_{k|k-1} C^T (C P_{k|k-1} C^T + R)^{-1}
\]

\[
\hat{x}_k = \hat{x}_{k|k-1} + L_k (y_k - C\hat{x}_{k|k-1})
\]

\[
P_k = (I - L_k C) P_{k|k-1}
\]


% ============================================================
\subsubsection{Ganancia en Régimen Permanente}
% ============================================================

En régimen estacionario, la ganancia converge a:

\[
L = A P C^T (C P C^T + R)^{-1}
\]

donde $P$ satisface la ecuación de Riccati discreta:

\[
P = A P A^T - A P C^T (C P C^T + R)^{-1} C P A^T + Q
\]

Ganancia obtenida:

\[
L = \textbf{[Completar]}
\]


% ============================================================
\subsubsection{Resultados de Estimación}
% ============================================================

Se evaluó la calidad de estimación comparando:

\begin{itemize}
	\item Estados reales vs estados estimados.
	\item Error cuadrático medio.
	\item Respuesta frente a ruido.
\end{itemize}

\insertarfigura{img/Estados/estimacion_kalman.png}
{Comparación entre estados reales y estimados mediante Filtro de Kalman.}
{fig:kalman_estimacion}{1}


% ============================================================
\subsubsection{Discusión}
% ============================================================

El filtro de Kalman presenta las siguientes ventajas:

\begin{itemize}
	\item Minimiza la varianza del error de estimación.
	\item Permite modelar explícitamente ruido de proceso y medición.
	\item Se integra naturalmente con LQR formando el esquema LQG.
\end{itemize}

Limitaciones:

\begin{itemize}
	\item Dependencia de una correcta elección de $Q$ y $R$.
	\item Sensibilidad a errores estructurales del modelo.
	\item Supone ruido blanco gaussiano.
\end{itemize}
