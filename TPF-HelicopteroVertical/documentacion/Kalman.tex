% ============================================================
\subsubsection{Modelo con Ruido}
% ============================================================

Para incorporar incertidumbre y modelar explícitamente la presencia de
perturbaciones no modeladas y ruido del sensor, se adopta la siguiente
representación estocástica discreta:

\[
x_{k+1} = A x_k + B u_k + w_k
\]

\[
y_k = C x_k + v_k
\]

donde:

\begin{itemize}
	\item $w_k \sim \mathcal{N}(0,Q)$ representa el ruido de proceso,
	asociado a dinámica no modelada, perturbaciones aerodinámicas
	y simplificaciones del modelo identificado.
	\item $v_k \sim \mathcal{N}(0,R)$ representa el ruido de medición,
	proveniente del sensor láser de distancia.
\end{itemize}

\paragraph{Estimación de $R$ (ruido de medición)}

La varianza del ruido de medición se obtuvo mediante ensayos empíricos,
midiendo la dispersión de la señal del sensor con la planta en reposo.
Siendo $\sigma_v$ la desviación estándar medida (en cm), se adopta:

\[
R = \sigma_v^2
\]

En los ensayos realizados se obtuvo:

\[
\sigma_v = 2.043\ \text{cm}
\qquad\Longrightarrow\qquad
R = 4.174\ \text{cm}^2
\]

\paragraph{Parametrización y sintonización de $Q$ (ruido de proceso)}

La matriz de covarianza del ruido de proceso se parametrizó como:

\[
Q = q\,I_n
\]

donde $q$ es un escalar positivo ajustable e $I_n$ es la matriz identidad
de dimensión $n$. El valor de $q$ se determinó mediante consistencia
estadística de la innovación normalizada (ver Apéndice~\ref{ap:kalman_tuning}).

El valor óptimo obtenido fue:

\[
q = 50.8022\times 10^{-3}
\qquad\Longrightarrow\qquad
Q = q\,I_n
\]

% ============================================================
\subsubsection{Filtro de Kalman en régimen permanente}
% ============================================================

Se utilizó un estimador de Kalman discreto en su variante \textit{current estimator}
(\texttt{kalman(...,'current')}). La ganancia en régimen permanente \(L\) se obtiene
a partir de la solución estacionaria \(P\) de la ecuación de Riccati discreta:

\[
P = A P A^T - A P C^T \left(C P C^T + R\right)^{-1} C P A^T + Q
\]

y la ganancia queda:

\[
L = A P C^T \left(C P C^T + R\right)^{-1}
\]

La ganancia obtenida en MATLAB para el modelo discretizado fue:

\[
L =
\begin{bmatrix}
	97.4968\\
	192.3014\\
	94.7820
\end{bmatrix}
\]

\paragraph{Polos del observador}

La estabilidad del observador se verifica a partir de los polos del sistema
estimador. Para el \textit{current estimator}, la dinámica del error queda
determinada por \(A - LCA\), por lo que se reportan los polos:

\[
\lambda(A - LCA) =
\begin{aligned}
	&0.9730274 + 0.0308455\,j \\
	&0.9730274 - 0.0308455\,j \\
	&0.9694430
\end{aligned}
\]

% ============================================================
\subsubsection{Control integral y realimentación de estados (LQI)}
% ============================================================

Con el objetivo de eliminar el error en régimen permanente ante referencias tipo escalón,
se incorporó un integrador de error. Definiendo el estado integral \(\xi_k\):

\[
\xi_{k+1} = \xi_k + \left(r_k - y_k\right)
\]

Se diseñó una ley de control tipo LQI:

\[
u_k = -K_x\,\hat{x}_k + K_i\,\xi_k
\]

donde \(\hat{x}_k\) proviene del estimador de Kalman.

Los valores obtenidos mediante \texttt{dlqr} (con pesos heurísticos) fueron:

\[
K_x =
\begin{bmatrix}
	16.5343 & -15.8046 & 15.1271
\end{bmatrix}
\]

\[
K_i = 3.2067
\]

\paragraph{Polos de la planta y del lazo cerrado}

Los polos de la planta discretizada (sin control) fueron:

\[
\lambda(A) =
\begin{aligned}
	&0.9999972 \\
	&0.9720409 + 0.0241158\,j \\
	&0.9720409 - 0.0241158\,j
\end{aligned}
\]

Los polos del lazo cerrado del sistema aumentado (planta + integrador + control) fueron:

\[
\lambda(A_{\text{cl}}) =
\begin{aligned}
	&0.9735939 + 0.0318308\,j \\
	&0.9735939 - 0.0318308\,j \\
	&0.9671966 \\
	&0
\end{aligned}
\]

% ============================================================
\subsubsection{Sistema aumentado usado para \texttt{pzmap}}
% ============================================================

Además del cálculo teórico de polos del lazo cerrado, en el script de validación se armó
explícitamente un \emph{sistema aumentado discreto} para analizar la estabilidad en el plano-\(z\)
con \texttt{pzmap}. Este sistema corresponde a la dinámica conjunta de planta e integrador,
con la ley de control LQI aplicada.

En el código se definieron las matrices del sistema aumentado como:

\[
A_{\text{aug}} =
\begin{bmatrix}
	A & B \\
	K_x - K_xA - K_iCA & 1 - K_xB - K_iCB
\end{bmatrix}
\]

\[
B_{\text{aug}} =
\begin{bmatrix}
	0\\
	0\\
	0\\
	K_i
\end{bmatrix}
\]

\[
C_{\text{aug}} =
\begin{bmatrix}
	C & 0
\end{bmatrix}
\qquad
D_{\text{aug}} = 0
\]

y se construyó el modelo:

\[
\texttt{sysDaug = ss(Aaug,Baug,Caug,0,Ts)}.
\]

Finalmente, se utilizó \texttt{pzmap(sysDaug)} para visualizar los polos del sistema aumentado
en el plano-\(z\) y verificar que la dinámica resultante permanezca estable (polos dentro del
círculo unidad) en el caso nominal.

\paragraph{Nota}

Esta construcción se incluye tal cual se implementa en el script, con el objetivo de
replicar el análisis práctico realizado y obtener directamente el mapa de polos asociado
a la estructura planta + integrador + control.

% ============================================================
\subsubsection{Nota práctica sobre saturación}
% ============================================================

En la implementación práctica se aplica saturación al mando \(u_k\) para respetar
los límites del actuador (PWM). La presencia de saturación puede degradar el
cumplimiento exacto de la dinámica diseñada y producir integrador acumulado, por lo
que en firmware se complementa con estrategias de anti-windup cuando corresponde.

% ============================================================
\subsubsection{Resultados con ruido y validación}
% ============================================================

Para evaluar el desempeño se simuló el sistema incluyendo tanto ruido de medición
(como dispersión del sensor) como ruido de proceso (dinámica no modelada). Se comparó:

\begin{itemize}
	\item Respuesta de salida con ruido, y seguimiento de referencia.
	\item Esfuerzo de control requerido en presencia de ruido.
	\item Estabilidad del sistema mediante mapa de polos y ceros.
\end{itemize}

Las siguientes figuras muestran los resultados obtenidos:

\insertarfigura{img/LQGi/LQGi_step_ruido.png}
{Respuesta temporal de la salida en presencia de ruido (medición y proceso) y comparación con la referencia.}
{fig:lqgi_step_ruido}{1}

\insertarfigura{img/LQGi/LQGi_esfuerzo_ruido.png}
{Esfuerzo de control (señal de mando) en presencia de ruido.}
{fig:lqgi_esfuerzo_ruido}{1}

\insertarfigura{img/LQGi/LQGi_practico.png}
{Resultado experimental obtenido con la implementación práctica sobre la planta real.}
{fig:lqgi_practico}{1}

\insertarfigura{img/LQGi/LQGi_pzmap_NaranajaOBS_AzulPLANTA.png}
{Mapa de polos y ceros. En naranja se muestran los polos del observador y en azul los polos asociados a la planta/sistema.}
{fig:lqgi_pzmap}{1}

% ============================================================

El código completo utilizado para calcular \(L\), \(K_x\), \(K_i\), los polos de la planta,
los polos del observador y los polos del lazo cerrado se incluye en el
Apéndice~\ref{ap:lqg_ganancias}.
