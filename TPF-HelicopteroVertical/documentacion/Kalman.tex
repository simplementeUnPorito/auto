% ============================================================
\subsubsection{Modelo con Ruido}
% ============================================================

Para incorporar incertidumbre y modelar explícitamente la presencia de
perturbaciones no modeladas y ruido del sensor, se adopta la siguiente
representación estocástica discreta:

\[
x_{k+1} = A x_k + B u_k + w_k
\]

\[
y_k = C x_k + v_k
\]

donde:

\begin{itemize}
	\item $w_k \sim \mathcal{N}(0,Q)$ representa el ruido de proceso,
	asociado a dinámica no modelada, perturbaciones aerodinámicas
	y simplificaciones del modelo identificado.
	\item $v_k \sim \mathcal{N}(0,R)$ representa el ruido de medición,
	proveniente del sensor láser de distancia.
\end{itemize}

% ============================================================
\paragraph{Estimación de $R$ (ruido de medición)}
% ============================================================

La varianza del ruido de medición se obtuvo mediante ensayos empíricos,
midiendo la dispersión de la señal del sensor con la planta en reposo.
Siendo $\sigma_v$ la desviación estándar medida (en cm), se adopta:

\[
R = \sigma_v^2
\]

En los ensayos realizados se obtuvo:

\[
\sigma_v = 2.043\ \text{cm}
\qquad\Longrightarrow\qquad
R = 4.174\ \text{cm}^2
\]

% ============================================================
\paragraph{Parametrización y sintonización de $Q$ (ruido de proceso)}
% ============================================================

La matriz de covarianza del ruido de proceso se parametrizó como:

\[
Q = q\,I_n
\]

donde $q$ es un escalar positivo ajustable e $I_n$ es la matriz identidad
de dimensión $n$. El valor de $q$ se determinó mediante consistencia
estadística de la innovación normalizada (ver Apéndice~\ref{ap:kalman_tuning}).

El valor óptimo obtenido fue:

\[
q = 50.8022\times 10^{-3}
\qquad\Longrightarrow\qquad
Q = q\,I_n
\]

% ============================================================
\subsubsection{Filtro de Kalman en régimen permanente}
% ============================================================

Se utilizó un estimador de Kalman discreto en su variante \textit{current estimator}
(\texttt{kalman(...,'current')}). La ganancia en régimen permanente \(L\) se obtiene
a partir de la solución estacionaria \(P\) de la ecuación de Riccati discreta:

\[
P = A P A^T - A P C^T \left(C P C^T + R\right)^{-1} C P A^T + Q
\]

y la ganancia queda:

\[
L = A P C^T \left(C P C^T + R\right)^{-1}
\]

La ganancia obtenida en MATLAB para el modelo discretizado fue:

\[
L =
\begin{bmatrix}
	97.4968\\
	192.3014\\
	94.7820
\end{bmatrix}
\]

% ============================================================
\paragraph{Polos del observador}
% ============================================================

Para el \textit{current estimator}, la dinámica del error queda
determinada por \(A - LCA\). Los polos obtenidos fueron:

\[
\lambda(A - LCA) =
\begin{aligned}
	&0.9730274 + 0.0308455\,j \\
	&0.9730274 - 0.0308455\,j \\
	&0.9694430
\end{aligned}
\]

Todos los polos se ubican dentro del círculo unitario,
garantizando estabilidad del estimador.

% ============================================================
\subsubsection{Control integral y realimentación de estados (LQGI)}
% ============================================================

Con el objetivo de eliminar el error en régimen permanente ante referencias tipo escalón,
se incorporó un integrador de error. Definiendo el estado integral \(\xi_k\):

\[
\xi_{k+1} = \xi_k + \left(r_k - y_k\right)
\]

Se diseñó una ley de control tipo LQI:

\[
u_k = -K_x\,\hat{x}_k + K_i\,\xi_k
\]

donde \(\hat{x}_k\) proviene del estimador de Kalman.

Los valores obtenidos mediante \texttt{dlqr} fueron:

\[
K_x =
\begin{bmatrix}
	16.5343 & -15.8046 & 15.1271
\end{bmatrix}
\]

\[
K_i = 3.2067
\]

% ============================================================
\paragraph{Polos de la planta y del lazo cerrado}
% ============================================================

Los polos de la planta discretizada (sin control) fueron:

\[
\lambda(A) =
\begin{aligned}
	&0.9999972 \\
	&0.9720409 + 0.0241158\,j \\
	&0.9720409 - 0.0241158\,j
\end{aligned}
\]

Los polos del lazo cerrado del sistema aumentado (planta + integrador + control) fueron:

\[
\lambda(A_{\text{cl}}) =
\begin{aligned}
	&0.9735939 + 0.0318308\,j \\
	&0.9735939 - 0.0318308\,j \\
	&0.9671966 \\
	&0
\end{aligned}
\]

Se verifica estabilidad discreta y presencia del polo en cero
asociado a la acción integral.

% ============================================================
\subsubsection{Sistema aumentado usado para \texttt{pzmap}}
% ============================================================

En el script de validación se construyó explícitamente el sistema aumentado:

\[
A_{\text{aug}} =
\begin{bmatrix}
	A & B \\
	K_x - K_xA - K_iCA & 1 - K_xB - K_iCB
\end{bmatrix}
\]

\[
B_{\text{aug}} =
\begin{bmatrix}
	0\\
	0\\
	0\\
	K_i
\end{bmatrix}
\]

\[
C_{\text{aug}} =
\begin{bmatrix}
	C & 0
\end{bmatrix}
\qquad
D_{\text{aug}} = 0
\]

Construyéndose el modelo:

\[
\texttt{sysDaug = ss(Aaug,Baug,Caug,0,Ts)}
\]

y visualizando los polos mediante \texttt{pzmap(sysDaug)}.


\insertarfigura{img/LQGi/LQGi_step_ruido.png}
{Respuesta temporal en presencia de ruido.}
{fig:lqgi_step_ruido}{1}

\insertarfigura{img/LQGi/LQGi_esfuerzo_ruido.png}
{Esfuerzo de control en presencia de ruido.}
{fig:lqgi_esfuerzo_ruido}{1}

\insertarfigura{img/LQGi/LQGi_practico.png}
{Resultado experimental sobre la planta real.}
{fig:lqgi_practico}{1}

\insertarfigura{img/LQGi/LQGi_pzmap_NaranajaOBS_AzulPLANTA.png}
{Mapa de polos: observador (naranja) y sistema (azul).}
{fig:lqgi_pzmap}{1}

% ============================================================
\subsubsection{Discusión}
% ============================================================

En régimen dinámico, el tiempo de subida experimental fue aproximadamente
\(t_r^{\text{exp}} \approx 0.65\,\text{s}\), con sobreimpulso reducido
(del orden del 10\% o menor en la mayoría de los ensayos).
La simulación con ruido predijo \(t_r^{\text{sim}} \approx 0.5\,\text{s}\)
y sobreimpulso cercano al 8\%, mostrando buena concordancia.

El esfuerzo de control en simulación presenta ciertos picos,
producto de la inyección explícita de ruido gaussiano. En la práctica,
los picos resultaron más consistentes y de menor amplitud relativa,
especialmente en comparación con implementaciones anteriores sin Kalman, cumpliendo de muy buena manera su objetivo.

Un aspecto destacable es que, gracias a la acción integral,
el error en régimen permanente converge sistemáticamente a cero,
a diferencia de prácticas previas donde persistían derivas
incluso utilizando prefiltros o compensaciones adicionales.

% ============================================================
\subsubsection{Conclusión}
% ============================================================

El esquema LQGI (LQR + Kalman + Integrador) constituye la síntesis más
completa implementada en este trabajo.

A diferencia de la simple ubicación arbitraria de polos, aquí los polos
del lazo cerrado resultan de una optimización basada en criterios
energéticos bien definidos por el ingeniero.

Aunque la selección de \(Q\) y \(R\) no es intuitiva y requiere
criterio y ajuste iterativo, una vez correctamente definidos,
el método produce resultados consistentes, robustos y coherentes
con el modelo.

La incorporación del filtro de Kalman permitió seleccionar las ganancias
del observador de forma óptima en presencia de ruido, reduciendo la
amplificación observada en enfoques anteriores.

Finalmente, la inclusión del integrador garantizó eliminación del error
estacionario y otorgó el mejor desempeño global entre todas las técnicas
implementadas, tanto en simulación como en la planta real.
