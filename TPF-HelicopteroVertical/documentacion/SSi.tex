% ============================================================
\subsubsection{Ubicación arbitraria de polos con Integrador}
% ============================================================

En esta sección se implementa el procedimiento propuesto en Ogata (Ec. 6.19)
para incorporar acción integral al sistema en espacio de estados, utilizando
únicamente un integrador externo y realimentación de estados estimados.

La planta discreta utilizada (orden $n=3$) se obtuvo mediante discretización
por ZOH con período de muestreo:

\[
T_s = 0.02\ \text{s}
\qquad (50\ \text{Hz})
\]

% ============================================================
\paragraph{Sistema aumentado con integrador}
% ============================================================

Para eliminar el error en régimen permanente frente a referencias constantes,
se define el estado integral:

\[
v_{k+1} = v_k + (r_k - y_k)
\]

donde $y_k = Cx_k$.

Siguiendo el desarrollo de Ogata, se construye el sistema aumentado:

\[
\hat{A} =
\begin{bmatrix}
	A & B \\
	0 & 0
\end{bmatrix},
\qquad
\hat{B} =
\begin{bmatrix}
	0 \\
	1
\end{bmatrix}
\]

y se ubican los polos deseados del sistema aumentado:

\[
p_{\text{ctrl}} =
\{0.95 \pm 0.15j,\; 0.98\},
\qquad
p_i = 0.96
\]

De esta forma se obtienen las ganancias equivalentes:

\[
K_2 =
\begin{bmatrix}
	0.7820721 & -0.5678709 & 0.3779423
\end{bmatrix}
\]

\[
K_1 = 0.1469358
\]

donde:

- $K_2$ actúa sobre el estado estimado $\hat{x}_k$.
- $K_1$ actúa sobre el estado integral $v_k$.

La ley de control implementada es:

\[
u_k = K_1 v_k - K_2 \hat{x}_k
\]

Los polos del sistema aumentado cerrado fueron correctamente ubicados
(según \texttt{place}) en:

\[
p_{K_{21}} = 13
\]

(lo que confirma la correcta asignación interna en MATLAB).

% ============================================================
\paragraph{Diseño del observador}
% ============================================================

Se diseñaron dos variantes de observador con los mismos polos deseados:

\[
p_{\text{obs}} =
\{0.8 \pm 0.25j,\; 0.9\}
\]

\paragraph{Observador predictor}

Dinámica del error:

\[
A - L_{\text{pred}} C
\]

Ganancia obtenida:

\[
L_{\text{pred}} =
\begin{bmatrix}
	167.3535 \\
	253.6822 \\
	86.0310
\end{bmatrix}
\]

\[
p_{L_{\text{pred}}} = 12
\]

\paragraph{Observador actual}

Dinámica del error:

\[
A - L_{\text{actual}} C A
\]

Ganancia obtenida:

\[
L_{\text{actual}} =
\begin{bmatrix}
	126.8411 \\
	172.0621 \\
	45.1301
\end{bmatrix}
\]

\[
p_{L_{\text{actual}}} = 13
\]

% ============================================================
\paragraph{Estructura completa del sistema aumentado cerrado}
% ============================================================

Para analizar explícitamente la dinámica del sistema completo
(planta + integrador + control), el script construye el modelo aumentado:

\[
A_{\text{aug}} =
\begin{bmatrix}
	A & B \\
	K_2 - K_2A - K_1 C A & 1 - K_2 B - K_1 C B
\end{bmatrix}
\]

\[
B_{\text{aug}} =
\begin{bmatrix}
	0 \\
	0 \\
	0 \\
	K_1
\end{bmatrix}
\]

\[
C_{\text{aug}} =
\begin{bmatrix}
	C & 0
\end{bmatrix}
\]

Este sistema representa la dinámica cerrada resultante de aplicar
la ley de control integral derivada mediante Ogata 6.19.

Además, para comparar la dinámica del error de estimación,
se construye el sistema:

\[
A_{\text{obs}} = A - L_{\text{actual}} C A
\]

y se superponen ambos mapas de polos mediante \texttt{pzmap}.

% ============================================================
\subsubsection{Resultados de simulación}
% ============================================================

Se evaluó el comportamiento del sistema para una referencia escalón
de 25 unidades, comparando:

\begin{itemize}
	\item Observador predictor
	\item Observador actual
	\item Esfuerzo de control
	\item Evolución del estado integral
\end{itemize}

Las figuras correspondientes se muestran a continuación:

\insertarfigura{img/SSi/SSi_step.png}
{Respuesta temporal $y(t)$ con integrador — comparación predictor vs actual.}
{fig:ssi_step}{1}

\insertarfigura{img/SSi/SSi_esfuerzo.png}
{Esfuerzo de control $u(t)$ con integrador.}
{fig:ssi_esfuerzo}{1}

\insertarfigura{img/SSi/SSi_pzmap.png}
{Mapa de polos: planta, sistema aumentado y dinámica del observador.}
{fig:ssi_pzmap}{1}

% ============================================================
\subsubsection{Discusión}
% ============================================================

El método de Ogata 6.19 permite incorporar acción integral sin modificar
la estructura original de la planta, utilizando un estado adicional y
realimentación adecuada.

Se observa que:

\begin{itemize}
	\item El integrador elimina el error en régimen permanente.
	\item La dinámica del observador es más rápida que la del sistema controlado.
	\item Ambos esquemas (predictor y actual) resultan estables,
	con polos dentro del círculo unidad.
	\item El sistema aumentado conserva estabilidad global.
\end{itemize}

El código completo utilizado para esta sección se incluye en el
Apéndice~\ref{ap:ogata_integrador_obs}.
