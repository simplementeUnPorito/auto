En esta sección se implementa el procedimiento propuesto en Ogata (Ec. 6.19) \cite{Ogata1996}
para incorporar acción integral al sistema en espacio de estados, utilizando
únicamente un integrador externo y realimentación de estados estimados.

La planta discreta utilizada (orden $n=3$) se obtuvo mediante discretización
por ZOH con período de muestreo:

\[
T_s = 0.02\ \text{s}
\qquad (50\ \text{Hz})
\]

% ============================================================
\paragraph{Sistema aumentado con integrador}
% ============================================================

Para eliminar el error en régimen permanente frente a referencias constantes,
se define el estado integral:

\[
v_{k+1} = v_k + (r_k - y_k)
\]

donde $y_k = Cx_k$.

Siguiendo el desarrollo de Ogata, se construye el sistema aumentado
(planta + integrador) en forma estructural:

\[
\hat{A} =
\begin{bmatrix}
	A & B \\
	0 & 0
\end{bmatrix},
\qquad
\hat{B} =
\begin{bmatrix}
	0 \\
	1
\end{bmatrix}
\]

Se seleccionaron los polos deseados del sistema aumentado como:

\[
p_{\text{ctrl}} =
\{0.95 \pm 0.15j,\; 0.98\},
\qquad
p_i = 0.96
\]

La determinación de las ganancias se realizó a partir de la relación:

\[
\text{Aux} =
\begin{bmatrix}
	A - I & B \\
	C A   & C B
\end{bmatrix}
\]

\[
K_{2,1} =
\frac{K_{\text{hat}} + [\,0\;\; 0\;\; 0\;\; 1\,]}{\text{Aux}}
\]

obteniéndose las ganancias equivalentes:

\[
K_2 =
\begin{bmatrix}
	0.7820721 & -0.5678709 & 0.3779423
\end{bmatrix}
\]

\[
K_1 = 0.1469358
\]

donde:

\begin{itemize}
	\item $K_2$ actúa sobre el estado estimado $\hat{x}_k$.
	\item $K_1$ actúa sobre el estado integral $v_k$.
\end{itemize}

La ley de control implementada es:

\[
u_k = K_1 v_k - K_2 \hat{x}_k
\]

% ============================================================
\paragraph{Diseño del observador}
% ============================================================

Dado que únicamente se mide la altura, el resto de estados debe estimarse.
Se diseñaron dos variantes de observador con los mismos polos deseados:

\[
p_{\text{obs}} =
\{0.8 \pm 0.25j,\; 0.9\}
\]

\paragraph{Observador predictor}

\[
\hat{x}_{k+1}
=
A\hat{x}_k + Bu_k + L_{\text{pred}}\big(y_k - C\hat{x}_k\big)
\]

Dinámica del error:

\[
e_{k+1} = (A - L_{\text{pred}} C)\,e_k
\]

Ganancia obtenida:

\[
L_{\text{pred}} =
\begin{bmatrix}
	167.3535 \\
	253.6822 \\
	86.0310
\end{bmatrix}
\]

\paragraph{Observador actual}

\[
z_{k+1} = A\hat{x}_k + Bu_k
\]

\[
\hat{x}_{k+1} =
z_{k+1}
+
L_{\text{act}}
\big(
y_{k+1} - C z_{k+1}
\big)
\]

Dinámica del error:

\[
e_{k+1} = (A - L_{\text{act}} C A)\,e_k
\]

Ganancia obtenida:

\[
L_{\text{act}} =
\begin{bmatrix}
	126.8411 \\
	172.0621 \\
	45.1301
\end{bmatrix}
\]

% ============================================================
\paragraph{Sistema aumentado cerrado}
% ============================================================

Para analizar explícitamente la dinámica cerrada (planta + integrador),
se construyó el sistema aumentado:

\[
A_{\text{aug}} =
\begin{bmatrix}
	A & B \\
	K_2 - K_2 A - K_1 C A & 1 - K_2 B - K_1 C B
\end{bmatrix}
\]

\[
C_{\text{aug}} =
\begin{bmatrix}
	C & 0
\end{bmatrix}
\]

lo que permite visualizar el mapa polo–cero del sistema completo.

% ============================================================
\subsubsection{Resultados de simulación}
% ============================================================

\insertarfigura{img/SSi/SSi_step.png}
{Respuesta temporal $y(t)$ con integrador — comparación predictor vs actual.}
{fig:ssi_step}{1}

\insertarfigura{img/SSi/SSi_esfuerzo.png}
{Esfuerzo de control $u(t)$ con integrador.}
{fig:ssi_esfuerzo}{1}

\insertarfigura{img/SSi/SSi_pzmap.png}
{Mapa de polos: planta, sistema aumentado y dinámica del observador.}
{fig:ssi_pzmap}{1}

En simulación se observa ausencia de sobreimpulso y un tiempo de subida
aproximado de:

\[
t_r^{\text{sim}} \approx 2.6\ \text{s}
\]

tanto para el observador predictor como para el actual.

% ============================================================
\subsubsection{Resultados experimentales}
% ============================================================

\insertarfigura{img/SSi/SSi_practico_pred.png}
{Resultado experimental con observador predictor.}
{fig:ssi_practico_pred}{1}

\insertarfigura{img/SSi/SSi_practico_act.png}
{Resultado experimental con observador actual.}
{fig:ssi_practico_act}{1}

Experimentalmente se obtuvo:

\[
t_r^{\text{exp}} = 2.02\ \text{s} \quad (\text{predictor})
\]

\[
t_r^{\text{exp}} = 2.36\ \text{s} \quad (\text{actual})
\]

Se observa buena concordancia cualitativa con la simulación,
aunque el sistema real presenta mayor rapidez que la predicha por
el modelo lineal.

A diferencia de implementaciones sin acción integral,
no se observaron derivas estacionarias sostenidas.
El sistema converge naturalmente a la referencia.

El predictor mostró picos de esfuerzo más continuos,
consistentes con mayor sensibilidad al ruido de medición.
El observador actual presentó un esfuerzo más progresivo
y menos errático, atribuido al filtrado implícito de la planta
entre $u_k$ y $y_{k+1}$.

% ============================================================
\subsubsection{Discusión}
% ============================================================

La incorporación de acción integral permitió eliminar el error en régimen
permanente de forma estructural, superando limitaciones observadas cuando
se utilizaba únicamente $N_{\text{bar}}$.

En particular:

\begin{itemize}
	\item El sistema alcanza la referencia incluso ante variaciones
	de batería o pequeñas perturbaciones constantes.
	\item La respuesta práctica se aproxima notablemente a la simulada.
	\item La estabilidad discreta se mantiene, con polos dentro del
	círculo unitario.
\end{itemize}

Persisten efectos tipo derivativos asociados al observador,
producto de una ubicación de polos subóptima que amplifica ruido.
Este aspecto constituye un punto claro de mejora futura.

% ============================================================
\subsubsection{Conclusión}
% ============================================================

El método de Ogata con acción integral demostró ser
estructuralmente más robusto frente a variaciones reales de la planta
que los enfoques sin integración.

Si bien depende del modelo identificado, la inclusión del estado
integral introduce una propiedad correctiva que compensa
desajustes moderados y elimina el error estacionario
sin necesidad de correcciones adicionales por software.


En conjunto, la realimentación de estados con acción integral
proporcionó el mejor compromiso entre estabilidad, eliminación
de error permanente y coherencia entre simulación y práctica
dentro de los métodos en espacio de estados evaluados.
