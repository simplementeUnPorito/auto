\section{Planteamiento del problema}

Los sistemas de control de altura basados en propulsión vertical presentan desafíos particulares desde el punto de vista del control automático, debido a su dinámica inherentemente inestable y a la presencia de múltiples no idealidades físicas. En este tipo de plantas, la variable de interés depende directamente del empuje generado por un actuador aerodinámico, el cual se encuentra limitado por saturaciones, retardos dinámicos y variaciones paramétricas asociadas a la alimentación eléctrica y a las condiciones de operación.

La planta experimental considerada en este trabajo consiste en un cuerpo móvil guiado mecánicamente en el eje vertical, cuya altura es determinada por el equilibrio entre la fuerza de empuje generada por un motor brushless y la fuerza gravitatoria. La dinámica del sistema se ve influenciada por fenómenos tales como el rozamiento en las guías, perturbaciones externas y eventos ocasionales de contacto mecánico, lo que dificulta la obtención de un comportamiento predecible y perfectamente reproducible.

Adicionalmente, la medición de la altura se realiza mediante un sensor de distancia láser, el cual introduce efectos de ruido y latencia que deben ser tenidos en cuenta en el diseño del sistema de control. Estas características convierten al sistema en una planta no ideal, donde los modelos teóricos simplificados resultan insuficientes para describir completamente su comportamiento real.

En este contexto, surge el problema de lograr una regulación estable y confiable de la altura del sistema, garantizando un desempeño aceptable frente a perturbaciones y variaciones de los parámetros físicos. La complejidad del sistema exige un análisis cuidadoso de su dinámica y una evaluación crítica de las distintas estrategias de control aplicables, considerando tanto su viabilidad práctica como su desempeño sobre una planta real.
