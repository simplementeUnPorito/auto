\section{Planteamiento del problema}

Los sistemas de regulación de altura basados en propulsión vertical
constituyen un caso particularmente exigente dentro del control automático,
debido a la interacción entre dinámica inestable, restricciones físicas
del actuador y presencia de no idealidades significativas.

La planta considerada consiste en un cuerpo móvil guiado en el eje vertical,
cuya posición depende del equilibrio entre la fuerza gravitatoria y el empuje
generado por un motor brushless accionado mediante un controlador electrónico
de velocidad (ESC). En lazo abierto, el sistema presenta comportamiento
inestable, dinámica de orden superior con integrador y características
no mínimas que condicionan la respuesta transitoria.

Adicionalmente, el sistema físico opera bajo restricciones reales,
entre ellas saturaciones estrictas del esfuerzo de control,
variaciones del punto de operación asociadas a la descarga de la batería,
fricción no uniforme en los rieles de guiado,
dinámica propia del conjunto motor–ESC
y ruido de medición proveniente del sensor de distancia.

Estas características introducen limitaciones estructurales
en el desempeño alcanzable y dificultan la obtención de un modelo
lineal simple que represente adecuadamente la dinámica dominante
alrededor del punto de operación.

En consecuencia, el problema central consiste en modelar y controlar
un sistema vertical inherentemente inestable, sujeto a restricciones
físicas y perturbaciones reales, garantizando estabilidad,
seguimiento de referencia y desempeño transitorio aceptable
en condiciones experimentales.
