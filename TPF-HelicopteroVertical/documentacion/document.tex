% !TeX program = pdflatex
\documentclass[conference]{IEEEtran}

% ======= Idioma y codificación =======
\usepackage[utf8]{inputenc}
\usepackage[T1]{fontenc}
\usepackage[spanish, es-nodecimaldot]{babel}
\usepackage{microtype}

% ======= Matemática =======
\usepackage{amsmath, amssymb, amsfonts}

% ======= Colores y gráficos =======
\usepackage{xcolor}
\usepackage{graphicx}
% \usepackage{subcaption}
\usepackage[caption=false,font=footnotesize]{subfig}
\usepackage{float}
\usepackage{booktabs}

% ======= Números y unidades =======
\usepackage{siunitx}
\sisetup{
output-decimal-marker = {,},
group-separator = {.},
group-minimum-digits = 4,
per-mode = symbol,
round-mode = places,
round-precision = 3,
table-number-alignment = center
}

% ======= Maquetación =======

% ======= Listas =======
\usepackage{enumitem}
\setlist{leftmargin=*, itemsep=2pt, topsep=4pt}

% ======= Bibliografía =======
\usepackage[numbers]{natbib}
\bibliographystyle{IEEEtranN}

% ======= Listados =======
\usepackage{listings}
\usepackage{listingsutf8}

% Config listings
\lstset{
inputencoding=utf8/latin1,
extendedchars=true,
upquote=true,
breaklines=true,
columns=fullflexible,
keepspaces=true,
frame=single,
numbers=left, numberstyle=\tiny, numbersep=6pt,
xleftmargin=1em,
captionpos=b,
literate=
{á}{{\'a}}1 {é}{{\'e}}1 {í}{{\'\i}}1 {ó}{{\'o}}1 {ú}{{\'u}}1
{Á}{{\'A}}1 {É}{{\'E}}1 {Í}{{\'I}}1 {Ó}{{\'O}}1 {Ú}{{\'U}}1
{ñ}{{\~n}}1 {Ñ}{{\~N}}1
{ü}{{\"u}}1 {Ü}{{\"U}}1
{¿}{{\textquestiondown}}1 {¡}{{\textexclamdown}}1
{º}{{$^\circ$}}1
{“}{{``}}1 {”}{{''}}1 {‘}{{`}}1 {’}{{'}}1
{—}{{---}}1 {–}{{--}}1
{→}{{\textrightarrow}}1
{≈}{{$\approx$}}1
{µ}{{\textmu}}1
}

% Lenguaje Matlab
\lstdefinelanguage{Matlab}{
morekeywords={break,case,catch,continue,else,elseif,end,for,function,global,if,otherwise,persistent,return,switch,try,while,classdef,properties,methods},
sensitive=true,
morecomment=[l]\%,
morecomment=[s]{\%\{} {\%},
morestring=[m]'
}

\lstdefinestyle{matlabstyle}{
language=Matlab,
basicstyle=\ttfamily\small,
keywordstyle=\bfseries\color{blue!70!black},
commentstyle=\itshape\color{green!40!black},
stringstyle=\color{red!60!black},
numbers=left, numberstyle=\tiny, numbersep=6pt,
frame=single,
breaklines=true,
columns=fullflexible,
keepspaces=true,
captionpos=b,
linewidth=\columnwidth,
xleftmargin=1em
}

% ======= Macros =======
\newcommand{\zetaD}{\zeta}
\newcommand{\wn}{\omega_n}
\newcommand{\Ts}{T_s}
\newcommand{\ESS}{\mathrm{ESS}}
\newcommand{\Gz}{G(z)}
\newcommand{\Gs}{G(s)}
\newcommand{\Nbar}{N_{\mathrm{bar}}}
\newcommand{\Ki}{K_i}
\newcommand{\Aaug}{A_{\mathrm{aug}}}
\newcommand{\Baug}{B_{\mathrm{aug}}}
\newcommand{\Caug}{C_{\mathrm{aug}}}

% ======= Macros figuras =======
\newcommand{\insertarfigura}[4]{%
\begin{figure}[!h]
\centering
\includegraphics[width=#4\linewidth]{#1}
\caption{#2}
\label{#3}
\end{figure}
}

% ======= Hyperref =======
\usepackage[unicode,hidelinks]{hyperref}

% ======= Metadatos LAB 9 =======
\title{Trabajo Práctico: Helicóptero vertical de un solo eje.}

\author{
\IEEEauthorblockN{Elías Álvarez}
\IEEEauthorblockA{Carrera de Ing. Electrónica\\Universidad Católica Nuestra Señora de la Asunción\\Asunción, Paraguay\\Email: elias.alvarez@universidadcatolica.edu.py}
\and
\IEEEauthorblockN{Tania Romero}
\IEEEauthorblockA{Carrera de Ing. Electrónica\\Universidad Católica Nuestra Señora de la Asunción\\Asunción, Paraguay\\Email: tania.romero@universidadcatolica.edu.py}
}

\begin{document}
	\maketitle
	
	\begin{abstract}
		En este trabajo práctico se estudia el control de altura de una planta experimental basada en un sistema de propulsión vertical mediante un motor brushless controlado electrónicamente. La planta consiste en un cuerpo móvil guiado mecánicamente en el eje vertical, cuya posición es medida mediante un sensor de distancia láser y regulada a través de una señal PWM aplicada a un controlador electrónico de velocidad (ESC). 
		
		A partir del modelado físico y matemático del sistema, se desarrollan e implementan distintas estrategias de control vistas a lo largo de la materia Automatizaciones. Se analizan controladores clásicos, tales como el PID y métodos de diseño basados en el lugar de las raíces, diagramas de Bode, síntesis directa y ubicación arbitraria de polos. Asimismo, se aborda el control en el espacio de estados, incorporando control integral para sistemas de seguimiento y el uso de estimadores de estado tanto de predicción como de actualización.
		
		Finalmente, se diseñan y evalúan controladores óptimos que incluyen integrador y estimadores basados en el filtro de Kalman, culminando en la implementación de un regulador LQG. Los distintos enfoques de control son comparados en términos de desempeño dinámico, estabilidad y robustez frente a perturbaciones y variaciones paramétricas, utilizando resultados obtenidos mediante ensayos experimentales sobre la planta real.
	\end{abstract}
	
	\begin{IEEEkeywords}
		control de altura, sistemas dinámicos, identificación de sistemas, control PID, control en espacio de estados, LQG, PSoC, TFMini, ESC, PWM
	\end{IEEEkeywords}
	
	
	 \section{Introducción}

En esta práctica se aborda el diseño de controladores mediante la técnica de \textbf{ubicación arbitraria de polos}, un método fundamental en el análisis y síntesis de sistemas de control en el espacio de estados. El propósito principal es modificar la dinámica de una planta determinada para que el sistema en lazo cerrado cumpla con \textit{especificaciones deseadas de respuesta transitoria}, tales como el tiempo de establecimiento y el coeficiente de amortiguamiento.

A través del cálculo de las matrices del sistema, su discretización e implementación en un \textbf{controlador digital basado en el PSoC}, se busca comprender cómo la realimentación de estados permite alterar el comportamiento dinámico y mejorar el desempeño del sistema. 

Finalmente, se comparan los resultados obtenidos mediante simulación en \texttt{MATLAB} con los resultados experimentales, analizando la efectividad del diseño y las posibles diferencias entre el modelo teórico y la práctica.

\section*{Objetivos}

\begin{itemize}
	\item Diseñar un controlador que modifique la dinámica de la planta para satisfacer condiciones específicas de la respuesta transitoria del sistema de control en lazo cerrado.
	\item Asegurar que el sistema regulado sea estable.
	\item Observar y analizar los efectos del controlador en el comportamiento dinámico del sistema.
	\item Considerar distintos métodos para el ajuste de los parámetros del controlador y analizar los resultados obtenidos.
	\item Diseñar el sistema de control en \texttt{MATLAB} e implementar la ecuación en diferencia correspondiente en el \texttt{PSoC}.
\end{itemize}

	
	\section{Planteamiento del problema}

Los sistemas de control de altura basados en propulsión vertical presentan desafíos particulares desde el punto de vista del control automático, debido a su dinámica inherentemente inestable y a la presencia de múltiples no idealidades físicas. En este tipo de plantas, la variable de interés depende directamente del empuje generado por un actuador aerodinámico, el cual se encuentra limitado por saturaciones, retardos dinámicos y variaciones paramétricas asociadas a la alimentación eléctrica y a las condiciones de operación.

La planta experimental considerada en este trabajo consiste en un cuerpo móvil guiado mecánicamente en el eje vertical, cuya altura es determinada por el equilibrio entre la fuerza de empuje generada por un motor brushless y la fuerza gravitatoria. La dinámica del sistema se ve influenciada por fenómenos tales como el rozamiento en las guías, perturbaciones externas y eventos ocasionales de contacto mecánico, lo que dificulta la obtención de un comportamiento predecible y perfectamente reproducible.

Adicionalmente, la medición de la altura se realiza mediante un sensor de distancia láser, el cual introduce efectos de ruido y latencia que deben ser tenidos en cuenta en el diseño del sistema de control. Estas características convierten al sistema en una planta no ideal, donde los modelos teóricos simplificados resultan insuficientes para describir completamente su comportamiento real.

En este contexto, surge el problema de lograr una regulación estable y confiable de la altura del sistema, garantizando un desempeño aceptable frente a perturbaciones y variaciones de los parámetros físicos. La complejidad del sistema exige un análisis cuidadoso de su dinámica y una evaluación crítica de las distintas estrategias de control aplicables, considerando tanto su viabilidad práctica como su desempeño sobre una planta real.

	
	\section{Objetivos}

\subsection{Objetivo general}

Desarrollar, analizar e implementar un sistema de control de altura para una planta experimental basada en un motor brushless, aplicando y comparando distintas estrategias de control automático estudiadas en la materia Automatizaciones, a partir del modelado físico y matemático del sistema y su validación experimental sobre la planta real.

\subsection{Objetivos específicos}

\begin{itemize}
	\item Caracterizar físicamente la planta experimental y describir su comportamiento dinámico a partir de sus componentes mecánicos, eléctricos y de sensado.
	
	\item Desarrollar un modelo físico y matemático del sistema que represente adecuadamente la dinámica vertical de la planta y sirva como base para el diseño de controladores.
	
	\item Identificar experimentalmente los parámetros relevantes del modelo, considerando las no idealidades propias del sistema real.
	
	\item Diseñar e implementar controladores clásicos de altura, incluyendo el controlador PID y métodos de diseño basados en el lugar de las raíces, diagramas de Bode, síntesis directa y ubicación arbitraria de polos.
	
	\item Diseñar controladores en el espacio de estados para sistemas de seguimiento, incorporando control integral para la eliminación del error estacionario.
	
	\item Implementar y evaluar estimadores de estado tanto de tipo predictivo como de actualización, analizando su influencia en el desempeño del sistema.
	
	\item Diseñar y aplicar estrategias de control óptimo, incluyendo el uso de integrador y estimadores basados en el filtro de Kalman, culminando en la implementación de un regulador LQG.
	
	\item Comparar el desempeño de las distintas estrategias de control implementadas en términos de estabilidad, respuesta transitoria, error en régimen permanente y robustez frente a perturbaciones y variaciones paramétricas.
	
	\item Validar experimentalmente los resultados obtenidos mediante ensayos sobre la planta real, contrastando el comportamiento observado con las predicciones del modelo.
\end{itemize}

	

	\section{Caracterización física de la planta}
	\subsection{Descripción general de la planta}
La planta desarrollada corresponde a un sistema mecatrónico de \textbf{un grado de libertad}, cuyo movimiento está restringido a la \textbf{dirección vertical}. El principio de funcionamiento se basa en la generación de empuje aerodinámico mediante un \textbf{motor brushless con hélice}, controlado electrónicamente, que permite regular la altura de un cuerpo móvil guiado mecánicamente.

El sistema fue concebido como una planta experimental para el análisis y diseño de estrategias de control en altura, incorporando sensado directo de posición y actuadores eléctricos de rápida respuesta.

\subsection{Estructura física de la planta}

\insertarfigura{img/Planta/planta.jpeg}{Vista general de la planta física.}{fig:plantaFisica}{0.5}

La planta experimental fue diseñada y construida específicamente para el desarrollo de las prácticas de control de altura previstas en el presente trabajo. El diseño de la estructura respondió a la necesidad de disponer de una altura útil suficiente para la correcta evaluación de las distintas estrategias de control, manteniendo al mismo tiempo un esquema constructivo simple, robusto y de fácil implementación. Durante el desarrollo del proyecto se atravesaron distintas etapas de diseño, las cuales se describen con mayor detalle en el Apéndice~D.

Inicialmente, la planta fue concebida con una altura total de aproximadamente \(80\,\text{cm}\). Sin embargo, dicha dimensión resultó insuficiente para la realización de todas las prácticas propuestas. En consecuencia, y por recomendación del profesor, se decidió extender la estructura hasta alcanzar una altura total de \(165\,\text{cm}\), permitiendo una altura útil de movimiento del cuerpo móvil de aproximadamente \(134\,\text{cm}\).

La estructura se apoya sobre una base de madera de dimensiones \(50 \times 45\,\text{cm}\) y un espesor aproximado de \(2\,\text{cm}\), la cual proporciona estabilidad al conjunto. En la parte superior se dispone de un techo de madera de \(50 \times 50\,\text{cm}\) y un espesor de \(0{,}6\,\text{cm}\), que actúa como elemento de cierre y soporte estructural. La base y el techo se encuentran unidos mediante tres columnas verticales de madera, cuya función principal es aportar rigidez al conjunto y limitar las deformaciones de la estructura.

El \textbf{movimiento vertical} del sistema se guía mediante tres rieles metálicos dispuestos en paralelo, construidos a partir de vigas de metal de aproximadamente \(0{,}6\,\text{cm}\) de diámetro. Estas vigas presentan deformaciones inherentes al material y a su longitud, por lo que las columnas de madera cumplen un rol fundamental en evitar que dichas deformaciones se acentúen durante el funcionamiento del sistema.

El \textbf{cuerpo móvil} se desplaza a lo largo de los rieles mediante piezas impresas en 3D de tipo abrazadera. Estas abrazaderas poseen un diámetro aproximado de \(1{,}5\,\text{cm}\), superior al de las vigas, con el objetivo de permitir cierto grado de libertad angular y evitar perturbaciones en el movimiento vertical causadas por las deformaciones de los rieles. Las abrazaderas se unen al cuerpo móvil mediante un sistema de tornillos, arandelas y tuercas, funcionando como articulaciones tipo ``muñeca'', que facilitan el guiado sin generar atascamientos.

\insertarfigura{img/Disenos/abrazadera.png}{Diseño 3D final de la abrazadera con agarre tipo ``muñeca''.}{fig:abrazadera}{0.5}

\insertarfigura{img/Disenos/brazo.png}{Diseño 3D final del brazo de unión del cuerpo móvil.}{fig:brazo}{0.5}

El cuerpo móvil está compuesto por dos \textbf{soportes principales}: un soporte superior que constituye la base de montaje del motor brushless y un soporte inferior que sostiene el conjunto estructural. Ambos soportes se encuentran unidos mediante tres brazos impresos en 3D, de aproximadamente \(1\,\text{cm}\) de altura, \(1{,}3\,\text{cm}\) de ancho y \(23\,\text{cm}\) de longitud. La unión de los distintos componentes del cuerpo móvil se realiza mediante roscas de aproximadamente \(3\,\text{cm}\) de diámetro y tuercas, evitando el uso de tornillería adicional. El conjunto incluye además un soporte para la batería, en cuya base se encuentra montado el ESC, equipado con dos disipadores térmicos laterales. La masa total del cuerpo móvil es de aproximadamente \(360\,\text{g}\).

\insertarfigura{img/Disenos/soporte-cuerpo.png}{Diseño 3D final del soportes principales del cuerpo móvil (soporte superior e inferior).}{fig:soporte-cuerpo}{0.5}

Con el fin de proteger el sistema ante caídas y movimientos bruscos, se incorporaron \textbf{elementos de seguridad adicionales}. En la parte inferior de la estructura se dispuso papel burbuja enrollado alrededor de los rieles, actuando como amortiguación ante impactos. Asimismo, se añadieron tubos de PVC tanto en la base como en la parte superior de la estructura, los cuales funcionan como topes mecánicos que limitan el recorrido del cuerpo móvil y evitan colisiones con la base o el techo. Adicionalmente, se incorporó una cuerda de seguridad destinada a restringir levantamientos excesivos durante las prácticas, reduciendo el riesgo de movimientos abruptos.

Finalmente, se añadió una cinta métrica a lo largo de la estructura con el objetivo de \textbf{facilitar la visualización} del desplazamiento vertical y permitir una referencia directa de la altura durante el funcionamiento del sistema.

El diseño de la estructura \textbf{priorizó} la rigidez del cuerpo móvil, de modo que pueda soportar tanto su propio peso como eventuales caídas desde alturas cercanas a un metro. Asimismo, se buscó una solución de fácil construcción, utilizando materiales de adquisición accesible en el contexto local y adaptándose a los recursos disponibles y al tiempo de desarrollo del trabajo. La visibilidad del movimiento y el correcto funcionamiento mecánico del sistema fueron considerados aspectos clave para su utilización como planta experimental en este trabajo práctico.



\subsection{Cuerpo móvil}
\begin{itemize}
	\item \textbf{Masa total móvil:}
	\[
	m = 0.360\ \text{kg}
	\]
	(incluye motor, hélice, soporte, cableado y elementos solidarios al movimiento).
	\item \textbf{Tipo de movimiento:} traslación puramente vertical.
	\item \textbf{Altura inicial típica:} aproximadamente $12.5\ \text{cm}$.
	\item \textbf{Altura máxima disponible en la estructura:} aproximadamente $134\ \text{cm}$.
\end{itemize}

La ausencia de contrapesos implica que el sistema depende exclusivamente del empuje generado por la hélice para vencer la fuerza gravitatoria y los efectos de rozamiento.

\subsection{Sistema de actuación (propulsión)}
El sistema de actuación está compuesto por:
\begin{itemize}
	\item \textbf{Motor:} brushless A2212/5T, $2450\ \text{KV}$ \cite{es_motor_a2212}.
	\insertarfigura{img/Planta/motor-a2212-2450kv.jpg}{Motor}{fig:motor2212}{0.5}
	\item \textbf{Hélice:}
	\begin{itemize}
		\item Diámetro: $25\ \text{cm}$.
		%\item El \textit{pitch} no se encuentra especificado.
	\end{itemize}
	\insertarfigura{img/Planta/heliceMedida.jpeg}{Hélice}{fig:HeliceM}{1}
	\item \textbf{Controlador electrónico (ESC) \cite{ESC_40A}.:}
	\begin{itemize}
		\item Corriente continua: $40\ \text{A}$.
		\item Corriente máxima de corta duración: $55\ \text{A}$.
	\end{itemize}
	\insertarfigura{img/Planta/ESC-40A.jpg}{ESC 40A}{fig:ESC40A}{0.5}
	\item \textbf{Batería:} LiPo 3S, \cite{LiPo3S}.
	\begin{itemize}
		\item Tensión inicial típica: $12.5\ \text{V}$.
		\item La tensión disminuye de forma apreciable durante la operación, dependiendo del régimen de empuje y la duración de la práctica.
	\end{itemize}
	\insertarfigura{img/Planta/bateria2200.jpg}{Bateria LiPo 3S Ovonic}{fig:bateria3s}{0.5}
\end{itemize}

El empuje generado por el sistema depende fuertemente del comando aplicado, de la hélice y del voltaje instantáneo de la batería, lo que introduce una \textbf{no linealidad significativa} en la planta.

\subsection{Señal de control}
\begin{itemize}
	\item \textbf{Tipo de señal:} PWM tipo servo.
	\item \textbf{Frecuencia:} $50\ \text{Hz}$.
	\item \textbf{Rango:} $1000\ \mu s$ -- $2000\ \mu s$.
\end{itemize}

Esta señal actúa como la \textbf{entrada manipulada} del sistema, regulando indirectamente el empuje generado por el motor y la hélice a través del ESC.

\insertarfigura{img/Planta/motor2212.png}{Características del motor A2212/5T, 2450KV \cite{es_motor_a2212}}{fig:motor2212}{1}

\subsection{Sistema de sensado}

\begin{itemize}
	\item \textbf{Sensor de altura:} TFMini Plus \cite{TFminiPlusDatasheet}.
	\item \textbf{Variable medida:} posición vertical del cuerpo móvil \(z(t)\).
	\item \textbf{Frecuencia de lectura:} configurable, típicamente en el rango de \(1\,\text{Hz}\) hasta \(1000\,\text{Hz}\), según la configuración utilizada durante las distintas prácticas.
\end{itemize}

La señal de medición presenta efectos de \textbf{ruido}, \textbf{cuantización} y \textbf{latencia}, propios del sistema de sensado y del procesamiento digital, los cuales deben ser considerados tanto en el diseño del sistema de control como en el tratamiento de la señal medida.

\insertarfigura{img/Planta/TFminiPlus.png}{Sensor óptico de distancia TFMini Plus.}{fig:sensor}{1}


\subsection{Variables del sistema}
\begin{itemize}
	\item \textbf{Entrada del sistema:}
	\[
	u(t) = \text{PWM} \in [1000,2000]\ \mu s
	\]
	\item \textbf{Salida del sistema:}
	\[
	y(t) = z(t)
	\]
	\item \textbf{Disturbios principales:}
	\begin{itemize}
		\item Variaciones del voltaje de la batería durante la operación.
		\item Rozamiento mecánico en las guías.
		\item Perturbaciones aerodinámicas externas.
		\item Vibraciones estructurales.
	\end{itemize}
\end{itemize}

\subsection{Limitaciones físicas y no idealidades}
La planta presenta las siguientes características no ideales:
\begin{itemize}
	\item \textbf{Saturación del actuador}, limitada por el rango de PWM y la corriente máxima del ESC.
	\item \textbf{Dinámica no instantánea del empuje}, asociada a la respuesta del ESC, del motor y de la hélice.
	\item \textbf{Variabilidad paramétrica}, principalmente debida a la caída de tensión de la batería bajo carga.
\end{itemize}

	
	\section{Diagrama del Sistema}
La \textbf{Figura~\ref{fig:DiagramaControl}} muestra el diagrama de bloques del sistema de control de altura en lazo cerrado. La referencia de altura \(z_{\text{ref}}\) es comparada con la señal de salida medida \(z(t)\), generando el error \(e(t)\), el cual es procesado por el controlador.

El controlador representa el algoritmo de control implementado, pudiendo corresponder a un controlador estudiado en el semestre. A partir del error, el controlador genera la señal de control \(u(t)\), que constituye el esfuerzo aplicado al sistema.

La señal \(u(t)\) actúa sobre el actuador, conformado por el motor brushless y la hélice, el cual convierte la señal de control en empuje mecánico. Dicho empuje excita la planta física, cuya dinámica vertical determina la altura real del sistema \(y(t)\).

La altura es medida mediante un sensor TFMini Plus, que proporciona la señal \(z(t)\) utilizada para cerrar el lazo de control. En el diagrama se indican explícitamente las fuentes de ruido asociadas tanto al controlador como al sensor, reflejando las perturbaciones y no idealidades presentes en el sistema real.

\insertarfigura{img/Planta/Diagrama.jpg}{Diagrama de bloques de Control.}{fig:DiagramaControl}{1}

La \textbf{Figura~\ref{fig:Diagrama2}} muestra un diagrama de bloques que representa la implementación física y funcional completa del sistema de control de altura. En dicho diagrama se identifican claramente los siguientes subsistemas:
\begin{itemize}
	\item la cadena de energía (\texttt{batería} $\rightarrow$ \texttt{ESC} $\rightarrow$ \texttt{motor}),
	\item la cadena de control (\texttt{MATLAB} $\rightarrow$ \texttt{PSoC} $\rightarrow$ \texttt{PWM}),
	\item la planta física correspondiente al sistema bajo control,
	\item y la cadena de medición y realimentación (\texttt{planta} $\rightarrow$ \texttt{sensor} $\rightarrow$ \texttt{PSoC} $\rightarrow$ \texttt{MATLAB}).
\end{itemize}



\insertarfigura{img/Planta/Diagrama-Page-2.jpg}{Diagrama de bloques, implementación física.}{fig:Diagrama2}{1}

\subsection{Flujo de energía (parte electro--energética)}

La batería utilizada es una LiPo de tres celdas (3S), con tensión nominal de \(11{,}1\,\text{V}\). 
Cada celda presenta una tensión máxima de \(4{,}2\,\text{V}\), por lo que la batería completamente cargada alcanza:

\[
V_{\text{max}} = 12{,}6\,\text{V}
\]

Con el objetivo de preservar la integridad química de la batería y evitar degradación prematura, el rango operativo se restringe aproximadamente a:

\[
V \in [11{,}5,\; 12{,}5]\,\text{V}
\]

Evitar descargas por debajo de \(11{,}5\,\text{V}\) resulta fundamental, ya que tensiones inferiores pueden producir daño irreversible en las celdas LiPo.

La batería alimenta directamente al controlador electrónico de velocidad (ESC) de \(40\,\text{A}\), el cual convierte la tensión continua en señales trifásicas moduladas para accionar el motor brushless. 

El ESC incorpora disipadores térmicos integrados sobre los dispositivos de potencia (MOSFETs), cuya función es evacuar el calor generado durante la conmutación y conducción de corriente. Dado que el sistema opera en un régimen de potencia relativamente elevado para el tamaño del conjunto, la gestión térmica resulta crítica para evitar sobrecalentamientos que puedan producir fallas o reducción de eficiencia.

La presencia de disipadores mejora la transferencia térmica hacia el ambiente, aumentando la confiabilidad del sistema durante los intervalos de operación de aproximadamente \(5\,\text{min}\) continuos.

A medida que la batería se descarga, se observa una disminución progresiva de la tensión disponible, lo que impacta directamente en la capacidad de generación de empuje del motor. Este fenómeno es perceptible incluso en aplicaciones aeronáuticas (como drones), donde la pérdida de tensión se traduce en menor capacidad de sustentación.

En la práctica experimental, el tiempo de operación continua es del orden de \(5\,\text{min}\) como máximo. Esto se debe a que la batería utilizada posee una capacidad limitada para el nivel de potencia demandado por el sistema. 

El tiempo típico de recarga completa es aproximadamente \(1{,}5\,\text{h}\), lo que introduce una restricción significativa en la repetibilidad de los ensayos experimentales, constituyendo un factor limitante en el desarrollo de la práctica.

En esta etapa del sistema no se realiza acción de control propiamente dicha, sino únicamente conversión y transferencia de potencia hacia el actuador.



\subsection{Flujo de control (parte de mando)}

El PSoC actúa como el controlador embebido del sistema. A partir de los algoritmos de control implementados (PID, control en espacio de estados, LQG, entre otros), el PSoC genera una señal PWM de tipo servo que es enviada al ESC.

Esta señal PWM representa la variable de control \(u(t)\) del sistema y determina el nivel de empuje aplicado al motor. Para asegurar una correcta referencia eléctrica y el funcionamiento adecuado del sistema, el PSoC y el ESC comparten una conexión de tierra común (GND).

El sistema MATLAB se comunica con el PSoC mediante una interfaz de datos seriales, lo que permite:
\begin{itemize}
	\item enviar referencias de altura,
	\item modificar parámetros de control,
	\item recibir y visualizar datos del sistema en tiempo real.
\end{itemize}

En este esquema, MATLAB cumple una función de supervisión, ajuste y análisis experimental, mientras que el PSoC ejecuta el control en tiempo real.

\subsection{Medición y realimentación (cierre del lazo)}

La planta física, al desplazarse verticalmente, genera una altura real que es medida mediante el sensor de distancia láser TFMini Plus. Este sensor entrega la medición de altura al PSoC a través de su interfaz de recepción (RX).

La señal medida es utilizada por el PSoC para calcular el error entre la referencia y la salida real del sistema, cerrando de esta manera el lazo de control de altura. Adicionalmente, los datos medidos pueden ser enviados a MATLAB para su visualización, almacenamiento y análisis experimental.



		
	\section{Modelado físico del sistema}
	\subsection{Variables y convenciones}

Se definen a continuación las variables y convenciones utilizadas para el modelado del sistema:

\begin{itemize}
	\item Eje vertical \(z\) [m], definido positivo hacia arriba.
	
	\item Masa móvil:
	\[
	m = 0.360\ \text{kg}
	\]
	
	\item Aceleración de la gravedad:
	\[
	g = 9.81\ \text{m/s}^2
	\]
	
	\item Peso del cuerpo móvil:
	\[
	mg = 3.924\ \text{N}
	\]
	
	\item Entrada del sistema: señal PWM tipo servo a \(50\ \text{Hz}\),
	\[
	u \in [1000,2000]\ \mu s
	\]
	
	Si bien el rango eléctrico nominal del protocolo PWM se encuentra entre
	$1000$ y $2000\ \mu s$, durante la operación en vuelo la señal se restringe
	intencionalmente al intervalo
	
	\[
	u \in [1100,1700]\ \mu s
	\]
	
	Esta limitación surge de criterios de confiabilidad experimental.
	En ensayos previos realizados con ESCs de menor capacidad nominal
	($30\ \text{A}$), se observaron fallas térmicas al operar en valores
	superiores a aproximadamente $1600\ \mu s$ bajo carga sostenida.
	
	Dado que no se realizaron mediciones directas de corriente ni
	caracterizaciones térmicas detalladas del conjunto motor--ESC--hélice,
	se adoptó un margen de seguridad conservador que evita la operación
	prolongada en regímenes de alta demanda energética.
	
	Asimismo, el límite inferior de $1100\ \mu s$ se fija con el objetivo
	de evitar regiones cercanas a la detención del motor, donde pueden
	presentarse comportamientos fuertemente no lineales y pérdida abrupta
	de sustentación.
	
	En consecuencia, el modelo identificado y las estrategias de control
	desarrolladas se consideran válidos únicamente dentro de este rango
	operativo seguro.
	
	\item Normalización de la entrada:
	\[
	\hat u = \mathrm{sat}\!\left(u-u_0,\,1100-u_0,\,1700-u_0\right)
	\]
	
	Siendo $u_0$ el valor de PWM necesario para generar un empuje
	equivalente al peso del sistema, es decir, la condición de equilibrio
	vertical estacionario.
	
	\item Salida medida:
	\[
	y = z
	\]
	
	correspondiente a la altura del cuerpo móvil medida mediante el sensor
	TFMini Plus, cuya resolución efectiva se encuentra en el orden de
	centímetros.
\end{itemize}

\subsubsection{Modelo físico simplificado}

Desde un punto de vista físico, el movimiento vertical del cuerpo móvil puede describirse, en primera aproximación, mediante:

\[
\dot z = v
\]

\[
m\dot v = T - mg
\]

Este modelo corresponde al caso ideal sin disipación, en el cual la
dinámica posición--empuje presenta una estructura de doble integración.

En una aproximación más realista, puede incorporarse el efecto de
rozamiento de las guías mediante un término viscoso proporcional a la
velocidad:

\[
m\dot v = T - mg - b\,v
\]

donde \(b\) [N\,s/m] representa un coeficiente equivalente de fricción
viscosa.

La inclusión de este término modifica la estructura ideal de doble
integrador, convirtiéndola en un sistema con integración amortiguada.
Desde el punto de vista de función de transferencia continua, el
doble polo en el origen deja de ser estrictamente doble, introduciéndose
un amortiguamiento mecánico que desplaza uno de los polos hacia el
semiplano izquierdo.

Por lo tanto, la presencia de un único integrador dominante en el modelo
simplificado resulta coherente con la física del sistema cuando se
consideran pérdidas mecánicas.

\subsubsection{Modelo dinámico equivalente del actuador (BLDC)}

A efectos de análisis lineal, el conjunto ESC--motor BLDC puede
aproximarse mediante un modelo promedio equivalente al de un motor DC
de imanes permanentes:

\[
v(t)=R\,i(t) + L\,\dot i(t) + K_e\,\omega(t)
\]

\[
J\,\dot \omega(t)=K_t\,i(t) - B\,\omega(t) - \tau_L(t)
\]

Este modelo introduce una dinámica electromecánica adicional respecto
del modelo puramente mecánico de la masa móvil, justificando la
aparición de un polo adicional en la relación entrada--salida.

En muchos casos, la constante de tiempo eléctrica
\(\tau_e=L/R\) es considerablemente menor que la mecánica,
permitiendo despreciar \(L\) y obtener una dinámica dominante de
segundo orden asociada al actuador.

\subsubsection{Función de transferencia continua equivalente}

La representación continua obtenida a partir del modelo identificado
puede expresarse inicialmente como:

\[
G(s)=
\frac{-0.12107\,(s-14)(s+10.62)}
{(s+0.0002797)\,(s^2+5.61s+14.02)}
\]

El polo ubicado en \(s=-0.0002797\) corresponde a una dinámica
extremadamente lenta respecto de las restantes constantes de tiempo
del sistema.

Considerando que:

\begin{itemize}
	\item la resolución del sensor de altura se encuentra en el orden
	de centímetros,
	\item no se dispone de mediciones de alta precisión submilimétrica,
	\item el identificador polinomial puede ajustar polos muy cercanos
	al origen para capturar pequeñas tendencias de deriva,
\end{itemize}

se interpreta que dicho polo próximo a cero no representa una dinámica
física real dominante, sino un posible efecto de sobreajuste
(\textit{overfitting}) del procedimiento de identificación.

En consecuencia, y en coherencia con el modelo físico que contempla
amortiguamiento mecánico, se adopta la siguiente simplificación:

\[
G(s)=
\frac{-0.12107\,(s-14)(s+10.62)}
{s\,(s^2+5.61s+14.02)}
\]

En esta forma:

\begin{itemize}
	\item El polo en el origen representa el carácter integrador dominante
	de la posición vertical.
	\item El segundo orden \(s^2+5.61s+14.02\) modela la dinámica
	electromecánica del actuador.
	\item Los ceros se interpretan como parámetros de ajuste que capturan
	efectos agregados del actuador, discretización y linealización
	alrededor del punto de operación.
\end{itemize}

\subsubsection{Validez del modelo}

El modelo adoptado constituye una aproximación coherente con la física
del sistema y adecuada para el diseño de control dentro del rango
operativo \([1100,1700]\ \mu s\).

Fuera de dicho intervalo, el sistema presenta comportamientos no lineales
significativos (aerodinámica, fricción no lineal, saturaciones y posibles
limitaciones térmicas del actuador) que no son capturados por el modelo
lineal simplificado.

	
	% ============================================================
	\section{Métodos Clásicos de Control}
	% ============================================================
	% ============================================================
\subsection{Introducción}
% ============================================================

Los métodos clásicos de control se fundamentan en el análisis de sistemas lineales mediante funciones de transferencia y herramientas del dominio de la frecuencia y del plano complejo.

En este enfoque, la dinámica del sistema se representa como:

\[
G(s)=\frac{Y(s)}{U(s)}
\quad \text{o en tiempo discreto} \quad
G(z)=\frac{Y(z)}{U(z)}
\]

El objetivo del diseño consiste en definir un controlador \(C\) tal que el lazo cerrado:

\[
G_{cl}=\frac{C G}{1+CG}
\]

cumpla simultáneamente:

\begin{itemize}
	\item Estabilidad.
	\item Respuesta transitoria adecuada.
	\item Error estacionario reducido.
	\item Robustez frente a incertidumbres.
\end{itemize}

En este trabajo se implementaron y evaluaron las siguientes técnicas clásicas:

\begin{itemize}
	\item Controlador PID.
	\item Diseño por Lugar de Raíces.
	\item Diseño mediante Respuesta en Frecuencia (Bode).
	\item Síntesis Directa (Truxal--Ragazzini).
\end{itemize}

% ============================================================
\subsection{Modelo de la Planta}
% ============================================================

El diseño clásico parte de la función de transferencia discreta identificada:

\[
G(z) = \frac{
	b_0 + b_1 z^{-1} + b_2 z^{-2} + b_3 z^{-3}
}{
	1 + a_1 z^{-1} + a_2 z^{-2} + a_3 z^{-3}
}
\]

El tiempo de muestreo utilizado depende de la experiencia considerada, ya que los distintos controladores fueron desarrollados en paralelo por los miembros del grupo con diferentes enfoques de diseño.

% ============================================================
\subsection{Criterio general de estabilidad}
% ============================================================

Para sistemas discretos, la estabilidad en lazo cerrado requiere que todos los polos satisfagan:

\[
|z_i| < 1
\]

Este criterio será verificado en cada uno de los métodos desarrollados.

% ============================================================
\subsection{Controlador PID}
% ============================================================


Durante las etapas iniciales de diseño se intentó sintonizar un controlador PID utilizando la herramienta \texttt{PID Tuner} de MATLAB. Sin embargo, el desempeño obtenido no resultó adecuado para la planta bajo estudio, por lo que se decidió adoptar una formulación alternativa que permitiera un mayor control estructural sobre el comportamiento dinámico del sistema.

En consecuencia, se implementó un controlador PID basado en la formulación propuesta por Åström, directamente en su versión discreta. Esta decisión permitió diseñar el controlador coherentemente con el tiempo de muestreo del sistema, evitando discretizaciones posteriores y manteniendo consistencia entre simulación e implementación embebida.

El controlador opera en coordenadas relativas al punto de hover previamente estimado, es decir, la señal de control generada corresponde a una variación \(\Delta u\) respecto del equilibrio.

% ============================================================
\subsubsection{Formulación del PID de Åström \cite{PID-TESIS}}
% ============================================================

La estructura implementada separa explícitamente las acciones proporcional, integral y derivativa.

La acción proporcional se define como:

\[
P(k) = K\big(b\,r(k) - y(k)\big)
\]

donde \(K\) es la ganancia proporcional y \(b\) pondera la contribución de la referencia en la acción proporcional. En este trabajo se adoptó deliberadamente \(b=1\) por simplicidad estructural, evitando introducir grados adicionales de libertad innecesarios.

La acción derivativa se implementa mediante un filtro de primer orden:

\[
D(k) =
\frac{T_d}{T_d + N h}\,D(k-1)
-
\frac{K T_d N}{T_d + N h}\big(y(k) - y(k-1)\big)
\]

donde \(T_d\) es la constante derivativa, \(N\) limita el ancho de banda del término derivativo y \(h\) es el período de muestreo.

La inclusión del parámetro \(N\) resulta fundamental para evitar la amplificación excesiva del ruido de medición en altas frecuencias, fenómeno relevante dado que el sensor TFMini presenta cuantización del orden de centímetros.

La acción integral se describe como:

\[
I(k) = I(k-1) + \frac{K h}{T_i}\,e(k)
\]

donde \(T_i\) es la constante integral y \(e(k)=r(k)-y(k)\) es el error de control.

La señal de control total es:

\[
u(k) = P(k) + I(k) + D(k)
\]

% ============================================================
\subsubsection{Antiwindup}
% ============================================================

La acción integral se encuentra condicionada mediante un esquema de antiwindup por integración condicional. El término integral se actualiza únicamente cuando la señal de control no se encuentra saturada, o cuando el error contribuye a desaturar el actuador.

Este mecanismo evita acumulación indebida del estado integral y previene comportamientos abruptos ante saturaciones del PWM, mejorando la estabilidad práctica del sistema.

% ============================================================
\subsubsection{Criterios de sintonización}
% ============================================================

La sintonización se realizó de forma iterativa directamente sobre la estructura discreta del controlador, variando los parámetros en el siguiente orden:

\begin{enumerate}
	\item Ajuste de la ganancia proporcional \(K\) hasta aproximar el sistema al límite de estabilidad para obtener una respuesta rápida.
	\item Incorporación y ajuste del término integral \(T_i\) para eliminar error estacionario sin introducir oscilaciones excesivas.
	\item Incorporación del término derivativo \(T_d\) para mejorar amortiguamiento y reducir sobreimpulso.
	\item Ajuste del parámetro \(N\) como compromiso entre efectividad de la acción derivativa y rechazo de ruido.
\end{enumerate}

Durante todo el proceso se monitoreó cuidadosamente el esfuerzo de control. Como restricción experimental de diseño se impuso que la variación de la señal PWM no excediera aproximadamente \(10\,\mu s\) por centímetro de incremento en la altura, garantizando que el actuador no ingresara en saturación ni se expusiera la planta a condiciones potencialmente dañinas.

% ============================================================
\subsubsection{Resultados}
% ============================================================

Los parámetros implementados fueron:

\[
K_p = 2.5, \quad
T_i = 5, \quad
T_d = 0.1, \quad
N = 3
\]

Con un tiempo de muestreo $T_s = 1ms$.

En la simulación del modelo lineal se obtuvo:

\[
\%OS_{\text{sim}} \approx 46\%, \quad
t_r^{\text{sim}} = 0.397 \text{ s}
\]

En la implementación experimental se observaron los siguientes valores de sobreimpulso según la altura de referencia:

\begin{itemize}
	\item Para $57\,\text{cm}$: $\%OS = 28.07\%$
	\item Para $78\,\text{cm}$: $\%OS = 11.54\%$
	\item Para $90\,\text{cm}$: $\%OS \approx 0\%$
\end{itemize}

El tiempo de subida experimental fue:

\[
t_r^{\text{exp}} = 0.393 \text{ s}
\]

Se observa una coincidencia prácticamente exacta entre el tiempo de subida simulado y el experimental, lo cual valida la capacidad del modelo lineal para capturar la dinámica dominante del sistema.

Por otra parte, el sobreimpulso experimental disminuye progresivamente a medida que aumenta la altura de operación. Este comportamiento se asocia a variaciones del punto de operación y a no linealidades no completamente capturadas por el modelo lineal identificado.

\insertarfigura{img/PID/RstaPID.png}{Respuesta al escalón con el controlador PID.}{fig:rstaPID}{1}

\insertarfigura{img/PID/EsfuerzoPID.png}{Esfuerzo de control con el PID implementado.}{fig:EsfPID}{1}

\insertarfigura{img/PID/PID_Practico.png}
{Implementación práctica del controlador PID.}
{fig:PID_practica}{1}
% ============================================================
\subsubsection{Conclusiones sobre el método}
% ============================================================

Desde el punto de vista práctico, el controlador PID basado en la formulación de Åström demostró ser una herramienta sumamente eficaz para el control de la planta.

Uno de los aspectos más destacables es la relativa simplicidad con la que puede obtenerse una respuesta dinámica deseada. A diferencia de otros enfoques que requieren un conocimiento detallado de la estructura interna del sistema, el PID permite alcanzar un comportamiento satisfactorio mediante ajuste iterativo de un número reducido de parámetros, sin necesidad de un modelado exhaustivo ni de una comprensión profunda de todos los fenómenos físicos involucrados.

En este trabajo, aun considerando la presencia de dinámicas no modeladas, el PID logró reproducir con notable precisión el tiempo de subida predicho por el modelo y mantener estabilidad robusta en la implementación experimental.

Las diferencias observadas en el sobreimpulso entre simulación y práctica pueden atribuirse principalmente a características no modeladas ya mencionadas con anterioridad. Sin embargo, incluso bajo estas condiciones, el comportamiento general del sistema se mantuvo cercano al previsto teóricamente.

Asimismo, la inclusión del parámetro \(N\) en la acción derivativa permitió limitar la amplificación de ruido de alta frecuencia, resultando en un esfuerzo de control significativamente más limpio que el obtenido mediante otras metodologías analizadas más adelante. Este aspecto resulta particularmente atractivo en una gran variedad de casos reales.

En síntesis, el controlador PID demostró ofrecer una solución de alta relación beneficio–complejidad: requiere bajo conocimiento estructural del sistema, es sencillo de implementar en hardware embebido y permite obtener un desempeño dinámico competitivo dentro del rango operativo seguro de la planta.
\balance

\balance
\clearpage

% ============================================================
\subsection{Diseño por Lugar de Raíces}
% ============================================================


Dado que la planta identificada presenta polos ubicados fuera del círculo unitario, el sistema en lazo abierto resulta inestable en el dominio discreto. En consecuencia, cualquier diseño de control debe garantizar que los polos del lazo cerrado queden estrictamente dentro del círculo unitario para asegurar estabilidad interna.

\insertarfigura{img/CircUnitarioPlanta.png}
{Ubicación de los polos de la planta sin compensar en el plano $z$. Se observa que al menos uno de ellos se encuentra fuera del círculo unitario, lo que implica inestabilidad discreta.}
{fig:PolosPlantaSinCompensar}{1}

El objetivo del diseño consistió en modificar la dinámica del sistema mediante compensación, de modo que:

\begin{itemize}
	\item todos los polos del lazo cerrado queden dentro del círculo unitario,
	\item se logre un compromiso adecuado entre rapidez de respuesta y amortiguamiento,
	\item el esfuerzo de control permanezca dentro de límites físicamente realizables.
\end{itemize}

\subsubsection{Elección de la estructura del compensador}

Para estabilizar el sistema se adoptó una estructura de tipo \textbf{lag--lead} (atraso--adelanto). Esta configuración permite actuar simultáneamente sobre la estabilidad relativa y el desempeño en régimen permanente.

El término \textit{lead} (adelanto) se empleó para aumentar el margen de fase y desplazar los polos dominantes del lazo cerrado hacia regiones del plano \(z\) asociadas con mayor amortiguamiento y mejor desempeño transitorio. Por otro lado, el término \textit{lag} (atraso) permitió ajustar la ganancia en bajas frecuencias, mejorando el comportamiento estacionario sin comprometer significativamente la estabilidad.

La adecuada ubicación de ceros permitió modificar la geometría del lugar de raíces, atrayendo las ramas hacia la región estable del plano discreto, mientras que los polos adicionales modelaron el compromiso dinámico requerido.

\subsubsection{Determinación de la ganancia \(K\)}

Una vez definida la estructura del compensador, se analizó el lugar de raíces del sistema compensado. La determinación manual de la ganancia \(K\) resultó particularmente sensible, ya que pequeños incrementos en su valor provocaban que las trayectorias de los polos abandonaran el círculo unitario antes de satisfacer las especificaciones dinámicas deseadas.

Esta sensibilidad está directamente relacionada con la naturaleza inestable de la planta y con la fuerte dependencia de la ubicación de los polos del lazo cerrado respecto a la ganancia del compensador.

El compensador finalmente adoptado fue:

\begin{equation}
	C(z)=
	-0.0173 \;
	\frac{(z-1.0140)(z-0.5)}
	{(z-0.9522)(z-0.9894)}
\end{equation}

\subsubsection{Ajuste mediante \texttt{Optimization-Based Tuning}}

Con el fin de sistematizar el proceso de ajuste y garantizar el cumplimiento simultáneo de múltiples especificaciones (estabilidad, rapidez y esfuerzo de control), se utilizó la herramienta \texttt{Optimization-Based Tuning} de MATLAB.

Este enfoque permitió:

\begin{itemize}
	\item definir directamente especificaciones temporales (tiempo de establecimiento, sobreimpulso, etc.),
	\item ajustar automáticamente los parámetros del compensador,
	\item verificar la estabilidad del sistema en el dominio discreto.
\end{itemize}

El resultado fue un compensador lag--lead cuyos parámetros fueron obtenidos mediante optimización numérica, asegurando que los polos del lazo cerrado se mantengan dentro del círculo unitario y que el desempeño temporal cumpla con los objetivos establecidos para la planta experimental.

\insertarfigura{img/Rlocus/circuloUnitario.png}
{Ubicación de los polos del sistema compensado en el plano $z$. Se verifica que todos ellos se encuentran dentro del círculo unitario, garantizando estabilidad discreta.}
{fig:PolosSistemaCompensado}{1}

\insertarfigura{img/Rlocus/Esfuerzo.png}
{Esfuerzo de control en lazo cerrado con el compensador diseñado mediante Lugar de Raíces.}
{fig:EsfuerzoControlCompensado}{1}

\insertarfigura{img/Rlocus/rsta.png}
{Respuesta temporal del sistema en lazo cerrado con el compensador diseñado.}
{fig:RespuestaTemporalCompensada}{1}

\subsection{Práctica}
a completar lo que sale en la práctica
\insertarfigura{img/PID/RstaPIDPractica.png}{La respuesta con el controlador PID.}{fig:RstaPID_practica}{1}

\insertarfigura{img/PID/EsfuerzoPIDPractica.png}{El esfuerzo del con el controlador PID.}{fig:EsfPID_practica}{1}

\clearpage
\balance
% ============================================================
\subsection{Diseño por Respuesta en Frecuencia}
% ============================================================


\insertarfigura{img/Bode/bodeMat.png}
{Respuesta en frecuencia del lazo abierto del sistema identificado sin compensación.}
{fig:bodeMat}{1}

Para el diseño del controlador basado en el método de respuesta en frecuencia se utilizó directamente el modelo discreto identificado de la planta \(G(z)\), obtenido mediante identificación experimental y presentado en secciones anteriores.

El modelo fue incorporado al entorno \texttt{controlSystemDesigner} de MATLAB, lo que permitió analizar la respuesta en frecuencia del lazo abierto y ajustar el compensador de manera interactiva a partir de los diagramas de Bode.

% ============================================================
\subsubsection{Análisis del sistema sin compensar}
% ============================================================

En la Fig.~\ref{fig:bodeMatSinC} se presenta el diagrama de Bode correspondiente al lazo abierto conformado únicamente por la planta identificada.

A partir del análisis en frecuencia se obtuvieron los siguientes márgenes iniciales:

\begin{itemize}
	\item Margen de ganancia: \(13\,\text{dB}\),
	\item Margen de fase: \SI{61.7}{\degree}.
\end{itemize}

Si bien el sistema presenta margen de fase positivo, lo que implica estabilidad para ganancias moderadas, la frecuencia de cruce se encuentra relativamente baja, lo que se traduce en una respuesta temporal lenta.

La pendiente del módulo en la región de cruce evidencia la influencia de múltiples polos dominantes, coherentes con la dinámica de orden superior asociada al conjunto motor--ESC--hélice.

% ============================================================
\subsubsection{Diseño del compensador proporcional}
% ============================================================

En este caso particular se optó por implementar un compensador puramente proporcional:

\[
C(z) = K_p, \qquad K_p = 1.308
\]

Por lo tanto, el lazo abierto queda:

\[
L(z) = K_p\,G(z)
\]

La acción del controlador proporcional consiste exclusivamente en escalar la magnitud del lazo abierto sin introducir polos ni ceros adicionales. Desde el punto de vista del diagrama de Bode, esto implica un desplazamiento vertical del módulo, modificando la frecuencia de cruce y, en consecuencia, los márgenes de estabilidad.

El aumento de \(K_p\) incrementa la frecuencia de cruce, lo que produce:

\begin{itemize}
	\item Mayor ancho de banda del sistema.
	\item Reducción del tiempo de subida.
	\item Respuesta temporal más rápida.
\end{itemize}

Si bien no se introduce compensación dinámica de fase, el incremento de ganancia resulta suficiente para mejorar significativamente la rapidez de respuesta manteniendo márgenes aceptables.

% ============================================================
\subsubsection{Análisis del sistema compensado}
% ============================================================

En la Fig.~\ref{fig:bodeMatConC} se presenta el diagrama de Bode del sistema compensado.

Los márgenes obtenidos fueron:

\begin{itemize}
	\item \textbf{Sistema sin compensación:}
	\begin{itemize}
		\item Margen de ganancia: \(13\,\text{dB}\),
		\item Margen de fase: \SI{61.7}{\degree}.
	\end{itemize}
	\item \textbf{Sistema compensado:}
	\begin{itemize}
		\item Margen de ganancia: \(10.6\,\text{dB}\),
		\item Margen de fase: \SI{52.9}{\degree}.
	\end{itemize}
\end{itemize}

Se observa una reducción controlada del margen de fase como consecuencia del aumento de la frecuencia de cruce. No obstante, el sistema mantiene estabilidad relativa adecuada.
\balance
\onecolumn
\insertarfigura{img/Bode/bode_C.png}
{Diagrama de Bode y respuestas temporales del sistema con compensación.}
{fig:bodeMatSinC}{1}

\insertarfigura{img/Bode/bode_compensado.png}
{Diagrama de Bode y respuestas temporales del sistema con compensación.}
{fig:bodeMatConC}{1}
\twocolumn

La validación del diseño se realizó mediante la respuesta temporal en lazo cerrado. El sistema compensado presentó:

\begin{itemize}
	\item Reducción significativa del tiempo de subida.
	\item Sobreimpulso moderado.
	\item Esfuerzo dentro de límites aceptables.
\end{itemize}

% ============================================================
\subsection{Práctica}
% ============================================================

\insertarfigura{img/Bode/bode_practica.png}
{Respuesta experimental: altura y esfuerzo con el compensador proporcional.}
{fig:B_practica}{0.8}

Compensador utilizado:

\[
C_{Bode} = 1.3082
\]

Tiempo de muestreo:

\[
T_s = 0.0001\,\text{s}
\]

El sobreimpulso simulado fue aproximadamente \(20\%\).  
En la implementación experimental se observó:

\begin{itemize}
	\item Primer levantamiento: \(\%OS = 55.55\%\),
	\item Levantamiento posterior: \(\%OS \approx 37\%\),
	\item Tendencia decreciente al aumentar la altura.
\end{itemize}

Esta variabilidad sugiere dependencia del punto de operación, condiciones iniciales y no linealidades del empuje.

Se observa además que durante el primer levantamiento el esfuerzo aplicado es mayor al inicio, disminuyendo a medida que el sistema se aproxima al equilibrio dinámico.

% ============================================================
\subsubsection{Conclusión del método}
% ============================================================

El método basado en respuesta en frecuencia permitió evaluar de manera directa y sencilla la estabilidad relativa del sistema mediante los márgenes de fase y ganancia.

A diferencia del Lugar de Raíces, donde la estabilidad se interpreta geométricamente en el plano Z, el método de Bode permite cuantificar cuánto margen de estabilidad se posee y cuánto puede sacrificarse en favor de mayor rapidez.

El ajuste mediante un simple controlador proporcional demostró que, en este sistema particular, el aumento controlado de ganancia resulta suficiente para mejorar significativamente la respuesta temporal, incrementando el ancho de banda sin comprometer la estabilidad global.

Sin embargo, el método presenta una limitación conceptual: al trabajar exclusivamente en el dominio de la frecuencia se pierde información geométrica directa sobre la ubicación de polos y su relación exacta con parámetros transitorios como \(\zeta\) y \(\omega_n\). Si bien existen aproximaciones que relacionan margen de fase y amortiguamiento, estas no son exactas.

En la práctica, esta desventaja se ve mitigada por herramientas de simulación que permiten validar la respuesta temporal rápidamente, situando al método al mismo nivel práctico que los demás enfoques analizados.

En síntesis, el diseño por respuesta en frecuencia resultó sencillo, intuitivo y efectivo, permitiendo mejorar la rapidez del sistema mediante un ajuste mínimo del controlador y manteniendo márgenes de estabilidad adecuados para la planta experimental desarrollada.
\balance

\balance
\clearpage
% ============================================================
\subsection{Síntesis Directa \cite{Fadali2020}}
% ============================================================


\subsubsection{Síntesis directa (Truxal--Ragazzini)}

Se parte del modelo discreto identificado de la planta:

\begin{equation}
	\label{eq:GZAS}
	G_{ZAS}(z)=
	\frac{-0.0001205\,z^{-1}+0.0002415\,z^{-2}-0.0001209\,z^{-3}}
	{1-2.994\,z^{-1}+2.989\,z^{-2}-0.9944\,z^{-3}} .
\end{equation}

El método de Truxal--Ragazzini consiste en especificar explícitamente
una dinámica deseada en lazo cerrado $G_{cl}(z)$ y obtener el controlador
a partir de la relación:

\begin{equation}
	\label{eq:TR_general}
	G_{cl}(z)=\frac{C(z)G_{ZAS}(z)}{1+C(z)G_{ZAS}(z)}
	\quad \Longrightarrow \quad
	C(z)=\frac{1}{G_{ZAS}(z)}\frac{G_{cl}(z)}{1-G_{cl}(z)}.
\end{equation}

Este procedimiento implica una inversión explícita del modelo de la planta,
por lo que el diseño depende fuertemente de la exactitud del modelo identificado.

\paragraph{Método 1: respuesta \textit{deadbeat}}

Como primera aproximación se adoptó una dinámica deseada del tipo
\textit{deadbeat}, definida por:

\begin{equation}
	G_{cl}(z)=z^{-1}.
\end{equation}

Esta elección implica que la salida alcance el valor deseado
en un único período de muestreo, anulando el error en el menor tiempo posible.

Reemplazando en \eqref{eq:TR_general} se obtiene:

\begin{equation}
	\label{eq:C1}
	C_1(z)=\frac{1}{G_{ZAS}(z)}\frac{z^{-1}}{1-z^{-1}}
	=
	\frac{1}{G_{ZAS}(z)}\frac{1}{z-1}.
\end{equation}

Se observa que el controlador resultante contiene explícitamente
la inversa de la planta y un polo adicional en $z=1$,
lo que anticipa posibles problemas de magnitud del esfuerzo de control.

\subsubsection{Método 2: Deadbeat \textit{ripple-free}}

Como alternativa se evaluó la variante \textit{ripple-free},
cuyo controlador obtenido es:

\begin{equation}
	\label{eq:C2_z}
	C_2(z)=
	\frac{10.43038\,z^{2}-2.7635\,z+3.3344}
	{-0.33662\,z^{2}-0.66338\,z+1}.
\end{equation}

Multiplicando numerador y denominador por $(-1)$
y normalizando el coeficiente líder del denominador, se obtiene:

\begin{equation}
	\label{eq:C2_z_norm}
	C_2(z)=
	\frac{-30.9856\,z^{2}+8.2096\,z-9.9055}
	{z^{2}+1.9707\,z-2.9707}.
\end{equation}

Para implementación digital resulta conveniente expresarlo en términos de $z^{-1}$:

\begin{equation}
	\label{eq:C2_zinv}
	C_2(z)=
	\frac{-30.9856+8.2096\,z^{-1}-9.9055\,z^{-2}}
	{1+1.9707\,z^{-1}-2.9707\,z^{-2}}.
\end{equation}

\subsubsection{Resultados y Limitaciones Prácticas}

Las simulaciones mostraron que ambos controladores demandan esfuerzos
de control extremadamente elevados, alcanzando valores del orden de:

\[
|u_{\max}| \sim 10^{29},
\]

lo cual excede ampliamente las capacidades del actuador físico.

En la implementación real, la señal de control corresponde a una
señal PWM tipo servo a $50\,\text{Hz}$ acotada en el rango:

\begin{equation}
	u \in [1000,2000]\ \mu s .
\end{equation}

La magnitud desproporcionada del esfuerzo se explica por:

\begin{itemize}
	\item La inversión explícita del modelo $G_{ZAS}(z)$.
	\item La presencia de polos cercanos a $z=1$ en la planta.
	\item Alta sensibilidad a pequeñas incertidumbres del modelo.
	\item Cancelaciones exactas requeridas por el diseño.
\end{itemize}

En particular, la inversión de dinámicas cercanas al borde del círculo unitario produce amplificaciones significativas en la señal de control, haciendo que el diseño sea extremadamente sensible a variaciones como cambios en la tensión de batería, fricción, efectos aerodinámicos y dinámica no modelada del conjunto ESC--motor--hélice.

Por estas razones, si bien la síntesis directa resulta valiosa desde el punto de vista conceptual y didáctico, no se considera viable para implementación experimental en la planta real.

En consecuencia, para la etapa práctica se priorizan estrategias de menor orden y mayor robustez, que contemplen explícitamente las limitaciones del actuador y la saturación de la señal de control.


% ============================================================
\subsection{Comparación entre Métodos Clásicos}
% ============================================================


La aplicación de distintos métodos clásicos sobre la misma planta
permitió comparar directamente su aplicabilidad práctica,
su dependencia del modelo y la información que aportan
para el diseño.

Desde el punto de vista de simplicidad y rapidez de implementación,
el \textbf{PID} resultó claramente el método más eficiente.
Permitió obtener una respuesta satisfactoria
con escasa dependencia del modelo
y con un proceso de ajuste directo e intuitivo.
En términos puramente ingenieriles,
si el objetivo es lograr funcionamiento estable
con el menor esfuerzo analítico posible,
el PID se posiciona como la alternativa más práctica.

El \textbf{Lugar de Raíces}, en cambio,
aportó mayor profundidad conceptual.
Permite visualizar explícitamente la estabilidad
en el plano \(z\),
relacionar polos con desempeño transitorio
y comprender geométricamente el efecto de la ganancia.
Si bien exige mayor comprensión del modelo
y es más sensible a la elección del período de muestreo,
proporciona una cantidad de información estructural
superior a la del PID.

El método basado en \textbf{Bode}
resultó adecuado para analizar márgenes de estabilidad
y robustez,
pero menos intuitivo en relación con el comportamiento temporal.
Sin herramientas de simulación,
su aplicación manual se vuelve considerablemente más compleja.
En esta experiencia no ofreció ventajas decisivas
respecto a los métodos anteriores.

Por su parte,
la \textbf{Síntesis Directa (Truxal--Ragazzini)}
mostró gran elegancia analítica
y rigor matemático,
pero evidenció una dependencia extrema del modelo
y una tendencia a generar esfuerzos de control elevados.
En una planta física con saturaciones y no linealidades,
esta sensibilidad la vuelve riesgosa para implementación real,
quedando principalmente como herramienta académica.

En conclusión,
los métodos que demostraron mayor aplicabilidad práctica
en la planta experimental fueron el PID y el Lugar de Raíces:
el primero por su simplicidad y robustez,
el segundo por la riqueza de información que aporta.
Los demás métodos resultaron valiosos conceptualmente,
pero menos determinantes en la implementación experimental concreta.


\balance
\clearpage

	
	% ============================================================
	\section{Métodos Modernos de Control}
	% ============================================================
	
	
\subsection{Introducción}

A diferencia de los métodos clásicos basados en funciones de transferencia
y análisis en frecuencia, los métodos modernos de control se fundamentan
en la representación en espacio de estados del sistema dinámico.
\subsubsection{Modelo en espacio de estados continuo}

Las matrices del modelo continuo en representación
estado–espacio son:

\[
F =
\begin{bmatrix}
	-5.6102 & -3.5055 & -0.0314 \\
	4.0000  & 0       & 0       \\
	0       & 0.0312  & 0
\end{bmatrix}
\]

\[
G =
\begin{bmatrix}
	16 \\
	0 \\
	0
\end{bmatrix}
\]

\[
H =
\begin{bmatrix}
	-0.0076 & 0.0064 & 8.9975
\end{bmatrix}
\]

\[
J =
\begin{bmatrix}
	0
\end{bmatrix}
\]

En este enfoque, la dinámica se describe mediante:

\[
x_{k+1}=Ax_k+Bu_k
\]
\[
y_k=Cx_k+Du_k
\]

donde $x_k \in \mathbb{R}^n$ es el vector de estados,
$u_k$ la entrada de control y $y_k$ la salida medida.

Este formalismo permite:

\begin{itemize}
	\item Diseñar realimentación directa de estados.
	\item Ubicar polos del sistema de manera sistemática.
	\item Formular problemas de control óptimo.
	\item Incorporar estimadores de estado.
\end{itemize}


% ============================================================
\subsubsection{Modelo en Espacio de Estados}
% ============================================================

El modelo discreto obtenido:
\[
A = e^{F T_s}, 
\qquad
B = F^{-1}\!\left(e^{F T_s} - I\right) G,
\qquad
C = H,
\]
\[
D = J
\]


Tiempo de muestreo depende de la práctica

% ============================================================
\subsubsection{Análisis de Controlabilidad y Observabilidad}
% ============================================================

Matriz de controlabilidad:

\[
\mathcal{C}=[G\ FG\ F^2G\ \dots F^{n-1}G]
\]

\[
\text{rank}(\mathcal{C}) = \textbf{[3]}
\]

Matriz de observabilidad:

\[
\mathcal{O}=\begin{bmatrix}
	H \\ HF \\ HF^2 \\ \vdots \\ HF^{n-1}
\end{bmatrix}
\]

\[
\text{rank}(\mathcal{O}) = \textbf{[3]}
\]

Conclusión estructural:

Dado que el rango de ambas matrices coincide con el orden del sistema,
se concluye que el modelo es completamente controlable y completamente
observable.

Desde el punto de vista físico, esto implica que:

\begin{itemize}
	\item Existe una combinación adecuada de la señal de entrada que
	permite influenciar todos los estados internos del sistema.
	\item La salida medida contiene información suficiente para reconstruir
	completamente el vector de estados mediante un observador.
\end{itemize}

En consecuencia, el modelo identificado resulta estructuralmente apto
para el diseño de control por realimentación de estados, ubicación de polos,
LQR y estimación de estados mediante observador de Luenberger o filtro de Kalman.

No obstante, la validez práctica de dicha síntesis en tiempo discreto
depende de que el tiempo de muestreo \(T_s\) sea suficientemente pequeño
para capturar la dinámica relevante del sistema, evitando aliasing.
En particular, se asume que las componentes significativas de la señal de
salida y de las perturbaciones se encuentran por debajo de la frecuencia de
Nyquist \(f_N = \tfrac{1}{2T_s}\), y que el acondicionamiento analógico
(filtrado anti-alias) es consistente con esta hipótesis.

% ============================================================

\subsection{Ubicación Arbitraria de Polos}
% ============================================================
\subsection{Realimentación de Estados y Estimación}

\subsubsection{Realimentación de Estados}

Considerando el modelo discreto del sistema:

\[
x_{k+1} = A x_k + B u_k
\]
\[
y_k = C x_k
\]

se propone una ley de control por realimentación de estados:

\[
u_k = -K x_k
\]

lo que conduce a la dinámica en lazo cerrado:

\[
x_{k+1} = (A - BK)x_k
\]

El diseño por ubicación arbitraria de polos consiste en determinar la
matriz de ganancia \(K\) tal que:

\[
\lambda(A - BK) = \{p_1, p_2, \dots, p_n\}
\]

siendo \(|p_i| < 1\) condición necesaria para estabilidad discreta.

La determinación de \(K\) puede realizarse mediante el método de
Ackermann o utilizando la función \texttt{place()} de MATLAB, siempre
que el sistema sea completamente controlable.

\vspace{0.3cm}

\subsubsection{Estimador de Estados}

En situaciones donde no todos los estados son medibles, se introduce
un estimador de Luenberger para reconstruir el vector de estados a
partir de la entrada y la salida medida.

El estimador discreto se define como:

\[
\hat{x}_{k+1} = A\hat{x}_k + B u_k + L (y_k - \hat{y}_k)
\]

donde:

\[
\hat{y}_k = C \hat{x}_k
\]

La dinámica del error de estimación:

\[
e_k = x_k - \hat{x}_k
\]

queda gobernada por:

\[
e_{k+1} = (A - LC)e_k
\]

Por lo tanto, la convergencia del estimador depende de la ubicación
de los autovalores de la matriz \(A - LC\), que pueden ser fijados
arbitrariamente siempre que el sistema sea observable:

\[
\lambda(A - LC) = \{p_{obs,1}, \dots, p_{obs,n}\}
\]

\vspace{0.3cm}

\subsubsection{Principio de Separación}

Cuando se combinan realimentación de estados y estimación, la ley de
control adopta la forma:

\[
u_k = -K\hat{x}_k
\]

y la dinámica total del sistema presenta autovalores dados por la
unión de los polos del controlador y los polos del estimador:

\[
\lambda_{\text{total}} =
\lambda(A - BK) \cup \lambda(A - LC)
\]

Este resultado, conocido como principio de separación, permite diseñar
independientemente el controlador y el estimador.


\clearpage

\subsection{Estimación de Estados}
% ============================================================

\subsubsection{Fundamento del Estimador de Estados}

En sistemas donde no es posible medir directamente todos los estados,
se recurre a un estimador (observador) que reconstruye el vector de
estados a partir de la señal de entrada y la salida medida.

El sistema dinámico discreto se describe mediante las ecuaciones \ref{eq:din_estado} y \ref{eq:din_salida}.


El estimador de Luenberger discreto se define como:

\[
\hat{x}_{k+1} = A\hat{x}_k + B u_k + L (y_k - \hat{y}_k)
\]

donde:

\[
\hat{y}_k = C\hat{x}_k
\]

y \(L\) es la matriz de ganancia del estimador.

\subsubsection{Dinámica del Error de Estimación}

Definiendo el error de estimación como:

\[
e_k = x_k - \hat{x}_k
\]

se obtiene la dinámica:

\[
e_{k+1} = (A - LC)e_k
\]

Por lo tanto, la convergencia del estimador depende únicamente de los
autovalores de la matriz \(A - LC\).

El objetivo del diseño consiste en ubicar arbitrariamente los polos del
observador:

\[
\lambda(A - LC) = \{p_{obs,1}, p_{obs,2}, \dots, p_{obs,n}\}
\]

\subsubsection{Selección de Polos del Observador}

Se seleccionaron los siguientes polos para el estimador:

\[
\{p_{obs}\} =
\left\{
0.8 + 0.25i,\;
0.8 - 0.25i,\;
0.9
\right\}
\]

Los criterios adoptados fueron:

\begin{itemize}
	\item Garantizar estabilidad discreta ($|p_{obs,i}|<1$).
	\item Asegurar una convergencia más rápida que la dinámica del
	sistema en lazo cerrado.
	\item Introducir amortiguamiento adecuado en el error de estimación.
\end{itemize}

\subsubsection{Cálculo de la Ganancia del Observador}

La ganancia \(L\) fue obtenida mediante el método dual de Ackermann,
utilizando la función \texttt{acker()} aplicada a la matriz transpuesta:

\[
L = \texttt{acker}(A^T, C^T, p_{obs})^T
\]

Para la estructura implementada, se obtuvieron las siguientes ganancias:

\[
L_{\text{actual}} =
\begin{bmatrix}
	126.8411 \\
	172.0621 \\
	45.1301
\end{bmatrix}
\]

y para la variante predictor:

\[
L_{\text{predictor}} =
\begin{bmatrix}
	167.3535 \\
	253.6822 \\
	86.0310
\end{bmatrix}
\]

\subsubsection{Verificación de Polos}

Se verificó que los autovalores de la matriz \(A - LC\) coinciden con
los polos deseados:

\[
\lambda(A - LC) =
\left\{
0.8 + 0.25i,\;
0.8 - 0.25i,\;
0.9
\right\}
\]

confirmando la correcta ubicación arbitraria de polos del estimador.

\subsubsection{Análisis del Comportamiento del Estimador}

La magnitud relativamente elevada de las ganancias del observador se
debe a la necesidad de forzar una convergencia rápida del error de
estimación. Polos más cercanos al origen implican mayores ganancias en
la matriz \(L\), acelerando la dinámica del error:

\[
e_k \sim p_{obs}^k
\]

Se observa que el estimador converge rápidamente hacia el estado real,
permitiendo su utilización en esquemas de control por realimentación
de estados estimados.

\subsubsection{Conclusión}

El estimador diseñado cumple con los requisitos de estabilidad y
convergencia rápida. La ubicación arbitraria de los polos del
observador garantiza que el error de estimación decae exponencialmente,
validando la implementación del estimador de Luenberger en el dominio
discreto.


\clearpage

% ============================================================
\subsection{Control Óptimo (LQR)}
% ============================================================

Se define el funcional de costo:

\[
J=\sum_{k=0}^{\infty}(x_k^TQx_k+u_k^TRu_k)
\]

Matrices de ponderación:

\[
Q=\textbf{[Completar]}
\]

\[
R=\textbf{[Completar]}
\]

La solución se obtiene resolviendo la ecuación de Riccati discreta:

\[
P=A^TPA - A^TPB(R+B^TPB)^{-1}B^TPA + Q
\]

Ganancia óptima:

\[
K=(R+B^TPB)^{-1}B^TPA
\]

Polos obtenidos:

\[
\lambda(A-BK)=\textbf{[Completar]}
\]


\clearpage
% ============================================================

% ============================================================
\subsection{Control Óptimo con Integrador}
% ============================================================

% ============================================================
\subsection{Control Óptimo con Integrador (LQI)}
% ============================================================

\subsubsection{Motivación}

El regulador cuadrático lineal (LQR) permite optimizar el compromiso
entre desempeño y esfuerzo de control. Sin embargo, no garantiza
error estacionario nulo ante referencias constantes.

Para eliminar el error permanente, se introduce un integrador
del error de seguimiento, ampliando el modelo del sistema.


% ============================================================
\subsubsection{Modelo Aumentado}
% ============================================================

Sistema original:

\[
x_{k+1} = Ax_k + Bu_k
\]

\[
y_k = Cx_k
\]

Se define el estado integrador:

\[
\xi_{k+1} = \xi_k + (r_k - y_k)
\]

Definiendo el estado aumentado:

\[
x_a =
\begin{bmatrix}
	x \\
	\xi
\end{bmatrix}
\]

el sistema aumentado queda:

\[
x_{a,k+1} =
\underbrace{
	\begin{bmatrix}
		A & 0 \\
		-C & 1
	\end{bmatrix}
}_{A_a}
x_{a,k}
+
\underbrace{
	\begin{bmatrix}
		B \\
		0
	\end{bmatrix}
}_{B_a}
u_k
+
\begin{bmatrix}
	0 \\
	1
\end{bmatrix}
r_k
\]


% ============================================================
\subsubsection{Diseño LQR sobre el Sistema Aumentado}
% ============================================================

Se define el funcional de costo:

\[
J=\sum_{k=0}^{\infty}
\left(
x_a^T Q_a x_a + u_k^T R u_k
\right)
\]

Matrices de ponderación:

\[
Q_a = \textbf{[Completar]}
\]

\[
R = \textbf{[Completar]}
\]

La solución se obtiene resolviendo la ecuación de Riccati discreta
para el sistema aumentado.

Ganancia obtenida:

\[
K_a =
\begin{bmatrix}
	K_x & K_i
\end{bmatrix}
\]

donde:

\begin{itemize}
	\item $K_x$ actúa sobre los estados originales.
	\item $K_i$ actúa sobre el estado integrador.
\end{itemize}

Ley de control final:

\[
u_k = -K_x x_k - K_i \xi_k
\]


% ============================================================
\subsubsection{Polos del Sistema Aumentado}
% ============================================================

Los polos del lazo cerrado resultan de:

\[
\lambda(A_a - B_a K_a)
=
\textbf{[Completar]}
\]

Se verifica estabilidad discreta:

\[
|p_i| < 1
\]


% ============================================================
\subsubsection{Resultados}
% ============================================================

\begin{itemize}
	\item Error estacionario: \textbf{[Completar]}
	\item Sobreimpulso: \textbf{[Completar]}
	\item Tiempo de establecimiento: \textbf{[Completar]}
	\item Pico de esfuerzo de control: \textbf{[Completar]}
\end{itemize}

\insertarfigura{img/Estados/respuesta_LQI.png}
{Respuesta temporal con LQI.}
{fig:resp_LQI}{1}

\insertarfigura{img/Estados/esfuerzo_LQI.png}
{Esfuerzo de control con LQI.}
{fig:esfuerzo_LQI}{1}


% ============================================================
\subsubsection{Discusión}
% ============================================================

La incorporación del integrador permite eliminar el error
estacionario frente a referencias constantes.

En comparación con:

\begin{itemize}
	\item Ubicación arbitraria de polos:
	el método LQI introduce un criterio explícito de optimización.
	\item LQR sin integrador:
	se logra seguimiento exacto de referencia.
\end{itemize}

Limitaciones:

\begin{itemize}
	\item Incremento del orden del sistema.
	\item Mayor sensibilidad a saturación si $K_i$ es elevado.
	\item Dependencia del modelo identificado.
\end{itemize}

\clearpage

% ============================================================
\subsection{Filtro de Kalman}
% ============================================================
% ============================================================
\subsubsection{Modelo con Ruido}
% ============================================================

Para incorporar incertidumbre y modelar explícitamente la presencia de
perturbaciones no modeladas y ruido del sensor, se adopta la siguiente
representación estocástica discreta:

\[
x_{k+1} = A x_k + B u_k + w_k
\]

\[
y_k = C x_k + v_k
\]

donde:

\begin{itemize}
	\item $w_k \sim \mathcal{N}(0,Q)$ representa el ruido de proceso,
	asociado a dinámica no modelada, perturbaciones aerodinámicas
	y simplificaciones del modelo identificado.
	\item $v_k \sim \mathcal{N}(0,R)$ representa el ruido de medición,
	proveniente del sensor láser de distancia.
\end{itemize}

% ============================================================
\paragraph{Estimación de $R$ (ruido de medición)}
% ============================================================

La varianza del ruido de medición se obtuvo mediante ensayos empíricos,
midiendo la dispersión de la señal del sensor con la planta en reposo.
Siendo $\sigma_v$ la desviación estándar medida (en cm), se adopta:

\[
R = \sigma_v^2
\]

En los ensayos realizados se obtuvo:

\[
\sigma_v = 2.043\ \text{cm}
\qquad\Longrightarrow\qquad
R = 4.174\ \text{cm}^2
\]

% ============================================================
\paragraph{Parametrización y sintonización de $Q$ (ruido de proceso)}
% ============================================================

La matriz de covarianza del ruido de proceso se parametrizó como:

\[
Q = q\,I_n
\]

donde $q$ es un escalar positivo ajustable e $I_n$ es la matriz identidad
de dimensión $n$. El valor de $q$ se determinó mediante consistencia
estadística de la innovación normalizada (ver Apéndice~\ref{ap:kalman_tuning}).

El valor óptimo obtenido fue:

\[
q = 50.8022\times 10^{-3}
\qquad\Longrightarrow\qquad
Q = q\,I_n
\]

% ============================================================
\subsubsection{Filtro de Kalman en régimen permanente}
% ============================================================

Se utilizó un estimador de Kalman discreto en su variante \textit{current estimator}
(\texttt{kalman(...,'current')}). La ganancia en régimen permanente \(L\) se obtiene
a partir de la solución estacionaria \(P\) de la ecuación de Riccati discreta:

\[
P = A P A^T - A P C^T \left(C P C^T + R\right)^{-1} C P A^T + Q
\]

y la ganancia queda:

\[
L = A P C^T \left(C P C^T + R\right)^{-1}
\]

La ganancia obtenida en MATLAB para el modelo discretizado fue:

\[
L =
\begin{bmatrix}
	97.4968\\
	192.3014\\
	94.7820
\end{bmatrix}
\]

% ============================================================
\paragraph{Polos del observador}
% ============================================================

Para el \textit{current estimator}, la dinámica del error queda
determinada por \(A - LCA\). Los polos obtenidos fueron:

\[
\lambda(A - LCA) =
\begin{aligned}
	&0.9730274 + 0.0308455\,j \\
	&0.9730274 - 0.0308455\,j \\
	&0.9694430
\end{aligned}
\]

Todos los polos se ubican dentro del círculo unitario,
garantizando estabilidad del estimador.

% ============================================================
\subsubsection{Control integral y realimentación de estados (LQGI)}
% ============================================================

Con el objetivo de eliminar el error en régimen permanente ante referencias tipo escalón,
se incorporó un integrador de error. Definiendo el estado integral \(\xi_k\):

\[
\xi_{k+1} = \xi_k + \left(r_k - y_k\right)
\]

Se diseñó una ley de control tipo LQI:

\[
u_k = -K_x\,\hat{x}_k + K_i\,\xi_k
\]

donde \(\hat{x}_k\) proviene del estimador de Kalman.

Los valores obtenidos mediante \texttt{dlqr} fueron:

\[
K_x =
\begin{bmatrix}
	16.5343 & -15.8046 & 15.1271
\end{bmatrix}
\]

\[
K_i = 3.2067
\]

% ============================================================
\paragraph{Polos de la planta y del lazo cerrado}
% ============================================================

Los polos de la planta discretizada (sin control) fueron:

\[
\lambda(A) =
\begin{aligned}
	&0.9999972 \\
	&0.9720409 + 0.0241158\,j \\
	&0.9720409 - 0.0241158\,j
\end{aligned}
\]

Los polos del lazo cerrado del sistema aumentado (planta + integrador + control) fueron:

\[
\lambda(A_{\text{cl}}) =
\begin{aligned}
	&0.9735939 + 0.0318308\,j \\
	&0.9735939 - 0.0318308\,j \\
	&0.9671966 \\
	&0
\end{aligned}
\]

Se verifica estabilidad discreta y presencia del polo en cero
asociado a la acción integral.

% ============================================================
\subsubsection{Sistema aumentado usado para \texttt{pzmap}}
% ============================================================

En el script de validación se construyó explícitamente el sistema aumentado:

\[
A_{\text{aug}} =
\begin{bmatrix}
	A & B \\
	K_x - K_xA - K_iCA & 1 - K_xB - K_iCB
\end{bmatrix}
\]

\[
B_{\text{aug}} =
\begin{bmatrix}
	0\\
	0\\
	0\\
	K_i
\end{bmatrix}
\]

\[
C_{\text{aug}} =
\begin{bmatrix}
	C & 0
\end{bmatrix}
\qquad
D_{\text{aug}} = 0
\]

Construyéndose el modelo:

\[
\texttt{sysDaug = ss(Aaug,Baug,Caug,0,Ts)}
\]

y visualizando los polos mediante \texttt{pzmap(sysDaug)}.


\insertarfigura{img/LQGi/LQGi_step_ruido.png}
{Respuesta temporal en presencia de ruido.}
{fig:lqgi_step_ruido}{1}

\insertarfigura{img/LQGi/LQGi_esfuerzo_ruido.png}
{Esfuerzo de control en presencia de ruido.}
{fig:lqgi_esfuerzo_ruido}{1}

\insertarfigura{img/LQGi/LQGi_practico.png}
{Resultado experimental sobre la planta real.}
{fig:lqgi_practico}{1}

\insertarfigura{img/LQGi/LQGi_pzmap_NaranajaOBS_AzulPLANTA.png}
{Mapa de polos: observador (naranja) y sistema (azul).}
{fig:lqgi_pzmap}{1}

% ============================================================
\subsubsection{Discusión}
% ============================================================

En régimen dinámico, el tiempo de subida experimental fue aproximadamente
\(t_r^{\text{exp}} \approx 0.65\,\text{s}\), con sobreimpulso reducido
(del orden del 10\% o menor en la mayoría de los ensayos).
La simulación con ruido predijo \(t_r^{\text{sim}} \approx 0.5\,\text{s}\)
y sobreimpulso cercano al 8\%, mostrando buena concordancia.

El esfuerzo de control en simulación presenta ciertos picos,
producto de la inyección explícita de ruido gaussiano. En la práctica,
los picos resultaron más consistentes y de menor amplitud relativa,
especialmente en comparación con implementaciones anteriores sin Kalman, cumpliendo de muy buena manera su objetivo.

Un aspecto destacable es que, gracias a la acción integral,
el error en régimen permanente converge sistemáticamente a cero,
a diferencia de prácticas previas donde persistían derivas
incluso utilizando prefiltros o compensaciones adicionales.

% ============================================================
\subsubsection{Conclusión}
% ============================================================

El esquema LQGI (LQR + Kalman + Integrador) constituye la síntesis más
completa implementada en este trabajo.

A diferencia de la simple ubicación arbitraria de polos, aquí los polos
del lazo cerrado resultan de una optimización basada en criterios
energéticos bien definidos por el ingeniero.

Aunque la selección de \(Q\) y \(R\) no es intuitiva y requiere
criterio y ajuste iterativo, una vez correctamente definidos,
el método produce resultados consistentes, robustos y coherentes
con el modelo.

La incorporación del filtro de Kalman permitió seleccionar las ganancias
del observador de forma óptima en presencia de ruido, reduciendo la
amplificación observada en enfoques anteriores.

Finalmente, la inclusión del integrador garantizó eliminación del error
estacionario y otorgó el mejor desempeño global entre todas las técnicas
implementadas, tanto en simulación como en la planta real.

	\section{Conclusiones}
	% 

La implementación de métodos clásicos y modernos sobre una misma
planta física permitió contrastar dos filosofías de diseño claramente
diferenciadas: una basada en aproximaciones entrada–salida y otra
basada en modelado estructural completo en espacio de estados.

Los \textbf{métodos clásicos} demostraron ser progresivamente más
sofisticados en análisis, pero mantienen una característica común:
son esencialmente aproximativos y heurísticos.
El PID ajusta directamente la respuesta observable;
Lugar de Raíces y Bode aportan mayor estructura,
pero continúan trabajando sobre simplificaciones
(dinámicas dominantes, márgenes, aproximaciones de segundo orden).
Incluso la Síntesis Directa, aunque analíticamente rigurosa,
termina siendo extremadamente dependiente del modelo ideal.

Su principal fortaleza es la robustez práctica:
al no apoyarse completamente en la estructura interna del sistema,
toleran mejor imprecisiones del modelo y variaciones paramétricas.
Son, en esencia, más “plug and play” y requieren menor
conocimiento profundo de la planta.

En contraste, los \textbf{métodos modernos} trasladan el núcleo del diseño
al modelo en espacio de estados.
Aquí ya no se moldea únicamente la salida,
sino la dinámica interna completa del sistema.
Esto implica una dependencia mucho mayor del modelo:
cualquier error de identificación se refleja directamente
en el desempeño del lazo cerrado.

Sin embargo, cuando el modelo es representativo,
el salto cualitativo es evidente.
La realimentación de estados permite ubicar dinámicas internas;
el LQR elimina la arbitrariedad en la selección de polos,
basando el diseño en un criterio óptimo explícito; y
el Filtro de Kalman reduce sistemáticamente la contaminación del esfuerzo por ruido.

La combinación:

\[
\text{LQR} + \text{Kalman} + \text{Integrador}
\]

representó el mejor desempeño global obtenido en el trabajo.
Este salto no provino de “hacer más cálculos por hacerlos”,
sino de incorporar estructura explícita (óptimo + estimación + rechazo de perturbaciones).
Aun así, el costo computacional \textbf{sí cambia} respecto a los métodos clásicos:
mientras un PID o un compensador discreto típico se implementan con unas pocas operaciones
(una ecuación en diferencias de bajo orden),
los métodos modernos requieren, por muestra,
\textbf{propagación de estados} ($A\hat{x}$),
\textbf{actualización del observador} (términos con $L$ y mediciones),
y en el caso del integrador una \textbf{acumulación adicional}.

En este trabajo, con $n=3$ y frecuencias de muestreo moderadas,
esa carga adicional resultó completamente abordable,
pero la comparación deja una lección general:
en sistemas con estados grandes ($n$ alto),
muestreos muy rápidos o microcontroladores muy limitados,
este incremento puede volverse un \textit{dealbreaker} y empujar a soluciones clásicas
más simples, aunque menos potentes.

La diferencia fundamental entre ambas vertientes no es
de implementación, sino de filosofía:
los métodos clásicos aceptan la incertidumbre y compensan
mediante ajustes progresivos;
los métodos modernos intentan modelar y optimizar explícitamente
la dinámica interna completa.

El aprendizaje central radica en comprender que
ningún método es universalmente superior.
Los enfoques clásicos ofrecen rapidez y tolerancia a la imprecisión.
Los enfoques modernos ofrecen mayor grado de libertad,
mejor desempeño potencial y control explícito sobre
ruido, esfuerzo y error estacionario,
a costa de mayor dependencia del modelo y criterio en su formulación.

En esta planta experimental,
la arquitectura moderna completa mostró una ventaja clara,
pero también evidenció que el éxito del diseño
depende tanto de la calidad del modelo
como del criterio ingenieril aplicado en su utilización.

	\appendices

\section{Parámetros del controlador}

\section{Códigos MATLAB}
% ============================================================
\appendix
\section{Script de LQR con observador y simulacion con saturacion y ruido}
\label{ap:lqr_obs_sim}
% ============================================================

En este apendice se incluye el script MATLAB utilizado para:
(i) cargar y discretizar la planta identificada,
(ii) definir las ponderaciones \(Q\) y \(R\) del LQR,
(iii) calcular la ganancia optima \(K\) (via \texttt{dlqr}/\texttt{dare}),
(iv) disenar observadores (predictor y actual) por ubicacion arbitraria de polos,
(v) calcular el prefiltro \(N_{\mathrm{bar}}\) para seguimiento de referencia,
y (vi) simular el lazo con/sin saturacion y con/sin ruido (incluyendo comparacion double vs \texttt{float32}).

\begin{lstlisting}[language=Matlab,caption={LQR + Observador (predictor/actual) + Nbar + Simulacion con saturacion y ruido},label={lst:lqr_obs_sim}]
	close all; clear; clc
	
	%% =========================
	% 1) CARGA + DISCRETIZACION
	%% =========================
	S = load('planta (1).mat');
	
	if isfield(S,'plantaC')
	plantaC = S.plantaC;
	elseif isfield(S,'sysC')
	plantaC = S.sysC;
	else
	error('No encuentro "plantaC" ni "sysC" dentro de planta (1).mat');
	end
	
	Ts   = 1/100;                 % sample time
	sysD = c2d(plantaC, Ts, 'zoh');
	[A,B,C,D] = ssdata(ss(sysD));
	n = size(A,1);
	
	fprintf('Ts=%.9f | n=%d\n', Ts, n);
	disp('A='); disp(A); disp('B='); disp(B); disp('C='); disp(C); disp('D='); disp(D);
	
	%% =========================
	% 2) PARAMETROS + PESOS (LQR)
	%% =========================
	% Observador (z-plane)
	p_obs  = [0.8 + 0.25i, 0.8 - 0.25i, 0.9];
	
	% Objetivo practico: step ~20 y u limitado a +/-300
	r_step = 20;
	u_max  = 300;
	
	% knobs
	wy = 20;          % subir => mas seguimiento (mas agresivo)
	wu = 500;         % subir => menos esfuerzo (mas timido)
	
	% Q y R coherentes
	Q = wy*(C'*C) + 1e-8*eye(n);
	R = wu/(u_max);
	
	%% =========================
	% 3) GANANCIAS: LQR + OBSERVADOR + Nbar
	%% =========================
	rc = rank(ctrb(A,B));
	if rc < n
	error('El par (A,B) NO es controlable (rank=%d < n=%d).', rc, n);
	end
	
	% --- LQR discreto (con fallback) ---
	if exist('dlqr','file') == 2
	[K, P, e_cl] = dlqr(A, B, Q, R);
	elseif exist('dare','file') == 2
	[P,~,~] = dare(A,B,Q,R);
	K = (R + B'*P*B)\(B'*P*A);
	e_cl = eig(A - B*K);
	else
	[P, K, e_cl, info] = dlqr_iter_nolic(A,B,Q,R);
	fprintf('DLQR sin toolbox: iters=%d, err=%.3e\n', info.iters, info.err);
	end
	
	fprintf('LQR: eig(A-BK) = \n'); disp(e_cl.');
	
	% --- Observador (place) ---
	Ke_pred = place(A', C', p_obs).';            % predictor: A - Ke*C
	Ke_act  = place(A', (C*A)', p_obs).';        % "actual": A - Ke*C*A
	
	% --- Nbar (SISO) ---
	if size(C,1) ~= 1
	error('Tu C no es SISO (tiene %d salidas). Elegi una fila de C.', size(C,1));
	end
	[~,~,Nbar] = refi(A, B, C, K);
	fprintf('Nbar=%.6g\n', Nbar);
	
	%% =========================
	% 4) SIMULACION: SIN RUIDO vs CON RUIDO (double y single)
	%% =========================
	N        = 200;
	ulim_sat = u_max;
	ulim_inf = Inf;
	
	% referencia
	r = zeros(1,N);
	r(2:end) = r_step;
	
	% --------- RUIDO (config) ----------
	rng(1);                % repetible
	
	noise.enable   = true;
	noise.sigma_y  = 2;    % ruido de medicion
	noise.sigma_u  = 2.0;  % jitter actuador
	noise.q_u      = 1.0;  % cuantizacion u
	noise.sigma_w  = 0.0;  % ruido de proceso
	
	noise.vy = noise.sigma_y * randn(1,N);
	noise.vu = noise.sigma_u * randn(1,N);
	noise.wx = noise.sigma_w * randn(n,N);
	
	noise_off = noise;
	noise_off.enable = false;
	noise_off.vy = zeros(1,N);
	noise_off.vu = zeros(1,N);
	noise_off.wx = zeros(n,N);
	
	% --- IDEAL (sin sat, sin ruido) ---
	out_pred_ideal = sim_obs_loop(A,B,C,D,K,Nbar,Ke_pred,Ke_act,r,Ts,ulim_inf,false,false,noise_off);
	out_act_ideal  = sim_obs_loop(A,B,C,D,K,Nbar,Ke_pred,Ke_act,r,Ts,ulim_inf,true ,false,noise_off);
	
	% --- REAL (sat, sin ruido) ---
	out_pred_sat_clean = sim_obs_loop(A,B,C,D,K,Nbar,Ke_pred,Ke_act,r,Ts,ulim_sat,false,false,noise_off);
	out_act_sat_clean  = sim_obs_loop(A,B,C,D,K,Nbar,Ke_pred,Ke_act,r,Ts,ulim_sat,true ,false,noise_off);
	
	% --- REAL (sat, con ruido) ---
	out_pred_sat_noise = sim_obs_loop(A,B,C,D,K,Nbar,Ke_pred,Ke_act,r,Ts,ulim_sat,false,false,noise);
	out_act_sat_noise  = sim_obs_loop(A,B,C,D,K,Nbar,Ke_pred,Ke_act,r,Ts,ulim_sat,true ,false,noise);
	
	% --- REAL (sat, con ruido) en FLOAT32 ---
	out_pred_sat_noise_f = sim_obs_loop(A,B,C,D,K,Nbar,Ke_pred,Ke_act,r,Ts,ulim_sat,false,true,noise);
	out_act_sat_noise_f  = sim_obs_loop(A,B,C,D,K,Nbar,Ke_pred,Ke_act,r,Ts,ulim_sat,true ,true,noise);
	
	t = (0:N-1)*Ts;
	
	%% =========================
	% 5) DIAGNOSTICOS
	%% =========================
	eyp = out_pred_sat_noise.y_meas - out_pred_sat_noise.y_true;
	eya = out_act_sat_noise.y_meas  - out_act_sat_noise.y_true;
	
	fprintf('\n--- CHECK RUIDO ---\n');
	fprintf('sigma_y=%.3g | RMS(y_meas-y_true) Pred=%.3g Act=%.3g\n', noise.sigma_y, rms(eyp), rms(eya));
	fprintf('sigma_u=%.3g | q_u=%g\n', noise.sigma_u, noise.q_u);
	
	fprintf('\n--- SATURACION ---\n');
	fprintf('Pred clean: max|u|=%.2f sat=%d\n', max(abs(out_pred_sat_clean.u)), sum(abs(out_pred_sat_clean.u) >= ulim_sat-1e-9));
	fprintf('Pred noise: max|u|=%.2f sat=%d\n', max(abs(out_pred_sat_noise.u)), sum(abs(out_pred_sat_noise.u) >= ulim_sat-1e-9));
	fprintf('Act  clean: max|u|=%.2f sat=%d\n', max(abs(out_act_sat_clean.u)),  sum(abs(out_act_sat_clean.u)  >= ulim_sat-1e-9));
	fprintf('Act  noise: max|u|=%.2f sat=%d\n', max(abs(out_act_sat_noise.u)),  sum(abs(out_act_sat_noise.u)  >= ulim_sat-1e-9));
	
	%% =========================
	% 6) PLOTS
	%% =========================
	figure('Name','Predictor: y (clean vs noise) [sat]');
	plot(t, out_pred_sat_clean.y_true, 'LineWidth',1.6); hold on;
	plot(t, out_pred_sat_noise.y_meas, '.-');
	plot(t, r, 'k--','LineWidth',1.2);
	grid on; xlabel('t [s]'); ylabel('y');
	legend('y true (clean)','y meas (noise)','r','Location','best');
	
	figure('Name','Predictor: u (clean vs noise) [sat]');
	plot(t, out_pred_sat_clean.u, 'LineWidth',1.6); hold on;
	plot(t, out_pred_sat_noise.u, '.-');
	yline(+ulim_sat,'k--'); yline(-ulim_sat,'k--');
	grid on; xlabel('t [s]'); ylabel('u');
	
	figure('Name','Actual: y (clean vs noise) [sat]');
	plot(t, out_act_sat_clean.y_true, 'LineWidth',1.6); hold on;
	plot(t, out_act_sat_noise.y_meas, '.-');
	plot(t, r, 'k--','LineWidth',1.2);
	grid on; xlabel('t [s]'); ylabel('y');
	
	figure('Name','Actual: u (clean vs noise) [sat]');
	plot(t, out_act_sat_clean.u, 'LineWidth',1.6); hold on;
	plot(t, out_act_sat_noise.u, '.-');
	yline(+ulim_sat,'k--'); yline(-ulim_sat,'k--');
	grid on; xlabel('t [s]'); ylabel('u');
	
	figure('Name','Ruido de medicion (y_meas - y_true)');
	plot(t, eyp, t, eya);
	grid on; xlabel('t [s]'); ylabel('error de medicion');
	legend('Pred','Act','Location','best');
	
	figure('Name','Float32 - Double (y_meas) [sat+noise]');
	plot(t, out_pred_sat_noise.y_meas - out_pred_sat_noise_f.y_meas, ...
	t, out_act_sat_noise.y_meas  - out_act_sat_noise_f.y_meas);
	grid on; xlabel('t [s]'); ylabel('double - float32');
	legend('Pred','Act','Location','best');
	
	%% =========================
	% 7) Z-PLANE (polos planta, CL, obs)
	%% =========================
	p_ol = eig(A);
	p_cl = eig(A - B*K);
	p_op = eig(A - Ke_pred*C);
	p_oa = eig(A - Ke_act*C*A);
	
	try
	z_plant = tzero(ss(A,B,C,D,Ts));
	catch
	z_plant = [];
	end
	
	plot_zplane('Z-plane: Predictor', p_ol, p_cl, p_op, z_plant);
	plot_zplane('Z-plane: Actual'   , p_ol, p_cl, p_oa, z_plant);
	
	%% ============================================================
	% ===================== FUNCIONES LOCALES ======================
	
	function out = sim_obs_loop(A,B,C,D,K,Nbar,Ke_pred,Ke_act,r,Ts,ulim,use_actual,use_single,noise)
	if nargin < 15 || isempty(noise)
	noise.enable=false; noise.vy=0; noise.vu=0; noise.wx=0; noise.q_u=Inf;
	end
	
	if use_single
	A=single(A); B=single(B); C=single(C); D=single(D);
	K=single(K); Nbar=single(Nbar);
	Ke_pred=single(Ke_pred); Ke_act=single(Ke_act);
	r=single(r); Ts=single(Ts); ulim=single(ulim);
	end
	
	n = size(A,1); N = numel(r);
	
	x  = zeros(n,N,'like',A);
	xh = zeros(n,N,'like',A);
	
	y_true = zeros(1,N,'like',A);
	y_meas = zeros(1,N,'like',A);
	
	u_cmd  = zeros(1,N,'like',A);
	u_app  = zeros(1,N,'like',A);
	
	vy = zeros(1,N,'like',A);
	vu = zeros(1,N,'like',A);
	wx = zeros(n,N,'like',A);
	
	if isfield(noise,'enable') && noise.enable
	vy = cast(noise.vy,'like',A);
	vu = cast(noise.vu,'like',A);
	wx = cast(noise.wx,'like',A);
	end
	
	q_u = Inf;
	if isfield(noise,'q_u'), q_u = noise.q_u; end
	q_u = cast(q_u,'like',A);
	
	y_true(1) = C*x(:,1) + D*0;
	y_meas(1) = y_true(1) + vy(1);
	
	for k=1:N-1
	u_unsat  = Nbar*r(k) - K*xh(:,k);
	u_cmd(k) = sat(u_unsat, ulim);
	
	u_app(k) = u_cmd(k) + vu(k);
	if isfinite(double(q_u))
	u_app(k) = round(u_app(k)/q_u)*q_u;
	end
	u_app(k) = sat(u_app(k), ulim);
	
	x(:,k+1) = A*x(:,k) + B*u_app(k) + wx(:,k);
	
	y_true(k)   = C*x(:,k)   + D*u_app(k);
	y_true(k+1) = C*x(:,k+1) + D*u_app(k);
	
	y_meas(k)   = y_true(k)   + vy(k);
	y_meas(k+1) = y_true(k+1) + vy(k+1);
	
	if ~use_actual
	yhat_k = C*xh(:,k) + D*u_app(k);
	xh(:,k+1) = A*xh(:,k) + B*u_app(k) + Ke_pred*( y_meas(k) - yhat_k );
	else
	z   = A*xh(:,k) + B*u_app(k);
	yzh = C*z + D*u_app(k);
	xh(:,k+1) = z + Ke_act*( y_meas(k+1) - yzh );
	end
	end
	
	u_cmd(N) = sat(Nbar*r(N) - K*xh(:,N), ulim);
	u_app(N) = u_cmd(N) + vu(N);
	if isfinite(double(q_u))
	u_app(N) = round(u_app(N)/q_u)*q_u;
	end
	u_app(N) = sat(u_app(N), ulim);
	
	y_true(N) = C*x(:,N) + D*u_app(N);
	y_meas(N) = y_true(N) + vy(N);
	
	out.x = x; out.xh = xh;
	out.y_true = y_true;
	out.y_meas = y_meas;
	out.u_cmd  = u_cmd;
	out.u_app  = u_app;
	
	out.y = y_meas;
	out.u = u_app;
	end
	
	function y = sat(u,lim)
	if isinf(lim)
	y = u;
	else
	y = min(max(u, -lim), lim);
	end
	end
	
	function plot_zplane(figName, p_ol, p_cl, p_obs, z_plant)
	figure('Name',figName,'NumberTitle','off');
	hold on; grid on; grid minor; axis equal;
	title('Z-plane');
	xlabel('Re{z}'); ylabel('Im{z}');
	
	th = linspace(0,2*pi,400);
	plot(cos(th), sin(th), 'k:'); % unit circle
	
	plot(real(p_ol),  imag(p_ol),  'o', 'LineWidth', 1.5);
	plot(real(p_cl),  imag(p_cl),  'x', 'LineWidth', 1.8);
	plot(real(p_obs), imag(p_obs), '^', 'LineWidth', 1.8);
	
	if ~isempty(z_plant)
	plot(real(z_plant), imag(z_plant), 's', 'LineWidth', 1.5);
	legend('unit circle','poles plant (A)','poles CL (A-BK)','poles obs','zeros plant','Location','bestoutside');
	else
	legend('unit circle','poles plant (A)','poles CL (A-BK)','poles obs','Location','bestoutside');
	end
	
	xlim([-1.2 1.2]); ylim([-1.2 1.2]);
	end
	
	function [Nx,Nu,Nbar] = refi(phi,gam,Hr,K)
	I=eye(size(phi));
	[m,n]=size(Hr);
	np=inv([phi-I gam;Hr zeros(m)])*([zeros(n,m);eye(m)]);
	Nx=np(1:n,:);
	Nu=np(n+1:n+m,:);
	Nbar=Nu+K*Nx;
	end
	
	function [P, K, e_cl, info] = dlqr_iter_nolic(A,B,Q,R)
	maxit = 5000;
	tol   = 1e-10;
	
	P = Q;
	err = Inf;
	
	for it = 1:maxit
	G = R + B'*P*B;
	Ktmp = G \ (B'*P*A);
	Pn = A'*P*A - A'*P*B*Ktmp + Q;
	
	err = norm(Pn - P, 'fro');
	P = Pn;
	
	if err < tol
	break;
	end
	end
	
	K = (R + B'*P*B) \ (B'*P*A);
	e_cl = eig(A - B*K);
	
	info.iters = it;
	info.err   = err;
	info.converged = (err < tol);
	
	if ~info.converged
	warning('Riccati iterativa NO convergio (err=%.3e).', err);
	end
	end
\end{lstlisting}

\input{estimarRuido.tex}
% ============================================================
\section{Apéndice: Cómputo de ganancias LQG/LQI y simulación (firmware friendly)}
\label{ap:lqg_ganancias}
% ============================================================

En este apéndice se incluye el script utilizado para obtener los parámetros
numéricos del esquema LQG/LQI empleado en el trabajo: ganancia del filtro de Kalman
en régimen permanente \(L\), ganancias de realimentación de estados \(K_x\) y del
integrador \(K_i\), junto con la verificación de polos del observador
\(\lambda(A-LC)\) y la simulación en presencia de ruido de proceso y medición.

El script fue construido con el objetivo de ser \textit{firmware friendly}, es decir,
replicar la lógica de estimación y control que luego se implementa en el microcontrolador:
\begin{itemize}
	\item \textbf{Estimador tipo \textit{current}}: se calcula \(L\) con \texttt{kalman(...,'current')}.
	\item \textbf{Ruido de proceso y medición}: se inyecta \(w_k\) en la dinámica del estado y \(v_k\) en la medición.
	\item \textbf{Control con integrador}: se emplea una acción integral \(\xi_k\) para asegurar error estacionario nulo.
\end{itemize}

\subsection{Estructura del diseño}

\paragraph{Planta discretizada}
A partir del modelo continuo identificado (\texttt{planta (1).mat}) se discretiza con ZOH a
\(\Ts = 0.01\ \text{s}\) (100 Hz), obteniendo el sistema discreto:

\[
x_{k+1} = A x_k + B u_k,
\qquad
y_k = C x_k + D u_k
\]

\paragraph{Covarianzas de ruido}
Las covarianzas utilizadas por el estimador se toman del archivo
\texttt{RQ\_tuning\_fixedR.mat} (Apéndice~\ref{ap:kalman_tuning}):

\[
Q_n = Q_{\text{best}},
\qquad
R_n = R
\]

Adicionalmente, para la simulación se utilizan las desviaciones estándar:
\(\sigma_v\) (medición) y \(\sigma_w\) (proceso), e inyección de ruido blanco gaussiano.

\paragraph{Filtro de Kalman (\textit{current estimator})}
Se construye un sistema extendido para el cálculo de Kalman, incorporando explícitamente
el canal de ruido de proceso \(w\):

\[
x_{k+1} = A x_k + B u_k + G w_k,
\qquad
y_k = C x_k + D u_k + H w_k + v_k
\]

En este trabajo se asumió:

\[
G = I_n,
\qquad
H = 0
\]

y se obtiene la ganancia \(L\) en régimen permanente (y sus polos):

\[
L = L_k,
\qquad
\lambda(A - LCA)
\]

\paragraph{Control LQI con integrador}
Se incorpora un integrador escalar \(\xi_k\) sobre el error:

\[
\xi_{k+1} = \xi_k + (r_k - y_k)
\]

y se adopta la ley de control:

\[
u_k = -K_x \hat{x}_k + K_i \xi_k
\]

Las ganancias se obtienen mediante \texttt{dlqr} aplicado al sistema aumentado, usando
pesos heurísticos basados en escalas prácticas:
paso deseado \(\Delta y \approx 20\) y esfuerzo máximo \(|u|\lesssim 300\).

\subsection{Script completo (LQG\_servo\_firmware\_friendly.m)}

\noindent A continuación se incluye el script completo utilizado para calcular
\(L\), \(K_x\), \(K_i\), reportar polos y simular en presencia de ruido.

\begin{lstlisting}[style=matlabstyle]
	%% =========================
	%  LQG SERVO “firmware friendly”
	%  (kalman + lqi + lqgtrack)
	%  + estimador CURRENT (default en discreto)
	%  + ruido de proceso y medición (sigma_w, sigma_v)
	%% =========================
	close all; clear; clc
	
	%% =========================
	% 1) CARGA + DISCRETIZACIÓN
	%% =========================
	S = load('planta (1).mat');
	
	if isfield(S,'plantaC')
	plantaC = S.plantaC;
	elseif isfield(S,'sysC')
	plantaC = S.sysC;
	else
	error('No encuentro "plantaC" ni "sysC" dentro de planta (1).mat');
	end
	
	Ts   = 1/100;                 % 100 Hz
	sysD = c2d(plantaC, Ts, 'zoh');
	[A,B,C,D] = ssdata(ss(sysD));
	
	n = size(A,1);
	if size(C,1) ~= 1
	C = C(1,:);
	D = D(1,:);
	end
	if size(B,2) ~= 1
	error('Este script asume SISO (1 entrada). size(B,2)=%d', size(B,2));
	end
	
	fprintf('Ts=%.9f | n=%d\n', Ts, n);
	
	%% =========================
	% 2) RUIDOS (sigma -> covarianzas)
	%% =========================
	S = load("RQ_tuning_fixedR.mat");
	
	Qn = S.Q_best;     % cov(w) : nxn
	Rn = S.R;          % cov(v) : 1x1
	sigma_v = S.sigma_v;
	sigma_w = S.best_q;
	Nn = zeros(n,1);   % cov(wv') asumimos 0
	
	%% =========================
	% 3) KALMAN: sys debe incluir el canal de ruido w
	%% =========================
	G = eye(n);        % w entra a todos los estados
	H = zeros(1,n);    % y NO depende directamente de w (H=0)
	
	sysK = ss(A, [B G], C, [D H], Ts);   % inputs: [u ; w]
	
	% Para tu MATLAB: TYPE va como ÚLTIMO argumento.
	% 'current' es el default, pero lo dejamos explícito.
	[kest, Lk, Pk, Mx, Z, My] = kalman(sysK, Qn, Rn, Nn, 'current');
	
	fprintf('\nLk (kalman) = '); disp(Lk);
	fprintf('poles(A-LC) = '); disp(eig(A - Lk*C).');
	
	%% =========================
	% 4) LQI: K = [Kx Ki] para u = -Kx*xhat - Ki*xi
	%% =========================
	m = 1;                      % integrador escalar
	Ahat = [A B; zeros(m,n+m)];
	Bhat = [zeros(n,m); eye(m)];
	Chat = [C zeros(1,m)];
	
	% --- interpretación física: "step ~20" y "u no pase ~300" ---
	y_step = 20;
	u_max  = 300;
	
	% knobs
	wy = 20;                    % subí => seguís más (más agresivo)
	wu = 50;                    % subí => penalizás más u (más tímido)
	wv = 0.00000000000001;      % subí => penalizás integrador
	
	% OJO: acá tu “escala física” es discutible. Esto es heurístico.
	qy = wy*(1/(y_step));
	ru = wu*(1/(u_max));
	qv = wv*(1/(y_step));
	
	Qa = blkdiag(qy*(C'*C) + 1e-8*eye(n), qv);
	R  = ru;
	
	[Khat,Pa,ecl] = dlqr(Ahat, Bhat, Qa, R);
	
	Aux  = [A-eye(size(A))  B;
	C*A             C*B];
	
	K2K1 = (Khat + [zeros(1,n) eye(m)]) / Aux;
	
	Kx = K2K1(1,1:n);
	Ki = K2K1(1,n+1);
	
	fprintf('\nKx (lqi) = '); disp(Kx);
	fprintf('Ki (lqi) = '); disp(Ki);
	
	%% =========================
	% 6) SIMULACIÓN (tu estilo) con ruido de proceso+medición
	%% =========================
	N  = 1200;
	t  = 0:Ts:(N-1)*Ts;
	
	r = ones(1,N)*25;
	r(1:30) = 0;
	
	umin = -u_max; umax = u_max;
	
	X  = zeros(n,N);   % estados reales
	Xh = zeros(n,N);   % estados estimados (los que usás en tu control)
	V  = zeros(1,N);   % integrador
	U  = zeros(1,N);   % control
	
	Y_true = zeros(1,N);
	Y_meas = zeros(1,N);
	
	for kidx = 1:N-1
	rk = r(kidx);
	
	% medición y[k]
	y_true_k = C*X(:,kidx);
	v_k      = sigma_v*randn;
	y_meas_k = y_true_k + v_k;
	
	Y_true(kidx) = y_true_k;
	Y_meas(kidx) = y_meas_k;
	
	% integrador xi[k+1] = xi[k] + (r-y)
	V(kidx+1) = V(kidx) + (rk - y_meas_k);
	
	% control: u = -Kx*xhat - Ki*xi
	u_unsat = -Kx*Xh(:,kidx) + Ki*V(kidx+1);
	u_k     = min(max(u_unsat, umin), umax);
	U(kidx) = u_k;
	
	% planta con ruido de proceso: x[k+1] = A x + B u + w
	w_k = sigma_w*randn(n,1);
	X(:,kidx+1) = A*X(:,kidx) + B*u_k + w_k;
	
	% estimador (usa Lk)
	Xh(:,kidx+1) = A*Xh(:,kidx) + B*u_k + ...
	Lk*(y_meas_k - (C*Xh(:,kidx) + D*u_k));
	end
	
	% último sample para plots
	Y_true(N) = C*X(:,N);
	Y_meas(N) = Y_true(N) + sigma_v*randn;
	U(N)      = U(N-1);
	
	%% =========================
	% 7) PLOTS
	%% =========================
	figure('Name','y(t) — LQG + integrador + ruido');
	plot(t, Y_true, 'LineWidth',1.2); hold on;
	plot(t, Y_meas, '.', 'LineWidth',1.0);
	plot(t, r, 'k--','LineWidth',1.1);
	grid on; grid minor;
	xlabel('t [s]'); ylabel('y');
	legend('y true','y meas','r','Location','best');
	
	figure('Name','u(t) — saturado');
	plot(t, U, 'LineWidth',1.4);
	grid on; grid minor;
	xlabel('t [s]'); ylabel('u');
	ylim([umin umax]);
	
	figure('Name','xi(t) — integrador');
	plot(t, V, 'LineWidth',1.4);
	grid on; grid minor;
	xlabel('t [s]'); ylabel('\xi');
	
	Aaug = [A,B;...
	Kx-Kx*A-Ki*C*A,1-Kx*B-Ki*C*B];
	Baug = [0;0;0;Ki];
	Caug = [C,0];
	D = 0;
	sysDaug = ss(Aaug,Baug,Caug,D,Ts);
	
	figure; pzmap(sysDaug); grid on; zgrid;
\end{lstlisting}



\section{Códigos PSoC}
 
\section{Evolución del diseño estructural de la planta}
\label{app:estructura}

Durante el desarrollo del trabajo práctico, la estructura física de la planta atravesó distintas etapas de diseño, las cuales permitieron identificar limitaciones mecánicas y realizar mejoras progresivas hasta alcanzar la configuración final utilizada en las prácticas experimentales. En este apéndice se describe la primera etapa de diseño de la estructura y se destacan las principales diferencias respecto de la versión final.

\subsection{Primera etapa de diseño}

La primera versión de la estructura fue concebida con una altura total aproximada de \(80\,\text{cm}\), utilizando la misma base y el mismo techo de madera que se mantienen en el diseño final. Debido a las dimensiones de estos elementos, la altura útil de movimiento del cuerpo móvil en esta etapa era de aproximadamente \(72\,\text{cm}\).

En esta configuración inicial, el diseño mecánico del cuerpo móvil era diferente al actual, presentando dimensiones ligeramente mayores. El sistema no contaba con elementos de seguridad adicionales, tales como topes mecánicos, amortiguación ante caídas ni cuerda de seguridad, dado que el recorrido vertical era considerablemente menor y el riesgo asociado a caídas desde grandes alturas resultaba limitado.

El guiado del cuerpo móvil se realizaba mediante vigas metálicas rectas y rígidas, las cuales no presentaban deformaciones apreciables. Debido a esta rigidez estructural, no fue necesario incorporar articulaciones pasivas tipo ``muñeca'' en las abrazaderas, ni estructuras auxiliares de madera para limitar deformaciones. En esta etapa, el contacto entre el cuerpo móvil y los rieles generaba fricción apreciable, la cual se manifestaba de forma consistente durante el movimiento vertical.

Cabe destacar que este comportamiento friccional, observable en la primera versión de la estructura, no se presenta de la misma manera en el diseño final. La incorporación de vigas metálicas de mayor longitud, junto con las deformaciones inherentes a las mismas y la inclusión de articulaciones pasivas en las abrazaderas, redujo significativamente la fricción directa entre el cuerpo móvil y los rieles, modificando así las características mecánicas del sistema.

En las figuras siguientes se presentan imágenes correspondientes a las primeras versiones de las piezas impresas en 3D utilizadas en esta etapa inicial del diseño, las cuales difieren de las empleadas en la configuración final de la planta.

\insertarfigura{img/Disenos/cuerpo1.png}{Primer diseño del soporte superior del motor.}{fig:cuerpo1}{0.5}

En la primera etapa de diseño, el soporte superior del cuerpo móvil presentaba una altura aproximada de \(3\,\text{cm}\) y un diámetro de \(4\,\text{cm}\). Las secciones sobresalientes destinadas al acople de los brazos contaban con una altura de aproximadamente \(1{,}5\,\text{cm}\). Dicho soporte incluía orificios dimensionados específicamente para el montaje del motor brushless, con un diámetro de \(3\,\text{mm}\), acorde al patrón de fijación del mismo.

\insertarfigura{img/Disenos/brazo_y_base_1.png}{Primeros diseños del soporte inferior y de los brazos estructurales.}{fig:brazosybase1}{1}

El soporte inferior del cuerpo móvil correspondía a una geometría espejo del soporte superior. En este componente se realizaba el encastre de los brazos estructurales, los cuales, en esta etapa inicial, presentaban dimensiones menores en comparación con el diseño final. Cada brazo tenía dimensiones aproximadas de \(1\,\text{cm} \times 1\,\text{cm} \times 20\,\text{cm}\).

El sistema de agarre de los brazos difería del implementado en la versión final de la planta. Inicialmente, el agarre no contaba con movilidad angular, aunque permitía un ajuste manual respecto a la posición de la viga metálica, lo que condicionaba el guiado del cuerpo móvil y su interacción con los rieles.

\subsection{Segunda etapa de diseño}

En la segunda etapa de diseño, el componente que presentó mayores modificaciones fue el brazo estructural. A partir de la experiencia obtenida en la etapa inicial, se introdujeron variaciones geométricas en el diseño del brazo, incorporando curvaturas con el objetivo de mejorar el encastre y la interacción con la estructura de guiado.

En esta versión, el brazo y el sistema de agarre fueron integrados en una única pieza impresa en 3D, eliminando la separación entre ambos componentes. Dado que en esta etapa las vigas metálicas utilizadas como rieles presentaban una geometría recta y una rigidez suficiente, no fue necesaria la incorporación de articulaciones pasivas tipo ``muñeca''. En consecuencia, el guiado del cuerpo móvil se realizaba mediante un agarre rígido, sin movilidad angular.

Cabe destacar que, durante esta etapa, el diseño del cuerpo móvil se mantuvo sin modificaciones significativas respecto a la versión anterior. Las mejoras se concentraron exclusivamente en el diseño de los brazos y del sistema de agarre, manteniendo constante la geometría general del conjunto móvil.


\insertarfigura{img/Disenos/brazo2.png}{Segundo diseño de brazo.}{fig:brazo2}{1}

\subsection{Tercera etapa de diseño}

En la tercera etapa de diseño se introdujo una modificación significativa en la estructura general de la planta, extendiendo su altura máxima hasta aproximadamente \(160\,\text{cm}\). Esta ampliación respondió a la necesidad de disponer de un mayor recorrido vertical para la realización de las prácticas de control, lo cual implicó nuevas exigencias mecánicas sobre el conjunto estructural y el cuerpo móvil.

Como consecuencia del aumento de altura de la estructura, el diseño de los brazos del cuerpo móvil volvió a ser modificado. En esta etapa, los brazos fueron rediseñados con mayor grosor y mayor altura, con el objetivo de incrementar su rigidez y capacidad de carga. Asimismo, se incorporaron aberturas longitudinales en los brazos, destinadas a permitir la inserción de elementos metálicos, con el fin de reforzar la estructura y mejorar su resistencia mecánica frente a esfuerzos y vibraciones.

El cuerpo móvil mantuvo su configuración general respecto a las etapas anteriores; sin embargo, el rediseño de los brazos resultó fundamental para adaptar el conjunto a las nuevas condiciones estructurales impuestas por la mayor altura de la planta.

\insertarfigura{img/Disenos/cuerpo3.jpeg}{Tercer diseño de brazos.}{fig:cuerpo3}{1}

\subsection{Cuarta etapa de diseño: configuración final}

La cuarta etapa de diseño corresponde a la configuración final de la estructura y del cuerpo móvil utilizada en las prácticas experimentales del trabajo. En esta etapa se introdujeron modificaciones orientadas principalmente a mejorar el guiado mecánico del cuerpo móvil y a reducir la masa total del conjunto.

Debido a que las vigas metálicas empleadas como rieles presentan deformaciones asociadas a su longitud, se incorporaron articulaciones pasivas tipo ``muñeca'' en el sistema de guiado. Estas articulaciones permiten un movimiento angular relativo entre el cuerpo móvil y los rieles, mejorando el desplazamiento vertical y evitando atascamientos o esfuerzos indeseados durante el recorrido.

Con el objetivo de reducir la masa del cuerpo móvil, se redimensionaron los soportes principales. El soporte superior, que en versiones anteriores presentaba una altura de \(3\,\text{cm}\), fue reducido a aproximadamente \(1{,}5\,\text{cm}\), mientras que el soporte inferior pasó de \(2\,\text{cm}\) a \(0{,}5\,\text{cm}\). A pesar de esta reducción dimensional, se conservó el sistema de encastre tanto en el soporte superior como en el inferior, asegurando la rigidez estructural del conjunto.

Adicionalmente, se incorporó un soporte específico para la batería, integrado al cuerpo móvil. En dicho soporte se colocaron almohadillas internas con el fin de proteger la batería frente a vibraciones e impactos durante el funcionamiento del sistema.

Estas modificaciones permitieron obtener un diseño final más liviano, adaptable a las deformaciones estructurales de los rieles y adecuado para la implementación de las distintas estrategias de control desarrolladas en el presente trabajo.

\insertarfigura{img/Disenos/cuerpo3.jpeg}{Tercer diseño de brazos.}{fig:cuerpo3}{1}

\section{Incidentes experimentales y fallas en los controladores ESC}
\label{app:esc}

Durante el desarrollo experimental del trabajo se presentaron fallas en los controladores electrónicos de velocidad (ESC) utilizados en las primeras etapas de prueba del sistema. En este apéndice se describen los incidentes observados, junto con el análisis de las posibles causas y las medidas adoptadas posteriormente.

\subsection{Primer incidente: ESC de 30\,A con alimentación externa}

En una primera instancia, se utilizó un ESC de \(30\,\text{A}\) alimentado mediante una batería para automóviles, con el objetivo de verificar el funcionamiento básico del sistema de propulsión. La conexión entre la fuente de alimentación y el ESC se realizó utilizando un cable unifilar de cobre de considerable longitud.

Durante las pruebas iniciales, el sistema logró generar empuje y el cuerpo móvil llegó a elevarse. Sin embargo, tras un período de funcionamiento, el ESC comenzó a emitir una secuencia de señales acústicas consistente en cuatro pitidos cortos seguidos de un pitido largo. En ese momento no se contaba con una interpretación clara del significado de dicha señalización.

Posteriormente, mediante la consulta de documentación y experiencias previas, se determinó que dicha secuencia de pitidos está asociada a condiciones de protección del ESC, tales como sobrecorriente o sobretemperatura. Esta hipótesis se vio reforzada por el hecho de que los cables unifilares utilizados para la alimentación se calentaron excesivamente y llegaron a derretirse, indicando una circulación de corriente elevada y pérdidas resistivas significativas.


\subsection{Segundo incidente: reinicios y falla del ESC de 30\,A}

En una segunda etapa de pruebas con el mismo ESC de \(30\,\text{A}\), se reemplazaron los cables de alimentación por conductores adecuados para altas corrientes, conectando el ESC directamente a la batería utilizada en la planta. En esta configuración, el sistema no lograba elevarse de forma sostenida y el ESC emitía una secuencia de sonidos correspondiente a un reinicio del controlador.

Con el fin de descartar un problema en la señal de control, se analizó la señal PWM generada por el PSoC mediante un osciloscopio, verificándose que la misma presentaba una forma adecuada y estable, sin perturbaciones significativas. En consecuencia, se descartó que la falla estuviera asociada a errores en la generación de la señal de control.

Ante la hipótesis de una posible caída de tensión en la alimentación del ESC durante los transitorios de corriente, se incorporaron capacitores de desacople en la línea de alimentación. Tras esta modificación, el sistema logró generar empuje y elevarse durante breves instantes. No obstante, luego de un corto período de funcionamiento, se produjo la falla definitiva del ESC, observándose la quema de un MOSFET correspondiente a una de las fases del motor.

\subsection{Análisis y consideraciones}

A partir de los incidentes descritos, se identificaron como causas probables la combinación de sobrecorriente, exigencias térmicas elevadas y condiciones de alimentación no ideales durante las primeras pruebas. La utilización de una fuente de alimentación inadecuada, conductores con alta resistencia y la ausencia inicial de medidas de protección contribuyeron a someter al ESC a esfuerzos superiores a sus límites operativos.

Estos eventos pusieron de manifiesto la importancia de considerar cuidadosamente los aspectos de potencia, disipación térmica y protección eléctrica en sistemas de propulsión basados en motores brushless, incluso en etapas preliminares de prueba.

Las lecciones aprendidas a partir de estas fallas motivaron la adopción de controladores de mayor capacidad de corriente, mejoras en el cableado de alimentación y la implementación de estrategias de operación más conservadoras, las cuales permitieron continuar con el desarrollo experimental del trabajo de manera segura y confiable. 
\insertarfigura{img/Planta/ESC30A1.jpeg}{ESC30A incidente 1.}{fig:inc30A1}{1}

\subsection{Tercer incidente: falla del ESC de 30\,A durante operación con batería LiPo}

En un tercer incidente, se utilizó un ESC de \(30\,\text{A}\) alimentado mediante una batería LiPo para drones. Durante esta prueba, se incrementó la señal PWM hasta aproximadamente \(1500\,\mu s\), logrando que el sistema generara empuje suficiente para elevar el cuerpo móvil hasta la parte superior de la estructura.

Al intentar detener el movimiento, se adoptó un procedimiento no óptimo, consistente en bloquear mecánicamente la hélice con el fin de evitar una colisión con el techo de la estructura. Esta acción provocó el trabado de la hélice durante el funcionamiento del motor, lo cual generó un incremento abrupto de la corriente demandada. Como consecuencia, el ESC sufrió una falla catastrófica, produciéndose la quema de múltiples componentes internos y la pérdida total del controlador.

Este incidente permitió identificar el riesgo asociado al bloqueo mecánico del rotor en sistemas de propulsión brushless, dado que dicha condición conduce a corrientes elevadas que superan rápidamente la capacidad de los dispositivos de conmutación del ESC.
\insertarfigura{img/Planta/ESC30A2.jpeg}{ESC30A incidente 2.}{fig:inc30A2}{1}


\subsection{Cuarto incidente: falla del ESC de 30\,A por sobrecorriente}

En un cuarto incidente, se realizaron pruebas controladas con un nuevo ESC de \(30\,\text{A}\), con el objetivo de determinar el valor máximo de PWM que el sistema podía soportar de manera segura. Durante esta prueba, el valor de PWM se incrementó progresivamente hasta alcanzar aproximadamente \(1600\,\mu s\).

En estas condiciones, el ESC volvió a presentar una falla similar a la observada en el segundo incidente, registrándose la quema de un MOSFET correspondiente a una de las fases del motor. Este comportamiento reforzó la hipótesis de que el controlador se encontraba operando cerca de sus límites de corriente, incluso sin que se produjera un bloqueo mecánico del rotor.

\insertarfigura{img/Planta/ESC30A3.jpeg}{ESC30A incidente 3.}{fig:inc30A3}{1}


\subsection{Medidas adoptadas}

La repetición de fallas en controladores de \(30\,\text{A}\), tanto bajo condiciones transitorias como en operación sostenida, llevó a concluir que dicho margen de corriente resultaba insuficiente para el motor utilizado y las exigencias mecánicas de la planta. Asimismo, se consideró la posible influencia de algoritmos internos del ESC y de su calidad de construcción, los cuales podrían limitar su capacidad de manejo de sobrecorrientes.

En función de estas observaciones, se decidió sobredimensionar el sistema de actuación mediante la adquisición de un ESC de \(40\,\text{A}\). Esta decisión permitió operar el motor con un mayor margen de seguridad, evitando la necesidad de reducir aún más la masa del cuerpo móvil y mejorando la confiabilidad del sistema durante las prácticas experimentales.



	\bibliographystyle{IEEEtran}
	\bibliography{refs}
	
\end{document}