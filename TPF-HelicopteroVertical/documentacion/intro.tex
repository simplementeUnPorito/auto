\section{Introducción}

En el presente trabajo práctico se estudia el control de altura de una planta experimental basada en un sistema de propulsión vertical mediante un motor brushless. La planta consiste en un cuerpo móvil que se desplaza exclusivamente en el eje vertical, cuya altura es regulada a partir del empuje generado por una hélice controlada electrónicamente. Este tipo de sistema presenta características dinámicas no lineales, saturaciones físicas y perturbaciones externas, lo que lo convierte en un banco de pruebas adecuado para el análisis y diseño de estrategias de control avanzadas.

El objetivo principal del trabajo es analizar, diseñar e implementar distintas técnicas de control vistas a lo largo del semestre en la materia \textit{Automatizaciones}, aplicándolas a una misma planta real. A partir del modelado físico y matemático del sistema, se evalúan múltiples enfoques de control, permitiendo comparar su desempeño tanto desde el punto de vista teórico como experimental.

En particular, se desarrollan y analizan controladores clásicos, incluyendo el controlador PID y métodos de diseño basados en el lugar de las raíces, diagramas de Bode, síntesis directa y ubicación arbitraria de polos. Asimismo, se aborda el control en el espacio de estados, incorporando control integral para sistemas de seguimiento, y se estudia la utilización de estimadores de estado tanto de tipo predictivo como actual.

Como parte del enfoque moderno de control, el trabajo incluye el diseño y evaluación de controladores óptimos, incorporando integrador y estimadores basados en el filtro de Kalman, culminando en la implementación de un regulador LQG (Linear Quadratic Gaussian). De este modo, se exploran distintas filosofías de control aplicadas a un mismo sistema, lo que permite comparar robustez, desempeño dinámico, sensibilidad a perturbaciones y complejidad de implementación.

El alcance del trabajo se centra en la regulación y seguimiento de la altura del sistema alrededor de puntos de operación definidos, haciendo énfasis en la validación experimental de los controladores diseñados. No se busca optimizar exhaustivamente el sistema desde el punto de vista energético o estructural, sino utilizar la planta como un medio para consolidar los conceptos teóricos de control automático y automatización vistos durante el curso.
