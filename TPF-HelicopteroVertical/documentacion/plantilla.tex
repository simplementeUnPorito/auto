% =========================
% PLANTILLA: Parámetros de controladores (para completar luego)
% =========================

% Recomendado en el preámbulo:
% \usepackage{xcolor}
% \usepackage{amsmath}   % para \text{...} (y mil cosas más)
% \usepackage{booktabs}
% \usepackage{siunitx}   % si querés \si{\degree}

% --- Helper para placeholders ---
% Texto normal (fuera de ecuaciones)
\providecommand{\TODO}[1]{\textcolor{red}{\textbf{TODO:} #1}}

% Math-safe: SIEMPRE devuelve texto “normal” dentro de math.
% (usa \mathrm{} en vez de \text{} para no depender de amsmath)
\providecommand{\TODOm}[1]{\mathrm{\TODO{#1}}}

% --- Si querés una tabla compacta ---
\newcommand{\coef}[1]{\ensuremath{\mathbf{#1}}}

\subsection{Parámetros de los controladores implementados}
\label{sec:param_ctrl}

En esta sección se documentan los parámetros finales utilizados en la práctica experimental para cada estrategia de control.
Todos los valores fueron ajustados considerando restricciones físicas del actuador (PWM) y criterios de seguridad (evitar saturación prolongada y esfuerzos excesivos).

\subsubsection{Convenciones de implementación}
\label{sec:conv_impl}

\begin{itemize}
	\item Señal aplicada al ESC (PWM tipo servo): \(u_{\text{PWM}} \in [1000,2000]\ \mu s\).
	\item Período de muestreo del control: \(T_s=\TODOm{indicar}\,\text{s}\) (o \(f_s=\TODOm{indicar}\,\text{Hz}\)).
	\item Si se utiliza control incremental: \(u_{\text{PWM}}[k]=\mathrm{sat}\!\left(u_0+\Delta u[k],\,1000,\,2000\right)\).
	\item Si se utiliza control absoluto: \(u_{\text{PWM}}[k]=\mathrm{sat}\!\left(u[k],\,1000,\,2000\right)\).
\end{itemize}

% =========================================================
% 1) BODE (Respuesta en frecuencia)
% =========================================================
\subsubsection{Controlador por respuesta en frecuencia (Bode)}
\label{sec:param_bode}

\paragraph{Estructura del controlador}
\begin{itemize}
	\item Tipo: \TODO{Proporcional / Lead / Lag / Lead--Lag}
	\item Forma general:
	\[
	C_{\text{Bode}}(z)=\TODOm{pegar expresión de \(C(z)\)}
	\]
\end{itemize}

\paragraph{Parámetros finales (a completar)}
\begin{itemize}
	\item Ganancia: \(K=\TODOm{valor}\)
	\item (Si aplica) Cero: \(z_0=\TODOm{valor}\)
	\item (Si aplica) Polo: \(p_0=\TODOm{valor}\)
	\item (Si aplica) Otros parámetros: \TODO{...}
\end{itemize}

\paragraph{Márgenes de estabilidad medidos en \texttt{controlSystemDesigner}}
\begin{itemize}
	\item \textbf{Antes de compensar:} \(M_g=\TODOm{dB}\), \(\phi_m=\TODOm{grados}\).
	\item \textbf{Después de compensar:} \(M_g=\TODOm{dB}\), \(\phi_m=\TODOm{grados}\).
\end{itemize}

% Si querés usar grados “lindos” con siunitx, activá esto y comentá las dos líneas de arriba:
% \item \textbf{Antes de compensar:} \(M_g=\TODOm{dB}\), \(\phi_m=\TODOm{}\si{\degree}\).
% \item \textbf{Después de compensar:} \(M_g=\TODOm{dB}\), \(\phi_m=\TODOm{}\si{\degree}\).

\paragraph{Ecuación en diferencias (si se implementa como filtro digital)}
Si el controlador se implementó como filtro IIR/FIR, indicar su forma recursiva:
\[
u[k]=\TODOm{b_0 e[k]+b_1 e[k-1]+\dots -a_1 u[k-1]-\dots}
\]

\paragraph{Notas de implementación}
\begin{itemize}
	\item Saturación: \(u_{\text{PWM}}=\mathrm{sat}(\cdot)\).
	\item \TODO{limitación de esfuerzo / filtro anti-ruido / etc.}
\end{itemize}

% =========================================================
% 2) LUGAR DE RAÍCES (Lag-Lead)
% =========================================================
\subsubsection{Controlador por lugar de raíces (Lag--Lead)}
\label{sec:param_rlocus}

\paragraph{Estructura del controlador}
Se diseñó un compensador Lag--Lead para estabilizar la planta e imponer un desempeño deseado en el plano-\(z\):
\[
C_{\text{RL}}(z)=K\,
\underbrace{\frac{(z-z_{L})(z-z_{D})}{(z-p_{L})(z-p_{D})}}_{\text{Lag--Lead}}
\]
donde \(z_{L},p_{L}\) corresponden a la red Lag y \(z_{D},p_{D}\) a la red Lead.

\paragraph{Parámetros finales (a completar)}
\begin{itemize}
	\item Ganancia: \(K=\TODOm{valor}\)
	\item Red Lag: \(z_{L}=\TODOm{valor}\), \(p_{L}=\TODOm{valor}\)
	\item Red Lead: \(z_{D}=\TODOm{valor}\), \(p_{D}=\TODOm{valor}\)
\end{itemize}

\paragraph{Polos dominantes objetivo}
\begin{itemize}
	\item Polos deseados: \(z_{d,1,2}=\TODOm{valor}\) (criterio: \TODO{\(t_s\), \(M_p\), \(\zeta\), etc.})
\end{itemize}

\paragraph{Método de ajuste}
\begin{itemize}
	\item Se utilizó: \TODO{\texttt{rlocus} manual / \texttt{Optimization Based Tuning}}
	\item Función objetivo (si aplica): \TODO{por ejemplo, minimizar \(J=\dots\)}
\end{itemize}

\paragraph{Ecuación en diferencias (implementación)}
\[
u[k]=\TODOm{expresión recursiva de \(C_{\text{RL}}(z)\)}
\]

\paragraph{Notas de implementación}
\begin{itemize}
	\item \TODO{saturación, anti-windup si hay integrador, filtro si hay derivada, etc.}
\end{itemize}

% =========================================================
% 3) PID (Åström)
% =========================================================
\subsubsection{Controlador PID (formulación de Åström)}
\label{sec:param_pid}

\paragraph{Estructura y parámetros}
\begin{itemize}
	\item Ganancia proporcional: \(K=\TODOm{valor}\)
	\item Peso de referencia proporcional: \(b=\TODOm{valor}\)
	\item Tiempo integral: \(T_i=\TODOm{valor}\)
	\item Tiempo derivativo: \(T_d=\TODOm{valor}\)
	\item Filtro derivativo: \(N=\TODOm{valor}\)
\end{itemize}

\paragraph{Ecuaciones discretas (para implementación)}
Sea \(h=T_s\) y \(e[k]=r[k]-y[k]\).
\[
P[k]=K\big(b\,r[k]-y[k]\big)
\]
\[
I[k]=I[k-1]+\frac{K h}{T_i}\,e[k-1]
\]
\[
D[k]=\frac{T_d}{T_d+Nh}D[k-1]-\frac{K T_d N}{T_d+Nh}\big(y[k]-y[k-1]\big)
\]
\[
u[k]=P[k]+I[k]+D[k]
\]

\paragraph{Anti-windup y saturación (si aplica)}
\begin{itemize}
	\item Saturación: \(u_{\text{PWM}}[k]=\mathrm{sat}(u_0+\Delta u[k],\,1000,\,2000)\) o \(u_{\text{PWM}}[k]=\mathrm{sat}(u[k],\,1000,\,2000)\).
	\item Anti-windup: \TODO{back-calculation / clamping / etc. (si se usó)}
\end{itemize}

\paragraph{Criterios experimentales de sintonización (resumen)}
\begin{itemize}
	\item \(K\): \TODO{criterio (ej. borde de inestabilidad)}
	\item \(T_i\): \TODO{criterio (error estacionario vs \(t_s\))}
	\item \(T_d\) y \(N\): \TODO{criterio (overshoot vs ruido)}
	\item Restricción de esfuerzo: \TODO{ej. \(\le 10\,\mu s/\text{cm}\)}
\end{itemize}

% =========================================================
% TABLA RESUMEN (opcional)
% =========================================================
\subsubsection{Resumen de coeficientes (para completar)}
\label{sec:tabla_resumen_coef}

\begin{table}[!t]
	\centering
	\caption{Resumen de parámetros finales de los controladores (completar).}
	\label{tab:coef_ctrl}
	\begin{tabular}{@{}l l@{}}
		\toprule
		\textbf{Controlador} & \textbf{Parámetros} \\
		\midrule
		Bode & \(K=\TODOm{}\), \TODO{polo/cero si aplica} \\
		Lugar de raíces & \(K=\TODOm{}\), \(z_L=\TODOm{}\), \(p_L=\TODOm{}\), \(z_D=\TODOm{}\), \(p_D=\TODOm{}\) \\
		PID (Åström) & \(K=\TODOm{}\), \(b=\TODOm{}\), \(T_i=\TODOm{}\), \(T_d=\TODOm{}\), \(N=\TODOm{}\) \\
		\bottomrule
	\end{tabular}
\end{table}

% =========================
% PLANTILLA: Control en espacio de estados (para completar luego)
% =========================

\subsection{Control en el espacio de estados}
\label{sec:ss_control}

\subsubsection{Modelo en espacio de estados utilizado}
\label{sec:ss_modelo}

Se empleó un modelo discreto en espacio de estados de orden \(n=\TODOm{1/2/3}\), obtenido a partir de la planta identificada \(G(z)\) mediante \texttt{tf2ss}/\texttt{ss} en MATLAB:
\begin{equation}
	\label{eq:ss_discreto}
	\begin{aligned}
		x[k+1] &= A\,x[k] + B\,u[k],\\
		y[k]   &= C\,x[k] + D\,u[k].
	\end{aligned}
\end{equation}
donde \(x[k]\in\mathbb{R}^{n}\), \(u[k]\in\mathbb{R}\) y \(y[k]\in\mathbb{R}\).

\paragraph{Matrices del modelo (a completar)}
\[
A=\TODOm{\text{matriz}},\qquad
B=\TODOm{\text{matriz}},\qquad
C=\TODOm{\text{matriz}},\qquad
D=\TODOm{\text{matriz}}.
\]

\paragraph{Período de muestreo}
\[
T_s=\TODOm{\text{valor}}\ \text{s}\qquad (f_s=\TODOm{\text{valor}}\ \text{Hz}).
\]
