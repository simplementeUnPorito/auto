% =========================
% PLANTILLA: Parámetros de controladores (para completar luego)
% =========================

% --- Helper para placeholders ---
\newcommand{\TODO}[1]{\textcolor{red}{\textbf{TODO:} #1}}

% --- Si querés una tabla compacta ---
\newcommand{\coef}[1]{\ensuremath{\mathbf{#1}}}

\subsection{Parámetros de los controladores implementados}
\label{sec:param_ctrl}

En esta sección se documentan los parámetros finales utilizados en la práctica experimental para cada estrategia de control. 
Todos los valores fueron ajustados considerando restricciones físicas del actuador (PWM) y criterios de seguridad (evitar saturación prolongada y esfuerzos excesivos).

\subsubsection{Convenciones de implementación}
\label{sec:conv_impl}

\begin{itemize}
	\item Señal aplicada al ESC (PWM tipo servo): \(u_{\text{PWM}} \in [1000,2000]\ \mu s\).
	\item Período de muestreo del control: \(T_s=\TODO{indicar }\,\text{s}\) (o \(f_s=\TODO{indicar }\,\text{Hz}\)).
	\item Si se utiliza control incremental: \(u_{\text{PWM}}[k]=\mathrm{sat}\!\left(u_0+\Delta u[k],\,1000,\,2000\right)\).
	\item Si se utiliza control absoluto: \(u_{\text{PWM}}[k]=\mathrm{sat}\!\left(u[k],\,1000,\,2000\right)\).
\end{itemize}

% =========================================================
% 1) BODE (Respuesta en frecuencia) - Control proporcional o compensador
% =========================================================
\subsubsection{Controlador por respuesta en frecuencia (Bode)}
\label{sec:param_bode}

\paragraph{Estructura del controlador}
\begin{itemize}
	\item Tipo: \TODO{Proporcional / Lead / Lag / Lead-Lag}
	\item Forma general:
	\[
	C_{\text{Bode}}(z)=\TODO{pegar expresión de C(z)}
	\]
\end{itemize}

\paragraph{Parámetros finales (a completar)}
\begin{itemize}
	\item Ganancia: \(K=\TODO{valor}\)
	\item (Si aplica) Cero: \(z_0=\TODO{valor}\)
	\item (Si aplica) Polo: \(p_0=\TODO{valor}\)
	\item (Si aplica) Otros parámetros: \TODO{...}
\end{itemize}

\paragraph{Márgenes de estabilidad medidos en \texttt{controlSystemDesigner}}
\begin{itemize}
	\item \textbf{Antes de compensar:} \(M_g=\TODO{dB}\), \(\phi_m=\TODO{\SI{}{\degree}}\).
	\item \textbf{Después de compensar:} \(M_g=\TODO{dB}\), \(\phi_m=\TODO{\SI{}{\degree}}\).
\end{itemize}

\paragraph{Ecuación en diferencias (si se implementa como filtro digital)}
Si el controlador se implementó como filtro IIR/FIR, indicar su forma recursiva:
\[
u[k]=\TODO{b_0 e[k]+b_1 e[k-1]+\dots -a_1 u[k-1]-\dots}
\]

\paragraph{Notas de implementación}
\begin{itemize}
	\item Saturación: \(u_{\text{PWM}}=\mathrm{sat}(\cdot)\).
	\item \TODO{limitación de esfuerzo / filtro anti-ruido / etc.}
\end{itemize}

% =========================================================
% 2) LUGAR DE RAÍCES (Lag-Lead)
% =========================================================
\subsubsection{Controlador por lugar de raíces (Lag--Lead)}
\label{sec:param_rlocus}

\paragraph{Estructura del controlador}
Se diseñó un compensador Lag--Lead para estabilizar la planta e imponer un desempeño deseado en el plano-\(z\):
\[
C_{\text{RL}}(z)=K\,
\underbrace{\frac{(z-z_{L})(z-z_{D})}{(z-p_{L})(z-p_{D})}}_{\text{Lag--Lead}}
\]
donde \(z_{L},p_{L}\) corresponden a la red Lag y \(z_{D},p_{D}\) a la red Lead.

\paragraph{Parámetros finales (a completar)}
\begin{itemize}
	\item Ganancia: \(K=\TODO{valor}\)
	\item Red Lag: \(z_{L}=\TODO{valor}\), \(p_{L}=\TODO{valor}\)
	\item Red Lead: \(z_{D}=\TODO{valor}\), \(p_{D}=\TODO{valor}\)
\end{itemize}

\paragraph{Polos dominantes objetivo}
\begin{itemize}
	\item Polos deseados: \(z_{d,1,2}=\TODO{valor}\) (criterio: \TODO{ts, Mp, \zeta, etc.})
\end{itemize}

\paragraph{Método de ajuste}
\begin{itemize}
	\item Se utilizó: \TODO{\texttt{rlocus} manual / \texttt{Optimization Based Tuning}}
	\item Función objetivo (si aplica): \TODO{por ejemplo, minimizar \(J=\dots\)}
\end{itemize}

\paragraph{Ecuación en diferencias (implementación)}
\[
u[k]=\TODO{\text{expresión recursiva de } C_{\text{RL}}(z)}
\]

\paragraph{Notas de implementación}
\begin{itemize}
	\item \TODO{saturación, anti-windup si hay integrador, filtro si hay derivada, etc.}
\end{itemize}

% =========================================================
% 3) PID (Åström)
% =========================================================
\subsubsection{Controlador PID (formulación de Åström)}
\label{sec:param_pid}

\paragraph{Estructura y parámetros}
\begin{itemize}
	\item Ganancia proporcional: \(K=\TODO{valor}\)
	\item Peso de referencia proporcional: \(b=\TODO{valor}\)
	\item Tiempo integral: \(T_i=\TODO{valor}\)
	\item Tiempo derivativo: \(T_d=\TODO{valor}\)
	\item Filtro derivativo: \(N=\TODO{valor}\)
\end{itemize}

\paragraph{Ecuaciones discretas (para implementación)}
Sea \(h=T_s\) y \(e[k]=r[k]-y[k]\).
\[
P[k]=K\big(b\,r[k]-y[k]\big)
\]
\[
I[k]=I[k-1]+\frac{K h}{T_i}\,e[k-1]
\]
\[
D[k]=\frac{T_d}{T_d+Nh}D[k-1]-\frac{K T_d N}{T_d+Nh}\big(y[k]-y[k-1]\big)
\]
\[
u[k]=P[k]+I[k]+D[k]
\]

\paragraph{Anti-windup y saturación (si aplica)}
\begin{itemize}
	\item Saturación: \(u_{\text{PWM}}[k]=\mathrm{sat}(u_0+\Delta u[k],\,1000,\,2000)\) o \(u_{\text{PWM}}[k]=\mathrm{sat}(u[k],\,1000,\,2000)\).
	\item Anti-windup: \TODO{back-calculation / clamping / etc. (si se usó)}
\end{itemize}

\paragraph{Criterios experimentales de sintonización (resumen)}
\begin{itemize}
	\item \(K\): \TODO{criterio (ej. borde de inestabilidad)}
	\item \(T_i\): \TODO{criterio (error estacionario vs ts)}
	\item \(T_d\) y \(N\): \TODO{criterio (overshoot vs ruido)}
	\item Restricción de esfuerzo: \TODO{ej. \(\le 10\,\mu s/\text{cm}\)}
\end{itemize}

% =========================================================
% TABLA RESUMEN (opcional)
% =========================================================
\subsubsection{Resumen de coeficientes (para completar)}
\label{sec:tabla_resumen_coef}

\begin{table}[!t]
	\centering
	\caption{Resumen de parámetros finales de los controladores (completar).}
	\label{tab:coef_ctrl}
	\begin{tabular}{@{}l l@{}}
		\toprule
		\textbf{Controlador} & \textbf{Parámetros} \\
		\midrule
		Bode & \(K=\TODO{}\), \TODO{polo/cero si aplica} \\
		Lugar de raíces & \(K=\TODO{}\), \(z_L=\TODO{}\), \(p_L=\TODO{}\), \(z_D=\TODO{}\), \(p_D=\TODO{}\) \\
		PID (Åström) & \(K=\TODO{}\), \(b=\TODO{}\), \(T_i=\TODO{}\), \(T_d=\TODO{}\), \(N=\TODO{}\) \\
		\bottomrule
	\end{tabular}
\end{table}


% =========================
% PLANTILLA: Control en espacio de estados (para completar luego)
% Ubicación arbitraria de polos, observadores, integral, LQI/Kalman
% =========================

\newcommand{\TODO}[1]{\textcolor{red}{\textbf{TODO:} #1}}

\subsection{Control en el espacio de estados}
\label{sec:ss_control}

\subsubsection{Modelo en espacio de estados utilizado}
\label{sec:ss_modelo}

Se empleó un modelo discreto en espacio de estados de orden \(n=\TODO{1/2/3}\), obtenido a partir de la planta identificada \(G(z)\) mediante \texttt{tf2ss}/\texttt{ss} en MATLAB:
\begin{equation}
	\label{eq:ss_discreto}
	\begin{aligned}
		x[k+1] &= A\,x[k] + B\,u[k],\\
		y[k]   &= C\,x[k] + D\,u[k],
	\end{aligned}
\end{equation}
donde \(x[k]\in\mathbb{R}^{n}\), \(u[k]\in\mathbb{R}\) y \(y[k]\in\mathbb{R}\).

\paragraph{Matrices del modelo (a completar)}
\[
A=\TODO{\text{matriz}},\qquad
B=\TODO{\text{matriz}},\qquad
C=\TODO{\text{matriz}},\qquad
D=\TODO{\text{matriz}}.
\]

\paragraph{Período de muestreo}
\[
T_s=\TODO{\text{valor}}\ \text{s}\qquad (f_s=\TODO{\text{valor}}\ \text{Hz}).
\]

\subsubsection{Convenciones de implementación}
\label{sec:ss_conv}

\begin{itemize}
	\item Entrada aplicada al actuador: PWM tipo servo \(u_{\text{PWM}}\in[1000,2000]\ \mu s\).
	\item Señal de control interna: \TODO{\(\Delta u\) incremental / \(u\) absoluto}.
	\item Saturación:
	\[
	u_{\text{PWM}}[k]=\mathrm{sat}\!\left(\TODO{u_0+\Delta u[k]\ \text{o}\ u[k]},\,1000,\,2000\right).
	\]
\end{itemize}

% =========================================================
% 1) UBICACIÓN ARBITRARIA DE POLOS (State feedback)
% =========================================================
\subsubsection{Ubicación arbitraria de polos (realimentación de estados)}
\label{sec:pole_placement}

Se diseñó un controlador por realimentación de estados de la forma:
\begin{equation}
	\label{eq:state_feedback}
	u[k] = -K\,x[k] + N_{\mathrm{bar}}\,r[k],
\end{equation}
donde \(K\in\mathbb{R}^{1\times n}\) es la ganancia de realimentación y \(N_{\mathrm{bar}}\) es un pre-filtro para seguimiento de referencia.

\paragraph{Polos deseados (a completar)}
Los polos de lazo cerrado fueron seleccionados como:
\[
\mathcal{P}_d=\{z_{d,1},z_{d,2},\dots,z_{d,n}\}=\TODO{\text{valores}}
\]
según el criterio de desempeño: \TODO{(tiempo de establecimiento, sobreimpulso, amortiguamiento, etc.)}.

\paragraph{Cálculo de \(K\)}
La ganancia \(K\) se obtuvo mediante:
\[
K=\texttt{place}(A,B,\mathcal{P}_d)\quad \text{o}\quad K=\texttt{acker}(A,B,\mathcal{P}_d).
\]
\[
K=\TODO{\text{vector}}
\]

\paragraph{Prefiltro \(N_{\mathrm{bar}}\)}
Para asegurar ganancia unitaria en régimen permanente (en el modelo lineal), se utilizó:
\[
N_{\mathrm{bar}}=\TODO{\text{método (dcgain, fórmula estándar, etc.)}}
\qquad\Rightarrow\qquad
N_{\mathrm{bar}}=\TODO{\text{valor}}.
\]

\paragraph{Notas de implementación}
\begin{itemize}
	\item \TODO{uso de \(x\) medido o estimado}
	\item saturación del PWM y manejo de límites.
\end{itemize}

% =========================================================
% 2) ESTIMADORES DE ESTADO: PREDICTOR VS ACTUAL
% =========================================================
\subsubsection{Estimación de estados: observador predictivo y observador actual}
\label{sec:obs_pred_act}

Dado que no todos los estados son medibles, se implementó un observador de orden completo. Se consideraron dos variantes:

\paragraph{Observador predictivo (predictor)}
El predictor estima \(x[k+1]\) a partir de \(x[k]\) y la innovación en \(k\):
\begin{equation}
	\label{eq:obs_pred}
	\hat x[k+1] = A\,\hat x[k] + B\,u[k] + L\big(y[k]-\hat y[k]\big),
	\qquad \hat y[k]=C\,\hat x[k] + D\,u[k].
\end{equation}

\paragraph{Observador actual (current estimator)}
El estimador actualiza el estado en el mismo instante \(k\):
\begin{equation}
	\label{eq:obs_act}
	\hat x[k] = \hat x^{-}[k] + L\big(y[k]-\hat y^{-}[k]\big),
	\qquad 
	\hat x^{-}[k]=A\,\hat x[k-1]+B\,u[k-1],
	\qquad 
	\hat y^{-}[k]=C\,\hat x^{-}[k]+D\,u[k-1].
\end{equation}

\paragraph{Selección de polos del observador / ganancia \(L\)}
Se eligieron polos del observador más rápidos que los del lazo cerrado:
\[
\mathcal{P}_o=\{z_{o,1},\dots,z_{o,n}\}=\TODO{\text{valores}}
\]
y se calculó:
\[
L=\texttt{place}(A^\top,C^\top,\mathcal{P}_o)^\top,
\qquad
L=\TODO{\text{matriz}}.
\]

\paragraph{Control con estados estimados}
El control se implementó reemplazando \(x[k]\) por \(\hat x[k]\):
\[
u[k] = -K\,\hat x[k] + N_{\mathrm{bar}}\,r[k].
\]

\paragraph{Notas prácticas}
\begin{itemize}
	\item Predictor vs actual: \TODO{cuál se usó en práctica y por qué}
	\item \TODO{efecto de ruido/retardo del sensor, filtrado, saturación}
\end{itemize}

% =========================================================
% 3) CONTROL INTEGRAL EN ESPACIO DE ESTADOS (Tracking)
% =========================================================
\subsubsection{Control integral en el espacio de estados (seguimiento)}
\label{sec:ss_integral}

Para eliminar error estacionario ante referencias constantes y perturbaciones, se añadió un integrador del error:
\begin{equation}
	\label{eq:int_state}
	x_i[k+1] = x_i[k] + \big(r[k]-y[k]\big).
\end{equation}

Definiendo el estado aumentado:
\[
x_a[k]=
\begin{bmatrix}
	x[k]\\x_i[k]
\end{bmatrix},
\]
el sistema aumentado queda:
\begin{equation}
	\label{eq:ss_aug}
	\begin{aligned}
		x_a[k+1] &=
		\underbrace{\begin{bmatrix}
				A & 0\\
				-C & 1
		\end{bmatrix}}_{A_{\mathrm{aug}}}
		x_a[k]
		+
		\underbrace{\begin{bmatrix}
				B\\
				0
		\end{bmatrix}}_{B_{\mathrm{aug}}}
		u[k]
		+
		\underbrace{\begin{bmatrix}
				0\\
				1
		\end{bmatrix}}_{E_{\mathrm{aug}}}
		r[k],\\
		y[k]&=\underbrace{\begin{bmatrix} C & 0\end{bmatrix}}_{C_{\mathrm{aug}}}x_a[k]+D\,u[k].
	\end{aligned}
\end{equation}

\paragraph{Ley de control con integrador}
\begin{equation}
	\label{eq:u_int}
	u[k] = -K_a\,x_a[k]
	=
	-\begin{bmatrix}K & K_i\end{bmatrix}
	\begin{bmatrix}x[k]\\x_i[k]\end{bmatrix}.
\end{equation}

\paragraph{Parámetros (a completar)}
\[
K=\TODO{\text{vector}},\qquad K_i=\TODO{\text{escalar}},\qquad K_a=\TODO{\text{vector aumentado}}.
\]

\paragraph{Notas de implementación}
\begin{itemize}
	\item Anti-windup: \TODO{si se aplicó (clamping/back-calculation)}
	\item Saturación PWM y protección.
\end{itemize}

% =========================================================
% 4) CONTROL ÓPTIMO CON INTEGRADOR + KALMAN (LQI + KF / LQG)
% =========================================================
\subsubsection{Control óptimo con integrador y filtro de Kalman (LQI/LQG)}
\label{sec:lqi_lqg}

Se implementó un enfoque óptimo que combina:
\begin{itemize}
	\item un regulador LQR/LQI para la realimentación óptima,
	\item y un estimador basado en filtro de Kalman para reconstrucción de estados.
\end{itemize}

\paragraph{Diseño LQI (regulador óptimo con integrador)}
Se plantea el costo cuadrático:
\begin{equation}
	\label{eq:J_lqi}
	J=\sum_{k=0}^{\infty}\left(x_a[k]^\top Q\,x_a[k] + u[k]^\top R\,u[k]\right),
\end{equation}
donde \(Q\succeq 0\) y \(R\succ 0\) son matrices de ponderación.

La ganancia óptima se obtiene como:
\[
K_a=\texttt{dlqr}(A_{\mathrm{aug}},B_{\mathrm{aug}},Q,R),
\qquad
K_a=\TODO{\text{vector}}.
\]

\paragraph{Matrices de ponderación (a completar)}
\[
Q=\TODO{\text{matriz}},\qquad R=\TODO{\text{escalar/matriz}}.
\]

\paragraph{Filtro de Kalman discreto}
Se modeló el sistema con ruido de proceso y medición:
\begin{equation}
	\label{eq:kf_model}
	\begin{aligned}
		x[k+1] &= A x[k] + B u[k] + w[k],\\
		y[k] &= C x[k] + D u[k] + v[k],
	\end{aligned}
\end{equation}
donde \(w[k]\sim\mathcal{N}(0,W)\) y \(v[k]\sim\mathcal{N}(0,V)\).

El estimador se calculó con:
\[
L_K=\texttt{dlqe}(A,\;I,\;C,\;W,\;V)\quad \text{o}\quad \texttt{kalman}(\cdot)
\]
y se obtuvo:
\[
L_K=\TODO{\text{matriz}}.
\]

\paragraph{Covarianzas (a completar)}
\[
W=\TODO{\text{matriz}},\qquad V=\TODO{\text{escalar/matriz}}.
\]

\paragraph{Ley de control LQG/LQI+KF}
Finalmente, se implementó:
\[
u[k]=-\begin{bmatrix}K & K_i\end{bmatrix}
\begin{bmatrix}\hat x[k]\\x_i[k]\end{bmatrix},
\]
donde \(\hat x[k]\) proviene del filtro de Kalman y \(x_i[k]\) integra el error \(r[k]-y[k]\).

\paragraph{Notas prácticas}
\begin{itemize}
	\item \TODO{si el integrador usa \(y\) medido o \(\hat y\)}
	\item \TODO{efecto del retardo/ruido del sensor en el KF}
	\item Saturación PWM y anti-windup en el integrador.
\end{itemize}

% =========================================================
% TABLA RESUMEN (opcional)
% =========================================================
\subsubsection{Resumen de coeficientes (para completar)}
\label{sec:ss_resumen_coef}

\begin{table}[!t]
	\centering
	\caption{Resumen de parámetros del control en EE (completar).}
	\label{tab:ss_coef}
	\begin{tabular}{@{}l l@{}}
		\toprule
		\textbf{Bloque} & \textbf{Parámetros} \\
		\midrule
		Pole placement & \(\mathcal{P}_d=\TODO{}\), \(K=\TODO{}\), \(N_{\mathrm{bar}}=\TODO{}\) \\
		Observador & \(\mathcal{P}_o=\TODO{}\), \(L=\TODO{}\) (predictor/actual: \TODO{}) \\
		Integral EE & \(K=\TODO{}\), \(K_i=\TODO{}\), anti-windup: \TODO{} \\
		LQI & \(Q=\TODO{}\), \(R=\TODO{}\), \(K_a=\TODO{}\) \\
		Kalman & \(W=\TODO{}\), \(V=\TODO{}\), \(L_K=\TODO{}\) \\
		\bottomrule
	\end{tabular}
\end{table}
