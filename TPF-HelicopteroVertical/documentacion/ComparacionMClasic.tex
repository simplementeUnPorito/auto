La aplicación de distintos métodos clásicos sobre la misma planta
permitió comparar directamente su aplicabilidad práctica,
su dependencia del modelo y la información que aportan
para el diseño.

Desde el punto de vista de simplicidad y rapidez de implementación,
el \textbf{PID} resultó claramente el método más eficiente.
Permitió obtener una respuesta satisfactoria
con escasa dependencia del modelo
y con un proceso de ajuste directo e intuitivo.
En términos puramente ingenieriles,
si el objetivo es lograr funcionamiento estable
con el menor esfuerzo analítico posible,
el PID se posiciona como la alternativa más práctica.

El \textbf{Lugar de Raíces}, en cambio,
aportó mayor profundidad conceptual.
Permite visualizar explícitamente la estabilidad
en el plano \(z\),
relacionar polos con desempeño transitorio
y comprender geométricamente el efecto de la ganancia.
Si bien exige mayor comprensión del modelo
y es más sensible a la elección del período de muestreo,
proporciona una cantidad de información estructural
superior a la del PID.

El método basado en \textbf{Bode}
resultó adecuado para analizar márgenes de estabilidad
y robustez,
pero menos intuitivo en relación con el comportamiento temporal.
Sin herramientas de simulación,
su aplicación manual se vuelve considerablemente más compleja.
En esta experiencia no ofreció ventajas decisivas
respecto a los métodos anteriores.

Por su parte,
la \textbf{Síntesis Directa (Truxal--Ragazzini)}
mostró gran elegancia analítica
y rigor matemático,
pero evidenció una dependencia extrema del modelo
y una tendencia a generar esfuerzos de control elevados.
En una planta física con saturaciones y no linealidades,
esta sensibilidad la vuelve riesgosa para implementación real,
quedando principalmente como herramienta académica.

En conclusión,
los métodos que demostraron mayor aplicabilidad práctica
en la planta experimental fueron el PID y el Lugar de Raíces:
el primero por su simplicidad y robustez,
el segundo por la riqueza de información que aporta.
Los demás métodos resultaron valiosos conceptualmente,
pero menos determinantes en la implementación experimental concreta.
