\subsubsection{Fundamento del Estimador de Estados}

En sistemas donde no es posible medir directamente todos los estados,
se recurre a un estimador (observador) que reconstruye el vector de
estados a partir de la señal de entrada y la salida medida.

El sistema dinámico discreto se describe mediante las ecuaciones \ref{eq:din_estado} y \ref{eq:din_salida}.


El estimador de Luenberger discreto se define como:

\[
\hat{x}_{k+1} = A\hat{x}_k + B u_k + L (y_k - \hat{y}_k)
\]

donde:

\[
\hat{y}_k = C\hat{x}_k
\]

y \(L\) es la matriz de ganancia del estimador.

\subsubsection{Dinámica del Error de Estimación}

Definiendo el error de estimación como:

\[
e_k = x_k - \hat{x}_k
\]

se obtiene la dinámica:

\[
e_{k+1} = (A - LC)e_k
\]

Por lo tanto, la convergencia del estimador depende únicamente de los
autovalores de la matriz \(A - LC\).

El objetivo del diseño consiste en ubicar arbitrariamente los polos del
observador:

\[
\lambda(A - LC) = \{p_{obs,1}, p_{obs,2}, \dots, p_{obs,n}\}
\]

\subsubsection{Selección de Polos del Observador}

Se seleccionaron los siguientes polos para el estimador:

\[
\{p_{obs}\} =
\left\{
0.8 + 0.25i,\;
0.8 - 0.25i,\;
0.9
\right\}
\]

Los criterios adoptados fueron:

\begin{itemize}
	\item Garantizar estabilidad discreta ($|p_{obs,i}|<1$).
	\item Asegurar una convergencia más rápida que la dinámica del
	sistema en lazo cerrado.
	\item Introducir amortiguamiento adecuado en el error de estimación.
\end{itemize}

\subsubsection{Cálculo de la Ganancia del Observador}

La ganancia \(L\) fue obtenida mediante el método dual de Ackermann,
utilizando la función \texttt{acker()} aplicada a la matriz transpuesta:

\[
L = \texttt{acker}(A^T, C^T, p_{obs})^T
\]

Para la estructura implementada, se obtuvieron las siguientes ganancias:

\[
L_{\text{actual}} =
\begin{bmatrix}
	126.8411 \\
	172.0621 \\
	45.1301
\end{bmatrix}
\]

y para la variante predictor:

\[
L_{\text{predictor}} =
\begin{bmatrix}
	167.3535 \\
	253.6822 \\
	86.0310
\end{bmatrix}
\]

\subsubsection{Verificación de Polos}

Se verificó que los autovalores de la matriz \(A - LC\) coinciden con
los polos deseados:

\[
\lambda(A - LC) =
\left\{
0.8 + 0.25i,\;
0.8 - 0.25i,\;
0.9
\right\}
\]

confirmando la correcta ubicación arbitraria de polos del estimador.

\subsubsection{Análisis del Comportamiento del Estimador}

La magnitud relativamente elevada de las ganancias del observador se
debe a la necesidad de forzar una convergencia rápida del error de
estimación. Polos más cercanos al origen implican mayores ganancias en
la matriz \(L\), acelerando la dinámica del error:

\[
e_k \sim p_{obs}^k
\]

Se observa que el estimador converge rápidamente hacia el estado real,
permitiendo su utilización en esquemas de control por realimentación
de estados estimados.

\subsubsection{Conclusión}

El estimador diseñado cumple con los requisitos de estabilidad y
convergencia rápida. La ubicación arbitraria de los polos del
observador garantiza que el error de estimación decae exponencialmente,
validando la implementación del estimador de Luenberger en el dominio
discreto.
