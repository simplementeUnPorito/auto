
\subsection{Introducción}

A diferencia de los métodos clásicos basados en funciones de transferencia
y análisis en frecuencia, los métodos modernos de control se fundamentan
en la representación en espacio de estados del sistema dinámico.

En este enfoque, la dinámica se describe mediante:

\[
x_{k+1}=Ax_k+Bu_k
\]
\[
y_k=Cx_k+Du_k
\]

donde $x_k \in \mathbb{R}^n$ es el vector de estados,
$u_k$ la entrada de control y $y_k$ la salida medida.

Este formalismo permite:

\begin{itemize}
	\item Diseñar realimentación directa de estados.
	\item Ubicar polos del sistema de manera sistemática.
	\item Formular problemas de control óptimo.
	\item Incorporar estimadores de estado.
\end{itemize}


% ============================================================
\subsection{Modelo en Espacio de Estados}
% ============================================================

El modelo discreto obtenido a partir de la identificación es:

\[
A = \textbf{[Completar]}
\]

\[
B = \textbf{[Completar]}
\]

\[
C = \textbf{[Completar]}
\]

\[
D = \textbf{[Completar]}
\]

Tiempo de muestreo:

\[
T_s = \textbf{[Completar]}
\]


% ============================================================
\subsection{Análisis de Controlabilidad y Observabilidad}
% ============================================================

Matriz de controlabilidad:

\[
\mathcal{C}=[B\ AB\ A^2B\ \dots A^{n-1}B]
\]

\[
\text{rank}(\mathcal{C}) = \textbf{[Completar]}
\]

Matriz de observabilidad:

\[
\mathcal{O}=\begin{bmatrix}
	C \\ CA \\ CA^2 \\ \vdots \\ CA^{n-1}
\end{bmatrix}
\]

\[
\text{rank}(\mathcal{O}) = \textbf{[Completar]}
\]

Conclusión estructural:

\textbf{[Completar: sistema controlable/observable]}


% ============================================================
\subsection{Realimentación de Estados}
% ============================================================

Se plantea la ley de control:

\[
u_k = -Kx_k
\]

Los polos del sistema en lazo cerrado quedan dados por:

\[
\lambda(A-BK)
\]

Objetivo:

\begin{itemize}
	\item Ubicación deseada de polos: \textbf{[Completar]}
\end{itemize}

Ganancia obtenida:

\[
K = \textbf{[Completar]}
\]


\subsection{Ubicación Arbitraria de Polos}
% ============================================================
\subsection{Realimentación de Estados y Estimación}

\subsubsection{Realimentación de Estados}

Considerando el modelo discreto del sistema:

\[
x_{k+1} = A x_k + B u_k
\]
\[
y_k = C x_k
\]

se propone una ley de control por realimentación de estados:

\[
u_k = -K x_k
\]

lo que conduce a la dinámica en lazo cerrado:

\[
x_{k+1} = (A - BK)x_k
\]

El diseño por ubicación arbitraria de polos consiste en determinar la
matriz de ganancia \(K\) tal que:

\[
\lambda(A - BK) = \{p_1, p_2, \dots, p_n\}
\]

siendo \(|p_i| < 1\) condición necesaria para estabilidad discreta.

La determinación de \(K\) puede realizarse mediante el método de
Ackermann o utilizando la función \texttt{place()} de MATLAB, siempre
que el sistema sea completamente controlable.

\vspace{0.3cm}

\subsubsection{Estimador de Estados}

En situaciones donde no todos los estados son medibles, se introduce
un estimador de Luenberger para reconstruir el vector de estados a
partir de la entrada y la salida medida.

El estimador discreto se define como:

\[
\hat{x}_{k+1} = A\hat{x}_k + B u_k + L (y_k - \hat{y}_k)
\]

donde:

\[
\hat{y}_k = C \hat{x}_k
\]

La dinámica del error de estimación:

\[
e_k = x_k - \hat{x}_k
\]

queda gobernada por:

\[
e_{k+1} = (A - LC)e_k
\]

Por lo tanto, la convergencia del estimador depende de la ubicación
de los autovalores de la matriz \(A - LC\), que pueden ser fijados
arbitrariamente siempre que el sistema sea observable:

\[
\lambda(A - LC) = \{p_{obs,1}, \dots, p_{obs,n}\}
\]

\vspace{0.3cm}

\subsubsection{Principio de Separación}

Cuando se combinan realimentación de estados y estimación, la ley de
control adopta la forma:

\[
u_k = -K\hat{x}_k
\]

y la dinámica total del sistema presenta autovalores dados por la
unión de los polos del controlador y los polos del estimador:

\[
\lambda_{\text{total}} =
\lambda(A - BK) \cup \lambda(A - LC)
\]

Este resultado, conocido como principio de separación, permite diseñar
independientemente el controlador y el estimador.


\clearpage

\subsection{Estimación de Estados}
% ============================================================

\subsubsection{Fundamento del Estimador de Estados}

En sistemas donde no es posible medir directamente todos los estados,
se recurre a un estimador (observador) que reconstruye el vector de
estados a partir de la señal de entrada y la salida medida.

El sistema dinámico discreto se describe mediante las ecuaciones \ref{eq:din_estado} y \ref{eq:din_salida}.


El estimador de Luenberger discreto se define como:

\[
\hat{x}_{k+1} = A\hat{x}_k + B u_k + L (y_k - \hat{y}_k)
\]

donde:

\[
\hat{y}_k = C\hat{x}_k
\]

y \(L\) es la matriz de ganancia del estimador.

\subsubsection{Dinámica del Error de Estimación}

Definiendo el error de estimación como:

\[
e_k = x_k - \hat{x}_k
\]

se obtiene la dinámica:

\[
e_{k+1} = (A - LC)e_k
\]

Por lo tanto, la convergencia del estimador depende únicamente de los
autovalores de la matriz \(A - LC\).

El objetivo del diseño consiste en ubicar arbitrariamente los polos del
observador:

\[
\lambda(A - LC) = \{p_{obs,1}, p_{obs,2}, \dots, p_{obs,n}\}
\]

\subsubsection{Selección de Polos del Observador}

Se seleccionaron los siguientes polos para el estimador:

\[
\{p_{obs}\} =
\left\{
0.8 + 0.25i,\;
0.8 - 0.25i,\;
0.9
\right\}
\]

Los criterios adoptados fueron:

\begin{itemize}
	\item Garantizar estabilidad discreta ($|p_{obs,i}|<1$).
	\item Asegurar una convergencia más rápida que la dinámica del
	sistema en lazo cerrado.
	\item Introducir amortiguamiento adecuado en el error de estimación.
\end{itemize}

\subsubsection{Cálculo de la Ganancia del Observador}

La ganancia \(L\) fue obtenida mediante el método dual de Ackermann,
utilizando la función \texttt{acker()} aplicada a la matriz transpuesta:

\[
L = \texttt{acker}(A^T, C^T, p_{obs})^T
\]

Para la estructura implementada, se obtuvieron las siguientes ganancias:

\[
L_{\text{actual}} =
\begin{bmatrix}
	126.8411 \\
	172.0621 \\
	45.1301
\end{bmatrix}
\]

y para la variante predictor:

\[
L_{\text{predictor}} =
\begin{bmatrix}
	167.3535 \\
	253.6822 \\
	86.0310
\end{bmatrix}
\]

\subsubsection{Verificación de Polos}

Se verificó que los autovalores de la matriz \(A - LC\) coinciden con
los polos deseados:

\[
\lambda(A - LC) =
\left\{
0.8 + 0.25i,\;
0.8 - 0.25i,\;
0.9
\right\}
\]

confirmando la correcta ubicación arbitraria de polos del estimador.

\subsubsection{Análisis del Comportamiento del Estimador}

La magnitud relativamente elevada de las ganancias del observador se
debe a la necesidad de forzar una convergencia rápida del error de
estimación. Polos más cercanos al origen implican mayores ganancias en
la matriz \(L\), acelerando la dinámica del error:

\[
e_k \sim p_{obs}^k
\]

Se observa que el estimador converge rápidamente hacia el estado real,
permitiendo su utilización en esquemas de control por realimentación
de estados estimados.

\subsubsection{Conclusión}

El estimador diseñado cumple con los requisitos de estabilidad y
convergencia rápida. La ubicación arbitraria de los polos del
observador garantiza que el error de estimación decae exponencialmente,
validando la implementación del estimador de Luenberger en el dominio
discreto.


\clearpage

% ============================================================
\subsection{Control Óptimo (LQR)}
% ============================================================

Se define el funcional de costo:

\[
J=\sum_{k=0}^{\infty}(x_k^TQx_k+u_k^TRu_k)
\]

Matrices de ponderación:

\[
Q=\textbf{[Completar]}
\]

\[
R=\textbf{[Completar]}
\]

La solución se obtiene resolviendo la ecuación de Riccati discreta:

\[
P=A^TPA - A^TPB(R+B^TPB)^{-1}B^TPA + Q
\]

Ganancia óptima:

\[
K=(R+B^TPB)^{-1}B^TPA
\]

Polos obtenidos:

\[
\lambda(A-BK)=\textbf{[Completar]}
\]


\clearpage
% ============================================================

% ============================================================
\subsection{Control Óptimo con Integrador}
% ============================================================


\clearpage

% ============================================================
\subsection{Filtro de Kalman}
% ============================================================
% ============================================================
\subsubsection{Modelo con Ruido}
% ============================================================

Para incorporar incertidumbre y modelar explícitamente la presencia de
perturbaciones no modeladas y ruido del sensor, se adopta la siguiente
representación estocástica discreta:

\[
x_{k+1} = A x_k + B u_k + w_k
\]

\[
y_k = C x_k + v_k
\]

donde:

\begin{itemize}
	\item $w_k \sim \mathcal{N}(0,Q)$ representa el ruido de proceso,
	asociado a dinámica no modelada, perturbaciones aerodinámicas
	y simplificaciones del modelo identificado.
	\item $v_k \sim \mathcal{N}(0,R)$ representa el ruido de medición,
	proveniente del sensor láser de distancia.
\end{itemize}

% ============================================================
\paragraph{Estimación de $R$ (ruido de medición)}
% ============================================================

La varianza del ruido de medición se obtuvo mediante ensayos empíricos,
midiendo la dispersión de la señal del sensor con la planta en reposo.
Siendo $\sigma_v$ la desviación estándar medida (en cm), se adopta:

\[
R = \sigma_v^2
\]

En los ensayos realizados se obtuvo:

\[
\sigma_v = 2.043\ \text{cm}
\qquad\Longrightarrow\qquad
R = 4.174\ \text{cm}^2
\]

% ============================================================
\paragraph{Parametrización y sintonización de $Q$ (ruido de proceso)}
% ============================================================

La matriz de covarianza del ruido de proceso se parametrizó como:

\[
Q = q\,I_n
\]

donde $q$ es un escalar positivo ajustable e $I_n$ es la matriz identidad
de dimensión $n$. El valor de $q$ se determinó mediante consistencia
estadística de la innovación normalizada (ver Apéndice~\ref{ap:kalman_tuning}).

El valor óptimo obtenido fue:

\[
q = 50.8022\times 10^{-3}
\qquad\Longrightarrow\qquad
Q = q\,I_n
\]

% ============================================================
\subsubsection{Filtro de Kalman en régimen permanente}
% ============================================================

Se utilizó un estimador de Kalman discreto en su variante \textit{current estimator}
(\texttt{kalman(...,'current')}). La ganancia en régimen permanente \(L\) se obtiene
a partir de la solución estacionaria \(P\) de la ecuación de Riccati discreta:

\[
P = A P A^T - A P C^T \left(C P C^T + R\right)^{-1} C P A^T + Q
\]

y la ganancia queda:

\[
L = A P C^T \left(C P C^T + R\right)^{-1}
\]

La ganancia obtenida en MATLAB para el modelo discretizado fue:

\[
L =
\begin{bmatrix}
	97.4968\\
	192.3014\\
	94.7820
\end{bmatrix}
\]

% ============================================================
\paragraph{Polos del observador}
% ============================================================

Para el \textit{current estimator}, la dinámica del error queda
determinada por \(A - LCA\). Los polos obtenidos fueron:

\[
\lambda(A - LCA) =
\begin{aligned}
	&0.9730274 + 0.0308455\,j \\
	&0.9730274 - 0.0308455\,j \\
	&0.9694430
\end{aligned}
\]

Todos los polos se ubican dentro del círculo unitario,
garantizando estabilidad del estimador.

% ============================================================
\subsubsection{Control integral y realimentación de estados (LQGI)}
% ============================================================

Con el objetivo de eliminar el error en régimen permanente ante referencias tipo escalón,
se incorporó un integrador de error. Definiendo el estado integral \(\xi_k\):

\[
\xi_{k+1} = \xi_k + \left(r_k - y_k\right)
\]

Se diseñó una ley de control tipo LQI:

\[
u_k = -K_x\,\hat{x}_k + K_i\,\xi_k
\]

donde \(\hat{x}_k\) proviene del estimador de Kalman.

Los valores obtenidos mediante \texttt{dlqr} fueron:

\[
K_x =
\begin{bmatrix}
	16.5343 & -15.8046 & 15.1271
\end{bmatrix}
\]

\[
K_i = 3.2067
\]

% ============================================================
\paragraph{Polos de la planta y del lazo cerrado}
% ============================================================

Los polos de la planta discretizada (sin control) fueron:

\[
\lambda(A) =
\begin{aligned}
	&0.9999972 \\
	&0.9720409 + 0.0241158\,j \\
	&0.9720409 - 0.0241158\,j
\end{aligned}
\]

Los polos del lazo cerrado del sistema aumentado (planta + integrador + control) fueron:

\[
\lambda(A_{\text{cl}}) =
\begin{aligned}
	&0.9735939 + 0.0318308\,j \\
	&0.9735939 - 0.0318308\,j \\
	&0.9671966 \\
	&0
\end{aligned}
\]

Se verifica estabilidad discreta y presencia del polo en cero
asociado a la acción integral.

% ============================================================
\subsubsection{Sistema aumentado usado para \texttt{pzmap}}
% ============================================================

En el script de validación se construyó explícitamente el sistema aumentado:

\[
A_{\text{aug}} =
\begin{bmatrix}
	A & B \\
	K_x - K_xA - K_iCA & 1 - K_xB - K_iCB
\end{bmatrix}
\]

\[
B_{\text{aug}} =
\begin{bmatrix}
	0\\
	0\\
	0\\
	K_i
\end{bmatrix}
\]

\[
C_{\text{aug}} =
\begin{bmatrix}
	C & 0
\end{bmatrix}
\qquad
D_{\text{aug}} = 0
\]

Construyéndose el modelo:

\[
\texttt{sysDaug = ss(Aaug,Baug,Caug,0,Ts)}
\]

y visualizando los polos mediante \texttt{pzmap(sysDaug)}.


\insertarfigura{img/LQGi/LQGi_step_ruido.png}
{Respuesta temporal en presencia de ruido.}
{fig:lqgi_step_ruido}{1}

\insertarfigura{img/LQGi/LQGi_esfuerzo_ruido.png}
{Esfuerzo de control en presencia de ruido.}
{fig:lqgi_esfuerzo_ruido}{1}

\insertarfigura{img/LQGi/LQGi_practico.png}
{Resultado experimental sobre la planta real.}
{fig:lqgi_practico}{1}

\insertarfigura{img/LQGi/LQGi_pzmap_NaranajaOBS_AzulPLANTA.png}
{Mapa de polos: observador (naranja) y sistema (azul).}
{fig:lqgi_pzmap}{1}

% ============================================================
\subsubsection{Discusión}
% ============================================================

En régimen dinámico, el tiempo de subida experimental fue aproximadamente
\(t_r^{\text{exp}} \approx 0.65\,\text{s}\), con sobreimpulso reducido
(del orden del 10\% o menor en la mayoría de los ensayos).
La simulación con ruido predijo \(t_r^{\text{sim}} \approx 0.5\,\text{s}\)
y sobreimpulso cercano al 8\%, mostrando buena concordancia.

El esfuerzo de control en simulación presenta ciertos picos,
producto de la inyección explícita de ruido gaussiano. En la práctica,
los picos resultaron más consistentes y de menor amplitud relativa,
especialmente en comparación con implementaciones anteriores sin Kalman, cumpliendo de muy buena manera su objetivo.

Un aspecto destacable es que, gracias a la acción integral,
el error en régimen permanente converge sistemáticamente a cero,
a diferencia de prácticas previas donde persistían derivas
incluso utilizando prefiltros o compensaciones adicionales.

% ============================================================
\subsubsection{Conclusión}
% ============================================================

El esquema LQGI (LQR + Kalman + Integrador) constituye la síntesis más
completa implementada en este trabajo.

A diferencia de la simple ubicación arbitraria de polos, aquí los polos
del lazo cerrado resultan de una optimización basada en criterios
energéticos bien definidos por el ingeniero.

Aunque la selección de \(Q\) y \(R\) no es intuitiva y requiere
criterio y ajuste iterativo, una vez correctamente definidos,
el método produce resultados consistentes, robustos y coherentes
con el modelo.

La incorporación del filtro de Kalman permitió seleccionar las ganancias
del observador de forma óptima en presencia de ruido, reduciendo la
amplificación observada en enfoques anteriores.

Finalmente, la inclusión del integrador garantizó eliminación del error
estacionario y otorgó el mejor desempeño global entre todas las técnicas
implementadas, tanto en simulación como en la planta real.
