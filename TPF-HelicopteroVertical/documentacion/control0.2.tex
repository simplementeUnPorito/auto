
\subsection{Introducción}

A diferencia de los métodos clásicos basados en funciones de transferencia
y análisis en frecuencia, los métodos modernos de control se fundamentan
en la representación en espacio de estados del sistema dinámico.

En este enfoque, la dinámica se describe mediante:

\[
x_{k+1}=Ax_k+Bu_k
\]
\[
y_k=Cx_k+Du_k
\]

donde $x_k \in \mathbb{R}^n$ es el vector de estados,
$u_k$ la entrada de control y $y_k$ la salida medida.

Este formalismo permite:

\begin{itemize}
	\item Diseñar realimentación directa de estados.
	\item Ubicar polos del sistema de manera sistemática.
	\item Formular problemas de control óptimo.
	\item Incorporar estimadores de estado.
\end{itemize}


% ============================================================
\subsection{Modelo en Espacio de Estados}
% ============================================================

El modelo discreto obtenido a partir de la identificación es:

\[
A = \textbf{[Completar]}
\]

\[
B = \textbf{[Completar]}
\]

\[
C = \textbf{[Completar]}
\]

\[
D = \textbf{[Completar]}
\]

Tiempo de muestreo:

\[
T_s = \textbf{[Completar]}
\]


% ============================================================
\subsection{Análisis de Controlabilidad y Observabilidad}
% ============================================================

Matriz de controlabilidad:

\[
\mathcal{C}=[B\ AB\ A^2B\ \dots A^{n-1}B]
\]

\[
\text{rank}(\mathcal{C}) = \textbf{[Completar]}
\]

Matriz de observabilidad:

\[
\mathcal{O}=\begin{bmatrix}
	C \\ CA \\ CA^2 \\ \vdots \\ CA^{n-1}
\end{bmatrix}
\]

\[
\text{rank}(\mathcal{O}) = \textbf{[Completar]}
\]

Conclusión estructural:

\textbf{[Completar: sistema controlable/observable]}


% ============================================================
\subsection{Realimentación de Estados}
% ============================================================

Se plantea la ley de control:

\[
u_k = -Kx_k
\]

Los polos del sistema en lazo cerrado quedan dados por:

\[
\lambda(A-BK)
\]

Objetivo:

\begin{itemize}
	\item Ubicación deseada de polos: \textbf{[Completar]}
\end{itemize}

Ganancia obtenida:

\[
K = \textbf{[Completar]}
\]


% ============================================================
\subsection{Control Óptimo (LQR)}
% ============================================================

Se define el funcional de costo:

\[
J=\sum_{k=0}^{\infty}(x_k^TQx_k+u_k^TRu_k)
\]

Matrices de ponderación:

\[
Q=\textbf{[Completar]}
\]

\[
R=\textbf{[Completar]}
\]

La solución se obtiene resolviendo la ecuación de Riccati discreta:

\[
P=A^TPA - A^TPB(R+B^TPB)^{-1}B^TPA + Q
\]

Ganancia óptima:

\[
K=(R+B^TPB)^{-1}B^TPA
\]

Polos obtenidos:

\[
\lambda(A-BK)=\textbf{[Completar]}
\]


% ============================================================
\subsection{Estimación de Estados}
% ============================================================

Dado que no todos los estados son medibles, se implementa un observador:

\[
\hat{x}_{k+1}=A\hat{x}_k+Bu_k+L(y_k-C\hat{x}_k)
\]

Ganancia del observador:

\[
L = \textbf{[Completar]}
\]

Polos del observador:

\[
\lambda(A-LC)=\textbf{[Completar]}
\]


% ============================================================
\subsection{Control Óptimo con Estimación (LQG)}
% ============================================================

Combinando LQR y estimador de estados se obtiene:

\[
u_k = -K\hat{x}_k
\]

Ventajas:

\begin{itemize}
	\item Diseño sistemático.
	\item Consideración explícita de estados internos.
	\item Optimización del esfuerzo de control.
\end{itemize}

Limitaciones:

\begin{itemize}
	\item Dependencia del modelo.
	\item Sensibilidad a incertidumbres no modeladas.
\end{itemize}
