
\subsection{Introducción}

A diferencia de los métodos clásicos, que se apoyan en funciones
de transferencia y análisis en el dominio de la frecuencia,
los métodos modernos de control se fundamentan en la
\textbf{representación en espacio de estados} del sistema dinámico.

En este enfoque, la dinámica del sistema discreto se describe como:

\begin{equation}
	x_{k+1}=Ax_k+Bu_k
	\label{eq:din_estado}
\end{equation}

\begin{equation}
	y_k=Cx_k+Du_k
	\label{eq:din_salida}
\end{equation}

donde:

\begin{itemize}
	\item $x_k \in \mathbb{R}^n$ es el vector de estados,
	\item $u_k$ es la entrada de control,
	\item $y_k$ es la salida medida.
\end{itemize}

Este formalismo presenta ventajas estructurales relevantes
respecto al enfoque clásico:

\begin{itemize}
	\item Permite diseñar realimentación directa de estados.
	\item Facilita la ubicación sistemática de polos.
	\item Habilita la formulación de problemas de control óptimo.
	\item Permite incorporar estimadores de estado cuando no todos
	los estados son medibles.
\end{itemize}

En el sistema desarrollado, únicamente se dispone de medición directa
de la altura, por lo que la reconstrucción de los estados internos
mediante observadores resulta un elemento central del diseño.

% ============================================================
\subsubsection{Modelo continuo en espacio de estados}
% ============================================================

Las matrices del modelo continuo identificado son:

\[
F =
\begin{bmatrix}
	-5.6102 & -3.5055 & -0.0314 \\
	4.0000  & 0       & 0       \\
	0       & 0.0312  & 0
\end{bmatrix}
\]

\[
G =
\begin{bmatrix}
	16 \\
	0 \\
	0
\end{bmatrix}
\]

\[
H =
\begin{bmatrix}
	-0.0076 & 0.0064 & 8.9975
\end{bmatrix}
\]

\[
J =
\begin{bmatrix}
	0
\end{bmatrix}
\]

Este modelo representa la dinámica linealizada del sistema
alrededor del punto de operación (hover),
en coordenadas relativas.

% ============================================================
\subsubsection{Modelo discreto}
% ============================================================

Para implementación digital en el PSoC,
el modelo continuo se discretiza con período de muestreo \(T_s\):

\[
A = e^{F T_s}, 
\qquad
B = F^{-1}\!\left(e^{F T_s} - I\right) G,
\qquad\\
C = H,
\qquad
D = J.
\]

El valor de \(T_s\) depende del método de diseño y de la práctica
experimental, ya que diferentes estrategias de control
requieren distintos compromisos entre velocidad de respuesta
y sensibilidad numérica.

% ============================================================
\subsubsection{Análisis de controlabilidad y observabilidad}
% ============================================================

Matriz de controlabilidad:

\[
\mathcal{C}=[G\ FG\ F^2G\ \dots F^{n-1}G]
\]

\[
\text{rank}(\mathcal{C}) = 3
\]

Matriz de observabilidad:

\[
\mathcal{O}=
\begin{bmatrix}
	H \\
	HF \\
	HF^2 \\
	\vdots \\
	HF^{n-1}
\end{bmatrix}
\]

\[
\text{rank}(\mathcal{O}) = 3
\]

Dado que ambos rangos coinciden con el orden del sistema,
el modelo es \textbf{completamente controlable y observable}.

Esto implica que:

\begin{itemize}
	\item Existe una señal de control capaz de influenciar todos los estados.
	\item La medición disponible contiene información suficiente
	para reconstruir el vector de estados mediante un observador.
\end{itemize}

Por lo tanto, el modelo es estructuralmente apto
para diseño mediante realimentación de estados,
ubicación arbitraria de polos,
control óptimo LQR
y estimación mediante observador de Luenberger
o filtro de Kalman.

No obstante,
la validez práctica en tiempo discreto
depende de la adecuada elección del período de muestreo \(T_s\),
de modo que la dinámica relevante del sistema
quede correctamente representada
y no se introduzcan efectos de aliasing
ni problemas numéricos asociados
a discretizaciones excesivamente finas.

% ============================================================
\subsection{Ubicación Arbitraria de Polos}
% ============================================================

\subsection{Realimentación de Estados y Estimación}

\subsubsection{Realimentación de Estados}

Considerando el modelo discreto del sistema:

\[
x_{k+1} = A x_k + B u_k
\]
\[
y_k = C x_k
\]

se propone una ley de control por realimentación de estados:

\[
u_k = -K x_k
\]

lo que conduce a la dinámica en lazo cerrado:

\[
x_{k+1} = (A - BK)x_k
\]

El diseño por ubicación arbitraria de polos consiste en determinar la
matriz de ganancia \(K\) tal que:

\[
\lambda(A - BK) = \{p_1, p_2, \dots, p_n\}
\]

siendo \(|p_i| < 1\) condición necesaria para estabilidad discreta.

La determinación de \(K\) puede realizarse mediante el método de
Ackermann o utilizando la función \texttt{place()} de MATLAB, siempre
que el sistema sea completamente controlable.

\vspace{0.3cm}

\subsubsection{Estimador de Estados}

En situaciones donde no todos los estados son medibles, se introduce
un estimador de Luenberger para reconstruir el vector de estados a
partir de la entrada y la salida medida.

El estimador discreto se define como:

\[
\hat{x}_{k+1} = A\hat{x}_k + B u_k + L (y_k - \hat{y}_k)
\]

donde:

\[
\hat{y}_k = C \hat{x}_k
\]

La dinámica del error de estimación:

\[
e_k = x_k - \hat{x}_k
\]

queda gobernada por:

\[
e_{k+1} = (A - LC)e_k
\]

Por lo tanto, la convergencia del estimador depende de la ubicación
de los autovalores de la matriz \(A - LC\), que pueden ser fijados
arbitrariamente siempre que el sistema sea observable:

\[
\lambda(A - LC) = \{p_{obs,1}, \dots, p_{obs,n}\}
\]

\vspace{0.3cm}

\subsubsection{Principio de Separación}

Cuando se combinan realimentación de estados y estimación, la ley de
control adopta la forma:

\[
u_k = -K\hat{x}_k
\]

y la dinámica total del sistema presenta autovalores dados por la
unión de los polos del controlador y los polos del estimador:

\[
\lambda_{\text{total}} =
\lambda(A - BK) \cup \lambda(A - LC)
\]

Este resultado, conocido como principio de separación, permite diseñar
independientemente el controlador y el estimador.

\balance
\clearpage
% ============================================================
\subsection{Ubicación Arbitraria de Polos con Integrador de error}
% ============================================================

En esta sección se implementa el procedimiento propuesto en Ogata (Ec. 6.19) \cite{Ogata1996}
para incorporar acción integral al sistema en espacio de estados, utilizando
únicamente un integrador externo y realimentación de estados estimados.

La planta discreta utilizada (orden $n=3$) se obtuvo mediante discretización
por ZOH con período de muestreo:

\[
T_s = 0.02\ \text{s}
\qquad (50\ \text{Hz})
\]

% ============================================================
\paragraph{Sistema aumentado con integrador}
% ============================================================

Para eliminar el error en régimen permanente frente a referencias constantes,
se define el estado integral:

\[
v_{k+1} = v_k + (r_k - y_k)
\]

donde $y_k = Cx_k$.

Siguiendo el desarrollo de Ogata, se construye el sistema aumentado
(planta + integrador) en forma estructural:

\[
\hat{A} =
\begin{bmatrix}
	A & B \\
	0 & 0
\end{bmatrix},
\qquad
\hat{B} =
\begin{bmatrix}
	0 \\
	1
\end{bmatrix}
\]

Se seleccionaron los polos deseados del sistema aumentado como:

\[
p_{\text{ctrl}} =
\{0.95 \pm 0.15j,\; 0.98\},
\qquad
p_i = 0.96
\]

La determinación de las ganancias se realizó a partir de la relación:

\[
\text{Aux} =
\begin{bmatrix}
	A - I & B \\
	C A   & C B
\end{bmatrix}
\]

\[
K_{2,1} =
\frac{K_{\text{hat}} + [\,0\;\; 0\;\; 0\;\; 1\,]}{\text{Aux}}
\]

obteniéndose las ganancias equivalentes:

\[
K_2 =
\begin{bmatrix}
	0.7820721 & -0.5678709 & 0.3779423
\end{bmatrix}
\]

\[
K_1 = 0.1469358
\]

donde:

\begin{itemize}
	\item $K_2$ actúa sobre el estado estimado $\hat{x}_k$.
	\item $K_1$ actúa sobre el estado integral $v_k$.
\end{itemize}

La ley de control implementada es:

\[
u_k = K_1 v_k - K_2 \hat{x}_k
\]

% ============================================================
\paragraph{Diseño del observador}
% ============================================================

Dado que únicamente se mide la altura, el resto de estados debe estimarse.
Se diseñaron dos variantes de observador con los mismos polos deseados:

\[
p_{\text{obs}} =
\{0.8 \pm 0.25j,\; 0.9\}
\]

\paragraph{Observador predictor}

\[
\hat{x}_{k+1}
=
A\hat{x}_k + Bu_k + L_{\text{pred}}\big(y_k - C\hat{x}_k\big)
\]

Dinámica del error:

\[
e_{k+1} = (A - L_{\text{pred}} C)\,e_k
\]

Ganancia obtenida:

\[
L_{\text{pred}} =
\begin{bmatrix}
	167.3535 \\
	253.6822 \\
	86.0310
\end{bmatrix}
\]

\paragraph{Observador actual}

\[
z_{k+1} = A\hat{x}_k + Bu_k
\]

\[
\hat{x}_{k+1} =
z_{k+1}
+
L_{\text{act}}
\big(
y_{k+1} - C z_{k+1}
\big)
\]

Dinámica del error:

\[
e_{k+1} = (A - L_{\text{act}} C A)\,e_k
\]

Ganancia obtenida:

\[
L_{\text{act}} =
\begin{bmatrix}
	126.8411 \\
	172.0621 \\
	45.1301
\end{bmatrix}
\]

% ============================================================
\paragraph{Sistema aumentado cerrado}
% ============================================================

Para analizar explícitamente la dinámica cerrada (planta + integrador),
se construyó el sistema aumentado:

\[
A_{\text{aug}} =
\begin{bmatrix}
	A & B \\
	K_2 - K_2 A - K_1 C A & 1 - K_2 B - K_1 C B
\end{bmatrix}
\]

\[
C_{\text{aug}} =
\begin{bmatrix}
	C & 0
\end{bmatrix}
\]

lo que permite visualizar el mapa polo–cero del sistema completo.

% ============================================================
\subsubsection{Resultados de simulación}
% ============================================================

\insertarfigura{img/SSi/SSi_step.png}
{Respuesta temporal $y(t)$ con integrador — comparación predictor vs actual.}
{fig:ssi_step}{1}

\insertarfigura{img/SSi/SSi_esfuerzo.png}
{Esfuerzo de control $u(t)$ con integrador.}
{fig:ssi_esfuerzo}{1}

\insertarfigura{img/SSi/SSi_pzmap.png}
{Mapa de polos: planta, sistema aumentado y dinámica del observador.}
{fig:ssi_pzmap}{1}

En simulación se observa ausencia de sobreimpulso y un tiempo de subida
aproximado de:

\[
t_r^{\text{sim}} \approx 2.6\ \text{s}
\]

tanto para el observador predictor como para el actual.

% ============================================================
\subsubsection{Resultados experimentales}
% ============================================================

\insertarfigura{img/SSi/SSi_practico_pred.png}
{Resultado experimental con observador predictor.}
{fig:ssi_practico_pred}{1}

\insertarfigura{img/SSi/SSi_practico_act.png}
{Resultado experimental con observador actual.}
{fig:ssi_practico_act}{1}

Experimentalmente se obtuvo:

\[
t_r^{\text{exp}} = 2.02\ \text{s} \quad (\text{predictor})
\]

\[
t_r^{\text{exp}} = 2.36\ \text{s} \quad (\text{actual})
\]

Se observa buena concordancia cualitativa con la simulación,
aunque el sistema real presenta mayor rapidez que la predicha por
el modelo lineal.

A diferencia de implementaciones sin acción integral,
no se observaron derivas estacionarias sostenidas.
El sistema converge naturalmente a la referencia.

El predictor mostró picos de esfuerzo más continuos,
consistentes con mayor sensibilidad al ruido de medición.
El observador actual presentó un esfuerzo más progresivo
y menos errático, atribuido al filtrado implícito de la planta
entre $u_k$ y $y_{k+1}$.

% ============================================================
\subsubsection{Discusión}
% ============================================================

La incorporación de acción integral permitió eliminar el error en régimen
permanente de forma estructural, superando limitaciones observadas cuando
se utilizaba únicamente $N_{\text{bar}}$.

En particular:

\begin{itemize}
	\item El sistema alcanza la referencia incluso ante variaciones
	de batería o pequeñas perturbaciones constantes.
	\item La respuesta práctica se aproxima notablemente a la simulada.
	\item La estabilidad discreta se mantiene, con polos dentro del
	círculo unitario.
\end{itemize}

Persisten efectos tipo derivativos asociados al observador,
producto de una ubicación de polos subóptima que amplifica ruido.
Este aspecto constituye un punto claro de mejora futura.

% ============================================================
\subsubsection{Conclusión}
% ============================================================

El método de Ogata con acción integral demostró ser
estructuralmente más robusto frente a variaciones reales de la planta
que los enfoques sin integración.

Si bien depende del modelo identificado, la inclusión del estado
integral introduce una propiedad correctiva que compensa
desajustes moderados y elimina el error estacionario
sin necesidad de correcciones adicionales por software.


En conjunto, la realimentación de estados con acción integral
proporcionó el mejor compromiso entre estabilidad, eliminación
de error permanente y coherencia entre simulación y práctica
dentro de los métodos en espacio de estados evaluados.

\balance
\clearpage
% ============================================================
\subsection{Control Óptimo (LQR)}
% ============================================================

Se define el funcional de costo:

\[
J=\sum_{k=0}^{\infty}(x_k^TQx_k+u_k^TRu_k)
\]

Matrices de ponderación:

\[
Q=\textbf{[Completar]}
\]

\[
R=\textbf{[Completar]}
\]

La solución se obtiene resolviendo la ecuación de Riccati discreta:

\[
P=A^TPA - A^TPB(R+B^TPB)^{-1}B^TPA + Q
\]

Ganancia óptima:

\[
K=(R+B^TPB)^{-1}B^TPA
\]

Polos obtenidos:

\[
\lambda(A-BK)=\textbf{[Completar]}
\]

\balance
\clearpage
% ============================================================
\subsection{Control Óptimo con Filtro de Kalman e Integrador de Error (LQGi)}
% ============================================================

% ============================================================
\subsubsection{Modelo con Ruido}
% ============================================================

Para incorporar incertidumbre y modelar explícitamente la presencia de
perturbaciones no modeladas y ruido del sensor, se adopta la siguiente
representación estocástica discreta:

\[
x_{k+1} = A x_k + B u_k + w_k
\]

\[
y_k = C x_k + v_k
\]

donde:

\begin{itemize}
	\item $w_k \sim \mathcal{N}(0,Q)$ representa el ruido de proceso,
	asociado a dinámica no modelada, perturbaciones aerodinámicas
	y simplificaciones del modelo identificado.
	\item $v_k \sim \mathcal{N}(0,R)$ representa el ruido de medición,
	proveniente del sensor láser de distancia.
\end{itemize}

% ============================================================
\paragraph{Estimación de $R$ (ruido de medición)}
% ============================================================

La varianza del ruido de medición se obtuvo mediante ensayos empíricos,
midiendo la dispersión de la señal del sensor con la planta en reposo.
Siendo $\sigma_v$ la desviación estándar medida (en cm), se adopta:

\[
R = \sigma_v^2
\]

En los ensayos realizados se obtuvo:

\[
\sigma_v = 2.043\ \text{cm}
\qquad\Longrightarrow\qquad
R = 4.174\ \text{cm}^2
\]

% ============================================================
\paragraph{Parametrización y sintonización de $Q$ (ruido de proceso)}
% ============================================================

La matriz de covarianza del ruido de proceso se parametrizó como:

\[
Q = q\,I_n
\]

donde $q$ es un escalar positivo ajustable e $I_n$ es la matriz identidad
de dimensión $n$. El valor de $q$ se determinó mediante consistencia
estadística de la innovación normalizada (ver Apéndice~\ref{ap:kalman_tuning}).

El valor óptimo obtenido fue:

\[
q = 50.8022\times 10^{-3}
\qquad\Longrightarrow\qquad
Q = q\,I_n
\]

% ============================================================
\subsubsection{Filtro de Kalman en régimen permanente}
% ============================================================

Se utilizó un estimador de Kalman discreto en su variante \textit{current estimator}
(\texttt{kalman(...,'current')}). La ganancia en régimen permanente \(L\) se obtiene
a partir de la solución estacionaria \(P\) de la ecuación de Riccati discreta:

\[
P = A P A^T - A P C^T \left(C P C^T + R\right)^{-1} C P A^T + Q
\]

y la ganancia queda:

\[
L = A P C^T \left(C P C^T + R\right)^{-1}
\]

La ganancia obtenida en MATLAB para el modelo discretizado fue:

\[
L =
\begin{bmatrix}
	97.4968\\
	192.3014\\
	94.7820
\end{bmatrix}
\]

% ============================================================
\paragraph{Polos del observador}
% ============================================================

Para el \textit{current estimator}, la dinámica del error queda
determinada por \(A - LCA\). Los polos obtenidos fueron:

\[
\lambda(A - LCA) =
\begin{aligned}
	&0.9730274 + 0.0308455\,j \\
	&0.9730274 - 0.0308455\,j \\
	&0.9694430
\end{aligned}
\]

Todos los polos se ubican dentro del círculo unitario,
garantizando estabilidad del estimador.

% ============================================================
\subsubsection{Control integral y realimentación de estados (LQGI)}
% ============================================================

Con el objetivo de eliminar el error en régimen permanente ante referencias tipo escalón,
se incorporó un integrador de error. Definiendo el estado integral \(\xi_k\):

\[
\xi_{k+1} = \xi_k + \left(r_k - y_k\right)
\]

Se diseñó una ley de control tipo LQI:

\[
u_k = -K_x\,\hat{x}_k + K_i\,\xi_k
\]

donde \(\hat{x}_k\) proviene del estimador de Kalman.

Los valores obtenidos mediante \texttt{dlqr} fueron:

\[
K_x =
\begin{bmatrix}
	16.5343 & -15.8046 & 15.1271
\end{bmatrix}
\]

\[
K_i = 3.2067
\]

% ============================================================
\paragraph{Polos de la planta y del lazo cerrado}
% ============================================================

Los polos de la planta discretizada (sin control) fueron:

\[
\lambda(A) =
\begin{aligned}
	&0.9999972 \\
	&0.9720409 + 0.0241158\,j \\
	&0.9720409 - 0.0241158\,j
\end{aligned}
\]

Los polos del lazo cerrado del sistema aumentado (planta + integrador + control) fueron:

\[
\lambda(A_{\text{cl}}) =
\begin{aligned}
	&0.9735939 + 0.0318308\,j \\
	&0.9735939 - 0.0318308\,j \\
	&0.9671966 \\
	&0
\end{aligned}
\]

Se verifica estabilidad discreta y presencia del polo en cero
asociado a la acción integral.

% ============================================================
\subsubsection{Sistema aumentado usado para \texttt{pzmap}}
% ============================================================

En el script de validación se construyó explícitamente el sistema aumentado:

\[
A_{\text{aug}} =
\begin{bmatrix}
	A & B \\
	K_x - K_xA - K_iCA & 1 - K_xB - K_iCB
\end{bmatrix}
\]

\[
B_{\text{aug}} =
\begin{bmatrix}
	0\\
	0\\
	0\\
	K_i
\end{bmatrix}
\]

\[
C_{\text{aug}} =
\begin{bmatrix}
	C & 0
\end{bmatrix}
\qquad
D_{\text{aug}} = 0
\]

Construyéndose el modelo:

\[
\texttt{sysDaug = ss(Aaug,Baug,Caug,0,Ts)}
\]

y visualizando los polos mediante \texttt{pzmap(sysDaug)}.


\insertarfigura{img/LQGi/LQGi_step_ruido.png}
{Respuesta temporal en presencia de ruido.}
{fig:lqgi_step_ruido}{1}

\insertarfigura{img/LQGi/LQGi_esfuerzo_ruido.png}
{Esfuerzo de control en presencia de ruido.}
{fig:lqgi_esfuerzo_ruido}{1}

\insertarfigura{img/LQGi/LQGi_practico.png}
{Resultado experimental sobre la planta real.}
{fig:lqgi_practico}{1}

\insertarfigura{img/LQGi/LQGi_pzmap_NaranajaOBS_AzulPLANTA.png}
{Mapa de polos: observador (naranja) y sistema (azul).}
{fig:lqgi_pzmap}{1}

% ============================================================
\subsubsection{Discusión}
% ============================================================

En régimen dinámico, el tiempo de subida experimental fue aproximadamente
\(t_r^{\text{exp}} \approx 0.65\,\text{s}\), con sobreimpulso reducido
(del orden del 10\% o menor en la mayoría de los ensayos).
La simulación con ruido predijo \(t_r^{\text{sim}} \approx 0.5\,\text{s}\)
y sobreimpulso cercano al 8\%, mostrando buena concordancia.

El esfuerzo de control en simulación presenta ciertos picos,
producto de la inyección explícita de ruido gaussiano. En la práctica,
los picos resultaron más consistentes y de menor amplitud relativa,
especialmente en comparación con implementaciones anteriores sin Kalman, cumpliendo de muy buena manera su objetivo.

Un aspecto destacable es que, gracias a la acción integral,
el error en régimen permanente converge sistemáticamente a cero,
a diferencia de prácticas previas donde persistían derivas
incluso utilizando prefiltros o compensaciones adicionales.

% ============================================================
\subsubsection{Conclusión}
% ============================================================

El esquema LQGI (LQR + Kalman + Integrador) constituye la síntesis más
completa implementada en este trabajo.

A diferencia de la simple ubicación arbitraria de polos, aquí los polos
del lazo cerrado resultan de una optimización basada en criterios
energéticos bien definidos por el ingeniero.

Aunque la selección de \(Q\) y \(R\) no es intuitiva y requiere
criterio y ajuste iterativo, una vez correctamente definidos,
el método produce resultados consistentes, robustos y coherentes
con el modelo.

La incorporación del filtro de Kalman permitió seleccionar las ganancias
del observador de forma óptima en presencia de ruido, reduciendo la
amplificación observada en enfoques anteriores.

Finalmente, la inclusión del integrador garantizó eliminación del error
estacionario y otorgó el mejor desempeño global entre todas las técnicas
implementadas, tanto en simulación como en la planta real.

\balance
\clearpage

\subsection{Comparación entre Métodos Modernos}


Los métodos modernos implementados (ubicación arbitraria de polos,
LQR, observadores por ubicación de polos y filtro de Kalman,
con y sin acción integral) demostraron en general
un desempeño satisfactorio y consistente sobre la planta real.

Sin embargo, la combinación de:

\begin{itemize}
	\item Regulador óptimo LQR,
	\item Observador basado en Filtro de Kalman,
	\item Acción integral (LQI),
\end{itemize}

representa una mejora sustancial respecto a las variantes
sin optimización explícita o sin estimación estocástica.

\paragraph{Ubicación arbitraria vs. LQR}

Desde el punto de vista computacional e implementativo,
no existe diferencia entre utilizar una ganancia
obtenida por \texttt{place()} o por \texttt{dlqr()}.
Ambos producen una matriz $K$ que se implementa exactamente
de la misma forma en firmware y no implican mayor carga
de cálculo en tiempo real ni similares.

La diferencia radica exclusivamente en el criterio de diseño:

\begin{itemize}
	\item En ubicación arbitraria, los polos se eligen manualmente.
	\item En LQR, los polos resultan de minimizar un funcional
	cuadrático que pondera explícitamente estados y esfuerzo.
\end{itemize}

Si bien la definición adecuada de las matrices $Q$ y $R$
requiere mayor reflexión y criterio ingenieril,
el resultado es un compromiso óptimo entre desempeño
y esfuerzo de control.

En sistemas donde los estados poseen interpretación física clara,
el LQR permite penalizar selectivamente variables,
algo que no puede hacerse directamente con
ubicación arbitraria de polos.

% ------------------------------------------------------------
\paragraph{Observador predictor vs. observador actual}
% ------------------------------------------------------------

Dentro de los esquemas de estimación determinística
(ubicación de polos), se evaluaron las variantes
\textit{predictor} y \textit{actual}.

El observador predictor corrige utilizando la medición $y_k$,
lo que implica una estructura ligeramente más simple
a nivel algorítmico.

El observador actual, en cambio, corrige con $y_{k+1}$,
incorporando implícitamente la dinámica de la planta
entre la predicción y la corrección.

En la práctica se observó que:

\begin{itemize}
	\item El predictor tiende a amplificar más el ruido de medición.
	\item El actual presenta un esfuerzo de control más limpio.
	\item La diferencia se vuelve más evidente cuando la señal medida
	presenta ruido significativo.
\end{itemize}

El costo computacional adicional del observador actual
es despreciable frente al período de muestreo utilizado
en este trabajo. Por lo tanto, siempre que el tiempo de cálculo
sea pequeño en comparación con $T_s$, el \textbf{observador actual}
resulta preferible debido a su efecto de filtrado natural
y mejor comportamiento práctico.
\paragraph{Observador por polos vs. Filtro de Kalman}

En cuanto a estimación de estados,
el Filtro de Kalman mostró una ventaja clara respecto
a la simple ubicación arbitraria de polos del observador.

Mientras que en el diseño por polos la selección
de ganancias es esencialmente heurística,
en Kalman las ganancias resultan de una optimización
basada en la relación estadística entre ruido de proceso ($Q$)
y ruido de medición ($R$).

En la práctica, esto se tradujo en:

\begin{itemize}
	\item Menor amplificación del ruido en el esfuerzo de control.
	\item Respuestas más limpias.
	\item Menor presencia de picos erráticos.
\end{itemize}

El principal inconveniente del Filtro de Kalman
es la necesidad de estimar adecuadamente las covarianzas
$Q$ y $R$, lo cual demanda experimentación adicional.
No obstante, una vez ajustadas,
el desempeño mejora de manera significativa.

\paragraph{Acción integral vs. Prefiltro $N_{\mathrm{bar}}$}

La incorporación del integrador constituye,
probablemente, el cambio más relevante
desde el punto de vista práctico.

El prefiltro $N_{\mathrm{bar}}$ garantiza seguimiento perfecto
únicamente bajo el modelo ideal.
En presencia de:

\begin{itemize}
	\item Derivas paramétricas,
	\item Descarga de batería,
	\item No linealidades del empuje,
	\item Envejecimiento del sistema,
\end{itemize}

la salida puede desviarse progresivamente de la referencia.

La acción integral, en cambio,
corrige sistemáticamente el error estacionario,
independientemente de pequeñas discrepancias del modelo,
siempre que el lazo permanezca estable y no exista saturación prolongada.

Aunque el integrador incrementa el orden del sistema
y agrega un estado adicional,
su impacto en recursos computacionales es despreciable
frente al beneficio obtenido en robustez de seguimiento.

% ------------------------------------------------------------
\paragraph{Conclusión comparativa}
% ------------------------------------------------------------

Todos los métodos modernos evaluados son funcionales
y técnicamente correctos.
Sin embargo, la combinación:

\[
\text{LQR} + \text{Kalman} + \text{Integrador}
\]

representa un salto cualitativo significativo.

\begin{itemize}
	\item LQR aporta un criterio óptimo explícito.
	\item Kalman reduce la contaminación del esfuerzo por ruido.
	\item El integrador elimina derivas en régimen permanente.
	\item El observador actual mejora la calidad del esfuerzo
	sin penalización práctica en recursos.
\end{itemize}

En conjunto, esta arquitectura ofrece
el mejor equilibrio entre desempeño dinámico,
robustez frente a incertidumbre
y calidad del esfuerzo de control,
constituyendo la solución más sólida
entre las implementadas en este trabajo.