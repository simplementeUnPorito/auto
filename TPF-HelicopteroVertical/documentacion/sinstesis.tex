\section{Código de síntesis directa y simulación intramuestra}
\label{ap:codigo_TR}

A continuación se presenta el script utilizado para la
discretización de la planta, resolución del sistema
polinómico asociado al diseño ripple-free y evaluación
intramuestra mediante retención de orden cero.

\begin{lstlisting}[language=Matlab]
	close all; clear; clc;
	
	S = load("D:\auto\TPF-HelicopteroVertical\Matlab\planta (1).mat");
	
	if isfield(S,'plantaC')
	plantaC = S.plantaC;
	elseif isfield(S,'sysC')
	plantaC = S.sysC;
	elseif isfield(S,'G')
	plantaC = S.G;
	else
	error("No se encontró la planta en el archivo .mat.");
	end
	
	Ts = 1/2;  % Fs = 2 Hz
	[num, den] = tfdata(c2d(plantaC,Ts,'zoh'),'v');
	
	n = length(num)-1;
	m = length(den)-1;
	k = max(n,m);
	
	num_z = [num zeros(1,k-n)];
	den_z = [den zeros(1,k-m)];
	
	Hz = tf(num_z, den_z, Ts, 'Variable','z');
	
	z = tf([1 0],1,Ts);
	
	% --- Controlador sin oscilaciones intramuestra ---
	C1 = (1.73*z^2 - 0.2763*z + 0.1047) / ...
	(z^2 + 0.6683*z + 0.04352);
	
	% --- Controlador con oscilaciones intramuestra ---
	C2 = (z^3 - 1.16*z^2 + 0.2202*z - 0.0605) / ...
	(0.1917*z^5 + 0.3612*z^4 + 0.02515*z^3 ...
	-0.1917*z^2 -0.3612*z -0.02515);
	
	% --- Simulación intramuestra ---
	[ud,n] = step(feedback(C1,Hz));
	uc = repelem(ud,40);
	dt = Ts/40;
	t  = (0:length(uc)-1)' * dt;
	[yc,t] = lsim(plantaC, uc, t);
\end{lstlisting}
