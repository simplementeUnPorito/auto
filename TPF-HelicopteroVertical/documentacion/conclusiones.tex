

La implementación de métodos clásicos y modernos sobre una misma
planta física permitió contrastar dos filosofías de diseño claramente
diferenciadas: una basada en aproximaciones entrada–salida y otra
basada en modelado estructural completo en espacio de estados.

Los \textbf{métodos clásicos} demostraron ser progresivamente más
sofisticados en análisis, pero mantienen una característica común:
son esencialmente aproximativos y heurísticos.
El PID ajusta directamente la respuesta observable;
Lugar de Raíces y Bode aportan mayor estructura,
pero continúan trabajando sobre simplificaciones
(dinámicas dominantes, márgenes, aproximaciones de segundo orden).
Incluso la Síntesis Directa, aunque analíticamente rigurosa,
termina siendo extremadamente dependiente del modelo ideal.

Su principal fortaleza es la robustez práctica:
al no apoyarse completamente en la estructura interna del sistema,
toleran mejor imprecisiones del modelo y variaciones paramétricas.
Son, en esencia, más “plug and play” y requieren menor
conocimiento profundo de la planta.

En contraste, los \textbf{métodos modernos} trasladan el núcleo del diseño
al modelo en espacio de estados.
Aquí ya no se moldea únicamente la salida,
sino la dinámica interna completa del sistema.
Esto implica una dependencia mucho mayor del modelo:
cualquier error de identificación se refleja directamente
en el desempeño del lazo cerrado.

Sin embargo, cuando el modelo es representativo,
el salto cualitativo es evidente.
La realimentación de estados permite ubicar dinámicas internas;
el LQR elimina la arbitrariedad en la selección de polos,
basando el diseño en un criterio óptimo explícito; y
el Filtro de Kalman reduce sistemáticamente la contaminación del esfuerzo por ruido.

La combinación:

\[
\text{LQR} + \text{Kalman} + \text{Integrador}
\]

representó el mejor desempeño global obtenido en el trabajo.
Este salto no provino de “hacer más cálculos por hacerlos”,
sino de incorporar estructura explícita (óptimo + estimación + rechazo de perturbaciones).
Aun así, el costo computacional \textbf{sí cambia} respecto a los métodos clásicos:
mientras un PID o un compensador discreto típico se implementan con unas pocas operaciones
(una ecuación en diferencias de bajo orden),
los métodos modernos requieren, por muestra,
\textbf{propagación de estados} ($A\hat{x}$),
\textbf{actualización del observador} (términos con $L$ y mediciones),
y en el caso del integrador una \textbf{acumulación adicional}.

En este trabajo, con $n=3$ y frecuencias de muestreo moderadas,
esa carga adicional resultó completamente abordable,
pero la comparación deja una lección general:
en sistemas con estados grandes ($n$ alto),
muestreos muy rápidos o microcontroladores muy limitados,
este incremento puede volverse un \textit{dealbreaker} y empujar a soluciones clásicas
más simples, aunque menos potentes.

La diferencia fundamental entre ambas vertientes no es
de implementación, sino de filosofía:
los métodos clásicos aceptan la incertidumbre y compensan
mediante ajustes progresivos;
los métodos modernos intentan modelar y optimizar explícitamente
la dinámica interna completa.

El aprendizaje central radica en comprender que
ningún método es universalmente superior.
Los enfoques clásicos ofrecen rapidez y tolerancia a la imprecisión.
Los enfoques modernos ofrecen mayor grado de libertad,
mejor desempeño potencial y control explícito sobre
ruido, esfuerzo y error estacionario,
a costa de mayor dependencia del modelo y criterio en su formulación.

En esta planta experimental,
la arquitectura moderna completa mostró una ventaja clara,
pero también evidenció que el éxito del diseño
depende tanto de la calidad del modelo
como del criterio ingenieril aplicado en su utilización.
