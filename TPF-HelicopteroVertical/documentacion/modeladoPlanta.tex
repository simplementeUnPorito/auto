\subsection{Variables y convenciones}

Se definen a continuación las variables y convenciones utilizadas para el modelado del sistema:

\begin{itemize}
	\item Eje vertical \(z\) [m], definido positivo hacia arriba.
	
	\item Masa móvil:
	\[
	m = 0.360\ \text{kg}
	\]
	
	\item Aceleración de la gravedad:
	\[
	g = 9.81\ \text{m/s}^2
	\]
	
	\item Peso del cuerpo móvil:
	\[
	mg = 3.924\ \text{N}
	\]
	
	\item Entrada del sistema: señal PWM tipo servo a \(50\ \text{Hz}\),
	\[
	u \in [1000,2000]\ \mu s
	\]
	
	Si bien el rango eléctrico nominal del protocolo PWM se encuentra entre
	$1000$ y $2000\ \mu s$, durante la operación en vuelo la señal se restringe
	intencionalmente al intervalo
	
	\[
	u \in [1100,1700]\ \mu s
	\]
	
	Esta limitación surge de criterios de confiabilidad experimental.
	En ensayos previos realizados con ESCs de menor capacidad nominal
	($30\ \text{A}$), se observaron fallas térmicas al operar en valores
	superiores a aproximadamente $1600\ \mu s$ bajo carga sostenida.
	
	Dado que no se realizaron mediciones directas de corriente ni
	caracterizaciones térmicas detalladas del conjunto motor--ESC--hélice,
	se adoptó un margen de seguridad conservador que evita la operación
	prolongada en regímenes de alta demanda energética.
	
	Asimismo, el límite inferior de $1100\ \mu s$ se fija con el objetivo
	de evitar regiones cercanas a la detención del motor, donde pueden
	presentarse comportamientos fuertemente no lineales y pérdida abrupta
	de sustentación.
	
	En consecuencia, el modelo identificado y las estrategias de control
	desarrolladas se consideran válidos únicamente dentro de este rango
	operativo seguro.
	
	\item Normalización de la entrada:
	\[
	\hat u = \mathrm{sat}\!\left(u-u_0,\,1100-u_0,\,1700-u_0\right)
	\]
	
	Siendo $u_0$ el valor de PWM necesario para generar un empuje
	equivalente al peso del sistema, es decir, la condición de equilibrio
	vertical estacionario.
	
	\item Salida medida:
	\[
	y = z
	\]
	
	correspondiente a la altura del cuerpo móvil medida mediante el sensor
	TFMini Plus, cuya resolución efectiva se encuentra en el orden de
	centímetros.
\end{itemize}

\subsubsection{Modelo físico simplificado}

Desde un punto de vista físico, el movimiento vertical del cuerpo móvil puede describirse, en primera aproximación, mediante:

\[
\dot z = v
\]

\[
m\dot v = T - mg
\]

Este modelo corresponde al caso ideal sin disipación, en el cual la
dinámica posición--empuje presenta una estructura de doble integración.

En una aproximación más realista, puede incorporarse el efecto de
rozamiento de las guías mediante un término viscoso proporcional a la
velocidad:

\[
m\dot v = T - mg - b\,v
\]

donde \(b\) [N\,s/m] representa un coeficiente equivalente de fricción
viscosa.

La inclusión de este término modifica la estructura ideal de doble
integrador, convirtiéndola en un sistema con integración amortiguada.
Desde el punto de vista de función de transferencia continua, el
doble polo en el origen deja de ser estrictamente doble, introduciéndose
un amortiguamiento mecánico que desplaza uno de los polos hacia el
semiplano izquierdo.

Por lo tanto, la presencia de un único integrador dominante en el modelo
simplificado resulta coherente con la física del sistema cuando se
consideran pérdidas mecánicas.

\subsubsection{Modelo dinámico equivalente del actuador (BLDC)}

A efectos de análisis lineal, el conjunto ESC--motor BLDC puede
aproximarse mediante un modelo promedio equivalente al de un motor DC
de imanes permanentes:

\[
v(t)=R\,i(t) + L\,\dot i(t) + K_e\,\omega(t)
\]

\[
J\,\dot \omega(t)=K_t\,i(t) - B\,\omega(t) - \tau_L(t)
\]

Este modelo introduce una dinámica electromecánica adicional respecto
del modelo puramente mecánico de la masa móvil, justificando la
aparición de un polo adicional en la relación entrada--salida.

En muchos casos, la constante de tiempo eléctrica
\(\tau_e=L/R\) es considerablemente menor que la mecánica,
permitiendo despreciar \(L\) y obtener una dinámica dominante de
segundo orden asociada al actuador.

\subsubsection{Función de transferencia continua equivalente}

La representación continua obtenida a partir del modelo identificado
puede expresarse inicialmente como:

\[
G(s)=
\frac{-0.12107\,(s-14)(s+10.62)}
{(s+0.0002797)\,(s^2+5.61s+14.02)}
\]

El polo ubicado en \(s=-0.0002797\) corresponde a una dinámica
extremadamente lenta respecto de las restantes constantes de tiempo
del sistema.

Considerando que:

\begin{itemize}
	\item la resolución del sensor de altura se encuentra en el orden
	de centímetros,
	\item no se dispone de mediciones de alta precisión submilimétrica,
	\item el identificador polinomial puede ajustar polos muy cercanos
	al origen para capturar pequeñas tendencias de deriva,
\end{itemize}

se interpreta que dicho polo próximo a cero no representa una dinámica
física real dominante, sino un posible efecto de sobreajuste
(\textit{overfitting}) del procedimiento de identificación.

En consecuencia, y en coherencia con el modelo físico que contempla
amortiguamiento mecánico, se adopta la siguiente simplificación:

\[
G(s)=
\frac{-0.12107\,(s-14)(s+10.62)}
{s\,(s^2+5.61s+14.02)}
\]

En esta forma:

\begin{itemize}
	\item El polo en el origen representa el carácter integrador dominante
	de la posición vertical.
	\item El segundo orden \(s^2+5.61s+14.02\) modela la dinámica
	electromecánica del actuador.
	\item Los ceros se interpretan como parámetros de ajuste que capturan
	efectos agregados del actuador, discretización y linealización
	alrededor del punto de operación.
\end{itemize}

\subsubsection{Validez del modelo}

El modelo adoptado constituye una aproximación coherente con la física
del sistema y adecuada para el diseño de control dentro del rango
operativo \([1100,1700]\ \mu s\).

Fuera de dicho intervalo, el sistema presenta comportamientos no lineales
significativos (aerodinámica, fricción no lineal, saturaciones y posibles
limitaciones térmicas del actuador) que no son capturados por el modelo
lineal simplificado.
