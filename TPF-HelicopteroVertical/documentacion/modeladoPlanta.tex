\subsection{Variables y convenciones}

Se definen a continuación las variables y convenciones utilizadas para el modelado del sistema:

\begin{itemize}
	\item Eje vertical \(z\) [m], definido positivo hacia arriba.
	\item Masa móvil:
	\[
	m = 0.360\ \text{kg}
	\]
	\item Aceleración de la gravedad:
	\[
	g = 9.81\ \text{m/s}^2
	\]
	\item Peso del cuerpo móvil:
	\[
	mg = 3.924\ \text{N}
	\]
	\item Entrada del sistema: señal PWM tipo servo a \(50\ \text{Hz}\),
	\[
	u \in [1000,2000]\ \mu s
	\]
	\item Normalización de la entrada:
	\[
	\hat u = \mathrm{sat}\!\left(\frac{u-1300}{1000},\,0,\,1\right)
	\]
	\item Salida medida:
	\[
	y = z
	\]
	correspondiente a la altura del cuerpo móvil medida mediante el sensor TFMini Plus.
\end{itemize}

\subsubsection{Modelo físico simplificado}

Desde un punto de vista físico, el movimiento vertical del cuerpo móvil puede describirse, en primera aproximación, mediante las siguientes ecuaciones:
\[
\dot z = v
\]
\[
m\dot v = T - mg
\]
donde \(T\) representa el empuje efectivo generado por el conjunto ESC--motor--hélice. Este modelo corresponde a una descripción idealizada del sistema y no contempla efectos tales como fricción, aerodinámica ni la dinámica interna del actuador.

Si bien este modelo permite capturar la estructura básica del sistema, resulta insuficiente para reproducir con precisión las dinámicas observadas durante las pruebas experimentales. Las principales fuentes de discrepancia se asocian a:
\begin{itemize}
	\item variaciones de masa efectiva y efectos de fricción mecánica,
	\item fenómenos aerodinámicos no lineales asociados al empuje generado por la hélice,
	\item dinámica propia del sistema de actuación, compuesta por el controlador ESC y el motor brushless.
\end{itemize}

\subsubsection{Identificación experimental del sistema}

Con el objetivo de obtener un modelo más representativo del comportamiento real de la planta, se recurrió a técnicas de identificación de sistemas utilizando la \textit{System Identification Toolbox} de MATLAB. A partir de datos experimentales de entrada--salida, se identificó un modelo discreto basado en estructuras polinomiales del tipo \textit{Box--Jenkins} / \textit{Polynomial Function}, seleccionadas por su capacidad para capturar dinámicas de mayor orden y retardos efectivos.

Durante el proceso de identificación se aplicaron escalones de amplitud suficiente en la señal PWM, con el fin de excitar los modos dominantes del sistema. Para el procesamiento de los datos se seleccionaron intervalos de operación aproximadamente lineales, evitando regiones dominadas por saturaciones, fricción estática u otros efectos fuertemente no lineales. Asimismo, se aplicaron técnicas de filtrado para reducir la influencia del ruido de medición, sin alterar la dinámica dominante.

\subsubsection{Función de transferencia identificada}

El modelo discreto resultante, que recuerda la relación entre la entrada \(u_1\) y la salida \(y_1\), queda expresado en términos de potencias de \(z^{-1}\) como:

\begin{equation}
	\label{eq:Gz_identificada}
	\scriptsize
	\begin{aligned}
		G(z) &= \frac{Y_1(z)}{U_1(z)} \\
		&= \frac{-1.205\times 10^{-4}z^{-1}+2.415\times 10^{-4}z^{-2}-1.209\times 10^{-4}z^{-3}}
		{1-2.994z^{-1}+2.989z^{-2}-0.9944z^{-3}}
	\end{aligned}
\end{equation}
\normalsize


Este modelo presenta un retardo de una muestra, evidenciado por el término dominante \(z^{-1}\) en el numerador, así como una dinámica de tercer orden en el denominador. La presencia de polos cercanos a \(z=1\) es consistente con el carácter integrador del sistema de altura, donde la posición resulta de la doble integración de la aceleración.

El polo adicional identificado respecto del modelo físico ideal puede asociarse a la dinámica electromecánica del conjunto motor--ESC--hélice, mientras que los ceros del sistema no pudieron ser vinculados de manera concluyente a un único fenómeno físico. No obstante, su presencia resulta compatible con efectos aerodinámicos y con la dinámica interna del sistema de actuación.

\subsubsection{Validez del modelo}

El modelo identificado constituye una aproximación válida del comportamiento del sistema en el rango de operación considerado y resulta adecuado para el diseño y la evaluación de las estrategias de control implementadas en este trabajo. Sin embargo, fuera de dicho rango, el sistema presenta comportamientos no lineales y efectos dependientes del punto de operación que no son capturados por el modelo lineal identificado.
