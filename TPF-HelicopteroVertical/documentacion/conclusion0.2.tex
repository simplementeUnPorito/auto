
Los métodos modernos implementados (ubicación arbitraria de polos,
LQR, observadores por ubicación de polos y filtro de Kalman,
con y sin acción integral) demostraron en general
un desempeño satisfactorio y consistente sobre la planta real.

Sin embargo, la combinación de:

\begin{itemize}
	\item Regulador óptimo LQR,
	\item Observador basado en Filtro de Kalman,
	\item Acción integral (LQI),
\end{itemize}

representa una mejora sustancial respecto a las variantes
sin optimización explícita o sin estimación estocástica.

\paragraph{Ubicación arbitraria vs. LQR}

Desde el punto de vista computacional e implementativo,
no existe diferencia entre utilizar una ganancia
obtenida por \texttt{place()} o por \texttt{dlqr()}.
Ambos producen una matriz $K$ que se implementa exactamente
de la misma forma en firmware y no implican mayor carga
de cálculo en tiempo real ni similares.

La diferencia radica exclusivamente en el criterio de diseño:

\begin{itemize}
	\item En ubicación arbitraria, los polos se eligen manualmente.
	\item En LQR, los polos resultan de minimizar un funcional
	cuadrático que pondera explícitamente estados y esfuerzo.
\end{itemize}

Si bien la definición adecuada de las matrices $Q$ y $R$
requiere mayor reflexión y criterio ingenieril,
el resultado es un compromiso óptimo entre desempeño
y esfuerzo de control.

En sistemas donde los estados poseen interpretación física clara,
el LQR permite penalizar selectivamente variables,
algo que no puede hacerse directamente con
ubicación arbitraria de polos.

% ------------------------------------------------------------
\paragraph{Observador predictor vs. observador actual}
% ------------------------------------------------------------

Dentro de los esquemas de estimación determinística
(ubicación de polos), se evaluaron las variantes
\textit{predictor} y \textit{actual}.

El observador predictor corrige utilizando la medición $y_k$,
lo que implica una estructura ligeramente más simple
a nivel algorítmico.

El observador actual, en cambio, corrige con $y_{k+1}$,
incorporando implícitamente la dinámica de la planta
entre la predicción y la corrección.

En la práctica se observó que:

\begin{itemize}
	\item El predictor tiende a amplificar más el ruido de medición.
	\item El actual presenta un esfuerzo de control más limpio.
	\item La diferencia se vuelve más evidente cuando la señal medida
	presenta ruido significativo.
\end{itemize}

El costo computacional adicional del observador actual
es despreciable frente al período de muestreo utilizado
en este trabajo. Por lo tanto, siempre que el tiempo de cálculo
sea pequeño en comparación con $T_s$, el \textbf{observador actual}
resulta preferible debido a su efecto de filtrado natural
y mejor comportamiento práctico.
\paragraph{Observador por polos vs. Filtro de Kalman}

En cuanto a estimación de estados,
el Filtro de Kalman mostró una ventaja clara respecto
a la simple ubicación arbitraria de polos del observador.

Mientras que en el diseño por polos la selección
de ganancias es esencialmente heurística,
en Kalman las ganancias resultan de una optimización
basada en la relación estadística entre ruido de proceso ($Q$)
y ruido de medición ($R$).

En la práctica, esto se tradujo en:

\begin{itemize}
	\item Menor amplificación del ruido en el esfuerzo de control.
	\item Respuestas más limpias.
	\item Menor presencia de picos erráticos.
\end{itemize}

El principal inconveniente del Filtro de Kalman
es la necesidad de estimar adecuadamente las covarianzas
$Q$ y $R$, lo cual demanda experimentación adicional.
No obstante, una vez ajustadas,
el desempeño mejora de manera significativa.

\paragraph{Acción integral vs. Prefiltro $N_{\mathrm{bar}}$}

La incorporación del integrador constituye,
probablemente, el cambio más relevante
desde el punto de vista práctico.

El prefiltro $N_{\mathrm{bar}}$ garantiza seguimiento perfecto
únicamente bajo el modelo ideal.
En presencia de:

\begin{itemize}
	\item Derivas paramétricas,
	\item Descarga de batería,
	\item No linealidades del empuje,
	\item Envejecimiento del sistema,
\end{itemize}

la salida puede desviarse progresivamente de la referencia.

La acción integral, en cambio,
corrige sistemáticamente el error estacionario,
independientemente de pequeñas discrepancias del modelo,
siempre que el lazo permanezca estable y no exista saturación prolongada.

Aunque el integrador incrementa el orden del sistema
y agrega un estado adicional,
su impacto en recursos computacionales es despreciable
frente al beneficio obtenido en robustez de seguimiento.

% ------------------------------------------------------------
\paragraph{Conclusión comparativa}
% ------------------------------------------------------------

Todos los métodos modernos evaluados son funcionales
y técnicamente correctos.
Sin embargo, la combinación:

\[
\text{LQR} + \text{Kalman} + \text{Integrador}
\]

representa un salto cualitativo significativo.

\begin{itemize}
	\item LQR aporta un criterio óptimo explícito.
	\item Kalman reduce la contaminación del esfuerzo por ruido.
	\item El integrador elimina derivas en régimen permanente.
	\item El observador actual mejora la calidad del esfuerzo
	sin penalización práctica en recursos.
\end{itemize}

En conjunto, esta arquitectura ofrece
el mejor equilibrio entre desempeño dinámico,
robustez frente a incertidumbre
y calidad del esfuerzo de control,
constituyendo la solución más sólida
entre las implementadas en este trabajo.