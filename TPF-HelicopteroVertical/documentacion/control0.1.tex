% ============================================================
\subsection{Introducción}
% ============================================================

Los métodos clásicos de control se fundamentan en el análisis de sistemas lineales mediante funciones de transferencia y herramientas del dominio de la frecuencia y del plano complejo.

En este enfoque, la dinámica del sistema se representa como:

\[
G(s)=\frac{Y(s)}{U(s)}
\quad \text{o en tiempo discreto} \quad
G(z)=\frac{Y(z)}{U(z)}
\]

El objetivo del diseño consiste en definir un controlador \(C\) tal que el lazo cerrado:

\[
G_{cl}=\frac{C G}{1+CG}
\]

cumpla simultáneamente:

\begin{itemize}
	\item Estabilidad.
	\item Respuesta transitoria adecuada.
	\item Error estacionario reducido.
	\item Robustez frente a incertidumbres.
\end{itemize}

En este trabajo se implementaron y evaluaron las siguientes técnicas clásicas:

\begin{itemize}
	\item Controlador PID.
	\item Diseño por Lugar de Raíces.
	\item Diseño mediante Respuesta en Frecuencia (Bode).
	\item Síntesis Directa (Truxal--Ragazzini).
\end{itemize}

% ============================================================
\subsection{Modelo de la Planta}
% ============================================================

El diseño clásico parte de la función de transferencia discreta identificada:

\[
G(z) = \frac{
	b_0 + b_1 z^{-1} + b_2 z^{-2} + b_3 z^{-3}
}{
	1 + a_1 z^{-1} + a_2 z^{-2} + a_3 z^{-3}
}
\]

El tiempo de muestreo utilizado depende de la experiencia considerada, ya que los distintos controladores fueron desarrollados en paralelo por los miembros del grupo con diferentes enfoques de diseño.

% ============================================================
\subsection{Criterio general de estabilidad}
% ============================================================

Para sistemas discretos, la estabilidad en lazo cerrado requiere que todos los polos satisfagan:

\[
|z_i| < 1
\]

Este criterio será verificado en cada uno de los métodos desarrollados.

% ============================================================
\subsection{Controlador PID}
% ============================================================


Durante las etapas iniciales de diseño se intentó sintonizar un controlador PID utilizando la herramienta \texttt{PID Tuner} de MATLAB. Sin embargo, el desempeño obtenido no resultó adecuado para la planta bajo estudio, por lo que se decidió adoptar una formulación alternativa que permitiera un mayor control estructural sobre el comportamiento dinámico del sistema.

En consecuencia, se implementó un controlador PID basado en la formulación propuesta por Åström, directamente en su versión discreta. Esta decisión permitió diseñar el controlador coherentemente con el tiempo de muestreo del sistema, evitando discretizaciones posteriores y manteniendo consistencia entre simulación e implementación embebida.

El controlador opera en coordenadas relativas al punto de hover previamente estimado, es decir, la señal de control generada corresponde a una variación \(\Delta u\) respecto del equilibrio.

% ============================================================
\subsubsection{Formulación del PID de Åström \cite{PID-TESIS}}
% ============================================================

La estructura implementada separa explícitamente las acciones proporcional, integral y derivativa.

La acción proporcional se define como:

\[
P(k) = K\big(b\,r(k) - y(k)\big)
\]

donde \(K\) es la ganancia proporcional y \(b\) pondera la contribución de la referencia en la acción proporcional. En este trabajo se adoptó deliberadamente \(b=1\) por simplicidad estructural, evitando introducir grados adicionales de libertad innecesarios.

La acción derivativa se implementa mediante un filtro de primer orden:

\[
D(k) =
\frac{T_d}{T_d + N h}\,D(k-1)
-
\frac{K T_d N}{T_d + N h}\big(y(k) - y(k-1)\big)
\]

donde \(T_d\) es la constante derivativa, \(N\) limita el ancho de banda del término derivativo y \(h\) es el período de muestreo.

La inclusión del parámetro \(N\) resulta fundamental para evitar la amplificación excesiva del ruido de medición en altas frecuencias, fenómeno relevante dado que el sensor TFMini presenta cuantización del orden de centímetros.

La acción integral se describe como:

\[
I(k) = I(k-1) + \frac{K h}{T_i}\,e(k)
\]

donde \(T_i\) es la constante integral y \(e(k)=r(k)-y(k)\) es el error de control.

La señal de control total es:

\[
u(k) = P(k) + I(k) + D(k)
\]

% ============================================================
\subsubsection{Antiwindup}
% ============================================================

La acción integral se encuentra condicionada mediante un esquema de antiwindup por integración condicional. El término integral se actualiza únicamente cuando la señal de control no se encuentra saturada, o cuando el error contribuye a desaturar el actuador.

Este mecanismo evita acumulación indebida del estado integral y previene comportamientos abruptos ante saturaciones del PWM, mejorando la estabilidad práctica del sistema.

% ============================================================
\subsubsection{Criterios de sintonización}
% ============================================================

La sintonización se realizó de forma iterativa directamente sobre la estructura discreta del controlador, variando los parámetros en el siguiente orden:

\begin{enumerate}
	\item Ajuste de la ganancia proporcional \(K\) hasta aproximar el sistema al límite de estabilidad para obtener una respuesta rápida.
	\item Incorporación y ajuste del término integral \(T_i\) para eliminar error estacionario sin introducir oscilaciones excesivas.
	\item Incorporación del término derivativo \(T_d\) para mejorar amortiguamiento y reducir sobreimpulso.
	\item Ajuste del parámetro \(N\) como compromiso entre efectividad de la acción derivativa y rechazo de ruido.
\end{enumerate}

Durante todo el proceso se monitoreó cuidadosamente el esfuerzo de control. Como restricción experimental de diseño se impuso que la variación de la señal PWM no excediera aproximadamente \(10\,\mu s\) por centímetro de incremento en la altura, garantizando que el actuador no ingresara en saturación ni se expusiera la planta a condiciones potencialmente dañinas.

% ============================================================
\subsubsection{Resultados}
% ============================================================

Los parámetros implementados fueron:

\[
K_p = 2.5, \quad
T_i = 5, \quad
T_d = 0.1, \quad
N = 3
\]

Con un tiempo de muestreo $T_s = 1ms$.

En la simulación del modelo lineal se obtuvo:

\[
\%OS_{\text{sim}} \approx 46\%, \quad
t_r^{\text{sim}} = 0.397 \text{ s}
\]

En la implementación experimental se observaron los siguientes valores de sobreimpulso según la altura de referencia:

\begin{itemize}
	\item Para $57\,\text{cm}$: $\%OS = 28.07\%$
	\item Para $78\,\text{cm}$: $\%OS = 11.54\%$
	\item Para $90\,\text{cm}$: $\%OS \approx 0\%$
\end{itemize}

El tiempo de subida experimental fue:

\[
t_r^{\text{exp}} = 0.393 \text{ s}
\]

Se observa una coincidencia prácticamente exacta entre el tiempo de subida simulado y el experimental, lo cual valida la capacidad del modelo lineal para capturar la dinámica dominante del sistema.

Por otra parte, el sobreimpulso experimental disminuye progresivamente a medida que aumenta la altura de operación. Este comportamiento se asocia a variaciones del punto de operación y a no linealidades no completamente capturadas por el modelo lineal identificado.

\insertarfigura{img/PID/RstaPID.png}{Respuesta al escalón con el controlador PID.}{fig:rstaPID}{1}

\insertarfigura{img/PID/EsfuerzoPID.png}{Esfuerzo de control con el PID implementado.}{fig:EsfPID}{1}

\insertarfigura{img/PID/PID_Practico.png}
{Implementación práctica del controlador PID.}
{fig:PID_practica}{1}
% ============================================================
\subsubsection{Conclusiones sobre el método}
% ============================================================

Desde el punto de vista práctico, el controlador PID basado en la formulación de Åström demostró ser una herramienta sumamente eficaz para el control de la planta.

Uno de los aspectos más destacables es la relativa simplicidad con la que puede obtenerse una respuesta dinámica deseada. A diferencia de otros enfoques que requieren un conocimiento detallado de la estructura interna del sistema, el PID permite alcanzar un comportamiento satisfactorio mediante ajuste iterativo de un número reducido de parámetros, sin necesidad de un modelado exhaustivo ni de una comprensión profunda de todos los fenómenos físicos involucrados.

En este trabajo, aun considerando la presencia de dinámicas no modeladas, el PID logró reproducir con notable precisión el tiempo de subida predicho por el modelo y mantener estabilidad robusta en la implementación experimental.

Las diferencias observadas en el sobreimpulso entre simulación y práctica pueden atribuirse principalmente a características no modeladas ya mencionadas con anterioridad. Sin embargo, incluso bajo estas condiciones, el comportamiento general del sistema se mantuvo cercano al previsto teóricamente.

Asimismo, la inclusión del parámetro \(N\) en la acción derivativa permitió limitar la amplificación de ruido de alta frecuencia, resultando en un esfuerzo de control significativamente más limpio que el obtenido mediante otras metodologías analizadas más adelante. Este aspecto resulta particularmente atractivo en una gran variedad de casos reales.

En síntesis, el controlador PID demostró ofrecer una solución de alta relación beneficio–complejidad: requiere bajo conocimiento estructural del sistema, es sencillo de implementar en hardware embebido y permite obtener un desempeño dinámico competitivo dentro del rango operativo seguro de la planta.
\balance

\balance
\clearpage

% ============================================================
\subsection{Diseño por Lugar de Raíces}
% ============================================================


Dado que la planta identificada presenta polos ubicados fuera del círculo unitario, el sistema en lazo abierto resulta inestable en el dominio discreto. En consecuencia, cualquier diseño de control debe garantizar que los polos del lazo cerrado queden estrictamente dentro del círculo unitario para asegurar estabilidad interna.

\insertarfigura{img/CircUnitarioPlanta.png}
{Ubicación de los polos de la planta sin compensar en el plano $z$. Se observa que al menos uno de ellos se encuentra fuera del círculo unitario, lo que implica inestabilidad discreta.}
{fig:PolosPlantaSinCompensar}{1}

El objetivo del diseño consistió en modificar la dinámica del sistema mediante compensación, de modo que:

\begin{itemize}
	\item todos los polos del lazo cerrado queden dentro del círculo unitario,
	\item se logre un compromiso adecuado entre rapidez de respuesta y amortiguamiento,
	\item el esfuerzo de control permanezca dentro de límites físicamente realizables.
\end{itemize}

\subsubsection{Elección de la estructura del compensador}

Para estabilizar el sistema se adoptó una estructura de tipo \textbf{lag--lead} (atraso--adelanto). Esta configuración permite actuar simultáneamente sobre la estabilidad relativa y el desempeño en régimen permanente.

El término \textit{lead} (adelanto) se empleó para aumentar el margen de fase y desplazar los polos dominantes del lazo cerrado hacia regiones del plano \(z\) asociadas con mayor amortiguamiento y mejor desempeño transitorio. Por otro lado, el término \textit{lag} (atraso) permitió ajustar la ganancia en bajas frecuencias, mejorando el comportamiento estacionario sin comprometer significativamente la estabilidad.

La adecuada ubicación de ceros permitió modificar la geometría del lugar de raíces, atrayendo las ramas hacia la región estable del plano discreto, mientras que los polos adicionales modelaron el compromiso dinámico requerido.

\subsubsection{Determinación de la ganancia \(K\)}

Una vez definida la estructura del compensador, se analizó el lugar de raíces del sistema compensado. La determinación manual de la ganancia \(K\) resultó particularmente sensible, ya que pequeños incrementos en su valor provocaban que las trayectorias de los polos abandonaran el círculo unitario antes de satisfacer las especificaciones dinámicas deseadas.

Esta sensibilidad está directamente relacionada con la naturaleza inestable de la planta y con la fuerte dependencia de la ubicación de los polos del lazo cerrado respecto a la ganancia del compensador.

El compensador finalmente adoptado fue:

\begin{equation}
	C(z)=
	-0.0173 \;
	\frac{(z-1.0140)(z-0.5)}
	{(z-0.9522)(z-0.9894)}
\end{equation}

\subsubsection{Ajuste mediante \texttt{Optimization-Based Tuning}}

Con el fin de sistematizar el proceso de ajuste y garantizar el cumplimiento simultáneo de múltiples especificaciones (estabilidad, rapidez y esfuerzo de control), se utilizó la herramienta \texttt{Optimization-Based Tuning} de MATLAB.

Este enfoque permitió:

\begin{itemize}
	\item definir directamente especificaciones temporales (tiempo de establecimiento, sobreimpulso, etc.),
	\item ajustar automáticamente los parámetros del compensador,
	\item verificar la estabilidad del sistema en el dominio discreto.
\end{itemize}

El resultado fue un compensador lag--lead cuyos parámetros fueron obtenidos mediante optimización numérica, asegurando que los polos del lazo cerrado se mantengan dentro del círculo unitario y que el desempeño temporal cumpla con los objetivos establecidos para la planta experimental.

\insertarfigura{img/Rlocus/circuloUnitario.png}
{Ubicación de los polos del sistema compensado en el plano $z$. Se verifica que todos ellos se encuentran dentro del círculo unitario, garantizando estabilidad discreta.}
{fig:PolosSistemaCompensado}{1}

\insertarfigura{img/Rlocus/Esfuerzo.png}
{Esfuerzo de control en lazo cerrado con el compensador diseñado mediante Lugar de Raíces.}
{fig:EsfuerzoControlCompensado}{1}

\insertarfigura{img/Rlocus/rsta.png}
{Respuesta temporal del sistema en lazo cerrado con el compensador diseñado.}
{fig:RespuestaTemporalCompensada}{1}

\subsection{Práctica}
a completar lo que sale en la práctica
\insertarfigura{img/PID/RstaPIDPractica.png}{La respuesta con el controlador PID.}{fig:RstaPID_practica}{1}

\insertarfigura{img/PID/EsfuerzoPIDPractica.png}{El esfuerzo del con el controlador PID.}{fig:EsfPID_practica}{1}

\clearpage
\balance
% ============================================================
\subsection{Diseño por Respuesta en Frecuencia}
% ============================================================


\insertarfigura{img/Bode/bodeMat.png}
{Respuesta en frecuencia del lazo abierto del sistema identificado sin compensación.}
{fig:bodeMat}{1}

Para el diseño del controlador basado en el método de respuesta en frecuencia se utilizó directamente el modelo discreto identificado de la planta \(G(z)\), obtenido mediante identificación experimental y presentado en secciones anteriores.

El modelo fue incorporado al entorno \texttt{controlSystemDesigner} de MATLAB, lo que permitió analizar la respuesta en frecuencia del lazo abierto y ajustar el compensador de manera interactiva a partir de los diagramas de Bode.

% ============================================================
\subsubsection{Análisis del sistema sin compensar}
% ============================================================

En la Fig.~\ref{fig:bodeMatSinC} se presenta el diagrama de Bode correspondiente al lazo abierto conformado únicamente por la planta identificada.

A partir del análisis en frecuencia se obtuvieron los siguientes márgenes iniciales:

\begin{itemize}
	\item Margen de ganancia: \(13\,\text{dB}\),
	\item Margen de fase: \SI{61.7}{\degree}.
\end{itemize}

Si bien el sistema presenta margen de fase positivo, lo que implica estabilidad para ganancias moderadas, la frecuencia de cruce se encuentra relativamente baja, lo que se traduce en una respuesta temporal lenta.

La pendiente del módulo en la región de cruce evidencia la influencia de múltiples polos dominantes, coherentes con la dinámica de orden superior asociada al conjunto motor--ESC--hélice.

% ============================================================
\subsubsection{Diseño del compensador proporcional}
% ============================================================

En este caso particular se optó por implementar un compensador puramente proporcional:

\[
C(z) = K_p, \qquad K_p = 1.308
\]

Por lo tanto, el lazo abierto queda:

\[
L(z) = K_p\,G(z)
\]

La acción del controlador proporcional consiste exclusivamente en escalar la magnitud del lazo abierto sin introducir polos ni ceros adicionales. Desde el punto de vista del diagrama de Bode, esto implica un desplazamiento vertical del módulo, modificando la frecuencia de cruce y, en consecuencia, los márgenes de estabilidad.

El aumento de \(K_p\) incrementa la frecuencia de cruce, lo que produce:

\begin{itemize}
	\item Mayor ancho de banda del sistema.
	\item Reducción del tiempo de subida.
	\item Respuesta temporal más rápida.
\end{itemize}

Si bien no se introduce compensación dinámica de fase, el incremento de ganancia resulta suficiente para mejorar significativamente la rapidez de respuesta manteniendo márgenes aceptables.

% ============================================================
\subsubsection{Análisis del sistema compensado}
% ============================================================

En la Fig.~\ref{fig:bodeMatConC} se presenta el diagrama de Bode del sistema compensado.

Los márgenes obtenidos fueron:

\begin{itemize}
	\item \textbf{Sistema sin compensación:}
	\begin{itemize}
		\item Margen de ganancia: \(13\,\text{dB}\),
		\item Margen de fase: \SI{61.7}{\degree}.
	\end{itemize}
	\item \textbf{Sistema compensado:}
	\begin{itemize}
		\item Margen de ganancia: \(10.6\,\text{dB}\),
		\item Margen de fase: \SI{52.9}{\degree}.
	\end{itemize}
\end{itemize}

Se observa una reducción controlada del margen de fase como consecuencia del aumento de la frecuencia de cruce. No obstante, el sistema mantiene estabilidad relativa adecuada.
\balance
\onecolumn
\insertarfigura{img/Bode/bode_C.png}
{Diagrama de Bode y respuestas temporales del sistema con compensación.}
{fig:bodeMatSinC}{1}

\insertarfigura{img/Bode/bode_compensado.png}
{Diagrama de Bode y respuestas temporales del sistema con compensación.}
{fig:bodeMatConC}{1}
\twocolumn

La validación del diseño se realizó mediante la respuesta temporal en lazo cerrado. El sistema compensado presentó:

\begin{itemize}
	\item Reducción significativa del tiempo de subida.
	\item Sobreimpulso moderado.
	\item Esfuerzo dentro de límites aceptables.
\end{itemize}

% ============================================================
\subsection{Práctica}
% ============================================================

\insertarfigura{img/Bode/bode_practica.png}
{Respuesta experimental: altura y esfuerzo con el compensador proporcional.}
{fig:B_practica}{0.8}

Compensador utilizado:

\[
C_{Bode} = 1.3082
\]

Tiempo de muestreo:

\[
T_s = 0.0001\,\text{s}
\]

El sobreimpulso simulado fue aproximadamente \(20\%\).  
En la implementación experimental se observó:

\begin{itemize}
	\item Primer levantamiento: \(\%OS = 55.55\%\),
	\item Levantamiento posterior: \(\%OS \approx 37\%\),
	\item Tendencia decreciente al aumentar la altura.
\end{itemize}

Esta variabilidad sugiere dependencia del punto de operación, condiciones iniciales y no linealidades del empuje.

Se observa además que durante el primer levantamiento el esfuerzo aplicado es mayor al inicio, disminuyendo a medida que el sistema se aproxima al equilibrio dinámico.

% ============================================================
\subsubsection{Conclusión del método}
% ============================================================

El método basado en respuesta en frecuencia permitió evaluar de manera directa y sencilla la estabilidad relativa del sistema mediante los márgenes de fase y ganancia.

A diferencia del Lugar de Raíces, donde la estabilidad se interpreta geométricamente en el plano Z, el método de Bode permite cuantificar cuánto margen de estabilidad se posee y cuánto puede sacrificarse en favor de mayor rapidez.

El ajuste mediante un simple controlador proporcional demostró que, en este sistema particular, el aumento controlado de ganancia resulta suficiente para mejorar significativamente la respuesta temporal, incrementando el ancho de banda sin comprometer la estabilidad global.

Sin embargo, el método presenta una limitación conceptual: al trabajar exclusivamente en el dominio de la frecuencia se pierde información geométrica directa sobre la ubicación de polos y su relación exacta con parámetros transitorios como \(\zeta\) y \(\omega_n\). Si bien existen aproximaciones que relacionan margen de fase y amortiguamiento, estas no son exactas.

En la práctica, esta desventaja se ve mitigada por herramientas de simulación que permiten validar la respuesta temporal rápidamente, situando al método al mismo nivel práctico que los demás enfoques analizados.

En síntesis, el diseño por respuesta en frecuencia resultó sencillo, intuitivo y efectivo, permitiendo mejorar la rapidez del sistema mediante un ajuste mínimo del controlador y manteniendo márgenes de estabilidad adecuados para la planta experimental desarrollada.
\balance

\balance
\clearpage
% ============================================================
\subsection{Síntesis Directa \cite{Fadali2020}}
% ============================================================


\subsubsection{Síntesis directa (Truxal--Ragazzini)}

Se parte del modelo discreto identificado de la planta:

\begin{equation}
	\label{eq:GZAS}
	G_{ZAS}(z)=
	\frac{-0.0001205\,z^{-1}+0.0002415\,z^{-2}-0.0001209\,z^{-3}}
	{1-2.994\,z^{-1}+2.989\,z^{-2}-0.9944\,z^{-3}} .
\end{equation}

El método de Truxal--Ragazzini consiste en especificar explícitamente
una dinámica deseada en lazo cerrado $G_{cl}(z)$ y obtener el controlador
a partir de la relación:

\begin{equation}
	\label{eq:TR_general}
	G_{cl}(z)=\frac{C(z)G_{ZAS}(z)}{1+C(z)G_{ZAS}(z)}
	\quad \Longrightarrow \quad
	C(z)=\frac{1}{G_{ZAS}(z)}\frac{G_{cl}(z)}{1-G_{cl}(z)}.
\end{equation}

Este procedimiento implica una inversión explícita del modelo de la planta,
por lo que el diseño depende fuertemente de la exactitud del modelo identificado.

\paragraph{Método 1: respuesta \textit{deadbeat}}

Como primera aproximación se adoptó una dinámica deseada del tipo
\textit{deadbeat}, definida por:

\begin{equation}
	G_{cl}(z)=z^{-1}.
\end{equation}

Esta elección implica que la salida alcance el valor deseado
en un único período de muestreo, anulando el error en el menor tiempo posible.

Reemplazando en \eqref{eq:TR_general} se obtiene:

\begin{equation}
	\label{eq:C1}
	C_1(z)=\frac{1}{G_{ZAS}(z)}\frac{z^{-1}}{1-z^{-1}}
	=
	\frac{1}{G_{ZAS}(z)}\frac{1}{z-1}.
\end{equation}

Se observa que el controlador resultante contiene explícitamente
la inversa de la planta y un polo adicional en $z=1$,
lo que anticipa posibles problemas de magnitud del esfuerzo de control.

\subsubsection{Método 2: Deadbeat \textit{ripple-free}}

Como alternativa se evaluó la variante \textit{ripple-free},
cuyo controlador obtenido es:

\begin{equation}
	\label{eq:C2_z}
	C_2(z)=
	\frac{10.43038\,z^{2}-2.7635\,z+3.3344}
	{-0.33662\,z^{2}-0.66338\,z+1}.
\end{equation}

Multiplicando numerador y denominador por $(-1)$
y normalizando el coeficiente líder del denominador, se obtiene:

\begin{equation}
	\label{eq:C2_z_norm}
	C_2(z)=
	\frac{-30.9856\,z^{2}+8.2096\,z-9.9055}
	{z^{2}+1.9707\,z-2.9707}.
\end{equation}

Para implementación digital resulta conveniente expresarlo en términos de $z^{-1}$:

\begin{equation}
	\label{eq:C2_zinv}
	C_2(z)=
	\frac{-30.9856+8.2096\,z^{-1}-9.9055\,z^{-2}}
	{1+1.9707\,z^{-1}-2.9707\,z^{-2}}.
\end{equation}

\subsubsection{Resultados y Limitaciones Prácticas}

Las simulaciones mostraron que ambos controladores demandan esfuerzos
de control extremadamente elevados, alcanzando valores del orden de:

\[
|u_{\max}| \sim 10^{29},
\]

lo cual excede ampliamente las capacidades del actuador físico.

En la implementación real, la señal de control corresponde a una
señal PWM tipo servo a $50\,\text{Hz}$ acotada en el rango:

\begin{equation}
	u \in [1000,2000]\ \mu s .
\end{equation}

La magnitud desproporcionada del esfuerzo se explica por:

\begin{itemize}
	\item La inversión explícita del modelo $G_{ZAS}(z)$.
	\item La presencia de polos cercanos a $z=1$ en la planta.
	\item Alta sensibilidad a pequeñas incertidumbres del modelo.
	\item Cancelaciones exactas requeridas por el diseño.
\end{itemize}

En particular, la inversión de dinámicas cercanas al borde del círculo unitario produce amplificaciones significativas en la señal de control, haciendo que el diseño sea extremadamente sensible a variaciones como cambios en la tensión de batería, fricción, efectos aerodinámicos y dinámica no modelada del conjunto ESC--motor--hélice.

Por estas razones, si bien la síntesis directa resulta valiosa desde el punto de vista conceptual y didáctico, no se considera viable para implementación experimental en la planta real.

En consecuencia, para la etapa práctica se priorizan estrategias de menor orden y mayor robustez, que contemplen explícitamente las limitaciones del actuador y la saturación de la señal de control.


% ============================================================
\subsection{Comparación entre Métodos Clásicos}
% ============================================================


La aplicación de distintos métodos clásicos sobre la misma planta
permitió comparar directamente su aplicabilidad práctica,
su dependencia del modelo y la información que aportan
para el diseño.

Desde el punto de vista de simplicidad y rapidez de implementación,
el \textbf{PID} resultó claramente el método más eficiente.
Permitió obtener una respuesta satisfactoria
con escasa dependencia del modelo
y con un proceso de ajuste directo e intuitivo.
En términos puramente ingenieriles,
si el objetivo es lograr funcionamiento estable
con el menor esfuerzo analítico posible,
el PID se posiciona como la alternativa más práctica.

El \textbf{Lugar de Raíces}, en cambio,
aportó mayor profundidad conceptual.
Permite visualizar explícitamente la estabilidad
en el plano \(z\),
relacionar polos con desempeño transitorio
y comprender geométricamente el efecto de la ganancia.
Si bien exige mayor comprensión del modelo
y es más sensible a la elección del período de muestreo,
proporciona una cantidad de información estructural
superior a la del PID.

El método basado en \textbf{Bode}
resultó adecuado para analizar márgenes de estabilidad
y robustez,
pero menos intuitivo en relación con el comportamiento temporal.
Sin herramientas de simulación,
su aplicación manual se vuelve considerablemente más compleja.
En esta experiencia no ofreció ventajas decisivas
respecto a los métodos anteriores.

Por su parte,
la \textbf{Síntesis Directa (Truxal--Ragazzini)}
mostró gran elegancia analítica
y rigor matemático,
pero evidenció una dependencia extrema del modelo
y una tendencia a generar esfuerzos de control elevados.
En una planta física con saturaciones y no linealidades,
esta sensibilidad la vuelve riesgosa para implementación real,
quedando principalmente como herramienta académica.

En conclusión,
los métodos que demostraron mayor aplicabilidad práctica
en la planta experimental fueron el PID y el Lugar de Raíces:
el primero por su simplicidad y robustez,
el segundo por la riqueza de información que aporta.
Los demás métodos resultaron valiosos conceptualmente,
pero menos determinantes en la implementación experimental concreta.


\balance
\clearpage
