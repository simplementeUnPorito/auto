\subsubsection{Síntesis directa (Truxal--Ragazzini)}

Se parte del modelo discreto identificado de la planta:

\begin{equation}
	\label{eq:GZAS}
	G_{ZAS}(z)=
	\frac{-0.0001205\,z^{-1}+0.0002415\,z^{-2}-0.0001209\,z^{-3}}
	{1-2.994\,z^{-1}+2.989\,z^{-2}-0.9944\,z^{-3}} .
\end{equation}

El método de Truxal--Ragazzini consiste en especificar explícitamente
una dinámica deseada en lazo cerrado $G_{cl}(z)$ y obtener el controlador
a partir de la relación:

\begin{equation}
	\label{eq:TR_general}
	G_{cl}(z)=\frac{C(z)G_{ZAS}(z)}{1+C(z)G_{ZAS}(z)}
	\quad \Longrightarrow \quad
	C(z)=\frac{1}{G_{ZAS}(z)}\frac{G_{cl}(z)}{1-G_{cl}(z)}.
\end{equation}

Este procedimiento implica una inversión explícita del modelo de la planta,
por lo que el diseño depende fuertemente de la exactitud del modelo identificado.

\paragraph{Método 1: respuesta \textit{deadbeat}}

Como primera aproximación se adoptó una dinámica deseada del tipo
\textit{deadbeat}, definida por:

\begin{equation}
	G_{cl}(z)=z^{-1}.
\end{equation}

Esta elección implica que la salida alcance el valor deseado
en un único período de muestreo, anulando el error en el menor tiempo posible.

Reemplazando en \eqref{eq:TR_general} se obtiene:

\begin{equation}
	\label{eq:C1}
	C_1(z)=\frac{1}{G_{ZAS}(z)}\frac{z^{-1}}{1-z^{-1}}
	=
	\frac{1}{G_{ZAS}(z)}\frac{1}{z-1}.
\end{equation}

Se observa que el controlador resultante contiene explícitamente
la inversa de la planta y un polo adicional en $z=1$,
lo que anticipa posibles problemas de magnitud del esfuerzo de control.

\subsubsection{Método 2: Deadbeat \textit{ripple-free}}

Como alternativa se evaluó la variante \textit{ripple-free},
cuyo controlador obtenido es:

\begin{equation}
	\label{eq:C2_z}
	C_2(z)=
	\frac{10.43038\,z^{2}-2.7635\,z+3.3344}
	{-0.33662\,z^{2}-0.66338\,z+1}.
\end{equation}

Multiplicando numerador y denominador por $(-1)$
y normalizando el coeficiente líder del denominador, se obtiene:

\begin{equation}
	\label{eq:C2_z_norm}
	C_2(z)=
	\frac{-30.9856\,z^{2}+8.2096\,z-9.9055}
	{z^{2}+1.9707\,z-2.9707}.
\end{equation}

Para implementación digital resulta conveniente expresarlo en términos de $z^{-1}$:

\begin{equation}
	\label{eq:C2_zinv}
	C_2(z)=
	\frac{-30.9856+8.2096\,z^{-1}-9.9055\,z^{-2}}
	{1+1.9707\,z^{-1}-2.9707\,z^{-2}}.
\end{equation}

\subsubsection{Resultados y Limitaciones Prácticas}

Las simulaciones mostraron que ambos controladores demandan esfuerzos
de control extremadamente elevados, alcanzando valores del orden de:

\[
|u_{\max}| \sim 10^{29},
\]

lo cual excede ampliamente las capacidades del actuador físico.

En la implementación real, la señal de control corresponde a una
señal PWM tipo servo a $50\,\text{Hz}$ acotada en el rango:

\begin{equation}
	u \in [1000,2000]\ \mu s .
\end{equation}

La magnitud desproporcionada del esfuerzo se explica por:

\begin{itemize}
	\item La inversión explícita del modelo $G_{ZAS}(z)$.
	\item La presencia de polos cercanos a $z=1$ en la planta.
	\item Alta sensibilidad a pequeñas incertidumbres del modelo.
	\item Cancelaciones exactas requeridas por el diseño.
\end{itemize}

En particular, la inversión de dinámicas cercanas al borde del círculo unitario produce amplificaciones significativas en la señal de control, haciendo que el diseño sea extremadamente sensible a variaciones como cambios en la tensión de batería, fricción, efectos aerodinámicos y dinámica no modelada del conjunto ESC--motor--hélice.

Por estas razones, si bien la síntesis directa resulta valiosa desde el punto de vista conceptual y didáctico, no se considera viable para implementación experimental en la planta real.

En consecuencia, para la etapa práctica se priorizan estrategias de menor orden y mayor robustez, que contemplen explícitamente las limitaciones del actuador y la saturación de la señal de control.
