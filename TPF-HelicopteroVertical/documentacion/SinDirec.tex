\subsubsection{Síntesis directa (Truxal--Ragazzini)}

Se parte del modelo discreto identificado de la planta:
\begin{equation}
	\label{eq:GZAS}
	G_{ZAS}(z)=
	\frac{-0.0001205\,z^{-1}+0.0002415\,z^{-2}-0.0001209\,z^{-3}}
	{1-2.994\,z^{-1}+2.989\,z^{-2}-0.9944\,z^{-3}} .
\end{equation}

El método de Truxal--Ragazzini propone especificar una dinámica deseada en lazo cerrado \(G_{cl}(z)\) y obtener el controlador a partir de:
\begin{equation}
	\label{eq:TR_general}
	G_{cl}(z)=\frac{C(z)\,G_{ZAS}(z)}{1+C(z)\,G_{ZAS}(z)}
	\quad\Longrightarrow\quad
	C(z)=\frac{1}{G_{ZAS}(z)}\frac{G_{cl}(z)}{1-G_{cl}(z)}.
\end{equation}

\paragraph{Método 1: respuesta ``deadbeat''}
Para el caso de oscilaciones entre muestras se adoptó:
\begin{equation}
	G_{cl}(z)=z^{-1}.
\end{equation}
Reemplazando en \eqref{eq:TR_general} se obtiene:
\begin{equation}
	\label{eq:C1}
	C_1(z)=\frac{1}{G_{ZAS}(z)}\frac{z^{-1}}{1-z^{-1}}
	=\frac{1}{G_{ZAS}(z)}\frac{1}{z-1}.
\end{equation}

\subsubsection{Resultado del Método 2 (Deadbeat ripple-free)}

El controlador obtenido para el Método 2 se expresa como:
\begin{equation}
	\label{eq:C2_z}
	C_2(z)=\frac{U(z)}{E(z)}=
	\frac{10.43038\,z^{2}-2.7635\,z+3.3344}{-0.33662\,z^{2}-0.66338\,z+1}.
\end{equation}

Multiplicando numerador y denominador por \((-1)\) y normalizando para que el
coeficiente líder del denominador sea \(1\), resulta:
\begin{equation}
	\label{eq:C2_z_norm}
	C_2(z)=
	\frac{-30.9856\,z^{2}+8.2096\,z-9.9055}{z^{2}+1.9707\,z-2.9707}.
\end{equation}

En forma de \(z^{-1}\) (conveniente para implementación digital):
\begin{equation}
	\label{eq:C2_zinv}
	C_2(z)=
	\frac{-30.9856+8.2096\,z^{-1}-9.9055\,z^{-2}}{1+1.9707\,z^{-1}-2.9707\,z^{-2}}.
\end{equation}

\subsubsection{Resultados y limitaciones prácticas}

Si bien la síntesis directa es útil para comprender la relación entre la respuesta deseada y la estructura del controlador, en esta planta el resultado no es implementable: los controladores obtenidos (M1 y M2) demandaron esfuerzos máximos del orden de \(10^{29}\) en simulación. Esto es incompatible con el actuador real, cuya acción de control se implementa como una señal PWM tipo servo a \(50\,\text{Hz}\) limitada a:
\begin{equation}
	u\in[1000,2000]\ \mu s .
\end{equation}

La magnitud extrema del esfuerzo se atribuye a la inversión explícita de \(G_{ZAS}(z)\) y a la presencia de dinámica cercana a \(z=1\), lo que vuelve al diseño altamente sensible a cancelaciones y a incertidumbres del modelo (tensión de batería, fricción, aerodinámica y dinámica no modelada del conjunto ESC--motor--hélice). En consecuencia, para implementación experimental se priorizan controladores de menor orden y mayor robustez, considerando explícitamente saturación y limitaciones del esfuerzo.


\subsection{Simulaciones}

\insertarfigura{img/Sintesis/Step_esfuerzoL.png}
{Respuesta temporal del esfuerzo de control para ambos métodos de síntesis.}
{fig:resp_esfuerzo_sintesis}{1}

\insertarfigura{img/Sintesis/Step_salidaL.png}
{Respuesta temporal de la salida en lazo cerrado para ambos métodos de síntesis.}
{fig:resp_salida_sintesis}{1}


\subsubsection{Modelo de la Planta}

Se parte del modelo discreto identificado:

\begin{equation}
	G(z)=\textbf{[Completar]}
\end{equation}

Tiempo de muestreo:

\[
T_s = \textbf{[Completar]}
\]

Polos de la planta:

\[
\{z_i\} = \textbf{[Completar]}
\]

Observaciones relevantes:

\begin{itemize}
	\item Presencia de polos cercanos a $z=1$: \textbf{[Completar]}
	\item Naturaleza inestable / marginalmente estable: \textbf{[Completar]}
\end{itemize}


% ============================================================
\subsubsection{Formulación General}
% ============================================================

La síntesis directa propone especificar una dinámica deseada en lazo cerrado:

\begin{equation}
	G_{cl}(z)=\frac{C(z)G(z)}{1+C(z)G(z)}
\end{equation}

Despejando el controlador:

\begin{equation}
	C(z)=
