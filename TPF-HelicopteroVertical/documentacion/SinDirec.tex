% ============================================================
\subsubsection{Síntesis directa (Truxal--Ragazzini) y evaluación intramuestra}
% ============================================================

El método de Síntesis Directa (Truxal--Ragazzini) se basa en el análisis
analítico del modelo identificado de la planta para obtener explícitamente
el compensador que produzca una respuesta deseada frente a una entrada
particular.

A diferencia de los métodos geométricos (Lugar de Raíces) o frecuenciales
(Bode), aquí se impone algebraicamente la dinámica deseada en lazo cerrado
a partir del modelo \(G_{ZAS}(z)\). El controlador surge de resolver
directamente la ecuación:

\[
G_{cl}(z)
=
\frac{C(z)G_{ZAS}(z)}{1 + C(z)G_{ZAS}(z)}.
\]

Este enfoque es extremadamente dependiente del modelo,
ya que implica esencialmente su inversión parcial o total.

% ============================================================
\paragraph{Frecuencia de muestreo}
% ============================================================

Se parte del modelo continuo previamente identificado,
cuyos polos dominantes son:

\[
p_{1,2} = -2.805 \pm j\,2.4804,
\qquad
p_3 = -0.0003.
\]

Para el par complejo dominante:

\[
\omega_n
=
\sqrt{(-2.805)^2 + (2.4804)^2}
\approx 3.745 \,\text{rad/s},
\]

\[
f_n
=
\frac{\omega_n}{2\pi}
\approx 0.596 \,\text{Hz}.
\]

Siguiendo el criterio de Nyquist, la frecuencia mínima de muestreo
debería ser superior a \(2f_n \approx 1.19\,\text{Hz}\).

Sin embargo, experimentalmente se observó que al aumentar
la frecuencia de muestreo el esfuerzo de control crece
de manera extremadamente elevada, producto de la inversión
explícita del modelo y la agresividad dinámica que introduce
el diseño.

Por esta razón se adoptó:

\[
F_s = 2\,\text{Hz}
\qquad
(T_s = 0.5\,\text{s}),
\]

como compromiso entre rapidez y esfuerzo físicamente realizable.

El procedimiento completo se detalla en el
Apéndice~\ref{ap:codigo_TR}.

% ============================================================
\subsubsection{Controlador con oscilaciones intramuestra}
% ============================================================

Si se resuelve directamente:

\[
G_{cl}(z)
=
\frac{C(z)G_{ZAS}(z)}{1+C(z)G_{ZAS}(z)}
=
\frac{Y(z)}{R(z)}
\]

para una referencia escalón con salida deseada retardada,
sin imponer condiciones estructurales adicionales,
se obtiene:

{\tiny
	\[
	C_2(z)=
	\frac{z^3 - 1.16z^2 + 0.2202z - 0.0605}
	{0.1917z^5 + 0.3612z^4 + 0.02515z^3
		- 0.1917z^2 - 0.3612z - 0.02515}.
	\]
}

\insertarfigura{img/Sintesis/Step_salidaConOscilaciones.png}
{Respuesta al escalón con oscilaciones intramuestra.}
{fig:conSalida}{1}

\insertarfigura{img/Sintesis/Step_esfuerzoConOscilaciones.png}
{Esfuerzo de control asociado a \(C_2(z)\).}
{fig:conEsfuerzo}{1}

En este caso se observan claramente
\textbf{oscilaciones intramuestra}, es decir,
variaciones significativas entre instantes de muestreo.

El esfuerzo de control resulta además de amplitud elevada,
mostrando alta sensibilidad frente a pequeñas variaciones
del modelo.

% ============================================================
\subsubsection{Sin oscilaciones intramuestra (Ripple-Free)}
% ============================================================

Sea la función de lazo:

\[
L(z) = C(z)G_{ZAS}(z) = \frac{N_L(z)}{D_L(z)}.
\]

Para eliminar oscilaciones intramuestra
(\textit{ripple-free response}) deben cumplirse simultáneamente:

\begin{enumerate}
	\item \(N_L(z) + D_L(z) = z^l,\quad l \ge n\).
	\item \(D_L(z)\) debe poseer una raíz en \(z=1\) (sistema tipo 1).
	\item No deben existir cancelaciones inestables.
\end{enumerate}

Resolviendo el sistema polinómico correspondiente se obtiene:

\[
C_1(z)=
\frac{1.73z^2 - 0.2763z + 0.1047}
{z^2 + 0.6683z + 0.04352}.
\]

\insertarfigura{img/Sintesis/Step_salidaSinOscilaciones.png}
{Respuesta al escalón sin oscilaciones intramuestra.}
{fig:sinSalida}{1}

\insertarfigura{img/Sintesis/Step_esfuerzoSinOscilaciones.png}
{Esfuerzo de control asociado a \(C_1(z)\).}
{fig:sinEsfuerzo}{1}

En este caso las oscilaciones intramuestra desaparecen,
pero el esfuerzo de control continúa siendo
considerablemente elevado.

% ============================================================
\subsubsection{Discusión y decisión experimental}
% ============================================================

Ambos controladores presentan una dinámica extremadamente agresiva.

El método, si bien matemáticamente elegante,
resulta excesivamente dependiente del modelo.
La inversión explícita amplifica errores de identificación,
no linealidades y saturaciones del actuador.

Si se reduce el tiempo de muestreo
el problema se mitiga parcialmente,
pero el esfuerzo tiende rápidamente a volverse
físicamente irrealizable.

Por esta razón, ninguno de los dos controladores
fue implementado experimentalmente.

Estos diseños deben considerarse
controladores teóricos de alta sensibilidad,
útiles desde el punto de vista académico,
pero no apropiados para la planta real.

Paradójicamente, el método que parece más “óptimo”
desde el punto de vista algebraico
resulta ser el menos robusto en la práctica.
