\subsubsection{Síntesis directa (Truxal--Ragazzini) y evaluación intramuestra}

Se parte del modelo continuo ya utilizado en los anteriores diseños.
Sus polos dominantes en tiempo continuo son:

\[
p_{1,2} = -2.805 \pm j\,2.4804,
\qquad
p_3 = -0.0003.
\]

Para el par complejo, la frecuencia natural resulta:

\[
\omega_n
=
\sqrt{(-2.805)^2 + (2.4804)^2}
\approx 3.745 \,\text{rad/s}.
\]

En términos de frecuencia:

\[
f_n
=
\frac{\omega_n}{2\pi}
\approx 0.596 \,\text{Hz}.
\]

Siguiendo el criterio de Nyquist, la frecuencia mínima de muestreo
debería ser superior a \(2f_n \approx 1.19\,\text{Hz}\).
Sin embargo, experimentalmente se observó que para frecuencias
de muestreo mayores a \(2\,\text{Hz}\)
el esfuerzo de control crece de manera extremadamente elevada,
producto de la inversión explícita del modelo y la mayor agresividad
dinámica introducida por el diseño.

Por esta razón se adoptó:

\[
F_s = 2\,\text{Hz}
\qquad
(T_s = 0.5\,\text{s}),
\]

valor que representa un compromiso entre velocidad de respuesta
y esfuerzo físicamente realizable.

El procedimiento completo de discretización, resolución
del sistema polinómico y simulación intramuestra
se encuentra detallado en el Apéndice~\ref{ap:codigo_TR}.


\paragraph{Fundamento teórico (Ogata 6.7.1)}

Sea la función de lazo:

\[
L(z) = C(z)G_{ZAS}(z) = \frac{N_L(z)}{D_L(z)}.
\]

Para eliminar oscilaciones intramuestra (ripple-free response)
y lograr error nulo luego de \(l\) pasos,
deben cumplirse simultáneamente:

\begin{enumerate}
	\item \(N_L(z) + D_L(z) = z^l,\quad l \ge n\).
	\item \(D_L(z)\) debe tener una raíz en \(z=1\) (sistema tipo 1).
	\item No deben existir cancelaciones inestables entre
	\(C(z)\) y \(G_{ZAS}(z)\).
\end{enumerate}

Resolviendo el sistema polinómico correspondiente se obtiene
el controlador sin oscilaciones intramuestra.

\subsubsection{Controlador sin oscilaciones intramuestra}

\[
C_1(z)=
\frac{1.73z^2 - 0.2763z + 0.1047}
{z^2 + 0.6683z + 0.04352}.
\]

\insertarfigura{img/Sintesis/Step_salidaSinOscilaciones.png}
{Respuesta al escalón sin oscilaciones intramuestra.}
{fig:sinSalida}{1}

\insertarfigura{img/Sintesis/Step_esfuerzoSinOscilaciones.png}
{Esfuerzo de control asociado a \(C_1(z)\).}
{fig:sinEsfuerzo}{1}

\subsubsection{Controlador con oscilaciones intramuestra}

Para evitar que la expresión exceda el ancho de columna,
se presenta en forma partida:

{\tiny
	\[
	C_2(z)=
	\frac{z^3 - 1.16z^2 + 0.2202z - 0.0605}
	{0.1917z^5 + 0.3612z^4 + 0.02515z^3
		- 0.1917z^2 - 0.3612z - 0.02515}
	\]
}


\insertarfigura{img/Sintesis/Step_salidaConOscilaciones.png}
{Respuesta al escalón con oscilaciones intramuestra.}
{fig:conSalida}{1}

\insertarfigura{img/Sintesis/Step_esfuerzoConOscilaciones.png}
{Esfuerzo de control asociado a \(C_2(z)\).}
{fig:conEsfuerzo}{1}

\subsubsection{Discusión y decisión experimental}

Ambos controladores presentan una dinámica extremadamente agresiva.
La inversión explícita del modelo genera amplificaciones significativas
del esfuerzo de control y alta sensibilidad ante incertidumbres,
no linealidades y saturaciones del actuador.

Si bien el controlador ripple-free elimina oscilaciones intramuestra,
los niveles de esfuerzo observados en simulación resultan
potencialmente destructivos para la planta real.

Por esta razón, ninguno de los dos controladores fue implementado
experimentalmente, priorizando la integridad física del sistema
y la seguridad del actuador.

Estos diseños deben considerarse controladores teóricos
altamente sensibles (“controladores de papel”),
útiles desde el punto de vista académico,
pero no apropiados para implementación directa
en la planta real.
