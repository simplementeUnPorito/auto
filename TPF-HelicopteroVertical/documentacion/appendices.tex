\onecolumn
\appendices

\section{Repositorios y evidencias}

Con el objetivo de garantizar la trazabilidad, reproducibilidad y transparencia
del desarrollo del trabajo práctico, se pone a disposición el repositorio
completo del proyecto, incluyendo:

\begin{itemize}
	\item Código fuente de todos los controladores implementados (PID, LQR, LQG, ubicación de polos, síntesis directa, etc.).
	\item Versiones preliminares y pruebas intermedias.
	\item Scripts de identificación en MATLAB.
	\item Firmware desarrollado para PSoC 5LP.
	\item Archivos históricos que permiten observar la evolución del diseño.
	\item Commits detallados que documentan cada etapa del proceso.
\end{itemize}

El repositorio completo del proyecto puede consultarse en:

\begin{center}
	\href{https://github.com/simplementeUnPorito/auto}{Repositorio oficial del proyecto (GitHub)}
\end{center}

La carpeta raíz del repositorio corresponde al proyecto presentado.
El historial completo de commits permite verificar las fechas de desarrollo,
las modificaciones realizadas y la progresión iterativa del trabajo.
La versión final disponible al momento de la entrega fue utilizada para
generar las figuras, simulaciones y resultados incluidos en este informe.

Adicionalmente, se proporciona un video demostrativo donde se puede
visualizar el funcionamiento práctico de los distintos controladores
implementados y las prácticas desarrolladas:

\begin{center}
	\href{https://onedrive.live.com/?qt=allmyphotos&photosData=%2Fshare%2F316EBACFC0E21183%21s88f2c40124b1443f88bc3d653bf2f29f%3Fithint%3Dvideo%26e%3DVpVrXT%26migratedtospo%3Dtrue&cid=316EBACFC0E21183&id=316EBACFC0E21183%21s88f2c40124b1443f88bc3d653bf2f29f&redeem=aHR0cHM6Ly8xZHJ2Lm1zL3YvYy8zMTZlYmFjZmMwZTIxMTgzL0lRQUJ4UEtJc1NRX1JJaThQV1U3OHZLZkFVdmNhbVpaQ2lFRHZWa1RYLURrX3hZP2U9VnBWclhU&v=photos}{Video demostrativo de las implementaciones}
\end{center}

En dicho material audiovisual se evidencia la implementación física
de los algoritmos desarrollados y el comportamiento dinámico real de la planta.

% ============================================================
\subsection{Interfaz gráfica de usuario}
\label{ap:gui_matlab}
% ============================================================

\begin{lstlisting}[language=Matlab,caption={Interfaz Gráfica de Usuario},label={lst:gui_matlab}]
function psoc_control_gui_hl_v3()
% PSoC Control GUI - UARTP single-file (NO external functions)
% Compatible con uartp_sw.c / uartp_sw.h (handshake eco-confirmado por word 4B).
%
% Protocolo (COMMAND):
%   cmd 'm' -> device: 'R' -> host sends 4B payload (mode + padding) eco/ACK -> device: 'K'
%   cmd 'c' -> device: 'R' -> host sends 100B payload (25 floats) eco/ACK -> device: 'K'
%   cmd 't' -> device: 'S' -> device sends 100B payload eco/ACK -> device: 'K'
%   cmd 'i' -> device: 'R' -> host sends 4B float32 eco/ACK -> device: 'K' then device entra CONTROL
%   cmd 's' -> device: 'K' (COMMAND)  / en CONTROL lo maneja ISR y responde 'K'
%   cmd 'r' -> device: 'K' y reset
%
% Telemetría CONTROL:
%   frames 8B: [u(float32) y(float32)] little-endian

% =========================
% Estado
% =========================
S = struct();
S.sp = [];
S.isConnected = false;

% Streaming
S.streamTimer  = [];
S.streamRxBuf  = uint8([]);
S.streamParseEnabled = false; % OFF hasta Start
S.nVec         = zeros(0,1);
S.uVec         = zeros(0,1);
S.yVec         = zeros(0,1);
S.framesTotal  = 0;
S.maxPoints    = 400000;
S.inTxn        = false;

% Auto-stop
S.autoStopEnabled   = false;
S.autoStopTarget    = 0;
S.autoStopArmed     = false;
S.autoStopStartBase = 0;
S.autoStopPending   = false;
S.autoStopReason    = "";

% Plot scaling
S.scaleU = 0.01;
S.scaleY = 1.0;

% ===== Debug =====
DBG_LOG_BYTES   = true;      % loguea RX/TX de comandos/payload
DBG_HEX_DUMP    = true;      % muestra hex de payloads (corto)
DBG_PRINT_COEFFS= true;      % imprime el vector de 25 floats

% ===== Protocolo fijo =====
TF_MAX_ORDER = 10;

% índices meta (MATLAB 1-based para arrays)
IDX_TF_ORDER = 23; % c23 (0-based 22)
IDX_META_N   = 24; % c24 (0-based 23)
IDX_META_FS  = 25; % c25 (0-based 24)

% ===== Constantes LL (debe coincidir con C) =====
RSP_READY_RX = uint8('R');
RSP_READY_TX = uint8('S');
RSP_OK       = uint8('K');
RSP_ERR      = uint8('!'); %#ok<NASGU>

CTL_ACK      = uint8('A');
CTL_NAK      = uint8('N');

STEP_TIMEOUT_S   = 0.5;      % ~500ms por step (PSoC usa ~300ms)
MAX_WORD_RETRIES = 50;       % igual que firmware

% =========================
% UI
% =========================
fig = uifigure('Name','PSoC Control GUI (UARTP single-file)','Position',[80 80 1280 720]);
fig.CloseRequestFcn = @onClose;

ax = uiaxes(fig, 'Position', [20 220 820 480]);
ax.XGrid = 'on'; ax.YGrid = 'on';
title(ax,'Streaming u,y');
xlabel(ax,'frame index'); ylabel(ax,'value');
hold(ax,'on');
hU = stairs(ax, nan, nan, 'DisplayName','u');
hY = plot(ax,   nan, nan, 'DisplayName','y');
legend(ax,'show','Location','best');
hold(ax,'off');

txtLog = uitextarea(fig,'Editable','off','Position',[20 20 820 180]);
txtLog.Value = strings(0,1);

RX = 860; RW = 400;

% Connection
pConn = uipanel(fig,'Title','Connection','Position',[RX 620 RW 95]);
uilabel(pConn,'Text','COM:','Position',[10 45 35 22]);
edtCom = uieditfield(pConn,'text','Value','COM19','Position',[50 45 90 22]);

edtBaud = uieditfield(pConn,'numeric','Value',115200,'Limits',[1200 2000000],'Position',[150 45 90 22]);
uilabel(pConn,'Text','baud','Position',[245 45 40 22]);

btnConnect = uibutton(pConn,'Text','Connect','Position',[10 10 90 26],'ButtonPushedFcn',@onConnectToggle);
lblStat = uilabel(pConn,'Text','DISCONNECTED','Position',[110 10 280 26]);

btnReset = uibutton(pConn,'Text','Reset (r)','Position',[310 10 80 26],'ButtonPushedFcn',@onReset);

% Mode
pMode = uipanel(fig,'Title','Mode','Position',[RX 435 RW 180]);

ddType = uidropdown(pMode,'Items',{'TF','SS','Open-loop'},'Value','Open-loop', ...
'Position',[10 130 120 24],'ValueChangedFcn',@(~,~)refreshVisibility());

ddObserver = uidropdown(pMode,'Items',{'Predictor','Actual'},'Position',[150 130 120 24]);
cbIntegrator = uicheckbox(pMode,'Text','Integrator','Position',[290 130 110 24]);

uilabel(pMode,'Text','N','Position',[10 95 20 24]);
edtN = uieditfield(pMode,'numeric','Limits',[0 1e9],'RoundFractionalValues','on','Value',1,'Position',[35 95 80 24]);

uilabel(pMode,'Text','Fs (Hz)','Position',[130 95 50 24]);
edtFs = uieditfield(pMode,'numeric','Limits',[0.001 1e9],'Value',1000,'Position',[180 95 90 24], ...
'ValueChangedFcn',@(~,~)onFsChanged());

lblFs = uilabel(pMode,'Text','Fs=?','Position',[10 70 360 20]);

btnSendMode   = uibutton(pMode,'Text','Send Mode (m)','Position',[280 95 110 24],'ButtonPushedFcn',@onSendMode);
btnSendCoeffs = uibutton(pMode,'Text','Send Coeffs (c)','Position',[10 25 140 30],'ButtonPushedFcn',@onSendCoeffs);
cbVerify      = uicheckbox(pMode,'Text','Verify (t)','Value',true,'Position',[160 30 120 24]);
btnGetCoeffs  = uibutton(pMode,'Text','Get Coeffs (t)','Position',[280 25 110 30],'ButtonPushedFcn',@onGetCoeffs);

% Control
pRef = uipanel(fig,'Title','Control','Position',[RX 315 RW 110]);
uilabel(pRef,'Text','u0 / ref','Position',[10 55 60 22]);
edtU0 = uieditfield(pRef,'numeric','Value',1000,'Position',[80 50 80 26]);

uilabel(pRef,'Text','r x','Position',[165 55 25 22]);
edtRmult = uieditfield(pRef,'numeric','Value',1.0,'Limits',[-1e12 1e12],'Position',[190 50 55 26]);

btnStart = uibutton(pRef,'Text','Start (i)','Position',[250 50 70 26],'ButtonPushedFcn',@onStart);
btnStop  = uibutton(pRef,'Text','Stop (s)','Position',[325 50 65 26],'ButtonPushedFcn',@onStop);

cbWaitBack = uicheckbox(pRef,'Text','wait back to COMMAND','Value',true,'Position',[10 15 190 24]); %#ok<NASGU>

cbAutoStop = uicheckbox(pRef,'Text','Auto-stop','Value',false,'Position',[205 15 75 24]);
uilabel(pRef,'Text','at frames','Position',[285 15 55 22]);
edtAutoStopN = uieditfield(pRef,'numeric','Limits',[0 1e12],'RoundFractionalValues','on','Value',0,'Position',[345 15 45 24]);

% Data
pData = uipanel(fig,'Title','Data','Position',[RX 5 RW 85]);
btnExportMAT = uibutton(pData,'Text','Export .mat','Position',[10 12 120 32],'ButtonPushedFcn',@onExportMAT);
btnClearData = uibutton(pData,'Text','Clear data','Position',[140 12 120 32],'ButtonPushedFcn',@onClearData);

lblDataInfo = uilabel(pData,'Text','n=0','Position',[70 64 80 22],'HorizontalAlignment','right');

uilabel(pData,'Text','u x','Position',[270 20 30 18]);
edtScaleU = uieditfield(pData,'numeric','Value',S.scaleU,'Limits',[-1e12 1e12],'Position',[300 17 70 22], ...
'ValueChangedFcn',@(~,~)onScaleChanged());

uilabel(pData,'Text','y x','Position',[375 20 30 18]);
edtScaleY = uieditfield(pData,'numeric','Value',S.scaleY,'Limits',[-1e12 1e12],'Position',[405 17 70 22], ...
'ValueChangedFcn',@(~,~)onScaleChanged());

% TF panel
pTF = uipanel(fig,'Title','TF (25): b0..b10, a0..a10, order, N, Fs','Position',[RX 90 RW 220]);
uilabel(pTF,'Text','k','Position',[10 175 20 18]);
uilabel(pTF,'Text','b(k)','Position',[60 175 60 18]);
uilabel(pTF,'Text','a(k)','Position',[185 175 60 18]);

tfData = zeros(TF_MAX_ORDER+1,2);
tfData(1,2) = 1; % a0
tfTable = uitable(pTF,'Data',tfData,'ColumnName',{'b','a'},'RowName',string(0:TF_MAX_ORDER), ...
'ColumnEditable',[true true],'Position',[10 50 260 125]);

uilabel(pTF,'Text','Order (0..10)','Position',[285 145 100 18]);
edtTFOrder = uieditfield(pTF,'numeric','Value',5,'Limits',[0 TF_MAX_ORDER], ...
'RoundFractionalValues','on','Position',[285 120 80 24]);

% SS panel (3 estados)
pSS = uipanel(fig,'Title','SS (3 estados / 25)','Position',[RX 90 RW 220]);

yLblTop = 170; y1=145; y2=120; y3=95; yLblMid=80; yMid=55; yK1=55; yK2=30; yK3=5;
uilabel(pSS,'Text','A (3x3)','Position',[10 yLblTop 60 18]);
edtA11 = uieditfield(pSS,'text','Value','1','Position',[10  y1 60 22]);
edtA12 = uieditfield(pSS,'text','Value','0','Position',[75  y1 60 22]);
edtA13 = uieditfield(pSS,'text','Value','0','Position',[140 y1 60 22]);
edtA21 = uieditfield(pSS,'text','Value','0','Position',[10  y2 60 22]);
edtA22 = uieditfield(pSS,'text','Value','1','Position',[75  y2 60 22]);
edtA23 = uieditfield(pSS,'text','Value','0','Position',[140 y2 60 22]);
edtA31 = uieditfield(pSS,'text','Value','0','Position',[10  y3 60 22]);
edtA32 = uieditfield(pSS,'text','Value','0','Position',[75  y3 60 22]);
edtA33 = uieditfield(pSS,'text','Value','1','Position',[140 y3 60 22]);

uilabel(pSS,'Text','B (3x1)','Position',[210 yLblTop 60 18]);
edtB1 = uieditfield(pSS,'text','Value','0','Position',[210 y1 60 22]);
edtB2 = uieditfield(pSS,'text','Value','0','Position',[210 y2 60 22]);
edtB3 = uieditfield(pSS,'text','Value','1','Position',[210 y3 60 22]);

uilabel(pSS,'Text','L (3x1)','Position',[280 yLblTop 60 18]);
edtL1 = uieditfield(pSS,'text','Value','0','Position',[280 y1 60 22]);
edtL2 = uieditfield(pSS,'text','Value','0','Position',[280 y2 60 22]);
edtL3 = uieditfield(pSS,'text','Value','0','Position',[280 y3 60 22]);

uilabel(pSS,'Text','C (1x3)','Position',[10 yLblMid 60 18]);
edtC1 = uieditfield(pSS,'text','Value','1','Position',[10  yMid 60 22]);
edtC2 = uieditfield(pSS,'text','Value','0','Position',[75  yMid 60 22]);
edtC3 = uieditfield(pSS,'text','Value','0','Position',[140 yMid 60 22]);

uilabel(pSS,'Text','D','Position',[210 yLblMid 30 18]);
edtD  = uieditfield(pSS,'text','Value','0','Position',[210 yMid 60 22]);

uilabel(pSS,'Text','Ki','Position',[210 yK2 30 18]);
edtKi = uieditfield(pSS,'text','Value','0','Position',[210 yK3 60 22]);

uilabel(pSS,'Text','K (3x1)','Position',[280 yLblMid 60 18]);
edtK1 = uieditfield(pSS,'text','Value','0','Position',[280 yK1 60 22]);
edtK2 = uieditfield(pSS,'text','Value','0','Position',[280 yK2 60 22]);
edtK3 = uieditfield(pSS,'text','Value','0','Position',[280 yK3 60 22]);

% init visuals
onFsChanged();
refreshVisibility();
updateDataInfo();
applyPlotScaling();

% =========================
% Helpers UI
% =========================
function logMsg(msg)
ts = string(datestr(now,'HH:MM:SS.FFF'));
line = "[" + ts + "] " + string(msg);
v = txtLog.Value; if ~isstring(v), v = string(v); end
v(end+1,1) = line;
if numel(v) > 600, v = v(end-600:end); end
txtLog.Value = v;
drawnow limitrate;
end

function ok = requireConn()
ok = S.isConnected && ~isempty(S.sp);
if ~ok, logMsg("Not connected."); end
end

function refreshVisibility()
typ = ddType.Value;
pTF.Visible = strcmp(typ,'TF');
pSS.Visible = strcmp(typ,'SS');
ddObserver.Enable   = strcmp(typ,'SS');
cbIntegrator.Enable = strcmp(typ,'SS');
end

function onFsChanged()
try
fs = double(edtFs.Value);
if ~isfinite(fs) || fs <= 0
lblFs.Text = "Fs inválida";
else
lblFs.Text = sprintf("Fs=%.6g Hz", fs);
end
catch
lblFs.Text = "Fs=?";
end
end

function updateDataInfo()
try, lblDataInfo.Text = sprintf("n=%d", numel(S.nVec)); catch, end
end

function onScaleChanged()
su = double(edtScaleU.Value);
sy = double(edtScaleY.Value);
if ~isfinite(su) || su==0, su=1; end
if ~isfinite(sy) || sy==0, sy=1; end
S.scaleU = su; S.scaleY = sy;
applyPlotScaling();
drawnow limitrate;
end

function applyPlotScaling()
su = double(S.scaleU); if ~isfinite(su) || su==0, su=1; end
sy = double(S.scaleY); if ~isfinite(sy) || sy==0, sy=1; end

if isempty(S.nVec)
set(hU,'XData',nan,'YData',nan);
set(hY,'XData',nan,'YData',nan);
else
set(hU,'XData',S.nVec,'YData',S.uVec*su);
set(hY,'XData',S.nVec,'YData',S.yVec*sy);
end
end

function setInTxnFalse()
S.inTxn = false;
end

% =========================
% Mode encoding
% =========================
function mode = computeMode()
typ = ddType.Value;
if strcmp(typ,'TF'), mode = 0; return; end
if strcmp(typ,'Open-loop'), mode = 5; return; end

isAct = strcmp(ddObserver.Value,'Actual');
hasI  = cbIntegrator.Value;

if ~isAct && ~hasI, mode = 1;
elseif isAct && ~hasI, mode = 2;
elseif ~isAct && hasI, mode = 3;
else, mode = 4;
end
end

function ord = getTFOrder()
ord = round(double(edtTFOrder.Value));
if ~isfinite(ord), ord = 5; end
ord = max(0, min(TF_MAX_ORDER, ord));
end

function [b,a] = readTFTable()
D = tfTable.Data;
if iscell(D)
D2 = nan(size(D));
for ii=1:size(D,1)
for jj=1:size(D,2)
v = D{ii,jj};
if isstring(v) || ischar(v)
x = str2double(strtrim(string(v)));
else
x = double(v);
end
D2(ii,jj) = x;
end
end
D = D2;
else
D = double(D);
end
if any(~isfinite(D),'all'), error("TF table: NaN/Inf o texto inválido."); end
b = D(:,1).'; a = D(:,2).';
end

% =========================
% UARTP LL (matching uartp_sw.c)
% =========================
function ll_flush()
if isempty(S.sp), return; end
try flush(S.sp); catch, end
end

function b = ll_readexact(n, timeout_s)
if nargin<2, timeout_s = STEP_TIMEOUT_S; end
t0 = tic;
buf = zeros(1,n,'uint8');
k = 0;
while k < n
if toc(t0) > timeout_s
error("Timeout leyendo %d bytes (tengo %d)", n, k);
end
avail = S.sp.NumBytesAvailable;
if avail > 0
m = min(avail, n-k);
tmp = read(S.sp, m, "uint8");
buf(k+1:k+m) = tmp(:);
k = k + m;
else
pause(0.001);
end
end
b = buf(:);
end

function rsp = ll_cmd_wait(cmd_char, timeout_s)
if nargin<2, timeout_s = STEP_TIMEOUT_S; end
valid = uint8(['R','S','K','!']);

tx = uint8(cmd_char);
write(S.sp, tx, "uint8");
logBytes("TX CMD", tx);

t0 = tic;
while true
if toc(t0) > timeout_s
error("Timeout esperando respuesta a '%s'", char(cmd_char));
end
if S.sp.NumBytesAvailable > 0
bb = read(S.sp, 1, "uint8"); bb = bb(1);
logBytes("RX RSP1", bb);
if any(bb == valid)
rsp = bb;
return;
end
else
pause(0.001);
end
end
end

% wrapper (seguridad)
function ll_send_payload(payload_u8) %#ok<DEFNU>
ll_send_payload2(payload_u8);
end

function ll_send_payload2(payload_u8)
payload_u8 = uint8(payload_u8(:));
logBytes("TX PAYLOAD", payload_u8);

n = numel(payload_u8);
idx = 1;

while idx <= n
w = uint8([0;0;0;0]);
take = min(4, n-idx+1);
w(1:take) = payload_u8(idx:idx+take-1);

tries = 0;
while true
tries = tries + 1;
if tries > MAX_WORD_RETRIES
error("Demasiados reintentos enviando word en idx=%d", idx);
end

write(S.sp, w, "uint8");

echo  = ll_readexact(4, STEP_TIMEOUT_S);
match = isequal(echo(:), w(:));

if match, write(S.sp, CTL_ACK, "uint8");
else,     write(S.sp, CTL_NAK, "uint8");
end

conf = ll_readexact(1, STEP_TIMEOUT_S); conf = conf(1);

if match && conf == CTL_ACK
break;
end
end

idx = idx + take;
end
end

function payload = ll_recv_payload2(nbytes)
payload = zeros(nbytes,1,'uint8');
off = 0;

while off < nbytes
rem = nbytes - off;
store = min(4, rem);

tries = 0;
while true
tries = tries + 1;
if tries > MAX_WORD_RETRIES
error("Demasiados reintentos recibiendo word (off=%d)", off);
end

w = ll_readexact(4, STEP_TIMEOUT_S);
write(S.sp, w, "uint8"); % eco

ctl = ll_readexact(1, STEP_TIMEOUT_S); ctl = ctl(1);

if ctl == CTL_ACK
payload(off+1:off+store) = w(1:store);
write(S.sp, CTL_ACK, "uint8");
off = off + store;
break;
else
write(S.sp, CTL_NAK, "uint8");
end
end
end

logBytes("RX PAYLOAD", payload);
end

function payload_u8 = ll_to_payload_bytes(coeffs)
if isa(coeffs,'uint8')
payload_u8 = coeffs(:);
assert(mod(numel(payload_u8),4)==0, "payload uint8 debe ser múltiplo de 4 bytes");
return;
end
c = single(coeffs(:));
payload_u8 = typecast(c,'uint8');
payload_u8 = payload_u8(:);
end

% =========================
% UARTP HL commands (local)
% =========================
function uartp_setmode_local(mode)
ll_flush();
rsp = ll_cmd_wait('m', STEP_TIMEOUT_S);
if rsp ~= RSP_READY_RX
error("m no devolvió R (rsp=%c)", char(rsp));
end
raw = uint8([mode;0;0;0]); % 4 bytes
ll_send_payload2(raw);
final = ll_readexact(1, STEP_TIMEOUT_S); final = final(1);
if final ~= RSP_OK
error("m no terminó con K (rsp=%c)", char(final));
end
end

function uartp_send_coeffs_local(coeffs25, verify_roundtrip)
if nargin<2, verify_roundtrip = true; end
ll_flush();
rsp = ll_cmd_wait('c', STEP_TIMEOUT_S);
if rsp ~= RSP_READY_RX
error("c no devolvió R (rsp=%c)", char(rsp));
end
payload = ll_to_payload_bytes(coeffs25);
if numel(payload) ~= 100
error("coeffs deben ser 25 floats => 100 bytes (tengo %d)", numel(payload));
end
ll_send_payload2(payload);
final = ll_readexact(1, STEP_TIMEOUT_S); final = final(1);
if final ~= RSP_OK
error("c no terminó con K (rsp=%c)", char(final));
end

if verify_roundtrip
rx = uartp_get_coeffs_raw_local();
if ~isequal(rx(:), payload(:))
error("Roundtrip 't' devolvió bytes distintos");
end
end
end

function rx = uartp_get_coeffs_raw_local()
ll_flush();
rsp = ll_cmd_wait('t', STEP_TIMEOUT_S);
if rsp ~= RSP_READY_TX
error("t no devolvió S (rsp=%c)", char(rsp));
end
rx = ll_recv_payload2(100); % 100 bytes = 25 floats
final = ll_readexact(1, STEP_TIMEOUT_S); final = final(1);
if final ~= RSP_OK
error("t no terminó con K (rsp=%c)", char(final));
end
end

function uartp_init_local(r_float)
ll_flush();
rsp = ll_cmd_wait('i', STEP_TIMEOUT_S);
if rsp ~= RSP_READY_RX
error("i no devolvió R (rsp=%c)", char(rsp));
end
payload = typecast(single(r_float),'uint8'); payload = payload(:);
ll_send_payload2(payload);
final = ll_readexact(1, STEP_TIMEOUT_S); final = final(1);
if final ~= RSP_OK
error("i no terminó con K (rsp=%c)", char(final));
end
end

function uartp_stop_local()
ll_flush();
rsp = ll_cmd_wait('s', STEP_TIMEOUT_S);
if rsp ~= RSP_OK
error("s no devolvió K (rsp=%c)", char(rsp));
end
end

function uartp_reset_local()
ll_flush();
rsp = ll_cmd_wait('r', STEP_TIMEOUT_S);
if rsp ~= RSP_OK
error("r no devolvió K (rsp=%c)", char(rsp));
end
end

% =========================
% Packets: TF25 / SS25 (3 estados)
% =========================
function coeffs = make_tf25(b, a, order, N, FsHz)
b = double(b(:).'); a = double(a(:).');
order = round(double(order));
order = max(0, min(TF_MAX_ORDER, order));

bb = zeros(1,TF_MAX_ORDER+1);
aa = zeros(1,TF_MAX_ORDER+1);
bb(1:min(numel(b),TF_MAX_ORDER+1)) = b(1:min(numel(b),TF_MAX_ORDER+1));
aa(1:min(numel(a),TF_MAX_ORDER+1)) = a(1:min(numel(a),TF_MAX_ORDER+1));

if order < TF_MAX_ORDER
bb(order+2:end) = 0;
aa(order+2:end) = 0;
end

coeffs = single(zeros(25,1));
coeffs(1:11) = single(bb(:));
coeffs(12:22)= single(aa(:));
coeffs(23)   = single(order);
coeffs(24)   = single(N);
coeffs(25)   = single(FsHz);
end

function coeffs = make_ss25_3x3(A,B,C,D,L,K,Ki,N,FsHz)
A=single(A); B=single(B(:)); C=single(C(:)).'; D=single(D);
L=single(L(:)); K=single(K(:)); Ki=single(Ki);

assert(isequal(size(A),[3 3]), "A debe ser 3x3");
assert(numel(B)==3, "B debe tener 3");
assert(numel(C)==3, "C debe tener 3");
assert(numel(L)==3, "L debe tener 3");
assert(numel(K)==3, "K debe tener 3");

coeffs = single(zeros(25,1));

% Layout SS25:
%  1..9   A row-major
% 10..12  B
% 13..15  C
% 16      D
% 17..19  L
% 20..22  K
% 23      Ki
% 24      N
% 25      FsHz

coeffs(1:9)   = reshape(A.',9,1);
coeffs(10:12) = B(:);
coeffs(13:15) = C(:);
coeffs(16)    = D;
coeffs(17:19) = L(:);
coeffs(20:22) = K(:);
coeffs(23)    = Ki;
coeffs(24)    = single(N);
coeffs(25)    = single(FsHz);
end

% =========================
% Actions
% =========================
function onConnectToggle(~,~)
if ~S.isConnected
com = strtrim(string(edtCom.Value));
if com == "", logMsg("COM vacío."); return; end
try
S.sp = serialport(com, edtBaud.Value);
configureTerminator(S.sp, "LF");
S.sp.Timeout = 0.1;

flush(S.sp);
pause(0.05);
ll_flush();

S.isConnected = true;
btnConnect.Text = "Disconnect";
lblStat.Text = "CONNECTED: " + com;

stopStreaming();
S.streamParseEnabled = false;
S.streamRxBuf = uint8([]);

logMsg("Connected to " + com + " (NO timer; stream OFF until Start)");
startStreaming();

catch e
logMsg("Connect error: " + string(e.message));
S.isConnected = false;
S.sp = [];
end
else
try
stopStreaming();
if ~isempty(S.sp)
try flush(S.sp); catch, end
delete(S.sp);
end
catch
end
S.sp = [];
S.isConnected = false;
btnConnect.Text = "Connect";
lblStat.Text = "DISCONNECTED";
logMsg("Disconnected.");
end
end

function onReset(~,~)
if ~requireConn(); return; end
try
S.inTxn = true; clean = onCleanup(@setInTxnFalse); %#ok<NASGU>
stopStreaming();
ll_flush();
uartp_reset_local();
logMsg("reset OK (r)");
catch e
logMsg("reset FAIL: " + string(e.message));
end
startStreaming();
end

function onSendMode(~,~)
if ~requireConn(); return; end
try
S.inTxn = true; clean = onCleanup(@setInTxnFalse); %#ok<NASGU>
stopStreaming();
ll_flush();

mode = computeMode();
uartp_setmode_local(uint8(mode));

S.streamParseEnabled = false;
S.streamRxBuf = uint8([]);
logMsg(sprintf("setmode OK: mode=%d (%s) (stream OFF until Start)", mode, ddType.Value));
catch e
logMsg("setmode FAIL: " + string(e.message));
end
startStreaming();
end

function onSendCoeffs(~,~)
if ~requireConn(); return; end
typ = ddType.Value;

try
S.inTxn = true; clean = onCleanup(@setInTxnFalse); %#ok<NASGU>
stopStreaming();
ll_flush();

Nval = single(round(double(edtN.Value)));
FsHz = double(edtFs.Value);
if ~isfinite(FsHz) || FsHz <= 0, error("Fs inválida"); end

if strcmp(typ,'TF')
[b,a] = readTFTable();
ord = getTFOrder();
coeffs = make_tf25(b,a,ord,Nval,FsHz);

elseif strcmp(typ,'SS')
A = [ str2double(edtA11.Value) str2double(edtA12.Value) str2double(edtA13.Value);
str2double(edtA21.Value) str2double(edtA22.Value) str2double(edtA23.Value);
str2double(edtA31.Value) str2double(edtA32.Value) str2double(edtA33.Value) ];
B = [ str2double(edtB1.Value); str2double(edtB2.Value); str2double(edtB3.Value) ];
Cc= [ str2double(edtC1.Value) str2double(edtC2.Value) str2double(edtC3.Value) ];
D =  str2double(edtD.Value);
L = [ str2double(edtL1.Value); str2double(edtL2.Value); str2double(edtL3.Value) ];
K = [ str2double(edtK1.Value); str2double(edtK2.Value); str2double(edtK3.Value) ];
Ki=  str2double(edtKi.Value);

coeffs = make_ss25_3x3(A,B,Cc,D,L,K,Ki,Nval,FsHz);

else
coeffs = single(zeros(25,1));
coeffs(IDX_META_N)  = Nval;
coeffs(IDX_META_FS) = single(FsHz);
end

uartp_send_coeffs_local(coeffs, cbVerify.Value);

S.streamParseEnabled = false;
S.streamRxBuf = uint8([]);
logMsg("coeffs OK (" + typ + ") (25 floats). stream OFF until Start.");

catch e
logMsg("coeffs FAIL: " + string(e.message));
end

startStreaming();
end

function onGetCoeffs(~,~)
if ~requireConn(); return; end
try
S.inTxn = true; clean = onCleanup(@setInTxnFalse); %#ok<NASGU>
stopStreaming();
ll_flush();

rx_raw = uartp_get_coeffs_raw_local();   % 100 bytes
logBytes("t RAW (100B)", rx_raw);

c = typecast(uint8(rx_raw(:)).','single'); % 25 singles
c = c(:);

prettyPrintCoeffs25(c, "RX coeffs");

typ = ddType.Value;
if strcmp(typ,'TF')
b = double(c(1:11)).';
a = double(c(12:22)).';
ord = double(c(IDX_TF_ORDER));
N   = double(c(IDX_META_N));
Fs  = double(c(IDX_META_FS));

tfTable.Data = [b(:), a(:)];
edtTFOrder.Value = min(max(round(ord),0),TF_MAX_ORDER);

logMsg("t OK (TF25)");
logMsg(sprintf("order=%g  N=%g  Fs=%.6g", ord, N, Fs));
logMsg("b = [" + strjoin(string(b), " ") + "]");
logMsg("a = [" + strjoin(string(a), " ") + "]");
else
N  = double(c(IDX_META_N));
Fs = double(c(IDX_META_FS));
logMsg("t OK (25)");
logMsg(sprintf("meta: N=%g Fs=%.6g", N, Fs));
end

catch e
logMsg("t FAIL: " + string(e.message));
end
startStreaming();
end

function onStart(~,~)
if ~requireConn(); return; end
try
S.inTxn = true; clean = onCleanup(@setInTxnFalse); %#ok<NASGU>
stopStreaming();
ll_flush();

r_ui   = double(edtU0.Value);
r_mult = double(edtRmult.Value); if ~isfinite(r_mult), r_mult = 1; end
r_send = r_ui * r_mult;

uartp_init_local(r_send);

S.autoStopEnabled = logical(cbAutoStop.Value);
tgt = round(double(edtAutoStopN.Value));
if ~isfinite(tgt) || tgt < 0, tgt = 0; end
S.autoStopTarget = tgt;
S.autoStopStartBase = S.framesTotal;
S.autoStopPending = false;
S.autoStopReason = "";
S.autoStopArmed = (S.autoStopEnabled && S.autoStopTarget > 0);

S.streamParseEnabled = true;
S.streamRxBuf = uint8([]);
logMsg(sprintf("init OK: r_sent=%.6g (stream ON)", r_send));

catch e
logMsg("init FAIL: " + string(e.message));
S.streamParseEnabled = false;
end
startStreaming();
end

function onStop(~,~)
if ~requireConn(); return; end
try
S.inTxn = true; clean = onCleanup(@setInTxnFalse); %#ok<NASGU>

S.autoStopPending = false;
S.autoStopArmed = false;

S.streamParseEnabled = false;
stopStreaming();
ll_flush();

uartp_stop_local();

S.streamRxBuf = uint8([]);
logMsg("stop OK (s). stream OFF.");
catch e
logMsg("stop FAIL: " + string(e.message));
end
startStreaming();
end

function onExportMAT(~,~)
try
if isempty(S.nVec), uialert(fig,"No hay datos.","Export"); return; end
[file,path] = uiputfile("*.mat","Guardar datos",".\psoc_stream.mat");
if isequal(file,0), logMsg("Export cancelado."); return; end
data = struct();
data.n = S.nVec; data.u = S.uVec; data.y = S.yVec;
data.timestamp = datestr(now,'yyyy-mm-dd HH:MM:SS.FFF');
save(fullfile(path,file), "-struct", "data");
logMsg("Export OK: " + string(fullfile(path,file)));
catch e
logMsg("Export FAIL: " + string(e.message));
end
end

function onClearData(~,~)
S.nVec=zeros(0,1); S.uVec=zeros(0,1); S.yVec=zeros(0,1);
S.streamRxBuf=uint8([]); S.framesTotal=0;
applyPlotScaling(); updateDataInfo();
logMsg("Data cleared.");
end

% =========================
% Streaming
% =========================
function startStreaming()
stopStreaming();
if ~S.isConnected || isempty(S.sp), return; end

S.streamTimer = timer('ExecutionMode','fixedSpacing','Period',0.02, ...
'TimerFcn',@onStreamTick,'BusyMode','drop');
start(S.streamTimer);

if S.streamParseEnabled
logMsg("Streaming reader ON (parsing ENABLED) (frames 8B: u,y)");
else
logMsg("Streaming reader ON (parsing DISABLED) (discarding UART bytes)");
end
end

function stopStreaming()
try
if ~isempty(S.streamTimer) && isvalid(S.streamTimer)
stop(S.streamTimer); delete(S.streamTimer);
end
catch
end
S.streamTimer = [];
end

function onStreamTick(~,~)
if S.inTxn, return; end
if ~S.isConnected || isempty(S.sp), return; end

try
if S.autoStopPending
S.autoStopPending = false;
onStop();
return;
end

if ~S.streamParseEnabled
return;
end

nAvail = S.sp.NumBytesAvailable;
if nAvail <= 0, return; end

raw = read(S.sp, nAvail, "uint8");
if isempty(raw), return; end

S.streamRxBuf = [S.streamRxBuf; uint8(raw(:))];

n = numel(S.streamRxBuf);
nFrames = floor(n/8);
if nFrames <= 0, return; end

take = 8*nFrames;
blk = S.streamRxBuf(1:take);
S.streamRxBuf = S.streamRxBuf(take+1:end);

frames = reshape(blk, 8, []).';

u_u32 = uint32(frames(:,1)) ...
+ bitshift(uint32(frames(:,2)),8) ...
+ bitshift(uint32(frames(:,3)),16) ...
+ bitshift(uint32(frames(:,4)),24);

y_u32 = uint32(frames(:,5)) ...
+ bitshift(uint32(frames(:,6)),8) ...
+ bitshift(uint32(frames(:,7)),16) ...
+ bitshift(uint32(frames(:,8)),24);

u = typecast(u_u32,'single');
y = typecast(y_u32,'single');

idx0 = S.framesTotal;
nIdx = (idx0 + (1:numel(u))).';

S.nVec = [S.nVec; nIdx];
S.uVec = [S.uVec; double(u)];
S.yVec = [S.yVec; double(y)];

if numel(S.nVec) > S.maxPoints
k0 = numel(S.nVec)-S.maxPoints+1;
S.nVec = S.nVec(k0:end);
S.uVec = S.uVec(k0:end);
S.yVec = S.yVec(k0:end);
end

S.framesTotal = S.framesTotal + numel(u);

applyPlotScaling();
updateDataInfo();

if S.autoStopArmed
framesSinceStart = S.framesTotal - S.autoStopStartBase;
if framesSinceStart >= S.autoStopTarget
S.autoStopArmed = false;
S.autoStopPending = true;
S.autoStopReason = sprintf("framesSinceStart=%d >= %d", framesSinceStart, S.autoStopTarget);
logMsg("Auto-stop TRIGGER: " + S.autoStopReason);
end
end

drawnow limitrate;

catch e
S.streamRxBuf = uint8([]);
logMsg("stream WARN: " + string(e.message));
end
end

function onClose(~,~)
try
stopStreaming();
if S.isConnected && ~isempty(S.sp)
try flush(S.sp); catch, end
try delete(S.sp); catch, end
end
catch
end
delete(fig);
end

% =========================
% Debug helpers
% =========================
function s = hexDump(u8, maxLen)
if nargin<2, maxLen = 64; end
u8 = uint8(u8(:));
n = numel(u8);
m = min(n, maxLen);
hh = upper(string(dec2hex(u8(1:m),2)));
s = strjoin(hh.', ' ');
if n > m
s = s + " ... (+" + string(n-m) + " bytes)";
end
end

function logBytes(tag, u8)
if ~DBG_LOG_BYTES, return; end
if isempty(u8), logMsg(tag + ": <empty>"); return; end
u8 = uint8(u8(:));
if DBG_HEX_DUMP
logMsg(tag + " [" + string(numel(u8)) + "B] HEX: " + hexDump(u8, 80));
else
logMsg(tag + " [" + string(numel(u8)) + "B]");
end
end

function prettyPrintCoeffs25(c, label)
if nargin<2, label="coeffs"; end
if ~DBG_PRINT_COEFFS, return; end
c = double(c(:));
logMsg(label + " (25 floats):");
M = reshape(c, 5, 5).';
for r=1:5
line = sprintf("[%2d..%2d]  % .7g  % .7g  % .7g  % .7g  % .7g", ...
(r-1)*5+1, (r-1)*5+5, M(r,1), M(r,2), M(r,3), M(r,4), M(r,5));
logMsg(line);
end
end

end

\end{lstlisting}

\section{Código de síntesis directa y simulación intramuestra}
\label{ap:codigo_TR}

A continuación se presenta el script utilizado para la
discretización de la planta, resolución del sistema
polinómico asociado al diseño ripple-free y evaluación
intramuestra mediante retención de orden cero.

\begin{lstlisting}[language=Matlab]
	close all; clear; clc;
	
	S = load("D:\auto\TPF-HelicopteroVertical\Matlab\planta (1).mat");
	
	if isfield(S,'plantaC')
	plantaC = S.plantaC;
	elseif isfield(S,'sysC')
	plantaC = S.sysC;
	elseif isfield(S,'G')
	plantaC = S.G;
	else
	error("No se encontró la planta en el archivo .mat.");
	end
	
	Ts = 1/2;  % Fs = 2 Hz
	[num, den] = tfdata(c2d(plantaC,Ts,'zoh'),'v');
	
	n = length(num)-1;
	m = length(den)-1;
	k = max(n,m);
	
	num_z = [num zeros(1,k-n)];
	den_z = [den zeros(1,k-m)];
	
	Hz = tf(num_z, den_z, Ts, 'Variable','z');
	
	z = tf([1 0],1,Ts);
	
	% --- Controlador sin oscilaciones intramuestra ---
	C1 = (1.73*z^2 - 0.2763*z + 0.1047) / ...
	(z^2 + 0.6683*z + 0.04352);
	
	% --- Controlador con oscilaciones intramuestra ---
	C2 = (z^3 - 1.16*z^2 + 0.2202*z - 0.0605) / ...
	(0.1917*z^5 + 0.3612*z^4 + 0.02515*z^3 ...
	-0.1917*z^2 -0.3612*z -0.02515);
	
	% --- Simulación intramuestra ---
	[ud,n] = step(feedback(C1,Hz));
	uc = repelem(ud,40);
	dt = Ts/40;
	t  = (0:length(uc)-1)' * dt;
	[yc,t] = lsim(plantaC, uc, t);
\end{lstlisting}

% ============================================================
\section{Espacio de estado con integrador)}
\label{ap:ogata_integrador_obs}
% ============================================================

En este apendice se incluye el script completo utilizado para implementar el metodo de Ogata 6.19
con un integrador externo (accion integral) y dos variantes de observador en tiempo discreto:
predictor y actual. El codigo calcula las ganancias \(K_1\) y \(K_2\) mediante \texttt{place} sobre el
sistema aumentado, disena los observadores con los polos \(p_{\text{obs}}\), y genera las simulaciones y
graficos asociados.

\begin{lstlisting}[language=Matlab,caption={OGATA 6.19 - SOLO INTEGRADOR + Observador PREDICTOR y ACTUAL (planta orden 3).},label={lst:ogata619_integrador_pred_act}]
	%% =========================
	%  OGATA 6.19 - SOLO INTEGRADOR
	%  + Observador PREDICTOR y ACTUAL
	%  (Planta discreta desde .mat, orden 3)
	%% =========================
	close all; clear; clc
	
	%% =========================
	% 1) CARGA + DISCRETIZACION
	%% =========================
	S = load('planta (1).mat');
	
	if isfield(S,'plantaC')
	plantaC = S.plantaC;
	elseif isfield(S,'sysC')
	plantaC = S.sysC;
	else
	error('No encuentro "plantaC" ni "sysC" dentro de planta (1).mat');
	end
	
	Ts   = 1/50;                 % ajusta si queres
	sysD = c2d(plantaC, Ts, 'zoh');
	[A,B,C,D] = ssdata(ss(sysD));
	
	n = size(A,1);
	if n ~= 3
	error('Esta plantilla asume planta de orden 3. n=%d', n);
	end
	
	% Asegurar SISO (una salida)
	if size(C,1) ~= 1
	C = C(1,:);
	end
	
	fprintf('Ts=%.9f | n=%d\n', Ts, n);
	
	%% =========================
	% 2) POLOS DESEADOS (FIJOS)
	%% =========================
	p_ctrl = [0.95 + 0.15i, 0.95 - 0.15i, 0.98];   % control
	p_obs  = [0.8 + 0.25i, 0.8 - 0.25i, 0.9];
	%p_obs  = [0.4 + 0.25i, 0.4 - 0.25i, 0.6];      % observador
	p_i    = 0.96;                                  % polo integrador
	
	%% =========================
	% 3) OGATA 6.19 (K1 y K2) TAL CUAL
	%% =========================
	m = 1;
	
	Ahat = [A B; zeros(m,n+m)];
	Bhat = [zeros(n,m); eye(m)];
	Chat = [C zeros(1,m)];
	
	polos_i = [p_ctrl p_i];           % 3 polos control + 1 integrador
	[Khat,pKhat]    = place(Ahat, Bhat, polos_i);
	
	Aux  = [A-eye(size(A))  B;
	C*A             C*B];
	
	K2K1 = (Khat + [zeros(1,n) eye(m)]) / Aux;
	K2   = K2K1(1,1:n);               % sobre xhat (1x3)
	K1   = K2K1(1,n+1:end);           % sobre v   (1x1)
	
	fprintf('\nK2 = '); disp(K2);
	fprintf('K1 = '); disp(K1); fprintf('\tpK21 = '); disp(pKhat);
	
	%% =========================
	% 4) OBSERVADORES (PRED / ACT) - mismo p_obs
	%% =========================
	[L_pred, pL_pred]     = place(A', C',     p_obs);   L_pred   = L_pred.';    % predictor: A - L*C
	[L_actual, pL_actual] = place(A', (C*A)', p_obs);   L_actual = L_actual.';  % actual:    A - L*C*A
	
	fprintf('L_pred = '); disp(L_pred); fprintf('\tpL_pred = '); disp(pL_pred);
	fprintf('\nL_actual = '); disp(L_actual); fprintf('\tpL_actual = '); disp(pL_actual);
	
	%% =========================
	% 5) SIMULACION (SOLO CON INTEGRADOR)
	%% =========================
	N  = 300;
	t  = 0:Ts:(N-1)*Ts;
	
	r = ones(1,N)*25;
	r(1:30) = 0;
	
	% ruido en medicion (igual que tu estilo)
	w = [zeros(1,75), 0.01*ones(1,N-75)];
	
	% ----- PREDICTIVO CON integrador -----
	X_p    = zeros(n,N);      % estado real
	Xh_p   = zeros(n,N);      % estado estimado
	V_p    = zeros(1,N);      % integrador
	U_p    = zeros(1,N);      % control
	
	% ----- ACTUAL CON integrador -----
	X_a    = zeros(n,N);
	Xh_a   = zeros(n,N);
	V_a    = zeros(1,N);
	U_a    = zeros(1,N);
	
	for k = 1:N-1
	rk = r(k);
	
	%% ===== 1) PREDICTIVO + integrador (MISMA FORMA) =====
	y_p          = C*X_p(:,k);
	V_p(k+1)     = V_p(k) + (rk - y_p);                 % integrador real
	u_p          = K1*V_p(k+1) - K2*Xh_p(:,k);
	X_p(:,k+1)   = A*X_p(:,k) + B*u_p;% + w(k);
	Xh_p(:,k+1)  = A*Xh_p(:,k) + B*u_p + L_pred*(y_p - C*Xh_p(:,k));
	U_p(k)       = u_p;
	
	%% ===== 2) ACTUAL + integrador (MISMA FORMA) =====
	y_a          = C*X_a(:,k);
	V_a(k+1)     = V_a(k) + (rk - y_a);
	u_a          = K1*V_a(k+1) - K2*Xh_a(:,k);
	X_a(:,k+1)   = A*X_a(:,k) + B*u_a;% + w(k);
	y_next       = C*X_a(:,k+1);% + w(k);                 % misma muestra de ruido
	z_a          = A*Xh_a(:,k) + B*u_a;
	Xh_a(:,k+1)  = z_a + L_actual*(y_next - C*z_a);
	U_a(k)       = u_a;
	end
	
	%% =========================
	% 6) PLOTS RAPIDOS
	%% =========================
	figure('Name','y(t) - integrador + predictor vs actual');
	plot(t, (C*X_p).', t, (C*X_a).', t, r, 'k--','LineWidth',1.2);
	grid on; grid minor;
	xlabel('t [s]'); ylabel('y');
	legend('Pred (planta)','Act (planta)','r','Location','best');
	
	figure('Name','u(t) - integrador + predictor vs actual');
	plot(t(1:end-1), U_p(1:end-1), 'LineWidth',1.4); hold on;
	plot(t(1:end-1), U_a(1:end-1), '--', 'LineWidth',1.4);
	grid on; grid minor;
	xlabel('t [s]'); ylabel('u');
	legend('Pred','Act','Location','best');
	
	figure('Name','v(t) - integrador (estado integral)');
	plot(t, V_p, 'LineWidth',1.4); hold on;
	plot(t, V_a, '--', 'LineWidth',1.4);
	grid on; grid minor;
	xlabel('t [s]'); ylabel('v');
	legend('Pred','Act','Location','best');
	
	%% =========================
	% 7) MAPA DE POLOS (resumen)
	%% =========================
	figure('Name','Z-plane - planta, aumentado e info obs','Position',[100 100 1000 400]);
	
	subplot(1,3,1); zgrid; hold on; grid on; box on;
	p_planta = eig(A);
	plot(real(p_planta), imag(p_planta), 'ko','MarkerSize',8,'LineWidth',1.4);
	title('Planta (A)'); xlabel('Re\{z\}'); ylabel('Im\{z\}');
	
	subplot(1,3,2); zgrid; hold on; grid on; box on;
	p_aug    = eig(Ahat);
	p_aug_cl = eig(Ahat - Bhat*Khat);
	plot(real(p_aug),    imag(p_aug),    'm^','MarkerSize',8,'LineWidth',1.4);
	plot(real(p_aug_cl), imag(p_aug_cl), 'c*','MarkerSize',10,'LineWidth',1.6);
	title('Aumentado: Ahat y Ahat-BhatKhat'); xlabel('Re\{z\}'); ylabel('Im\{z\}');
	
	subplot(1,3,3); zgrid; hold on; grid on; box on;
	p_obs_pred = eig(A - L_pred*C);
	p_obs_act  = eig(A - L_actual*C*A);
	plot(real(p_obs_pred), imag(p_obs_pred), 'bs','MarkerSize',9,'LineWidth',1.5);
	plot(real(p_obs_act),  imag(p_obs_act),  'rx','MarkerSize',9,'LineWidth',1.5);
	title('Obs error'); xlabel('Re\{z\}'); ylabel('Im\{z\}');
	legend('A-LC (pred)','A-LCA (act)','Location','best');
	
	sgtitle('Ogata 6.19 - Solo integrador + predictor/actual');
	
	Aaug = [A,B;...
	K2-K2*A-K1*C*A,1-K2*B-K1*C*B];
	Baug = [0;0;0;K1];
	Caug    = [C,0];
	D = 0;
	sysDaug = ss(Aaug,Baug,Caug,D,Ts);
	sysObs = ss((A-L_actual*C*A),B,C,D,Ts);
	figure; pzmap(sysDaug); grid on; zgrid; hold on; pzmap(sysObs);
\end{lstlisting}

% ============================================================
\appendix
\section{Script de LQR con observador y simulacion con saturacion y ruido}
\label{ap:lqr_obs_sim}
% ============================================================

En este apendice se incluye el script MATLAB utilizado para:
(i) cargar y discretizar la planta identificada,
(ii) definir las ponderaciones \(Q\) y \(R\) del LQR,
(iii) calcular la ganancia optima \(K\) (via \texttt{dlqr}/\texttt{dare}),
(iv) disenar observadores (predictor y actual) por ubicacion arbitraria de polos,
(v) calcular el prefiltro \(N_{\mathrm{bar}}\) para seguimiento de referencia,
y (vi) simular el lazo con/sin saturacion y con/sin ruido (incluyendo comparacion double vs \texttt{float32}).

\begin{lstlisting}[language=Matlab,caption={LQR + Observador (predictor/actual) + Nbar + Simulacion con saturacion y ruido},label={lst:lqr_obs_sim}]
	close all; clear; clc
	
	%% =========================
	% 1) CARGA + DISCRETIZACION
	%% =========================
	S = load('planta (1).mat');
	
	if isfield(S,'plantaC')
	plantaC = S.plantaC;
	elseif isfield(S,'sysC')
	plantaC = S.sysC;
	else
	error('No encuentro "plantaC" ni "sysC" dentro de planta (1).mat');
	end
	
	Ts   = 1/100;                 % sample time
	sysD = c2d(plantaC, Ts, 'zoh');
	[A,B,C,D] = ssdata(ss(sysD));
	n = size(A,1);
	
	fprintf('Ts=%.9f | n=%d\n', Ts, n);
	disp('A='); disp(A); disp('B='); disp(B); disp('C='); disp(C); disp('D='); disp(D);
	
	%% =========================
	% 2) PARAMETROS + PESOS (LQR)
	%% =========================
	% Observador (z-plane)
	p_obs  = [0.8 + 0.25i, 0.8 - 0.25i, 0.9];
	
	% Objetivo practico: step ~20 y u limitado a +/-300
	r_step = 20;
	u_max  = 300;
	
	% knobs
	wy = 20;          % subir => mas seguimiento (mas agresivo)
	wu = 500;         % subir => menos esfuerzo (mas timido)
	
	% Q y R coherentes
	Q = wy*(C'*C) + 1e-8*eye(n);
	R = wu/(u_max);
	
	%% =========================
	% 3) GANANCIAS: LQR + OBSERVADOR + Nbar
	%% =========================
	rc = rank(ctrb(A,B));
	if rc < n
	error('El par (A,B) NO es controlable (rank=%d < n=%d).', rc, n);
	end
	
	% --- LQR discreto (con fallback) ---
	if exist('dlqr','file') == 2
	[K, P, e_cl] = dlqr(A, B, Q, R);
	elseif exist('dare','file') == 2
	[P,~,~] = dare(A,B,Q,R);
	K = (R + B'*P*B)\(B'*P*A);
	e_cl = eig(A - B*K);
	else
	[P, K, e_cl, info] = dlqr_iter_nolic(A,B,Q,R);
	fprintf('DLQR sin toolbox: iters=%d, err=%.3e\n', info.iters, info.err);
	end
	
	fprintf('LQR: eig(A-BK) = \n'); disp(e_cl.');
	
	% --- Observador (place) ---
	Ke_pred = place(A', C', p_obs).';            % predictor: A - Ke*C
	Ke_act  = place(A', (C*A)', p_obs).';        % "actual": A - Ke*C*A
	
	% --- Nbar (SISO) ---
	if size(C,1) ~= 1
	error('Tu C no es SISO (tiene %d salidas). Elegi una fila de C.', size(C,1));
	end
	[~,~,Nbar] = refi(A, B, C, K);
	fprintf('Nbar=%.6g\n', Nbar);
	
	%% =========================
	% 4) SIMULACION: SIN RUIDO vs CON RUIDO (double y single)
	%% =========================
	N        = 200;
	ulim_sat = u_max;
	ulim_inf = Inf;
	
	% referencia
	r = zeros(1,N);
	r(2:end) = r_step;
	
	% --------- RUIDO (config) ----------
	rng(1);                % repetible
	
	noise.enable   = true;
	noise.sigma_y  = 2;    % ruido de medicion
	noise.sigma_u  = 2.0;  % jitter actuador
	noise.q_u      = 1.0;  % cuantizacion u
	noise.sigma_w  = 0.0;  % ruido de proceso
	
	noise.vy = noise.sigma_y * randn(1,N);
	noise.vu = noise.sigma_u * randn(1,N);
	noise.wx = noise.sigma_w * randn(n,N);
	
	noise_off = noise;
	noise_off.enable = false;
	noise_off.vy = zeros(1,N);
	noise_off.vu = zeros(1,N);
	noise_off.wx = zeros(n,N);
	
	% --- IDEAL (sin sat, sin ruido) ---
	out_pred_ideal = sim_obs_loop(A,B,C,D,K,Nbar,Ke_pred,Ke_act,r,Ts,ulim_inf,false,false,noise_off);
	out_act_ideal  = sim_obs_loop(A,B,C,D,K,Nbar,Ke_pred,Ke_act,r,Ts,ulim_inf,true ,false,noise_off);
	
	% --- REAL (sat, sin ruido) ---
	out_pred_sat_clean = sim_obs_loop(A,B,C,D,K,Nbar,Ke_pred,Ke_act,r,Ts,ulim_sat,false,false,noise_off);
	out_act_sat_clean  = sim_obs_loop(A,B,C,D,K,Nbar,Ke_pred,Ke_act,r,Ts,ulim_sat,true ,false,noise_off);
	
	% --- REAL (sat, con ruido) ---
	out_pred_sat_noise = sim_obs_loop(A,B,C,D,K,Nbar,Ke_pred,Ke_act,r,Ts,ulim_sat,false,false,noise);
	out_act_sat_noise  = sim_obs_loop(A,B,C,D,K,Nbar,Ke_pred,Ke_act,r,Ts,ulim_sat,true ,false,noise);
	
	% --- REAL (sat, con ruido) en FLOAT32 ---
	out_pred_sat_noise_f = sim_obs_loop(A,B,C,D,K,Nbar,Ke_pred,Ke_act,r,Ts,ulim_sat,false,true,noise);
	out_act_sat_noise_f  = sim_obs_loop(A,B,C,D,K,Nbar,Ke_pred,Ke_act,r,Ts,ulim_sat,true ,true,noise);
	
	t = (0:N-1)*Ts;
	
	%% =========================
	% 5) DIAGNOSTICOS
	%% =========================
	eyp = out_pred_sat_noise.y_meas - out_pred_sat_noise.y_true;
	eya = out_act_sat_noise.y_meas  - out_act_sat_noise.y_true;
	
	fprintf('\n--- CHECK RUIDO ---\n');
	fprintf('sigma_y=%.3g | RMS(y_meas-y_true) Pred=%.3g Act=%.3g\n', noise.sigma_y, rms(eyp), rms(eya));
	fprintf('sigma_u=%.3g | q_u=%g\n', noise.sigma_u, noise.q_u);
	
	fprintf('\n--- SATURACION ---\n');
	fprintf('Pred clean: max|u|=%.2f sat=%d\n', max(abs(out_pred_sat_clean.u)), sum(abs(out_pred_sat_clean.u) >= ulim_sat-1e-9));
	fprintf('Pred noise: max|u|=%.2f sat=%d\n', max(abs(out_pred_sat_noise.u)), sum(abs(out_pred_sat_noise.u) >= ulim_sat-1e-9));
	fprintf('Act  clean: max|u|=%.2f sat=%d\n', max(abs(out_act_sat_clean.u)),  sum(abs(out_act_sat_clean.u)  >= ulim_sat-1e-9));
	fprintf('Act  noise: max|u|=%.2f sat=%d\n', max(abs(out_act_sat_noise.u)),  sum(abs(out_act_sat_noise.u)  >= ulim_sat-1e-9));
	
	%% =========================
	% 6) PLOTS
	%% =========================
	figure('Name','Predictor: y (clean vs noise) [sat]');
	plot(t, out_pred_sat_clean.y_true, 'LineWidth',1.6); hold on;
	plot(t, out_pred_sat_noise.y_meas, '.-');
	plot(t, r, 'k--','LineWidth',1.2);
	grid on; xlabel('t [s]'); ylabel('y');
	legend('y true (clean)','y meas (noise)','r','Location','best');
	
	figure('Name','Predictor: u (clean vs noise) [sat]');
	plot(t, out_pred_sat_clean.u, 'LineWidth',1.6); hold on;
	plot(t, out_pred_sat_noise.u, '.-');
	yline(+ulim_sat,'k--'); yline(-ulim_sat,'k--');
	grid on; xlabel('t [s]'); ylabel('u');
	
	figure('Name','Actual: y (clean vs noise) [sat]');
	plot(t, out_act_sat_clean.y_true, 'LineWidth',1.6); hold on;
	plot(t, out_act_sat_noise.y_meas, '.-');
	plot(t, r, 'k--','LineWidth',1.2);
	grid on; xlabel('t [s]'); ylabel('y');
	
	figure('Name','Actual: u (clean vs noise) [sat]');
	plot(t, out_act_sat_clean.u, 'LineWidth',1.6); hold on;
	plot(t, out_act_sat_noise.u, '.-');
	yline(+ulim_sat,'k--'); yline(-ulim_sat,'k--');
	grid on; xlabel('t [s]'); ylabel('u');
	
	figure('Name','Ruido de medicion (y_meas - y_true)');
	plot(t, eyp, t, eya);
	grid on; xlabel('t [s]'); ylabel('error de medicion');
	legend('Pred','Act','Location','best');
	
	figure('Name','Float32 - Double (y_meas) [sat+noise]');
	plot(t, out_pred_sat_noise.y_meas - out_pred_sat_noise_f.y_meas, ...
	t, out_act_sat_noise.y_meas  - out_act_sat_noise_f.y_meas);
	grid on; xlabel('t [s]'); ylabel('double - float32');
	legend('Pred','Act','Location','best');
	
	%% =========================
	% 7) Z-PLANE (polos planta, CL, obs)
	%% =========================
	p_ol = eig(A);
	p_cl = eig(A - B*K);
	p_op = eig(A - Ke_pred*C);
	p_oa = eig(A - Ke_act*C*A);
	
	try
	z_plant = tzero(ss(A,B,C,D,Ts));
	catch
	z_plant = [];
	end
	
	plot_zplane('Z-plane: Predictor', p_ol, p_cl, p_op, z_plant);
	plot_zplane('Z-plane: Actual'   , p_ol, p_cl, p_oa, z_plant);
	
	%% ============================================================
	% ===================== FUNCIONES LOCALES ======================
	
	function out = sim_obs_loop(A,B,C,D,K,Nbar,Ke_pred,Ke_act,r,Ts,ulim,use_actual,use_single,noise)
	if nargin < 15 || isempty(noise)
	noise.enable=false; noise.vy=0; noise.vu=0; noise.wx=0; noise.q_u=Inf;
	end
	
	if use_single
	A=single(A); B=single(B); C=single(C); D=single(D);
	K=single(K); Nbar=single(Nbar);
	Ke_pred=single(Ke_pred); Ke_act=single(Ke_act);
	r=single(r); Ts=single(Ts); ulim=single(ulim);
	end
	
	n = size(A,1); N = numel(r);
	
	x  = zeros(n,N,'like',A);
	xh = zeros(n,N,'like',A);
	
	y_true = zeros(1,N,'like',A);
	y_meas = zeros(1,N,'like',A);
	
	u_cmd  = zeros(1,N,'like',A);
	u_app  = zeros(1,N,'like',A);
	
	vy = zeros(1,N,'like',A);
	vu = zeros(1,N,'like',A);
	wx = zeros(n,N,'like',A);
	
	if isfield(noise,'enable') && noise.enable
	vy = cast(noise.vy,'like',A);
	vu = cast(noise.vu,'like',A);
	wx = cast(noise.wx,'like',A);
	end
	
	q_u = Inf;
	if isfield(noise,'q_u'), q_u = noise.q_u; end
	q_u = cast(q_u,'like',A);
	
	y_true(1) = C*x(:,1) + D*0;
	y_meas(1) = y_true(1) + vy(1);
	
	for k=1:N-1
	u_unsat  = Nbar*r(k) - K*xh(:,k);
	u_cmd(k) = sat(u_unsat, ulim);
	
	u_app(k) = u_cmd(k) + vu(k);
	if isfinite(double(q_u))
	u_app(k) = round(u_app(k)/q_u)*q_u;
	end
	u_app(k) = sat(u_app(k), ulim);
	
	x(:,k+1) = A*x(:,k) + B*u_app(k) + wx(:,k);
	
	y_true(k)   = C*x(:,k)   + D*u_app(k);
	y_true(k+1) = C*x(:,k+1) + D*u_app(k);
	
	y_meas(k)   = y_true(k)   + vy(k);
	y_meas(k+1) = y_true(k+1) + vy(k+1);
	
	if ~use_actual
	yhat_k = C*xh(:,k) + D*u_app(k);
	xh(:,k+1) = A*xh(:,k) + B*u_app(k) + Ke_pred*( y_meas(k) - yhat_k );
	else
	z   = A*xh(:,k) + B*u_app(k);
	yzh = C*z + D*u_app(k);
	xh(:,k+1) = z + Ke_act*( y_meas(k+1) - yzh );
	end
	end
	
	u_cmd(N) = sat(Nbar*r(N) - K*xh(:,N), ulim);
	u_app(N) = u_cmd(N) + vu(N);
	if isfinite(double(q_u))
	u_app(N) = round(u_app(N)/q_u)*q_u;
	end
	u_app(N) = sat(u_app(N), ulim);
	
	y_true(N) = C*x(:,N) + D*u_app(N);
	y_meas(N) = y_true(N) + vy(N);
	
	out.x = x; out.xh = xh;
	out.y_true = y_true;
	out.y_meas = y_meas;
	out.u_cmd  = u_cmd;
	out.u_app  = u_app;
	
	out.y = y_meas;
	out.u = u_app;
	end
	
	function y = sat(u,lim)
	if isinf(lim)
	y = u;
	else
	y = min(max(u, -lim), lim);
	end
	end
	
	function plot_zplane(figName, p_ol, p_cl, p_obs, z_plant)
	figure('Name',figName,'NumberTitle','off');
	hold on; grid on; grid minor; axis equal;
	title('Z-plane');
	xlabel('Re{z}'); ylabel('Im{z}');
	
	th = linspace(0,2*pi,400);
	plot(cos(th), sin(th), 'k:'); % unit circle
	
	plot(real(p_ol),  imag(p_ol),  'o', 'LineWidth', 1.5);
	plot(real(p_cl),  imag(p_cl),  'x', 'LineWidth', 1.8);
	plot(real(p_obs), imag(p_obs), '^', 'LineWidth', 1.8);
	
	if ~isempty(z_plant)
	plot(real(z_plant), imag(z_plant), 's', 'LineWidth', 1.5);
	legend('unit circle','poles plant (A)','poles CL (A-BK)','poles obs','zeros plant','Location','bestoutside');
	else
	legend('unit circle','poles plant (A)','poles CL (A-BK)','poles obs','Location','bestoutside');
	end
	
	xlim([-1.2 1.2]); ylim([-1.2 1.2]);
	end
	
	function [Nx,Nu,Nbar] = refi(phi,gam,Hr,K)
	I=eye(size(phi));
	[m,n]=size(Hr);
	np=inv([phi-I gam;Hr zeros(m)])*([zeros(n,m);eye(m)]);
	Nx=np(1:n,:);
	Nu=np(n+1:n+m,:);
	Nbar=Nu+K*Nx;
	end
	
	function [P, K, e_cl, info] = dlqr_iter_nolic(A,B,Q,R)
	maxit = 5000;
	tol   = 1e-10;
	
	P = Q;
	err = Inf;
	
	for it = 1:maxit
	G = R + B'*P*B;
	Ktmp = G \ (B'*P*A);
	Pn = A'*P*A - A'*P*B*Ktmp + Q;
	
	err = norm(Pn - P, 'fro');
	P = Pn;
	
	if err < tol
	break;
	end
	end
	
	K = (R + B'*P*B) \ (B'*P*A);
	e_cl = eig(A - B*K);
	
	info.iters = it;
	info.err   = err;
	info.converged = (err < tol);
	
	if ~info.converged
	warning('Riccati iterativa NO convergio (err=%.3e).', err);
	end
	end
\end{lstlisting}

\input{estimarRuido.tex}
% ============================================================
\section{Apéndice: Cómputo de ganancias LQG/LQI y simulación (firmware friendly)}
\label{ap:lqg_ganancias}
% ============================================================

En este apéndice se incluye el script utilizado para obtener los parámetros
numéricos del esquema LQG/LQI empleado en el trabajo: ganancia del filtro de Kalman
en régimen permanente \(L\), ganancias de realimentación de estados \(K_x\) y del
integrador \(K_i\), junto con la verificación de polos del observador
\(\lambda(A-LC)\) y la simulación en presencia de ruido de proceso y medición.

El script fue construido con el objetivo de ser \textit{firmware friendly}, es decir,
replicar la lógica de estimación y control que luego se implementa en el microcontrolador:
\begin{itemize}
	\item \textbf{Estimador tipo \textit{current}}: se calcula \(L\) con \texttt{kalman(...,'current')}.
	\item \textbf{Ruido de proceso y medición}: se inyecta \(w_k\) en la dinámica del estado y \(v_k\) en la medición.
	\item \textbf{Control con integrador}: se emplea una acción integral \(\xi_k\) para asegurar error estacionario nulo.
\end{itemize}

\subsection{Estructura del diseño}

\paragraph{Planta discretizada}
A partir del modelo continuo identificado (\texttt{planta (1).mat}) se discretiza con ZOH a
\(\Ts = 0.01\ \text{s}\) (100 Hz), obteniendo el sistema discreto:

\[
x_{k+1} = A x_k + B u_k,
\qquad
y_k = C x_k + D u_k
\]

\paragraph{Covarianzas de ruido}
Las covarianzas utilizadas por el estimador se toman del archivo
\texttt{RQ\_tuning\_fixedR.mat} (Apéndice~\ref{ap:kalman_tuning}):

\[
Q_n = Q_{\text{best}},
\qquad
R_n = R
\]

Adicionalmente, para la simulación se utilizan las desviaciones estándar:
\(\sigma_v\) (medición) y \(\sigma_w\) (proceso), e inyección de ruido blanco gaussiano.

\paragraph{Filtro de Kalman (\textit{current estimator})}
Se construye un sistema extendido para el cálculo de Kalman, incorporando explícitamente
el canal de ruido de proceso \(w\):

\[
x_{k+1} = A x_k + B u_k + G w_k,
\qquad
y_k = C x_k + D u_k + H w_k + v_k
\]

En este trabajo se asumió:

\[
G = I_n,
\qquad
H = 0
\]

y se obtiene la ganancia \(L\) en régimen permanente (y sus polos):

\[
L = L_k,
\qquad
\lambda(A - LCA)
\]

\paragraph{Control LQI con integrador}
Se incorpora un integrador escalar \(\xi_k\) sobre el error:

\[
\xi_{k+1} = \xi_k + (r_k - y_k)
\]

y se adopta la ley de control:

\[
u_k = -K_x \hat{x}_k + K_i \xi_k
\]

Las ganancias se obtienen mediante \texttt{dlqr} aplicado al sistema aumentado, usando
pesos heurísticos basados en escalas prácticas:
paso deseado \(\Delta y \approx 20\) y esfuerzo máximo \(|u|\lesssim 300\).

\subsection{Script completo (LQG\_servo\_firmware\_friendly.m)}

\noindent A continuación se incluye el script completo utilizado para calcular
\(L\), \(K_x\), \(K_i\), reportar polos y simular en presencia de ruido.

\begin{lstlisting}[style=matlabstyle]
	%% =========================
	%  LQG SERVO “firmware friendly”
	%  (kalman + lqi + lqgtrack)
	%  + estimador CURRENT (default en discreto)
	%  + ruido de proceso y medición (sigma_w, sigma_v)
	%% =========================
	close all; clear; clc
	
	%% =========================
	% 1) CARGA + DISCRETIZACIÓN
	%% =========================
	S = load('planta (1).mat');
	
	if isfield(S,'plantaC')
	plantaC = S.plantaC;
	elseif isfield(S,'sysC')
	plantaC = S.sysC;
	else
	error('No encuentro "plantaC" ni "sysC" dentro de planta (1).mat');
	end
	
	Ts   = 1/100;                 % 100 Hz
	sysD = c2d(plantaC, Ts, 'zoh');
	[A,B,C,D] = ssdata(ss(sysD));
	
	n = size(A,1);
	if size(C,1) ~= 1
	C = C(1,:);
	D = D(1,:);
	end
	if size(B,2) ~= 1
	error('Este script asume SISO (1 entrada). size(B,2)=%d', size(B,2));
	end
	
	fprintf('Ts=%.9f | n=%d\n', Ts, n);
	
	%% =========================
	% 2) RUIDOS (sigma -> covarianzas)
	%% =========================
	S = load("RQ_tuning_fixedR.mat");
	
	Qn = S.Q_best;     % cov(w) : nxn
	Rn = S.R;          % cov(v) : 1x1
	sigma_v = S.sigma_v;
	sigma_w = S.best_q;
	Nn = zeros(n,1);   % cov(wv') asumimos 0
	
	%% =========================
	% 3) KALMAN: sys debe incluir el canal de ruido w
	%% =========================
	G = eye(n);        % w entra a todos los estados
	H = zeros(1,n);    % y NO depende directamente de w (H=0)
	
	sysK = ss(A, [B G], C, [D H], Ts);   % inputs: [u ; w]
	
	% Para tu MATLAB: TYPE va como ÚLTIMO argumento.
	% 'current' es el default, pero lo dejamos explícito.
	[kest, Lk, Pk, Mx, Z, My] = kalman(sysK, Qn, Rn, Nn, 'current');
	
	fprintf('\nLk (kalman) = '); disp(Lk);
	fprintf('poles(A-LC) = '); disp(eig(A - Lk*C).');
	
	%% =========================
	% 4) LQI: K = [Kx Ki] para u = -Kx*xhat - Ki*xi
	%% =========================
	m = 1;                      % integrador escalar
	Ahat = [A B; zeros(m,n+m)];
	Bhat = [zeros(n,m); eye(m)];
	Chat = [C zeros(1,m)];
	
	% --- interpretación física: "step ~20" y "u no pase ~300" ---
	y_step = 20;
	u_max  = 300;
	
	% knobs
	wy = 20;                    % subí => seguís más (más agresivo)
	wu = 50;                    % subí => penalizás más u (más tímido)
	wv = 0.00000000000001;      % subí => penalizás integrador
	
	% OJO: acá tu “escala física” es discutible. Esto es heurístico.
	qy = wy*(1/(y_step));
	ru = wu*(1/(u_max));
	qv = wv*(1/(y_step));
	
	Qa = blkdiag(qy*(C'*C) + 1e-8*eye(n), qv);
	R  = ru;
	
	[Khat,Pa,ecl] = dlqr(Ahat, Bhat, Qa, R);
	
	Aux  = [A-eye(size(A))  B;
	C*A             C*B];
	
	K2K1 = (Khat + [zeros(1,n) eye(m)]) / Aux;
	
	Kx = K2K1(1,1:n);
	Ki = K2K1(1,n+1);
	
	fprintf('\nKx (lqi) = '); disp(Kx);
	fprintf('Ki (lqi) = '); disp(Ki);
	
	%% =========================
	% 6) SIMULACIÓN (tu estilo) con ruido de proceso+medición
	%% =========================
	N  = 1200;
	t  = 0:Ts:(N-1)*Ts;
	
	r = ones(1,N)*25;
	r(1:30) = 0;
	
	umin = -u_max; umax = u_max;
	
	X  = zeros(n,N);   % estados reales
	Xh = zeros(n,N);   % estados estimados (los que usás en tu control)
	V  = zeros(1,N);   % integrador
	U  = zeros(1,N);   % control
	
	Y_true = zeros(1,N);
	Y_meas = zeros(1,N);
	
	for kidx = 1:N-1
	rk = r(kidx);
	
	% medición y[k]
	y_true_k = C*X(:,kidx);
	v_k      = sigma_v*randn;
	y_meas_k = y_true_k + v_k;
	
	Y_true(kidx) = y_true_k;
	Y_meas(kidx) = y_meas_k;
	
	% integrador xi[k+1] = xi[k] + (r-y)
	V(kidx+1) = V(kidx) + (rk - y_meas_k);
	
	% control: u = -Kx*xhat - Ki*xi
	u_unsat = -Kx*Xh(:,kidx) + Ki*V(kidx+1);
	u_k     = min(max(u_unsat, umin), umax);
	U(kidx) = u_k;
	
	% planta con ruido de proceso: x[k+1] = A x + B u + w
	w_k = sigma_w*randn(n,1);
	X(:,kidx+1) = A*X(:,kidx) + B*u_k + w_k;
	
	% estimador (usa Lk)
	Xh(:,kidx+1) = A*Xh(:,kidx) + B*u_k + ...
	Lk*(y_meas_k - (C*Xh(:,kidx) + D*u_k));
	end
	
	% último sample para plots
	Y_true(N) = C*X(:,N);
	Y_meas(N) = Y_true(N) + sigma_v*randn;
	U(N)      = U(N-1);
	
	%% =========================
	% 7) PLOTS
	%% =========================
	figure('Name','y(t) — LQG + integrador + ruido');
	plot(t, Y_true, 'LineWidth',1.2); hold on;
	plot(t, Y_meas, '.', 'LineWidth',1.0);
	plot(t, r, 'k--','LineWidth',1.1);
	grid on; grid minor;
	xlabel('t [s]'); ylabel('y');
	legend('y true','y meas','r','Location','best');
	
	figure('Name','u(t) — saturado');
	plot(t, U, 'LineWidth',1.4);
	grid on; grid minor;
	xlabel('t [s]'); ylabel('u');
	ylim([umin umax]);
	
	figure('Name','xi(t) — integrador');
	plot(t, V, 'LineWidth',1.4);
	grid on; grid minor;
	xlabel('t [s]'); ylabel('\xi');
	
	Aaug = [A,B;...
	Kx-Kx*A-Ki*C*A,1-Kx*B-Ki*C*B];
	Baug = [0;0;0;Ki];
	Caug = [C,0];
	D = 0;
	sysDaug = ss(Aaug,Baug,Caug,D,Ts);
	
	figure; pzmap(sysDaug); grid on; zgrid;
\end{lstlisting}



\section{Códigos PSoC}
% ============================================================
\section{Control App}
\label{ap:control_app}

\begin{lstlisting}[style=cfirmware, caption={Control App (control_app.h + control_app.c)}, label={lst:control_app}]
/* =========================
control_app.h
========================= */
#ifndef CONTROL_APP_H
#define CONTROL_APP_H

#include <stdint.h>
#include <stdbool.h>

/* =========================================================
1) ACTUADOR (PWM absoluto que sale por write_u callback)
========================================================= */
#ifndef CONTROL_SAT_MIN
#define CONTROL_SAT_MIN (1000.0f)   /* us */
#endif

#ifndef CONTROL_SAT_MAX
#define CONTROL_SAT_MAX (1750.0f)   /* us */
#endif

/* Offset default (fallback). El real se auto-calibra en runtime (g_u0_offset_us) */
#ifndef CONTROL_U_OFFSET
#define CONTROL_U_OFFSET (1350.0f)  /* us */
#endif


/* =========================================================
2) REFERENCIAS (según modo)
=========================================================
- OPEN LOOP:
si CONTROL_OL_REF_IS_DELTA=1 -> ref = deltau [us] relativo a u0
si CONTROL_OL_REF_IS_DELTA=0 -> ref = PWM absoluto [us]
- CLOSED LOOP:
ref = altura física [cm]
========================================================= */
#ifndef CONTROL_OL_REF_IS_DELTA
#define CONTROL_OL_REF_IS_DELTA (1u)
#endif

/* Usados SOLO si CONTROL_OL_REF_IS_DELTA = 0 (PWM absoluto) */
#ifndef CONTROL_OL_REF_MIN
#define CONTROL_OL_REF_MIN (1000.0f) /* us */
#endif
#ifndef CONTROL_OL_REF_MAX
#define CONTROL_OL_REF_MAX (1700.0f) /* us */
#endif

/* Closed-loop (altura física) */
#ifndef CONTROL_CL_REF_MIN
#define CONTROL_CL_REF_MIN (12.0f)   /* cm */
#endif
#ifndef CONTROL_CL_REF_MAX
#define CONTROL_CL_REF_MAX (300.0f)  /* cm */
#endif


/* =========================================================
3) STOP SUAVE (descenso controlado)
=========================================================
- TARGET es deltau relativo (cmd), NO PWM absoluto.
- CUTOFF es PWM absoluto final de corte.
========================================================= */
#ifndef CONTROL_DESC_TARGET_CMD_US
#define CONTROL_DESC_TARGET_CMD_US   (-50.0f) /* deltau [us] relativo a u0 */
#endif

#ifndef CONTROL_DESC_CUTOFF_US
#define CONTROL_DESC_CUTOFF_US       (1000.0f) /* PWM absoluto [us] */
#endif

#ifndef CONTROL_DESC_MIN_Y_CM
#define CONTROL_DESC_MIN_Y_CM        (12.0f)   /* cm */
#endif

#ifndef CONTROL_DESC_STEP_US
#define CONTROL_DESC_STEP_US         (5.0f)    /* deltau [us] */
#endif

#ifndef CONTROL_DESC_DELAY_US
#define CONTROL_DESC_DELAY_US        (20000u)  /* 20ms -> 50Hz */
#endif

#ifndef CONTROL_DESC_MAX_ITERS
#define CONTROL_DESC_MAX_ITERS       (600u)    /* 600*20ms = 12s */
#endif

#ifndef CONTROL_DESC_MIN_HITS
#define CONTROL_DESC_MIN_HITS        (2u)
#endif

#ifndef CONTROL_DESC_HOLD_MS
#define CONTROL_DESC_HOLD_MS         (500u)    /* ms */
#endif

#ifndef CONTROL_DESC_HOLD_POLL_US
#define CONTROL_DESC_HOLD_POLL_US    (20000u)  /* us */
#endif


/* =========================================================
4) AUTO-CALIBRACIÓN u0 (RÁPIDA)
=========================================================
Objetivo: detectar el primer PWM (u_phy) donde hay movimiento real.

Nota: estos parámetros suponen fs_calib = 1000 Hz.
Si fs cambia, escalá HOLD/MIN_HITS/MAX_SAMPLES proporcionalmente.
========================================================= */
#ifndef CONTROL_ENABLE_AUTOCAL
#define CONTROL_ENABLE_AUTOCAL (1u)
#endif

/* Barrido PWM absoluto durante la rampa */
#ifndef CONTROL_CALIB_START_US
#define CONTROL_CALIB_START_US (1250.0f) /* us */
#endif

#ifndef CONTROL_CALIB_STEP_US
#define CONTROL_CALIB_STEP_US  (5.0f)    /* us por nivel (rápido) */
#endif

#ifndef CONTROL_CALIB_MAX_US
#define CONTROL_CALIB_MAX_US   (1500.0f) /* us */
#endif

/* Nivel: cuántas muestras se mantiene cada PWM antes de subir */
#ifndef CONTROL_CALIB_HOLD_SAMPLES
#define CONTROL_CALIB_HOLD_SAMPLES (200u) /* 120 ms por nivel (rápido pero usable) */
#endif

/* Detección de movimiento: |y - y0| >= DY durante MIN_HITS consecutivos */
#ifndef CONTROL_CALIB_DY_CM
#define CONTROL_CALIB_DY_CM    (2.0f)    /* cm (rápido) */
#endif

#ifndef CONTROL_CALIB_MIN_HITS
#define CONTROL_CALIB_MIN_HITS (20u)     /* 20 ms consecutivos */
#endif

/* Tiempo máximo total de calibración (en muestras) */
#ifndef CONTROL_CALIB_MAX_SAMPLES
#define CONTROL_CALIB_MAX_SAMPLES (12000u) /* 12 s (rápido, pero no se corta enseguida) */
#endif

/* Post-move: mantener u0 detectado un rato y luego soltar a deltau=0 */
#ifndef CONTROL_CALIB_POSTMOVE_MS
#define CONTROL_CALIB_POSTMOVE_MS (500u) /* ms */
#endif


/* =========================================================
5) ESTABILIZACIÓN POST-CALIB (SETTLE)
=========================================================
Dy pequeño por N muestras => estable.
========================================================= */
#ifndef CONTROL_U0_SETTLE_SAMPLES
#define CONTROL_U0_SETTLE_SAMPLES (200u) /* 200 ms estable */
#endif

#ifndef CONTROL_U0_SETTLE_DY_CM
#define CONTROL_U0_SETTLE_DY_CM   (0.4f) /* cm */
#endif


/* =========================================================
6) TF (IIR) - orden máximo soportado por el paquete TF
========================================================= */
#ifndef CONTROL_TF_MAX_ORDER
#define CONTROL_TF_MAX_ORDER (10u) /* b0..b10 / a0..a10 */
#endif


/* =========================================================
7) Callbacks IO
========================================================= */
typedef void (*control_write_u_fn_t)(float u_phy);

/* Corre DENTRO del ISR de muestreo.
Debe ser ultraliviano y terminar llamando:
control_sample_isr_push(y);
*/
typedef void (*control_sample_isr_fn_t)(void);


/* =========================================================
8) Variables ISR -> main
========================================================= */
extern volatile float   control_last_y;
extern volatile uint8_t control_sample_pending;
extern volatile float   control_ref_v;


/* =========================================================
9) API
========================================================= */
void control_register_io(control_sample_isr_fn_t sample_isr,
control_write_u_fn_t    write_u);

void control_on_sample_isr(void);
void control_sample_isr_push(float y);

void  control_set_reference(float r);
float control_get_reference(void);

void  control_set_sample_time(float Ts);
float control_get_sample_time(void);

/* START: inicia control + dispara autocalibración (si enabled) */
void  control_start(float ref0);

/* Stop suave: rampa hacia target y corta */
bool  control_stop_suave_step(void);

/* PAQUETE FIJO 25 floats:
TF:
c[0..10]  = b0..b10
c[11..21] = a0..a10
c[22]     = order (0..10)
c[23]     = N
c[24]     = FsHz
SS:
c[0..22]  = matrices/ganancias
c[23]     = N
c[24]     = FsHz
*/
void  control_apply_tf(const float* c, uint16_t n);
void  control_apply_ss(const float* c, uint16_t n);

void  control_step(void);
void  control_force_min(void);

/* Debug: leer u0 runtime */
float control_get_u0_offset_us(void);

/* Estado de autocal u0 */
bool control_is_calibrating(void);
bool control_u0_is_valid(void);

#endif /* CONTROL_APP_H */

/* =========================
control_app.c
(PAQUETE FIJO 25 floats)
=========================
Meta SIEMPRE al final:
c[23] = N
c[24] = FsHz
TF además usa:
c[22] = order (0..10)
*/

#include "control_app.h"

#include "project.h"
#include "uartp_sw.h"
#include "arm_math.h"
#include <string.h>

/* =======================
Defaults de estabilización post-calib (si no están en .h)
======================= */
#ifndef CONTROL_U0_SETTLE_SAMPLES
#define CONTROL_U0_SETTLE_SAMPLES (50u)
#endif
#ifndef CONTROL_U0_SETTLE_DY_CM
#define CONTROL_U0_SETTLE_DY_CM   (0.5f)
#endif

/* =======================
Variables ISR -> main
======================= */
volatile float   control_last_y = 0.0f;
volatile uint8_t control_sample_pending = 0u;
volatile float   control_ref_v = 0.0f;

/* =======================
IO callbacks
======================= */
static control_sample_isr_fn_t s_sample_isr_cb = 0;
static control_write_u_fn_t    s_write_u_cb    = 0;

/* =======================
Estado general
======================= */
static float g_ref = 0.0f;
static float g_Ts  = 1.0f;

/* Interno: deltau (us) */
static float g_u_cmd = 0.0f;  /* comando interno (deltau) */
static float g_u_out = 0.0f;  /* físico (PWM us) post offset+sat */

/* integrador */
static float g_vint  = 0.0f;

/* Offset runtime (u0). Se calcula en calibración o cae al default */
static float g_u0_offset_us = CONTROL_U_OFFSET;

float control_get_u0_offset_us(void)
{
	return g_u0_offset_us;
}

/* =======================
Sesión de control
======================= */
static uint8_t g_session_active = 0u; /* 0=parado, 1=en sesión */
static uint8_t g_u0_valid       = 0u; /* 0=pendiente calibrar, 1=u0 fijo en esta sesión */
static uint8_t g_first_start = 1u;

/* =======================
TF (IIR DF2T) hasta orden 10
======================= */
static float tf_b[CONTROL_TF_MAX_ORDER + 1u] = {0}; /* b0..b10 */
static float tf_a[CONTROL_TF_MAX_ORDER + 1u] = {0}; /* a0..a10 */
static float tf_w[CONTROL_TF_MAX_ORDER]      = {0}; /* w0..w9 (usa solo [0..order-1]) */
static uint8  tf_order = 0u;                         /* orden efectivo 0..10 */

/* =======================
SS (3 estados)
======================= */
static float ss_A[9] = {0};   /* fila-major 3x3 */
static float ss_B[3] = {0};   /* 3x1 */
static float ss_C[3] = {0};   /* 1x3 */
static float ss_D    = 0.0f;
static float ss_L[3] = {0};   /* 3x1 */
static float ss_K[3] = {0};   /* 1x3 */
static float ss_Kx   = 0.0f;  /* Ki (modo I) o Kr (modo NOI) */

static float xhat[3] = {0.0f, 0.0f, 0.0f};  /* posterior */
static float zhat[3] = {0.0f, 0.0f, 0.0f};  /* prior (para ACT) */

/* =======================
Auto-calibración u0 (estado)
======================= */
#if CONTROL_ENABLE_AUTOCAL
typedef enum {
	CALIB_IDLE = 0,
	CALIB_RAMP,
	CALIB_POSTMOVE,
	CALIB_SETTLE,
	CALIB_DONE,
	CALIB_FAIL
} calib_state_t;

static calib_state_t g_calib_state   = CALIB_IDLE;

/* Calibración por niveles */
static float    g_calib_u_phy         = CONTROL_CALIB_START_US;
static float    g_calib_y_level0      = 0.0f;
static uint16_t g_calib_hits          = 0u;
static uint16_t g_calib_level_samples = 0u;
static uint32_t g_calib_total_samples = 0u;
static uint16_t g_postmove_cnt   = 0u;
static uint16_t g_postmove_need  = 0u;
static float    g_postmove_u_phy = 0.0f;


/* Estabilización post-calib */
static float    g_settle_y_prev       = 0.0f;
static uint16_t g_settle_stable_cnt   = 0u;
static uint8_t  g_settle_first        = 1u;
#endif


static volatile uint8_t g_stop_busy      = 0u;
static volatile uint8_t g_start_queued   = 0u;
static volatile float   g_start_ref_q    = 0.0f;

/* =======================
Helpers
======================= */

static inline uint16_t samples_from_ms(uint16_t ms)
{
	float Ts = g_Ts;
	if (!(Ts > 0.0f)) Ts = 1.0f;
	
	float fs = 1.0f / Ts;
	float n  = fs * ((float)ms / 1000.0f);
	
	if (n < 1.0f) n = 1.0f;
	if (n > 65535.0f) n = 65535.0f;
	
	return (uint16_t)(n + 0.5f);
}


static inline uint16_t u16_from_float(float x)
{
	if (x <= 0.0f) return 0u;
	if (x >= 65535.0f) return 65535u;
	return (uint16_t)(x + 0.5f);
}

static inline uint8 tf_order_from_float(float x)
{
	int32_t o = (int32_t)(x + 0.5f);
	if (o < 0) o = 0;
	if (o > (int32_t)CONTROL_TF_MAX_ORDER) o = (int32_t)CONTROL_TF_MAX_ORDER;
	return (uint8)o;
}

static inline float dot3_cmsis(const float a[3], const float b[3])
{
	float out = 0.0f;
	arm_dot_prod_f32((float32_t*)a, (float32_t*)b, 3u, (float32_t*)&out);
	return out;
}

static inline float ss_yhat_from_xu(const float x[3], float u_cmd)
{
	return dot3_cmsis(ss_C, x) + ss_D * u_cmd;
}

static inline void ss_predict_from_xu(const float x[3], float u_cmd, float z[3])
{
	arm_dot_prod_f32((float32_t*)&ss_A[0], (float32_t*)x, 3u, (float32_t*)&z[0]);
	z[0] += ss_B[0] * u_cmd;
	
	arm_dot_prod_f32((float32_t*)&ss_A[3], (float32_t*)x, 3u, (float32_t*)&z[1]);
	z[1] += ss_B[1] * u_cmd;
	
	arm_dot_prod_f32((float32_t*)&ss_A[6], (float32_t*)x, 3u, (float32_t*)&z[2]);
	z[2] += ss_B[2] * u_cmd;
}

static inline float satf(float u, float umin, float umax)
{
	if (u > umax) return umax;
	if (u < umin) return umin;
	return u;
}

static inline uint8 ss_mode_has_integrator(uint8 impl)
{
	return (impl == UARTP_IMPL_SS_PRED_I) || (impl == UARTP_IMPL_SS_ACT_I);
}

static inline float ss_Kr_eff(void)
{
	return ss_mode_has_integrator((uint8)UARTP_Impl) ? 0.0f : ss_Kx;
}

static inline float ss_Ki_eff(void)
{
	return ss_mode_has_integrator((uint8)UARTP_Impl) ? ss_Kx : 0.0f;
}

/* Conversión: PWM físico (us) -> deltau (us) */
static inline float ucmd_from_uphy(float u_phy)
{
	return (u_phy - g_u0_offset_us);
}

/* Conversión: deltau (us) -> PWM físico (us) */
static inline float uphy_from_ucmd(float u_cmd)
{
	return (u_cmd + g_u0_offset_us);
}

/* Rango deltau permitido (dinámico) */
static inline float cmd_min_dyn(void) { return (CONTROL_SAT_MIN - g_u0_offset_us); }
static inline float cmd_max_dyn(void) { return (CONTROL_SAT_MAX - g_u0_offset_us); }

static inline float y_rel_from_y(float y_phys) { return (y_phys - (float)CONTROL_DESC_MIN_Y_CM); }
static inline float r_rel_from_r(float r_phys) { return (r_phys - (float)CONTROL_DESC_MIN_Y_CM); }

/* --- clamp de referencia según modo --- */
static inline float clamp_ref(float r)
{
	if (UARTP_Impl == UARTP_IMPL_OPENLOOP)
	{
		#if CONTROL_OL_REF_IS_DELTA
		return satf(r, cmd_min_dyn(), cmd_max_dyn());
		#else
		if (r < CONTROL_OL_REF_MIN) return CONTROL_OL_REF_MIN;
		if (r > CONTROL_OL_REF_MAX) return CONTROL_OL_REF_MAX;
		return r;
		#endif
	}
	else
	{
		if (r < CONTROL_CL_REF_MIN) return CONTROL_CL_REF_MIN;
		if (r > CONTROL_CL_REF_MAX) return CONTROL_CL_REF_MAX;
		return r;
	}
}

/* =======================
write_u (único actuador)
Entrada: deltau (us)
======================= */
static inline void write_u(float u_cmd)
{
	u_cmd = satf(u_cmd, cmd_min_dyn(), cmd_max_dyn());
	g_u_cmd = u_cmd;
	
	float u_phy = uphy_from_ucmd(u_cmd);
	u_phy = satf(u_phy, CONTROL_SAT_MIN, CONTROL_SAT_MAX);
	g_u_out = u_phy;
	
	if (s_write_u_cb) s_write_u_cb(u_phy);
}

void control_force_min(void)
{
	write_u(ucmd_from_uphy(CONTROL_SAT_MIN));
}

/* =======================
Observadores
======================= */
static inline void ss_observer_pred_step(float y, float u_prev_cmd, float u_k_cmd)
{
	const float innov = y - ss_yhat_from_xu(xhat, u_prev_cmd);
	
	float z[3];
	ss_predict_from_xu(xhat, u_k_cmd, z);
	
	float Linv[3];
	arm_scale_f32((float32_t*)ss_L, innov, (float32_t*)Linv, 3u);
	
	arm_add_f32((float32_t*)z, (float32_t*)Linv, (float32_t*)xhat, 3u);
}

static inline void ss_observer_act_correct(float y, float u_prev_cmd)
{
	const float innov = y - ss_yhat_from_xu(zhat, u_prev_cmd);
	
	float Linv[3];
	arm_scale_f32((float32_t*)ss_L, innov, (float32_t*)Linv, 3u);
	
	arm_add_f32((float32_t*)zhat, (float32_t*)Linv, (float32_t*)xhat, 3u);
}

static inline void ss_observer_act_predict(float u_k_cmd)
{
	ss_predict_from_xu(xhat, u_k_cmd, zhat);
}

/* =======================
TF DF2T step (usa SOLO el orden efectivo)
======================= */
static inline float tf_step(float x)
{
	const uint8 ord = tf_order;
	
	if (ord == 0u) {
		return tf_b[0] * x;
	}
	
	float y = tf_b[0] * x + tf_w[0];
	
	for (uint8 i = 0u; i < (uint8)(ord - 1u); i++) {
		tf_w[i] = tf_w[i + 1u] + tf_b[i + 1u] * x - tf_a[i + 1u] * y;
	}
	
	tf_w[ord - 1u] = tf_b[ord] * x - tf_a[ord] * y;
	
	return y;
}

/* =======================
IO / ISR glue
======================= */
void control_register_io(control_sample_isr_fn_t sample_isr,
control_write_u_fn_t    write_u_fn)
{
	s_sample_isr_cb = sample_isr;
	s_write_u_cb    = write_u_fn;
}

void control_on_sample_isr(void)
{
	if (s_sample_isr_cb) {
		s_sample_isr_cb();
	} else {
		control_sample_pending = 1u;
	}
}

void control_sample_isr_push(float y)
{
	control_last_y = y;
	control_sample_pending = 1u;
}

/* =======================
Referencia y Ts
======================= */
void control_set_reference(float r)
{
	uint8 intr = CyEnterCriticalSection();
	g_ref = r;
	control_ref_v = r;
	CyExitCriticalSection(intr);
}

float control_get_reference(void)
{
	return g_ref;
}

void control_set_sample_time(float Ts)
{
	if (Ts <= 0.0f) Ts = 1.0f;
	g_Ts = Ts;
}

float control_get_sample_time(void)
{
	return g_Ts;
}

/* =======================
START: 1x por sesión
======================= */
void control_start(float ref0)
{
	
	control_set_reference(ref0);
	
	/* ===== FIX RÁPIDO: limpiar estado en el primer START ===== */
	if (g_first_start)
	{
		g_first_start   = 0u;
		
		/* reset duro de estados que pueden quedar "a medio camino" */
		g_session_active = 0u;
		g_stop_busy      = 0u;
		g_start_queued   = 0u;
		
		#if CONTROL_ENABLE_AUTOCAL
		g_calib_state = CALIB_IDLE;
		#endif
		g_u0_valid = 0u;                 /* fuerza calibración */
		g_u0_offset_us = CONTROL_U_OFFSET;
		
		g_u_cmd = 0.0f;
		g_u_out = CONTROL_SAT_MIN;       /* o CONTROL_U_OFFSET si preferís */
		g_vint  = 0.0f;
		
		xhat[0]=0; xhat[1]=0; xhat[2]=0;
		zhat[0]=0; zhat[1]=0; zhat[2]=0;
		memset(tf_w, 0, sizeof(tf_w));
	}
	/* ======================================================== */
	
	#if CONTROL_ENABLE_AUTOCAL
	if (g_stop_busy)
	{
		g_start_queued = 1u;
		g_start_ref_q  = ref0;
		return;
	}
	#endif
	
	if (g_session_active) return;
	g_session_active = 1u;
	
	
	xhat[0] = 0.0f; xhat[1] = 0.0f; xhat[2] = 0.0f;
	zhat[0] = 0.0f; zhat[1] = 0.0f; zhat[2] = 0.0f;
	
	memset(tf_w, 0, sizeof(tf_w));
	g_vint = 0.0f;
	
	#if CONTROL_ENABLE_AUTOCAL
	if (!g_u0_valid)
	{
		g_calib_state          = CALIB_RAMP;
		g_calib_u_phy          = CONTROL_CALIB_START_US;
		g_calib_hits           = 0u;
		g_calib_level_samples  = 0u;
		g_calib_total_samples  = 0u;
		g_calib_y_level0       = 0.0f;
		
		g_u0_offset_us = CONTROL_U_OFFSET;
		
		write_u(ucmd_from_uphy(g_calib_u_phy));
		return;
	}
	#endif
	
	/* bumpless desde u actual */
	{
		float u0_cmd = ucmd_from_uphy(g_u_out);
		
		if (UARTP_Impl != UARTP_IMPL_TF && UARTP_Impl != UARTP_IMPL_OPENLOOP) {
			float Ki = ss_Ki_eff();
			if (Ki != 0.0f) g_vint = -u0_cmd / Ki;
		}
		
		write_u(u0_cmd);
		
		if (UARTP_Impl != UARTP_IMPL_TF && UARTP_Impl != UARTP_IMPL_OPENLOOP) {
			ss_predict_from_xu(xhat, u0_cmd, zhat);
		}
	}
}

/* =======================
STOP suave: termina sesión
(FIX: target es deltau directo, no PWM absoluto)
======================= */
bool control_stop_suave_step(void)
{
	g_stop_busy = 1u;
	
	/* TARGET es deltau relativo (cmd) */
	float u_target_cmd = (float)CONTROL_DESC_TARGET_CMD_US;
	u_target_cmd = satf(u_target_cmd, cmd_min_dyn(), cmd_max_dyn());
	
	for (uint32_t i = 0u; i < (uint32_t)CONTROL_DESC_MAX_ITERS; i++)
	{
		float u = g_u_cmd;
		
		if (u < u_target_cmd) {
			break;
		} else if (u > u_target_cmd) {
			u -= CONTROL_DESC_STEP_US;
			if (u < u_target_cmd) u = u_target_cmd;
			write_u(u);
		} else {
			break;
		}
		
		CyDelayUs(CONTROL_DESC_DELAY_US);
	}
	
	write_u(u_target_cmd);
	
	uint32_t hold_loops = (CONTROL_DESC_HOLD_MS * 1000u) / CONTROL_DESC_HOLD_POLL_US;
	if (hold_loops < 1u) hold_loops = 1u;
	
	for (uint32_t k = 0u; k < hold_loops; k++) {
		write_u(u_target_cmd);
		CyDelayUs(CONTROL_DESC_HOLD_POLL_US);
	}
	
	/* CUTOFF es PWM absoluto (us) -> convertir a deltau */
	write_u(ucmd_from_uphy((float)CONTROL_DESC_CUTOFF_US));
	
	g_session_active = 0u;
	g_u0_valid       = 0u;
	#if CONTROL_ENABLE_AUTOCAL
	g_calib_state    = CALIB_IDLE;
	#endif
	
	g_stop_busy = 0u;
	
	#if CONTROL_ENABLE_AUTOCAL
	/* Si llegó un START durante el STOP, ejecutalo ahora */
	if (g_start_queued)
	{
		float r0 = g_start_ref_q;
		g_start_queued = 0u;
		control_start(r0);
	}
	#endif
	
	return true;
	
}

/* =======================
Carga coeficientes (PAQUETE FIJO 25 floats)
======================= */

void control_apply_tf(const float* c, uint16_t n)
{
	if (!c || n < 25u) return;
	
	for (uint8 i = 0u; i < (uint8)(CONTROL_TF_MAX_ORDER + 1u); i++) {
		tf_b[i] = c[i];
		tf_a[i] = c[i + 11u];
	}
	
	tf_order = tf_order_from_float(c[22]);
	
	for (uint8 i = (uint8)(tf_order + 1u); i < (uint8)(CONTROL_TF_MAX_ORDER + 1u); i++) {
		tf_b[i] = 0.0f;
		tf_a[i] = 0.0f;
	}
	
	if (tf_a[0] == 0.0f) {
		tf_a[0] = 1.0f;
	}
	if (tf_a[0] != 1.0f) {
		const float a0 = tf_a[0];
		for (uint8 i = 0u; i <= tf_order; i++) {
			tf_b[i] /= a0;
		}
		for (uint8 i = 1u; i <= tf_order; i++) {
			tf_a[i] /= a0;
		}
		tf_a[0] = 1.0f;
	}
	
	UARTP_StreamN    = u16_from_float(c[23]);
	UARTP_StreamFsHz = c[24];
	
	memset(tf_w, 0, sizeof(tf_w));
}

void control_apply_ss(const float* c, uint16_t n)
{
	if (!c || n < 25u) return;
	
	ss_A[0] = c[0]; ss_A[1] = c[1]; ss_A[2] = c[2];
	ss_A[3] = c[3]; ss_A[4] = c[4]; ss_A[5] = c[5];
	ss_A[6] = c[6]; ss_A[7] = c[7]; ss_A[8] = c[8];
	
	ss_B[0] = c[9];  ss_B[1] = c[10]; ss_B[2] = c[11];
	
	ss_C[0] = c[12]; ss_C[1] = c[13]; ss_C[2] = c[14];
	
	ss_D    = c[15];
	
	ss_L[0] = c[16]; ss_L[1] = c[17]; ss_L[2] = c[18];
	
	ss_K[0] = c[19]; ss_K[1] = c[20]; ss_K[2] = c[21];
	
	ss_Kx   = c[22];
	
	UARTP_StreamN    = u16_from_float(c[23]);
	UARTP_StreamFsHz = c[24];
	
	xhat[0] = 0.0f; xhat[1] = 0.0f; xhat[2] = 0.0f;
	zhat[0] = 0.0f; zhat[1] = 0.0f; zhat[2] = 0.0f;
	g_vint  = 0.0f;
}



bool control_is_calibrating(void)
{
	#if CONTROL_ENABLE_AUTOCAL
	return (g_calib_state == CALIB_RAMP) ||
	(g_calib_state == CALIB_POSTMOVE) ||
	(g_calib_state == CALIB_SETTLE);
	
	#else
	return false;
	#endif
}

bool control_u0_is_valid(void)
{
	return (g_u0_valid != 0u);
}













/* =======================
CONTROL STEP (main)
+ Anti-windup (freeze) en:
- TF (congelando estados DF2T si satura y el error empuja hacia afuera)
- SS_PRED_I / SS_ACT_I (congelando g_vint si satura y el error empuja hacia afuera)
======================= */
void control_step(void)
{
	uint8 intr = CyEnterCriticalSection();
	controlPin_Write(1);
	
	float y_phys = control_last_y;
	control_sample_pending = 0u;
	
	float r_phys = g_ref;
	
	/* u_prev_cmd: deltau usado para yhat/innov (consistente con tu obs) */
	float u_prev_cmd = ucmd_from_uphy(g_u_out);
	
	CyExitCriticalSection(intr);
	
	r_phys = clamp_ref(r_phys);
	
	#if CONTROL_ENABLE_AUTOCAL
	/* ... TODO lo tuyo de autocal igual (sin cambios) ... */
	if (g_calib_state == CALIB_RAMP)
	{
		g_calib_total_samples++;
		
		if (g_calib_level_samples == 0u) {
			g_calib_y_level0 = y_phys;
			g_calib_hits = 0u;
		}
		
		g_calib_level_samples++;
		
		float dy = y_phys - g_calib_y_level0;
		if (dy < 0) dy = -dy;
		
		if (dy >= CONTROL_CALIB_DY_CM) {
			if (g_calib_hits < 65535u) g_calib_hits++;
		} else {
			g_calib_hits = 0u;
		}
		
		if (g_calib_hits >= (uint16_t)CONTROL_CALIB_MIN_HITS)
		{
			g_u0_offset_us = g_calib_u_phy;
			g_u0_valid     = 1u;
			
			g_postmove_u_phy = g_calib_u_phy;
			g_postmove_cnt   = 0u;
			g_postmove_need  = samples_from_ms((uint16_t)CONTROL_CALIB_POSTMOVE_MS);
			g_calib_state    = CALIB_POSTMOVE;
			
			write_u(ucmd_from_uphy(g_postmove_u_phy));
			
			UARTP_Telemetry_Push(g_u_cmd, y_phys);
			controlPin_Write(0);
			return;
		}
		
		if (g_calib_total_samples >= CONTROL_CALIB_MAX_SAMPLES || g_calib_u_phy >= CONTROL_CALIB_MAX_US)
		{
			g_calib_state   = CALIB_FAIL;
			g_u0_offset_us  = CONTROL_U_OFFSET;
			g_u0_valid      = 0u;
			
			write_u(0.0f);
			UARTP_Telemetry_Push(g_u_cmd, y_phys);
			controlPin_Write(0);
			return;
		}
		
		if (g_calib_level_samples < (uint16_t)CONTROL_CALIB_HOLD_SAMPLES)
		{
			write_u(ucmd_from_uphy(g_calib_u_phy));
			UARTP_Telemetry_Push(g_u_cmd, y_phys);
			controlPin_Write(0);
			return;
		}
		
		g_calib_u_phy += CONTROL_CALIB_STEP_US;
		if (g_calib_u_phy > CONTROL_CALIB_MAX_US) g_calib_u_phy = CONTROL_CALIB_MAX_US;
		
		g_calib_level_samples = 0u;
		
		write_u(ucmd_from_uphy(g_calib_u_phy));
		UARTP_Telemetry_Push(g_u_cmd, y_phys);
		controlPin_Write(0);
		return;
	}
	
	if (g_calib_state == CALIB_POSTMOVE)
	{
		write_u(ucmd_from_uphy(g_postmove_u_phy));
		
		if (g_postmove_cnt < 65535u) g_postmove_cnt++;
		
		if (g_postmove_cnt >= g_postmove_need)
		{
			write_u(0.0f);
			g_calib_state       = CALIB_SETTLE;
			g_settle_first      = 1u;
			g_settle_stable_cnt = 0u;
			
			UARTP_Telemetry_Push(g_u_cmd, y_phys);
			controlPin_Write(0);
			return;
		}
		
		UARTP_Telemetry_Push(g_u_cmd, y_phys);
		controlPin_Write(0);
		return;
	}
	
	if (g_calib_state == CALIB_SETTLE)
	{
		write_u(0.0f);
		
		if (g_settle_first) {
			g_settle_y_prev = y_phys;
			g_settle_first = 0u;
			g_settle_stable_cnt = 0u;
			UARTP_Telemetry_Push(g_u_cmd, y_phys);
			controlPin_Write(0);
			return;
		}
		
		float dy = y_phys - g_settle_y_prev;
		if (dy < 0) dy = -dy;
		g_settle_y_prev = y_phys;
		
		if (dy <= (float)CONTROL_U0_SETTLE_DY_CM) {
			if (g_settle_stable_cnt < 65535u) g_settle_stable_cnt++;
		} else {
			g_settle_stable_cnt = 0u;
		}
		
		if (g_settle_stable_cnt >= (uint16_t)CONTROL_U0_SETTLE_SAMPLES)
		{
			g_calib_state = CALIB_DONE;
			
			if (UARTP_Impl != UARTP_IMPL_TF && UARTP_Impl != UARTP_IMPL_OPENLOOP) {
				ss_predict_from_xu(xhat, 0.0f, zhat);
			}
			
			UARTP_Telemetry_Push(g_u_cmd, y_phys);
			controlPin_Write(0);
			return;
		}
		
		UARTP_Telemetry_Push(g_u_cmd, y_phys);
		controlPin_Write(0);
		return;
	}
	#endif
	
	/* referencias en modo closed-loop: en cm (físico) -> relativo */
	float y_rel = y_rel_from_y(y_phys);
	float r_rel = r_rel_from_r(r_phys);
	
	/* límites de deltau (cmd) (dinámicos por u0) */
	const float umin_cmd = cmd_min_dyn();
	const float umax_cmd = cmd_max_dyn();
	
	switch (UARTP_Impl)
	{
		case UARTP_IMPL_OPENLOOP:
		{
			#if CONTROL_OL_REF_IS_DELTA
			float u_cmd = satf(r_phys, umin_cmd, umax_cmd);
			write_u(u_cmd);
			#else
			float u_cmd = ucmd_from_uphy(r_phys);
			write_u(u_cmd);
			#endif
		} break;
		
		case UARTP_IMPL_TF:
		{
			/* Error en relativo (cm) */
			const float e = (r_rel - y_rel);
			
			/* TF devuelve comando relativo deltau (us) */
			float u_unsat = tf_step(e);
			
			/* Saturación en deltau (dinámica por u0) */
			float u_cmd = satf(u_unsat, umin_cmd, umax_cmd);
			
			/* write_u espera deltau */
			write_u(u_cmd);
		} break;
		
		
		case UARTP_IMPL_SS_PRED_NOI:
		case UARTP_IMPL_SS_PRED_I:
		{
			uint8 has_i = ss_mode_has_integrator((uint8)UARTP_Impl);
			
			if (has_i)
			{
				/* Ogata: v(k+1)=v(k)+e ; u = K1*v(k+1) - K2*xhat */
				const float K1 = ss_Kx;                 /* K1 (integrador) */
				const float e  = (r_rel - y_rel);
				
				const float v0 = g_vint;
				const float v1 = v0 + e;                /* si querés Ts: v0 + g_Ts*e */
				
				const float u_unsat = (K1 * v1) - dot3_cmsis(ss_K, xhat);  /* ss_K = K2 */
				const float u_cmd   = satf(u_unsat, umin_cmd, umax_cmd);
				
				/* Anti-windup: CONDITIONAL INTEGRATION
				- Integro si NO satura
				- Si satura: integro solo si el error empuja hacia adentro
				(saturado arriba => e<0 baja u; saturado abajo => e>0 sube u) */
				if (u_unsat == u_cmd)
				{
					g_vint = v1;
				}
				else
				{
					if ( (u_unsat > umax_cmd && e < 0.0f) ||
					(u_unsat < umin_cmd && e > 0.0f) )
					{
						g_vint = v1;   /* deja integrar: desatura */
					}
					else
					{
						g_vint = v0;   /* freeze */
					}
				}
				
				write_u(u_cmd);
			}
			else
			{
				/* NOI: NO TOCAR */
				float u_cmd = ss_Kr_eff()*r_rel - dot3_cmsis(ss_K, xhat);
				write_u(u_cmd);
			}
			
			/* predictor step (NO TOCAR) */
			{
				float u_k_cmd = ucmd_from_uphy(g_u_out);
				ss_observer_pred_step(y_rel, u_prev_cmd, u_k_cmd);
			}
		} break;
		
		case UARTP_IMPL_SS_ACT_NOI:
		case UARTP_IMPL_SS_ACT_I:
		default:
		{
			uint8 has_i = ss_mode_has_integrator((uint8)UARTP_Impl);
			
			/* corrección ACT (NO TOCAR) */
			ss_observer_act_correct(y_rel, u_prev_cmd);
			
			if (has_i)
			{
				/* Ogata: v(k+1)=v(k)+e ; u = K1*v(k+1) - K2*xhat */
				const float K1 = ss_Kx;
				const float e  = (r_rel - y_rel);
				
				const float v0 = g_vint;
				const float v1 = v0 + e;
				
				const float u_unsat = (K1 * v1) - dot3_cmsis(ss_K, xhat);
				const float u_cmd   = satf(u_unsat, umin_cmd, umax_cmd);
				
				/* Anti-windup: CONDITIONAL INTEGRATION (igual que arriba) */
				if (u_unsat == u_cmd)
				{
					g_vint = v1;
				}
				else
				{
					if ( (u_unsat > umax_cmd && e < 0.0f) ||
					(u_unsat < umin_cmd && e > 0.0f) )
					{
						g_vint = v1;
					}
					else
					{
						g_vint = v0;
					}
				}
				
				write_u(u_cmd);
			}
			else
			{
				/* NOI: NO TOCAR */
				float u_cmd = ss_Kr_eff()*r_rel - dot3_cmsis(ss_K, xhat);
				write_u(u_cmd);
			}
			
			/* predicción ACT (NO TOCAR) */
			{
				float u_k_cmd = ucmd_from_uphy(g_u_out);
				ss_observer_act_predict(u_k_cmd);
			}
		} break;
		
		
	}
	
	UARTP_Telemetry_Push(g_u_cmd, y_phys);
	controlPin_Write(0);
}
	
\end{lstlisting}

% ============================================================
\section{Comunicación y máquina de estados}
\label{ap:uartp_sw}
% ============================================================

\begin{lstlisting}[style=cfirmware, caption={UARTP: comunicación y máquina de estados (uartp_sw.h + uartp_sw.c)}, label={lst:uartp_sw}]
/* uartp_sw.h */
#ifndef UARTP_SW_H
#define UARTP_SW_H

#include "project.h"
#include "cytypes.h"
#include <stdint.h>
#include <string.h>
#include <stdbool.h>

/* ===== UART mapping (ajustá si tu componente no se llama UART) ===== */
#ifndef UARTP_UART_Start
#define UARTP_UART_Start()            UART_Start()
#endif
#ifndef UARTP_UART_ClearRxBuffer
#define UARTP_UART_ClearRxBuffer()    UART_ClearRxBuffer()
#endif
#ifndef UARTP_UART_ClearTxBuffer
#define UARTP_UART_ClearTxBuffer()    UART_ClearTxBuffer()
#endif
#ifndef UARTP_UART_GetRxBufferSize
#define UARTP_UART_GetRxBufferSize()  UART_GetRxBufferSize()
#endif
#ifndef UARTP_UART_ReadRxData
#define UARTP_UART_ReadRxData()       UART_ReadRxData()
#endif
#ifndef UARTP_UART_ReadRxStatus
#define UARTP_UART_ReadRxStatus()     UART_ReadRxStatus()
#endif
#ifndef UARTP_UART_PutChar
#define UARTP_UART_PutChar(c)         UART_PutChar((c))
#endif
#ifndef UARTP_UART_PutArray
#define UARTP_UART_PutArray(p,n)      UART_PutArray((p),(n))
#endif

#if defined(UART_TX_BUFFER_SIZE) && (UART_TX_BUFFER_SIZE > 0)
#ifndef UARTP_UART_GetTxBufferSize
#define UARTP_UART_GetTxBufferSize()  UART_GetTxBufferSize()
#endif
#endif

/* ===== ISR mapping (tu componente ISR se llama isr_rx?) ===== */
#ifndef UARTP_ISR_RX_StartEx
#define UARTP_ISR_RX_StartEx(f)       isr_rx_StartEx((f))
#endif
#ifndef UARTP_ISR_RX_Enable
#define UARTP_ISR_RX_Enable()         isr_rx_Enable()
#endif
#ifndef UARTP_ISR_RX_Disable
#define UARTP_ISR_RX_Disable()        isr_rx_Disable()
#endif

/* ===== sampling callbacks (abstracto) ===== */
typedef void (*uartp_sampling_enable_fn_t)(void);
typedef void (*uartp_sampling_disable_fn_t)(void);

/* La librería pasa Fs (Hz) y el MAIN decide cómo convertirlo a Period/ticks */
typedef void (*uartp_sampling_change_fs_fn_t)(float fs_hz);

typedef void (*uartp_sampling_clear_flags_fn_t)(void);

/* ===== Timer UART OneShot para timeouts (sin CyDelay) ===== */
#ifndef UARTP_TIMER_UART_Start
#define UARTP_TIMER_UART_Start()              timer_uart_Start()
#endif
#ifndef UARTP_TIMER_UART_Stop
#define UARTP_TIMER_UART_Stop()               timer_uart_Stop()
#endif
#ifndef UARTP_TIMER_UART_WritePeriod
#define UARTP_TIMER_UART_WritePeriod(p)       timer_uart_WritePeriod((p))
#endif
#ifndef UARTP_TIMER_UART_ReadStatusRegister
#define UARTP_TIMER_UART_ReadStatusRegister() timer_uart_ReadStatusRegister()
#endif
#ifndef UARTP_TIMER_UART_ReadCounter
#define UARTP_TIMER_UART_ReadCounter()        timer_uart_ReadCounter()
#endif
#ifndef UARTP_TIMER_UART_STATUS_TC
#define UARTP_TIMER_UART_STATUS_TC            (0x01u)
#endif

/* ===== comandos ===== */
#define UARTP_CMD_RESET     ((uint8)'r')
#define UARTP_CMD_STOP      ((uint8)'s')
#define UARTP_CMD_INIT      ((uint8)'i')
#define UARTP_CMD_COEFFS    ((uint8)'c')
#define UARTP_CMD_TXCOEF    ((uint8)'t')
#define UARTP_CMD_SETMODE   ((uint8)'m')

/* ===== respuestas ===== */
#define UARTP_RSP_READY_RX  ((uint8)'R')
#define UARTP_RSP_READY_TX  ((uint8)'S')
#define UARTP_RSP_OK        ((uint8)'K')
#define UARTP_RSP_ERR       ((uint8)'!')

/* ===== control por word (eco-confirmado) ===== */
#define UARTP_CTL_ACK       ((uint8)'A')
#define UARTP_CTL_NAK       ((uint8)'N')
#define UARTP_WORD_BYTES    (4u)

/* ===== tamaños fijos =====
Normalización:
- Siempre se envían 25 floats (100 bytes).
- Meta SIEMPRE al final:
c[23] = N     (float -> u16)
c[24] = FsHz  (float32)
- Para TF (nuevo layout):
c[0..10]  = b0..b10
c[11..21] = a0..a10
c[22]     = order (0..10)
*/
#define UARTP_COEF_COUNT    (25u)
#define UARTP_COEF_BYTES    (UARTP_COEF_COUNT*4u)
#define UARTP_INIT_BYTES    (4u)
#define UARTP_MODE_BYTES    (4u)

/* TF max order soportado por el protocolo (debe coincidir con control_app) */
#define UARTP_TF_MAX_ORDER  (10u)

/* Indices (0-based) dentro del vector de 25 floats */
#define UARTP_IDX_TF_ORDER      (22u)
#define UARTP_IDX_META_N        (23u)
#define UARTP_IDX_META_FSHZ     (24u)

/* ===== robustez ===== */
#define UARTP_STEP_TIMEOUT_MS   (300u)
#define UARTP_ABORT_TIMEOUTS    (6u)
#define UARTP_MAX_WORD_RETRIES  (50u)

/* ===== modos globales ===== */
typedef enum { UARTP_SYS_COMMAND=0, UARTP_SYS_CONTROL=1 } uartp_sysmode_t;
typedef enum { UARTP_CTRL_STOPPED=0, UARTP_CTRL_STOPPING=1, UARTP_CTRL_RUNNING=2 } uartp_ctrl_state_t;

typedef enum {
	UARTP_IMPL_TF            = 0,
	UARTP_IMPL_SS_PRED_NOI   = 1,
	UARTP_IMPL_SS_ACT_NOI    = 2,
	UARTP_IMPL_SS_PRED_I     = 3,
	UARTP_IMPL_SS_ACT_I      = 4,
	UARTP_IMPL_OPENLOOP      = 5
} uartp_impl_t;

/* ===== callbacks ===== */
typedef bool (*uartp_stop_step_fn_t)(void);
typedef void (*uartp_start_fn_t)(float u0);
typedef void (*uartp_apply_tf_fn_t)(const float* c, uint16 n);
typedef void (*uartp_apply_ss_fn_t)(const float* c, uint16 n);
typedef void (*uartp_force_min_fn_t)(void);

typedef struct {
	uartp_stop_step_fn_t stop_step;
	uartp_force_min_fn_t force_min;
	uartp_start_fn_t     start;
	uartp_apply_tf_fn_t  apply_tf;
	uartp_apply_ss_fn_t  apply_ss;
	
	uartp_sampling_enable_fn_t        sampling_enable;
	uartp_sampling_disable_fn_t       sampling_disable;
	uartp_sampling_change_fs_fn_t     sampling_change_fs;     /* Fs (Hz) */
	uartp_sampling_clear_flags_fn_t   sampling_clear_flags;   /* puede ser NULL */
} uartp_cfg_t;

/* ===== globals (extern) ===== */
extern volatile uartp_sysmode_t     UARTP_SysMode;
extern volatile uartp_ctrl_state_t  UARTP_CtrlState;

extern volatile uartp_impl_t        UARTP_Impl;
extern volatile uint8               UARTP_ModeValid;
extern volatile uint8               UARTP_CoeffsValid;

extern volatile uint8               UARTP_StopRequested;
extern volatile uint8               UARTP_LastCmd;

extern float                        UARTP_Coeffs[UARTP_COEF_COUNT];
extern float                        UARTP_InitValue;

/* meta streaming */
extern volatile uint16              UARTP_StreamN;
extern float                        UARTP_StreamFsHz;

/* ===== API ===== */
void UARTP_Init(const uartp_cfg_t* cfg);
bool UARTP_ProcessOnce(void);
void UARTP_ControlTick(void);
void UARTP_EnterCommandMode(void);
void UARTP_EnterControlMode(void);

/* ===== Telemetry streaming (u,y) ===== */
void UARTP_Telemetry_Reset(void);
void UARTP_Telemetry_Push(float u, float y);

#endif
/* uartp_sw.c */
#include "uartp_sw.h"

/* ================= globals ================= */
volatile uartp_sysmode_t    UARTP_SysMode   = UARTP_SYS_COMMAND;
volatile uartp_ctrl_state_t UARTP_CtrlState = UARTP_CTRL_STOPPED;

volatile uartp_impl_t UARTP_Impl        = UARTP_IMPL_TF;
volatile uint8        UARTP_ModeValid   = 0u;
volatile uint8        UARTP_CoeffsValid = 0u;

static volatile uint8 g_telem_enable = 0u;

volatile uint8 UARTP_StopRequested = 0u;
volatile uint8 UARTP_LastCmd = 0u;

float UARTP_Coeffs[UARTP_COEF_COUNT];
float UARTP_InitValue = 0.0f;

/* meta streaming:
N y FsHz se toman SIEMPRE de:
c[23] = N
c[24] = FsHz
*/
volatile uint16 UARTP_StreamN    = 0u;
float           UARTP_StreamFsHz = 0.0f;

static uartp_cfg_t g_cfg;

/* update 'i' en CONTROL */
static volatile uint8  g_init_update_pending = 0u;
static uint8           g_init_update_raw[UARTP_INIT_BYTES];

/* ================= helpers ================= */
static void sampling_disable(void)
{
	if (g_cfg.sampling_disable) g_cfg.sampling_disable();
}
static void sampling_enable(void)
{
	if (g_cfg.sampling_enable) g_cfg.sampling_enable();
}
static void sampling_change_fs(float fs_hz)
{
	if (g_cfg.sampling_change_fs) g_cfg.sampling_change_fs(fs_hz);
}
static void sampling_clear_flags(void)
{
	if (g_cfg.sampling_clear_flags) g_cfg.sampling_clear_flags();
}
static void flush_rx(void)
{
	while (UARTP_UART_GetRxBufferSize() > 0u) (void)UARTP_UART_ReadRxData();
}

/* float -> u16 (para N) */
static uint16 u16_from_float(float x)
{
	if (x <= 0.0f) return 0u;
	if (x >= 65535.0f) return 65535u;
	return (uint16)(x + 0.5f);
}

/* Clamp/sanitize order TF (float -> int 0..UARTP_TF_MAX_ORDER) */
static void sanitize_tf_order_in_coeffs(void)
{
	float ord_f = UARTP_Coeffs[UARTP_IDX_TF_ORDER];
	
	/* NaN check: (NaN != NaN) */
	if (!(ord_f == ord_f)) ord_f = 0.0f;
	
	int32 ord_i = (int32)(ord_f + 0.5f);
	if (ord_i < 0) ord_i = 0;
	if (ord_i > (int32)UARTP_TF_MAX_ORDER) ord_i = (int32)UARTP_TF_MAX_ORDER;
	
	UARTP_Coeffs[UARTP_IDX_TF_ORDER] = (float)ord_i;
}

/* Lee N/Fs desde el vector normalizado (25 floats) */
static void update_stream_meta_from_coeffs(void)
{
	UARTP_StreamN    = u16_from_float(UARTP_Coeffs[UARTP_IDX_META_N]);
	UARTP_StreamFsHz = UARTP_Coeffs[UARTP_IDX_META_FSHZ];
}

/* ---- timer oneshot timeout (sin CyDelay) ---- */
static void uart_timer_start_ms(uint16 ms)
{
	if (ms == 0u) ms = 1u;
	
	UARTP_TIMER_UART_Stop();
	(void)UARTP_TIMER_UART_ReadStatusRegister();
	UARTP_TIMER_UART_WritePeriod(ms);
	UARTP_TIMER_UART_Start();
}
static bool uart_timer_expired(void)
{
	uint8 st = (uint8)UARTP_TIMER_UART_ReadStatusRegister();
	return ((st & UARTP_TIMER_UART_STATUS_TC) != 0u);
}
static bool read_byte_timeout(uint8* out, uint16 timeout_ms)
{
	uart_timer_start_ms(timeout_ms);
	
	while (!uart_timer_expired())
	{
		if (UARTP_UART_GetRxBufferSize() > 0u) {
			*out = UARTP_UART_ReadRxData();
			UARTP_TIMER_UART_Stop();
			(void)UARTP_TIMER_UART_ReadStatusRegister();
			return true;
		}
	}
	
	UARTP_TIMER_UART_Stop();
	(void)UARTP_TIMER_UART_ReadStatusRegister();
	return false;
}
static bool read_n_timeout(uint8* out, uint16 n, uint16 timeout_ms_each)
{
	for (uint16 i=0u; i<n; i++) {
		if (!read_byte_timeout(&out[i], timeout_ms_each)) return false;
	}
	return true;
}

static void write_byte(uint8 b){ UARTP_UART_PutChar(b); }
static void write_n(const uint8* p, uint16 n){ UARTP_UART_PutArray(p,n); }

/* ===== eco-confirmado RX word ===== */
static bool rx_word_try(uint8* dst4_or_partial, uint16 bytes_to_store, bool* did_timeout)
{
	*did_timeout = false;
	uint8 w[4];
	uint8 ctl;
	
	if (!read_n_timeout(w, 4u, UARTP_STEP_TIMEOUT_MS)) {
		*did_timeout = true;
		return false;
	}
	
	write_n(w, 4u);
	
	if (!read_byte_timeout(&ctl, UARTP_STEP_TIMEOUT_MS)) {
		*did_timeout = true;
		return false;
	}
	
	if (ctl == UARTP_CTL_ACK) {
		for (uint16 i=0u; i<bytes_to_store; i++) dst4_or_partial[i] = w[i];
		write_byte(UARTP_CTL_ACK);
		return true;
	}
	
	write_byte(UARTP_CTL_NAK);
	return false;
}

static bool recv_payload(uint8* dst, uint16 nbytes)
{
	uint16 off = 0u;
	uint8 timeouts_in_row = 0u;
	
	while (off < nbytes)
	{
		uint16 rem   = (uint16)(nbytes - off);
		uint16 store = (rem >= 4u) ? 4u : rem;
		
		uint16 tries = 0u;
		for (;;)
		{
			bool did_timeout = false;
			bool ok = rx_word_try(&dst[off], store, &did_timeout);
			
			if (ok) { off = (uint16)(off + store); timeouts_in_row = 0u; break; }
			
			if (did_timeout) {
				if (++timeouts_in_row >= UARTP_ABORT_TIMEOUTS) { flush_rx(); return false; }
			}
			if (++tries >= UARTP_MAX_WORD_RETRIES) { flush_rx(); return false; }
		}
	}
	return true;
}

/* ===== eco-confirmado TX word ===== */
static bool tx_word_once(const uint8 w[4], bool* accepted, bool* did_timeout)
{
	*did_timeout = false;
	uint8 echo[4];
	uint8 host_confirm;
	
	write_n(w, 4u);
	
	if (!read_n_timeout(echo, 4u, UARTP_STEP_TIMEOUT_MS)) {
		*did_timeout = true;
		return false;
	}
	
	bool match = (memcmp(echo, w, 4u) == 0);
	write_byte(match ? UARTP_CTL_ACK : UARTP_CTL_NAK);
	
	if (!read_byte_timeout(&host_confirm, UARTP_STEP_TIMEOUT_MS)) {
		*did_timeout = true;
		return false;
	}
	
	if (host_confirm != (match ? UARTP_CTL_ACK : UARTP_CTL_NAK)) {
		*accepted = false;
		return false;
	}
	
	*accepted = match;
	return true;
}

static bool send_payload(const uint8* src, uint16 nbytes)
{
	uint16 off = 0u;
	uint8 timeouts_in_row = 0u;
	
	while (off < nbytes)
	{
		uint8 w[4] = {0,0,0,0};
		uint16 rem  = (uint16)(nbytes - off);
		uint16 take = (rem >= 4u) ? 4u : rem;
		
		for (uint16 i=0u; i<take; i++) w[i] = src[off+i];
		
		uint16 tries = 0u;
		for (;;)
		{
			bool accepted = false;
			bool did_timeout = false;
			
			bool ok = tx_word_once(w, &accepted, &did_timeout);
			if (ok && accepted) { off = (uint16)(off + take); timeouts_in_row = 0u; break; }
			
			if (did_timeout) {
				if (++timeouts_in_row >= UARTP_ABORT_TIMEOUTS) { flush_rx(); return false; }
			}
			if (++tries >= UARTP_MAX_WORD_RETRIES) { flush_rx(); return false; }
		}
	}
	return true;
}

/* ================= RX ISR: STOP y INIT en CONTROL ================= */
CY_ISR(UARTP_RxIsr)
{
	while (UARTP_UART_GetRxBufferSize() > 0u)
	{
		uint8 b = UARTP_UART_ReadRxData();
		
		if (b == UARTP_CMD_STOP) {
			g_telem_enable = 0u;
			
			UARTP_StopRequested = 1u;
			if (UARTP_CtrlState == UARTP_CTRL_RUNNING) UARTP_CtrlState = UARTP_CTRL_STOPPING;
			write_byte(UARTP_RSP_OK);
			UARTP_ISR_RX_Disable();
		}
		else if (b == UARTP_CMD_INIT) {
			g_telem_enable = 0u;
			
			if (UARTP_CtrlState == UARTP_CTRL_RUNNING) {
				g_init_update_pending = 1u;
				write_byte(UARTP_RSP_READY_RX);
				UARTP_ISR_RX_Disable();
			} else {
				write_byte(UARTP_RSP_ERR);
			}
		}
		else {
			write_byte(UARTP_RSP_ERR);
		}
	}
	(void)UARTP_UART_ReadRxStatus();
}

/* ================= transitions ================= */
void UARTP_EnterCommandMode(void)
{
	sampling_disable();
	sampling_clear_flags();
	
	g_telem_enable = 0u;
	UARTP_ISR_RX_Disable();
	
	UARTP_StopRequested = 0u;
	UARTP_CtrlState = UARTP_CTRL_STOPPED;
	UARTP_SysMode = UARTP_SYS_COMMAND;
	
	flush_rx();
}

void UARTP_EnterControlMode(void)
{
	UARTP_StopRequested = 0u;
	g_telem_enable = 1u;
	
	sampling_disable();
	sampling_clear_flags();
	
	/* pasamos FsHz al MAIN (si es válido) */
	if (UARTP_StreamFsHz > 0.0f) {
		sampling_change_fs(UARTP_StreamFsHz);
	}
	
	sampling_enable();
	
	UARTP_SysMode = UARTP_SYS_CONTROL;
	UARTP_CtrlState = UARTP_CTRL_RUNNING;
	UARTP_ISR_RX_Enable();
	UARTP_Telemetry_Reset();
}

/* ================= public ================= */
void UARTP_Init(const uartp_cfg_t* cfg)
{
	UARTP_UART_Start();
	UARTP_UART_ClearRxBuffer();
	UARTP_UART_ClearTxBuffer();
	flush_rx();
	
	UARTP_TIMER_UART_Stop();
	(void)UARTP_TIMER_UART_ReadStatusRegister();
	
	memset(&g_cfg, 0, sizeof(g_cfg));
	if (cfg) g_cfg = *cfg;
	
	UARTP_SysMode = UARTP_SYS_COMMAND;
	UARTP_CtrlState = UARTP_CTRL_STOPPED;
	
	UARTP_Impl = UARTP_IMPL_TF;
	UARTP_ModeValid = 0u;
	UARTP_CoeffsValid = 0u;
	
	UARTP_StopRequested = 0u;
	UARTP_LastCmd = 0u;
	
	memset((void*)UARTP_Coeffs, 0, sizeof(UARTP_Coeffs));
	UARTP_InitValue = 0.0f;
	
	UARTP_StreamN = 0u;
	UARTP_StreamFsHz = 0.0f;
	
	g_init_update_pending = 0u;
	memset((void*)g_init_update_raw, 0, sizeof(g_init_update_raw));
	
	UARTP_ISR_RX_StartEx(UARTP_RxIsr);
	UARTP_ISR_RX_Disable();
	
	if (g_cfg.force_min) {
		g_cfg.force_min();
	}
	
	sampling_disable();
	sampling_clear_flags();
}

bool UARTP_ProcessOnce(void)
{
	if (UARTP_SysMode != UARTP_SYS_COMMAND) return false;
	if (UARTP_UART_GetRxBufferSize() == 0u) return false;
	
	uint8 cmd = UARTP_UART_ReadRxData();
	UARTP_LastCmd = cmd;
	
	if (UARTP_CtrlState != UARTP_CTRL_STOPPED) {
		if (cmd == UARTP_CMD_STOP) { write_byte(UARTP_RSP_OK); return true; }
		write_byte(UARTP_RSP_ERR);
		return true;
	}
	
	switch (cmd)
	{
		case UARTP_CMD_RESET:
		write_byte(UARTP_RSP_OK);
		CySoftwareReset();
		return true;
		
		case UARTP_CMD_STOP:
		write_byte(UARTP_RSP_OK);
		return true;
		
		case UARTP_CMD_SETMODE:
		{
			write_byte(UARTP_RSP_READY_RX);
			
			uint8 raw[UARTP_MODE_BYTES];
			if (!recv_payload(raw, (uint16)UARTP_MODE_BYTES)) { write_byte(UARTP_RSP_ERR); return true; }
			
			uint8 mode = raw[0];
			if (mode > 5u) { write_byte(UARTP_RSP_ERR); return true; }
			
			UARTP_Impl = (uartp_impl_t)mode;
			UARTP_ModeValid = 1u;
			
			write_byte(UARTP_RSP_OK);
			return true;
		}
		
		case UARTP_CMD_COEFFS:
		{
			if (UARTP_ModeValid == 0u) { write_byte(UARTP_RSP_ERR); return true; }
			
			write_byte(UARTP_RSP_READY_RX);
			
			uint8 rawc[UARTP_COEF_BYTES];
			if (!recv_payload(rawc, (uint16)UARTP_COEF_BYTES)) { write_byte(UARTP_RSP_ERR); return true; }
			
			memcpy((void*)UARTP_Coeffs, rawc, UARTP_COEF_BYTES);
			UARTP_CoeffsValid = 1u;
			
			/* meta streaming normalizada (siempre al final) */
			update_stream_meta_from_coeffs();
			
			/* si es TF, saneamos el order en c[22] */
			if (UARTP_Impl == UARTP_IMPL_TF) {
				sanitize_tf_order_in_coeffs();
			}
			
			if (UARTP_Impl == UARTP_IMPL_TF) {
				if (g_cfg.apply_tf) g_cfg.apply_tf((const float*)UARTP_Coeffs, UARTP_COEF_COUNT);
			} else if (UARTP_Impl == UARTP_IMPL_OPENLOOP) {
				/* nada */
			} else {
				if (g_cfg.apply_ss) g_cfg.apply_ss((const float*)UARTP_Coeffs, UARTP_COEF_COUNT);
			}
			
			write_byte(UARTP_RSP_OK);
			return true;
		}
		
		case UARTP_CMD_TXCOEF:
		{
			if (UARTP_CoeffsValid == 0u) { write_byte(UARTP_RSP_ERR); return true; }
			
			write_byte(UARTP_RSP_READY_TX);
			
			if (!send_payload((const uint8*)UARTP_Coeffs, (uint16)UARTP_COEF_BYTES)) { write_byte(UARTP_RSP_ERR); return true; }
			
			write_byte(UARTP_RSP_OK);
			return true;
		}
		
		case UARTP_CMD_INIT:
		{
			if (UARTP_ModeValid == 0u)   { write_byte(UARTP_RSP_ERR); return true; }
			if (UARTP_CoeffsValid == 0u) { write_byte(UARTP_RSP_ERR); return true; }
			
			write_byte(UARTP_RSP_READY_RX);
			
			uint8 rawi[UARTP_INIT_BYTES];
			if (!recv_payload(rawi, (uint16)UARTP_INIT_BYTES)) { write_byte(UARTP_RSP_ERR); return true; }
			
			memcpy((void*)&UARTP_InitValue, rawi, 4u);
			
			if (g_cfg.start) g_cfg.start(UARTP_InitValue);
			
			write_byte(UARTP_RSP_OK);
			
			UARTP_EnterControlMode();
			return true;
		}
		
		default:
		flush_rx();
		write_byte(UARTP_RSP_ERR);
		return true;
	}
}

void UARTP_ControlTick(void)
{
	if (UARTP_SysMode != UARTP_SYS_CONTROL) return;
	
	if (g_init_update_pending)
	{
		bool ok = recv_payload((uint8*)g_init_update_raw, (uint16)UARTP_INIT_BYTES);
		if (ok)
		{
			memcpy((void*)&UARTP_InitValue, (const void*)g_init_update_raw, 4u);
			if (g_cfg.start) g_cfg.start(UARTP_InitValue);
			write_byte(UARTP_RSP_OK);
		}
		else {
			write_byte(UARTP_RSP_ERR);
			flush_rx();
		}
		
		g_init_update_pending = 0u;
		g_telem_enable = 1u;
		UARTP_ISR_RX_Enable();
	}
	
	if (UARTP_CtrlState == UARTP_CTRL_STOPPING)
	{
		sampling_disable();
		sampling_clear_flags();
		
		bool done = true;
		if (g_cfg.stop_step) done = g_cfg.stop_step();
		if (!done) return;
		
		UARTP_Telemetry_Reset();
		
		UARTP_CtrlState = UARTP_CTRL_STOPPED;
		UARTP_SysMode = UARTP_SYS_COMMAND;
		UARTP_ISR_RX_Disable();
		g_telem_enable = 0u;
		flush_rx();
	}
}

/* ================= Telemetry (u,y) ================= */
static volatile uint16 g_tel_cnt = 0u;

void UARTP_Telemetry_Reset(void)
{
	g_tel_cnt = 0u;
}

void UARTP_Telemetry_Push(float u, float y)
{
	if (!g_telem_enable) return;
	if (UARTP_SysMode != UARTP_SYS_CONTROL) return;
	
	uint16 N = UARTP_StreamN;
	if (N == 0u) return;
	
	uint16 cnt = (uint16)(g_tel_cnt + 1u);
	g_tel_cnt = cnt;
	
	if (cnt < N) return;
	g_tel_cnt = 0u;
	
	uint8 pkt[8];
	memcpy(&pkt[0], &u, 4u);
	memcpy(&pkt[4], &y, 4u);
	
	#if defined(UART_TX_BUFFER_SIZE) && (UART_TX_BUFFER_SIZE > 0)
	if (UARTP_UART_GetTxBufferSize() > (UART_TX_BUFFER_SIZE - 8u)) {
		return;
	}
	#endif
	
	UARTP_UART_PutArray(pkt, 8u);
}
\end{lstlisting}
% ============================================================
% ============================================================
\section{Funciones de esfuerzo}
\label{ap:firmware_esfuerzo}
% ============================================================

\begin{lstlisting}[style=cfirmware, caption={ESC PWM (esc_pwm.h + esc_pwm.c)}, label={lst:esfuerzo}]
#ifndef ESC_PWM_H
#define ESC_PWM_H

#include <stdint.h>

#define ESC_MIN_US   (1000u)
#define ESC_MAX_US   (2000u)

/* Inicializa el PWM del ESC y deja en mínimo */
void esc_pwm_init(void);

/* Escribe pulso en microsegundos (satura 1000..2000) */
void esc_pwm_write_us(uint16_t us);

#endif
#include "project.h"
#include "esc_pwm.h"

/* Ajustá si tu ESC usa otros límites */


static uint16_t clamp_u16(uint16_t x)
{
	if (x < ESC_MIN_US) return ESC_MIN_US;
	if (x > ESC_MAX_US) return ESC_MAX_US;
	return x;
}

void esc_pwm_init(void)
{
	PWM_ESC_Start();
	esc_pwm_write_us(ESC_MIN_US);
}

void esc_pwm_write_us(uint16_t us)
{
	us = clamp_u16(us);
	PWM_ESC_WriteCompare(us);
}

	
\end{lstlisting}
% ============================================================
% ============================================================
\section{Sampling y sensor TFMini}
\label{ap:sampling}
% ============================================================

\begin{lstlisting}[style=cfirmware, caption={TFMini driver + calibración (tfmini\_psoc.h/.c y tfm\_calib\_simple.h)}, label={lst:sampling}]

#ifndef TFMINI_PSOC_H
#define TFMINI_PSOC_H

#include <stdint.h>
#include <stdbool.h>

#include "cytypes.h"
#include "CyLib.h"
#include "uart_TFminiPlus.h"
#include "isr_rx_TFminiPlus.h"

#define TFMINI_FRAME_SIZE (9u)
#define TFMINI_MIN_CM   (10u)
#define TFMINI_MAX_CM   (130u)

/* “hay muestra nueva lista” */
extern volatile uint8_t tfmini_sample_pending;

void  tfmini_init(void);
void  tfmini_clear_flags(void);

bool  tfmini_enable(void);
bool  tfmini_disable(void);
bool  tfmini_set_fps(uint16_t fps_hz);

/* Consume una muestra calibrada */
bool  tfmini_pop_cm(uint16_t *y_cm);

/* Calibración: SOLO distancia */
uint16_t tfmini_calibrate_cm(uint16_t dist_cm);

#endif /* TFMINI_PSOC_H */
#ifndef TFM_CALIB_SIMPLE_H_
#define TFM_CALIB_SIMPLE_H_

/* y = A + B*d + C*d^2  (ajustá desde MATLAB) */
#define TFM_CAL_A   (3.795825465480922f)
#define TFM_CAL_B   (0.999665425307861f)
#define TFM_CAL_C   (-0.001091976037850f)

static inline float tfmini_correct_distance_cm_simple(float d_cm)
{
	return (TFM_CAL_A + (TFM_CAL_B * d_cm) + (TFM_CAL_C * d_cm * d_cm));
}

#endif /* TFM_CALIB_SIMPLE_H_ */
#include "tfmini_psoc.h"
#include "tfm_calib_simple.h"

#include <string.h>

volatile uint8_t tfmini_sample_pending = 0u;

/* Parser state */
static volatile uint8_t  s_frame[TFMINI_FRAME_SIZE];
static volatile uint8_t  s_state = 0u; /* 0:wait H1, 1:wait H2, 2:collect */
static volatile uint8_t  s_idx   = 0u;

/* Última distancia calibrada */
static volatile uint16_t s_last_cm = 0u;

/* ===== TFMini Plus protocol ===== */
#define TF_CMD_HDR   (0x5Au)
#define TF_DATA_HDR  (0x59u)

#define CMD_SET_RATE (0x03u) /* payload: rate (uint16 LE) */
#define CMD_OUTPUT   (0x07u) /* payload: 0 disable, 1 enable */

/* =========================
Helpers
========================= */
static void parser_reset(void)
{
	s_state = 0u;
	s_idx   = 0u;
}

static uint8_t sum8_first8(const uint8_t* p9)
{
	uint16_t s = 0u;
	uint8_t i;
	for (i = 0u; i < 8u; i++) s += p9[i];
	return (uint8_t)s;
}

static void rx_flush_nolock(void)
{
	while (uart_TFminiPlus_GetRxBufferSize() > 0u) {
		(void)uart_TFminiPlus_ReadRxData();
	}
	(void)uart_TFminiPlus_ReadRxStatus();
}

static void rx_flush_and_reset_locked(void)
{
	uint8 intr = CyEnterCriticalSection();
	rx_flush_nolock();
	parser_reset();
	tfmini_sample_pending = 0u;
	isr_rx_TFminiPlus_ClearPending();
	CyExitCriticalSection(intr);
}

static bool send_cmd(uint8_t id, const uint8_t* payload, uint8_t plen)
{
	uint8_t  len;
	uint8_t  buf[8];
	uint16_t s;
	uint8_t  i;
	
	len = (uint8_t)(3u + plen + 1u);
	if (len > 8u) return false;
	
	buf[0] = TF_CMD_HDR;
	buf[1] = len;
	buf[2] = id;
	
	for (i = 0u; i < plen; i++) buf[3u + i] = payload[i];
	
	s = 0u;
	for (i = 0u; i < (uint8_t)(len - 1u); i++) s += buf[i];
	buf[len - 1u] = (uint8_t)s;
	
	rx_flush_and_reset_locked();
	for (i = 0u; i < len; i++) uart_TFminiPlus_PutChar(buf[i]);
	
	CyDelay(2u);
	rx_flush_and_reset_locked();
	
	return true;
}

/* =========================
Calibración: SOLO dist
========================= */
uint16_t tfmini_calibrate_cm(uint16_t dist_cm)
{
	float y = tfmini_correct_distance_cm_simple((float)dist_cm);
	
	/* guardas mínimas */
	if (!(y == y)) y = 0.0f;          /* NaN */
	if (y < 0.0f) y = 0.0f;
	if (y > 65535.0f) y = 65535.0f;
	
	return (uint16_t)(y + 0.5f);
}

/* =========================
Frame handler
========================= */
static void on_frame(const uint8_t* f)
{
	if (sum8_first8(f) != f[8]) return;
	
	/* dist = bytes[2..3] little-endian */
	uint16_t dist = (uint16_t)f[2] | ((uint16_t)f[3] << 8);
	
	dist = tfmini_calibrate_cm(dist);
	/* saturación a rango útil */
	if (dist < TFMINI_MIN_CM) dist = TFMINI_MIN_CM;
	else if (dist > TFMINI_MAX_CM) dist = TFMINI_MAX_CM;
	
	s_last_cm = dist;
	tfmini_sample_pending = 1u;
}

/* =========================
Byte parser
========================= */
static void parser_push(uint8_t b)
{
	if (s_state == 0u) {
		if (b == TF_DATA_HDR) {
			s_frame[0] = b;
			s_state = 1u;
		}
		return;
	}
	
	if (s_state == 1u) {
		if (b == TF_DATA_HDR) {
			s_frame[1] = b;
			s_idx = 2u;
			s_state = 2u;
		} else {
			s_state = 0u;
		}
		return;
	}
	
	/* collect */
	s_frame[s_idx] = b;
	s_idx++;
	
	if (s_idx >= TFMINI_FRAME_SIZE) {
		uint8_t fcopy[TFMINI_FRAME_SIZE];
		uint8_t i;
		for (i = 0u; i < TFMINI_FRAME_SIZE; i++) fcopy[i] = s_frame[i];
		on_frame(fcopy);
		parser_reset();
	}
}

/* =========================
ISR RX
========================= */
CY_ISR(isr_rx_tfmini_handler)
{
	isr_rx_TFminiPlus_ClearPending();
	
	#if defined(uart_TFminiPlus_RX_STS_FIFO_NOTEMPTY)
	for (;;) {
		uint8_t st = uart_TFminiPlus_ReadRxStatus();
		
		#if defined(uart_TFminiPlus_RX_STS_OVERRUN)
		if (st & uart_TFminiPlus_RX_STS_OVERRUN) {
			rx_flush_nolock();        /* ya estás en ISR */
			parser_reset();
			tfmini_sample_pending = 0u;
			break;
		}
		#endif
		
		if (st & uart_TFminiPlus_RX_STS_FIFO_NOTEMPTY) {
			uint8_t b = (uint8_t)uart_TFminiPlus_ReadRxData();
			parser_push(b);
			continue;
		}
		break;
	}
	#else
	while (uart_TFminiPlus_GetRxBufferSize() > 0u) {
		uint8_t b = (uint8_t)uart_TFminiPlus_ReadRxData();
		parser_push(b);
	}
	(void)uart_TFminiPlus_ReadRxStatus();
	#endif
}

/* =========================
API pública
========================= */
void tfmini_init(void)
{
	parser_reset();
	tfmini_sample_pending = 0u;
	s_last_cm = 0u;
	
	uart_TFminiPlus_Start();
	rx_flush_and_reset_locked();
	
	isr_rx_TFminiPlus_StartEx(isr_rx_tfmini_handler);
}

void tfmini_clear_flags(void)
{
	rx_flush_and_reset_locked();
}

bool tfmini_enable(void)
{
	uint8_t p = 1u;
	return send_cmd(CMD_OUTPUT, &p, 1u);
}

bool tfmini_disable(void)
{
	uint8_t p = 0u;
	return send_cmd(CMD_OUTPUT, &p, 1u);
}

bool tfmini_set_fps(uint16_t fps_hz)
{
	uint8_t p[2];
	
	if (fps_hz < 1u)    fps_hz = 1u;
	if (fps_hz > 1000u) fps_hz = 1000u;
	
	p[0] = (uint8_t)(fps_hz & 0xFFu);
	p[1] = (uint8_t)((fps_hz >> 8) & 0xFFu);
	
	return send_cmd(CMD_SET_RATE, p, 2u);
}

bool tfmini_pop_cm(uint16_t *y_cm)
{
	if (!y_cm) return false;
	
	uint8 intr = CyEnterCriticalSection();
	if (tfmini_sample_pending == 0u) {
		CyExitCriticalSection(intr);
		return false;
	}
	
	*y_cm = (uint16_t)s_last_cm;
	tfmini_sample_pending = 0u;
	CyExitCriticalSection(intr);
	
	return true;
}

	
\end{lstlisting}


% ============================================================
\section{Main}
\label{ap:main}
% ============================================================

\begin{lstlisting}[style=cfirmware, caption={Programa principal (main.c)}, label={lst:main}]
	
#include "project.h"
#include "uartp_sw.h"
#include "control_app.h"
#include "tfmini_psoc.h"
#include "esc_pwm.h"
#include <stdint.h>

#define TFMINI_FPS_DEFAULT   (1000u)
#define CALIB_FORCE_FS_HZ    (1000.0f)

/* =========================================================
Actuador: u_phy = microsegundos (1000..2000)
El control_app ya satura, pero acá también por seguridad.
========================================================= */
static void my_write_u(float u_us)
{
	if (u_us < 1000.0f) u_us = 1000.0f;
	if (u_us > 2000.0f) u_us = 2000.0f;
	
	esc_pwm_write_us((uint16_t)(u_us + 0.5f));
}

/* =========================================================
Helpers Fs
========================================================= */
static float   g_fs_requested_hz      = (float)TFMINI_FPS_DEFAULT;
static uint8_t g_fs_apply_after_calib = 0u;
static uint8_t g_fs_forcing_calib     = 0u;

/* Aplica Fs al sensor y mantiene Ts coherente */
static void apply_fs_now(float fs_hz)
{
	if (!(fs_hz > 0.0f)) return;
	
	uint16_t fps = (uint16_t)(fs_hz + 0.5f);
	if (fps < 1u) fps = 1u;
	if (fps > 1000u) fps = 1000u;
	
	if (tfmini_set_fps(fps)) {
		control_set_sample_time(1.0f / (float)fps);
	}
}

/* =========================================================
Callbacks para UARTP -> TFMini
========================================================= */
static void my_sampling_enable(void)      { (void)tfmini_enable(); }
static void my_sampling_disable(void)     { (void)tfmini_disable(); }
static void my_sampling_clear_flags(void) { tfmini_clear_flags(); }

/* Clave: durante autocal u0, forzamos 1000 Hz y pateamos el Fs real */
static void my_sampling_change_fs(float fs_hz)
{
	if (!(fs_hz > 0.0f)) return;
	
	/* Guardar siempre el Fs pedido por el host */
	g_fs_requested_hz = fs_hz;
	
	/* Si estamos calibrando u0 -> FORZAR 1000 Hz y aplicar después */
	if (control_is_calibrating())
	{
		g_fs_apply_after_calib = 1u;
		g_fs_forcing_calib     = 1u;
		apply_fs_now(CALIB_FORCE_FS_HZ);
		return;
	}
	
	/* Si no estamos calibrando -> aplicar Fs real */
	apply_fs_now(fs_hz);
}

/* =========================================================
MAIN
========================================================= */
int main(void)
{
	CyGlobalIntEnable;
	
	/* UARTP usa timer_uart (si tu implementación lo requiere) */
	timer_uart_Start();
	
	/* ===== ESC PWM ===== */
	esc_pwm_init();
	esc_pwm_write_us(1000u);  /* seguro al boot */
	
	/* ===== Control app =====
	- sample_isr_cb = NULL porque el muestreo lo manejamos por polling de tfmini_pop_cm()
	- write_u_cb = my_write_u
	*/
	control_register_io(NULL, my_write_u);
	control_set_sample_time(1.0f / (float)TFMINI_FPS_DEFAULT);
	
	/* ===== TFMini ===== */
	tfmini_init();
	(void)tfmini_set_fps(TFMINI_FPS_DEFAULT);
	(void)tfmini_enable();
	
	/* ===== UARTP ===== */
	uartp_cfg_t cfg = {
		.stop_step = control_stop_suave_step,
		.start     = control_start,
		.force_min = control_force_min,
		.apply_tf  = control_apply_tf,
		.apply_ss  = control_apply_ss,
		
		.sampling_enable      = my_sampling_enable,
		.sampling_disable     = my_sampling_disable,
		.sampling_change_fs   = my_sampling_change_fs,
		.sampling_clear_flags = my_sampling_clear_flags
	};
	UARTP_Init(&cfg);
	
	for (;;)
	{
		switch (UARTP_SysMode)
		{
			case UARTP_SYS_COMMAND:
			(void)UARTP_ProcessOnce();
			break;
			
			case UARTP_SYS_CONTROL:
			{
				uint16_t y_cm;
				
				/* Consumimos una muestra nueva del sensor */
				if (tfmini_pop_cm(&y_cm)) {
					control_sample_isr_push((float)y_cm);
				}
				
				/* Si hay muestra lista -> ejecutar control */
				if (control_sample_pending) {
					control_sample_pending = 0u;
					control_step();
				}
				
				/* Si veníamos forzando 1000 Hz por calibración y ya terminó -> aplicar Fs real */
				if (g_fs_forcing_calib && !control_is_calibrating())
				{
					g_fs_forcing_calib = 0u;
					
					if (g_fs_apply_after_calib)
					{
						g_fs_apply_after_calib = 0u;
						apply_fs_now(g_fs_requested_hz);
					}
				}
				
				UARTP_ControlTick();
				break;
			}
			
			default:
			UARTP_EnterCommandMode();
			break;
		}
	}
}
\end{lstlisting}

\twocolumn
\section{Evolución del diseño estructural de la planta}
\label{app:estructura}

Durante el desarrollo del trabajo práctico, la estructura física de la planta atravesó distintas etapas de diseño, las cuales permitieron identificar limitaciones mecánicas y realizar mejoras progresivas hasta alcanzar la configuración final utilizada en las prácticas experimentales. En este apéndice se describe la primera etapa de diseño de la estructura y se destacan las principales diferencias respecto de la versión final.

\subsection{Primera etapa de diseño}

La primera versión de la estructura fue concebida con una altura total aproximada de \(80\,\text{cm}\), utilizando la misma base y el mismo techo de madera que se mantienen en el diseño final. Debido a las dimensiones de estos elementos, la altura útil de movimiento del cuerpo móvil en esta etapa era de aproximadamente \(72\,\text{cm}\).

En esta configuración inicial, el diseño mecánico del cuerpo móvil era diferente al actual, presentando dimensiones ligeramente mayores. El sistema no contaba con elementos de seguridad adicionales, tales como topes mecánicos, amortiguación ante caídas ni cuerda de seguridad, dado que el recorrido vertical era considerablemente menor y el riesgo asociado a caídas desde grandes alturas resultaba limitado.

El guiado del cuerpo móvil se realizaba mediante vigas metálicas rectas y rígidas, las cuales no presentaban deformaciones apreciables. Debido a esta rigidez estructural, no fue necesario incorporar articulaciones pasivas tipo ``muñeca'' en las abrazaderas, ni estructuras auxiliares de madera para limitar deformaciones. En esta etapa, el contacto entre el cuerpo móvil y los rieles generaba fricción apreciable, la cual se manifestaba de forma consistente durante el movimiento vertical.

Cabe destacar que este comportamiento friccional, observable en la primera versión de la estructura, no se presenta de la misma manera en el diseño final. La incorporación de vigas metálicas de mayor longitud, junto con las deformaciones inherentes a las mismas y la inclusión de articulaciones pasivas en las abrazaderas, redujo significativamente la fricción directa entre el cuerpo móvil y los rieles, modificando así las características mecánicas del sistema.

En las figuras siguientes se presentan imágenes correspondientes a las primeras versiones de las piezas impresas en 3D utilizadas en esta etapa inicial del diseño, las cuales difieren de las empleadas en la configuración final de la planta.

\insertarfigura{img/Disenos/cuerpo1.png}{Primer diseño del soporte superior del motor.}{fig:cuerpo1}{0.5}

En la primera etapa de diseño, el soporte superior del cuerpo móvil presentaba una altura aproximada de \(3\,\text{cm}\) y un diámetro de \(4\,\text{cm}\). Las secciones sobresalientes destinadas al acople de los brazos contaban con una altura de aproximadamente \(1{,}5\,\text{cm}\). Dicho soporte incluía orificios dimensionados específicamente para el montaje del motor brushless, con un diámetro de \(3\,\text{mm}\), acorde al patrón de fijación del mismo.

\insertarfigura{img/Disenos/brazo_y_base_1.png}{Primeros diseños del soporte inferior y de los brazos estructurales.}{fig:brazosybase1}{1}

El soporte inferior del cuerpo móvil correspondía a una geometría espejo del soporte superior. En este componente se realizaba el encastre de los brazos estructurales, los cuales, en esta etapa inicial, presentaban dimensiones menores en comparación con el diseño final. Cada brazo tenía dimensiones aproximadas de \(1\,\text{cm} \times 1\,\text{cm} \times 20\,\text{cm}\).

El sistema de agarre de los brazos difería del implementado en la versión final de la planta. Inicialmente, el agarre no contaba con movilidad angular, aunque permitía un ajuste manual respecto a la posición de la viga metálica, lo que condicionaba el guiado del cuerpo móvil y su interacción con los rieles.

\subsection{Segunda etapa de diseño}

En la segunda etapa de diseño, el componente que presentó mayores modificaciones fue el brazo estructural. A partir de la experiencia obtenida en la etapa inicial, se introdujeron variaciones geométricas en el diseño del brazo, incorporando curvaturas con el objetivo de mejorar el encastre y la interacción con la estructura de guiado.

En esta versión, el brazo y el sistema de agarre fueron integrados en una única pieza impresa en 3D, eliminando la separación entre ambos componentes. Dado que en esta etapa las vigas metálicas utilizadas como rieles presentaban una geometría recta y una rigidez suficiente, no fue necesaria la incorporación de articulaciones pasivas tipo ``muñeca''. En consecuencia, el guiado del cuerpo móvil se realizaba mediante un agarre rígido, sin movilidad angular.

Cabe destacar que, durante esta etapa, el diseño del cuerpo móvil se mantuvo sin modificaciones significativas respecto a la versión anterior. Las mejoras se concentraron exclusivamente en el diseño de los brazos y del sistema de agarre, manteniendo constante la geometría general del conjunto móvil.


\insertarfigura{img/Disenos/brazo2.png}{Segundo diseño de brazo.}{fig:brazo2}{1}

\subsection{Tercera etapa de diseño}

En la tercera etapa de diseño se introdujo una modificación significativa en la estructura general de la planta, extendiendo su altura máxima hasta aproximadamente \(160\,\text{cm}\). Esta ampliación respondió a la necesidad de disponer de un mayor recorrido vertical para la realización de las prácticas de control, lo cual implicó nuevas exigencias mecánicas sobre el conjunto estructural y el cuerpo móvil.

Como consecuencia del aumento de altura de la estructura, el diseño de los brazos del cuerpo móvil volvió a ser modificado. En esta etapa, los brazos fueron rediseñados con mayor grosor y mayor altura, con el objetivo de incrementar su rigidez y capacidad de carga. Asimismo, se incorporaron aberturas longitudinales en los brazos, destinadas a permitir la inserción de elementos metálicos, con el fin de reforzar la estructura y mejorar su resistencia mecánica frente a esfuerzos y vibraciones.

El cuerpo móvil mantuvo su configuración general respecto a las etapas anteriores; sin embargo, el rediseño de los brazos resultó fundamental para adaptar el conjunto a las nuevas condiciones estructurales impuestas por la mayor altura de la planta.

\insertarfigura{img/Disenos/cuerpo3.jpeg}{Tercer diseño de brazos.}{fig:cuerpo3}{1}

\subsection{Cuarta etapa de diseño: configuración final}

La cuarta etapa de diseño corresponde a la configuración final de la estructura y del cuerpo móvil utilizada en las prácticas experimentales del trabajo. En esta etapa se introdujeron modificaciones orientadas principalmente a mejorar el guiado mecánico del cuerpo móvil y a reducir la masa total del conjunto.

Debido a que las vigas metálicas empleadas como rieles presentan deformaciones asociadas a su longitud, se incorporaron articulaciones pasivas tipo ``muñeca'' en el sistema de guiado. Estas articulaciones permiten un movimiento angular relativo entre el cuerpo móvil y los rieles, mejorando el desplazamiento vertical y evitando atascamientos o esfuerzos indeseados durante el recorrido.

Con el objetivo de reducir la masa del cuerpo móvil, se redimensionaron los soportes principales. El soporte superior, que en versiones anteriores presentaba una altura de \(3\,\text{cm}\), fue reducido a aproximadamente \(1{,}5\,\text{cm}\), mientras que el soporte inferior pasó de \(2\,\text{cm}\) a \(0{,}5\,\text{cm}\). A pesar de esta reducción dimensional, se conservó el sistema de encastre tanto en el soporte superior como en el inferior, asegurando la rigidez estructural del conjunto.

Adicionalmente, se incorporó un soporte específico para la batería, integrado al cuerpo móvil. En dicho soporte se colocaron almohadillas internas con el fin de proteger la batería frente a vibraciones e impactos durante el funcionamiento del sistema.

Estas modificaciones permitieron obtener un diseño final más liviano, adaptable a las deformaciones estructurales de los rieles y adecuado para la implementación de las distintas estrategias de control desarrolladas en el presente trabajo.

\insertarfigura{img/Disenos/cuerpo3.jpeg}{Tercer diseño de brazos.}{fig:cuerpo3}{1}

\section{Incidentes experimentales y fallas en los controladores ESC}
\label{app:esc}

Durante el desarrollo experimental del trabajo se presentaron fallas en los controladores electrónicos de velocidad (ESC) utilizados en las primeras etapas de prueba del sistema. En este apéndice se describen los incidentes observados, junto con el análisis de las posibles causas y las medidas adoptadas posteriormente.

\subsection{Primer incidente: ESC de 30\,A con alimentación externa}

En una primera instancia, se utilizó un ESC de \(30\,\text{A}\) alimentado mediante una batería para automóviles, con el objetivo de verificar el funcionamiento básico del sistema de propulsión. La conexión entre la fuente de alimentación y el ESC se realizó utilizando un cable unifilar de cobre de considerable longitud.

Durante las pruebas iniciales, el sistema logró generar empuje y el cuerpo móvil llegó a elevarse. Sin embargo, tras un período de funcionamiento, el ESC comenzó a emitir una secuencia de señales acústicas consistente en cuatro pitidos cortos seguidos de un pitido largo. En ese momento no se contaba con una interpretación clara del significado de dicha señalización.

Posteriormente, mediante la consulta de documentación y experiencias previas, se determinó que dicha secuencia de pitidos está asociada a condiciones de protección del ESC, tales como sobrecorriente o sobretemperatura. Esta hipótesis se vio reforzada por el hecho de que los cables unifilares utilizados para la alimentación se calentaron excesivamente y llegaron a derretirse, indicando una circulación de corriente elevada y pérdidas resistivas significativas.


\subsection{Segundo incidente: reinicios y falla del ESC de 30\,A}

En una segunda etapa de pruebas con el mismo ESC de \(30\,\text{A}\), se reemplazaron los cables de alimentación por conductores adecuados para altas corrientes, conectando el ESC directamente a la batería utilizada en la planta. En esta configuración, el sistema no lograba elevarse de forma sostenida y el ESC emitía una secuencia de sonidos correspondiente a un reinicio del controlador.

Con el fin de descartar un problema en la señal de control, se analizó la señal PWM generada por el PSoC mediante un osciloscopio, verificándose que la misma presentaba una forma adecuada y estable, sin perturbaciones significativas. En consecuencia, se descartó que la falla estuviera asociada a errores en la generación de la señal de control.

Ante la hipótesis de una posible caída de tensión en la alimentación del ESC durante los transitorios de corriente, se incorporaron capacitores de desacople en la línea de alimentación. Tras esta modificación, el sistema logró generar empuje y elevarse durante breves instantes. No obstante, luego de un corto período de funcionamiento, se produjo la falla definitiva del ESC, observándose la quema de un MOSFET correspondiente a una de las fases del motor.

\subsection{Análisis y consideraciones}

A partir de los incidentes descritos, se identificaron como causas probables la combinación de sobrecorriente, exigencias térmicas elevadas y condiciones de alimentación no ideales durante las primeras pruebas. La utilización de una fuente de alimentación inadecuada, conductores con alta resistencia y la ausencia inicial de medidas de protección contribuyeron a someter al ESC a esfuerzos superiores a sus límites operativos.

Estos eventos pusieron de manifiesto la importancia de considerar cuidadosamente los aspectos de potencia, disipación térmica y protección eléctrica en sistemas de propulsión basados en motores brushless, incluso en etapas preliminares de prueba.

Las lecciones aprendidas a partir de estas fallas motivaron la adopción de controladores de mayor capacidad de corriente, mejoras en el cableado de alimentación y la implementación de estrategias de operación más conservadoras, las cuales permitieron continuar con el desarrollo experimental del trabajo de manera segura y confiable. 
\insertarfigura{img/Planta/ESC30A1.jpeg}{ESC30A incidente 1.}{fig:inc30A1}{1}

\subsection{Tercer incidente: falla del ESC de 30\,A durante operación con batería LiPo}

En un tercer incidente, se utilizó un ESC de \(30\,\text{A}\) alimentado mediante una batería LiPo para drones. Durante esta prueba, se incrementó la señal PWM hasta aproximadamente \(1500\,\mu s\), logrando que el sistema generara empuje suficiente para elevar el cuerpo móvil hasta la parte superior de la estructura.

Al intentar detener el movimiento, se adoptó un procedimiento no óptimo, consistente en bloquear mecánicamente la hélice con el fin de evitar una colisión con el techo de la estructura. Esta acción provocó el trabado de la hélice durante el funcionamiento del motor, lo cual generó un incremento abrupto de la corriente demandada. Como consecuencia, el ESC sufrió una falla catastrófica, produciéndose la quema de múltiples componentes internos y la pérdida total del controlador.

Este incidente permitió identificar el riesgo asociado al bloqueo mecánico del rotor en sistemas de propulsión brushless, dado que dicha condición conduce a corrientes elevadas que superan rápidamente la capacidad de los dispositivos de conmutación del ESC.
\insertarfigura{img/Planta/ESC30A2.jpeg}{ESC30A incidente 2.}{fig:inc30A2}{1}


\subsection{Cuarto incidente: falla del ESC de 30\,A por sobrecorriente}

En un cuarto incidente, se realizaron pruebas controladas con un nuevo ESC de \(30\,\text{A}\), con el objetivo de determinar el valor máximo de PWM que el sistema podía soportar de manera segura. Durante esta prueba, el valor de PWM se incrementó progresivamente hasta alcanzar aproximadamente \(1600\,\mu s\).

En estas condiciones, el ESC volvió a presentar una falla similar a la observada en el segundo incidente, registrándose la quema de un MOSFET correspondiente a una de las fases del motor. Este comportamiento reforzó la hipótesis de que el controlador se encontraba operando cerca de sus límites de corriente, incluso sin que se produjera un bloqueo mecánico del rotor.

\insertarfigura{img/Planta/ESC30A3.jpeg}{ESC30A incidente 3.}{fig:inc30A3}{1}


\subsection{Medidas adoptadas}

La repetición de fallas en controladores de \(30\,\text{A}\), tanto bajo condiciones transitorias como en operación sostenida, llevó a concluir que dicho margen de corriente resultaba insuficiente para el motor utilizado y las exigencias mecánicas de la planta. Asimismo, se consideró la posible influencia de algoritmos internos del ESC y de su calidad de construcción, los cuales podrían limitar su capacidad de manejo de sobrecorrientes.

En función de estas observaciones, se decidió sobredimensionar el sistema de actuación mediante la adquisición de un ESC de \(40\,\text{A}\). Esta decisión permitió operar el motor con un mayor margen de seguridad, evitando la necesidad de reducir aún más la masa del cuerpo móvil y mejorando la confiabilidad del sistema durante las prácticas experimentales.


