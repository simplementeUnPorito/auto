Dado que la planta identificada presenta polos ubicados fuera del círculo unitario, el sistema en lazo abierto resulta inestable en el dominio discreto. En consecuencia, cualquier diseño de control debe garantizar que los polos del lazo cerrado queden estrictamente dentro del círculo unitario para asegurar estabilidad interna.

\insertarfigura{img/CircUnitarioPlanta.png}
{Ubicación de los polos de la planta sin compensar en el plano $z$. Se observa que al menos uno de ellos se encuentra fuera del círculo unitario, lo que implica inestabilidad discreta.}
{fig:PolosPlantaSinCompensar}{1}

El objetivo del diseño consistió en modificar la dinámica del sistema mediante compensación, de modo que:

\begin{itemize}
	\item todos los polos del lazo cerrado queden dentro del círculo unitario,
	\item se logre un compromiso adecuado entre rapidez de respuesta y amortiguamiento,
	\item el esfuerzo de control permanezca dentro de límites físicamente realizables.
\end{itemize}

\subsubsection{Elección de la estructura del compensador}

Para estabilizar el sistema se adoptó una estructura de tipo \textbf{lag--lead} (atraso--adelanto). Esta configuración permite actuar simultáneamente sobre la estabilidad relativa y el desempeño en régimen permanente.

El término \textit{lead} (adelanto) se empleó para aumentar el margen de fase y desplazar los polos dominantes del lazo cerrado hacia regiones del plano \(z\) asociadas con mayor amortiguamiento y mejor desempeño transitorio. Por otro lado, el término \textit{lag} (atraso) permitió ajustar la ganancia en bajas frecuencias, mejorando el comportamiento estacionario sin comprometer significativamente la estabilidad.

La adecuada ubicación de ceros permitió modificar la geometría del lugar de raíces, atrayendo las ramas hacia la región estable del plano discreto, mientras que los polos adicionales modelaron el compromiso dinámico requerido.

\subsubsection{Determinación de la ganancia \(K\)}

Una vez definida la estructura del compensador, se analizó el lugar de raíces del sistema compensado. La determinación manual de la ganancia \(K\) resultó particularmente sensible, ya que pequeños incrementos en su valor provocaban que las trayectorias de los polos abandonaran el círculo unitario antes de satisfacer las especificaciones dinámicas deseadas.

Esta sensibilidad está directamente relacionada con la naturaleza inestable de la planta y con la fuerte dependencia de la ubicación de los polos del lazo cerrado respecto a la ganancia del compensador.

El compensador finalmente adoptado fue:

\begin{equation}
	C(z)=
	-0.0173 \;
	\frac{(z-1.0140)(z-0.5)}
	{(z-0.9522)(z-0.9894)}
\end{equation}

\subsubsection{Ajuste mediante \texttt{Optimization-Based Tuning}}

Con el fin de sistematizar el proceso de ajuste y garantizar el cumplimiento simultáneo de múltiples especificaciones (estabilidad, rapidez y esfuerzo de control), se utilizó la herramienta \texttt{Optimization-Based Tuning} de MATLAB.

Este enfoque permitió:

\begin{itemize}
	\item definir directamente especificaciones temporales (tiempo de establecimiento, sobreimpulso, etc.),
	\item ajustar automáticamente los parámetros del compensador,
	\item verificar la estabilidad del sistema en el dominio discreto.
\end{itemize}

El resultado fue un compensador lag--lead cuyos parámetros fueron obtenidos mediante optimización numérica, asegurando que los polos del lazo cerrado se mantengan dentro del círculo unitario y que el desempeño temporal cumpla con los objetivos establecidos para la planta experimental.

\insertarfigura{img/Rlocus/circuloUnitario.png}
{Ubicación de los polos del sistema compensado en el plano $z$. Se verifica que todos ellos se encuentran dentro del círculo unitario, garantizando estabilidad discreta.}
{fig:PolosSistemaCompensado}{1}

\insertarfigura{img/Rlocus/Esfuerzo.png}
{Esfuerzo de control en lazo cerrado con el compensador diseñado mediante Lugar de Raíces.}
{fig:EsfuerzoControlCompensado}{1}

\insertarfigura{img/Rlocus/rsta.png}
{Respuesta temporal del sistema en lazo cerrado con el compensador diseñado.}
{fig:RespuestaTemporalCompensada}{1}

\subsection{Práctica}
a completar lo que sale en la práctica
\insertarfigura{img/PID/RstaPIDPractica.png}{La respuesta con el controlador PID.}{fig:RstaPID_practica}{1}

\insertarfigura{img/PID/EsfuerzoPIDPractica.png}{El esfuerzo del con el controlador PID.}{fig:EsfPID_practica}{1}