El método de Lugar de Raíces se fundamenta en el análisis directo de la ecuación característica del sistema en lazo cerrado:

\[
1 + C(z)G(z) = 0
\]

Las raíces de esta ecuación determinan los polos del sistema compensado.  
En el dominio discreto, la condición de estabilidad interna exige que:

\[
|z_i| < 1 \quad \forall i
\]

es decir, que todos los polos del lazo cerrado se encuentren estrictamente dentro del círculo unitario.

% ============================================================
\subsubsection{Inestabilidad discreta de la planta}
% ============================================================

La planta identificada presenta polos ubicados sobre el perímetro del círculo unitario, lo que implica que el sistema en lazo abierto es criticamente inestable tanto  en el dominio continuo como discreto.

\insertarfigura{img/CircUnitarioPlanta.png}
{Ubicación de los polos de la planta sin compensar en el plano $z$. Se observa la presencia de polos sobre o fuera del círculo unitario, indicando inestabilidad discreta.}
{fig:PolosPlantaSinCompensar}{1}

En consecuencia, el objetivo del diseño no consistió únicamente en mejorar el desempeño dinámico, sino en estabilizar activamente la planta mediante la adecuada selección de un compensador.

Los criterios de diseño adoptados fueron:

\begin{itemize}
	\item Desplazar todos los polos del lazo cerrado dentro del círculo unitario.
	\item Obtener una respuesta sobreamortiguada, sin sobreimpulso.
	\item Mantener el esfuerzo de control dentro de límites físicamente realizables.
\end{itemize}

% ============================================================
\subsubsection{Frecuencia de muestreo}
% ============================================================

Para este método en particular se determinó utilizar una frecuencia de muestreo de:

\[
F_s = 1000\,\text{Hz}
\]

Esta decisión se tomó inicialmente con el objetivo de capturar adecuadamente la dinámica identificada y aproximar el comportamiento discreto al continuo.

Sin embargo, la elección de una frecuencia de muestreo elevada produjo que los polos discretos asociados a las dinámicas relevantes quedaran extremadamente próximos a \(z=1\). Esto generó un fenómeno que puede interpretarse como un \textbf{desaprovechamiento geométrico del plano Z}, ya que la región efectiva donde se desarrollaba la dinámica quedó concentrada en un sector muy reducido cercano al borde del círculo unitario.

En estas condiciones, pequeñas variaciones en la ganancia del compensador producían desplazamientos significativos de los polos en el plano discreto, aumentando la sensibilidad numérica del diseño.

Posteriormente se verificó que el controlador también hubiese funcionado correctamente para frecuencias menores (200 Hz o 100 Hz), donde el plano Z resulta mejor aprovechado geométricamente y la sensibilidad disminuye. No obstante, el diseño a 1000 Hz permitió experimentar de manera directa los efectos teóricos asociados a la selección del período de muestreo.

% ============================================================
\subsubsection{Elección de la estructura del compensador}
% ============================================================

Para estabilizar el sistema se adoptó una estructura de tipo \textbf{lag--lead} (atraso--adelanto).

El término \textit{lead} permitió:

\begin{itemize}
	\item Incrementar el margen de fase.
	\item Desplazar los polos dominantes hacia regiones asociadas a mayor amortiguamiento.
	\item Mejorar el desempeño transitorio.
\end{itemize}

El término \textit{lag} permitió:

\begin{itemize}
	\item Ajustar la ganancia en bajas frecuencias.
	\item Mejorar el comportamiento estacionario.
	\item Reducir la sensibilidad global del sistema.
\end{itemize}

La adecuada ubicación de ceros modificó la geometría del lugar de raíces, atrayendo las trayectorias hacia el interior del círculo unitario.

% ============================================================
\subsubsection{Determinación de la ganancia}
% ============================================================

Una vez definida la estructura del compensador, se analizó el lugar de raíces del sistema compensado.

Se observó una elevada sensibilidad respecto a la ganancia \(K\).  
Pequeños incrementos en su valor provocaban que las trayectorias abandonaran la región estable antes de satisfacer las especificaciones dinámicas deseadas.

Esta sensibilidad se atribuye a:

\begin{itemize}
	\item La naturaleza originalmente inestable de la planta.
	\item La proximidad de los polos discretos a \(z=1\).
	\item La frecuencia de muestreo elevada.
\end{itemize}

El compensador finalmente adoptado fue:

\begin{equation}
	C(z)=
	-0.0173 \;
	\frac{(z-1.0140)(z-0.5)}
	{(z-0.9522)(z-0.9894)}
\end{equation}

% ============================================================
\subsubsection{Optimización basada en especificaciones}
% ============================================================

Con el fin de sistematizar el ajuste y evitar la extrema sensibilidad manual, se utilizó la herramienta \texttt{Optimization-Based Tuning} de MATLAB.

Este enfoque permitió:

\begin{itemize}
	\item Definir especificaciones temporales deseadas.
	\item Ajustar automáticamente los parámetros del compensador.
	\item Verificar estabilidad discreta.
\end{itemize}

El resultado fue un compensador cuyos polos en lazo cerrado se ubican completamente dentro del círculo unitario.

\insertarfigura{img/Rlocus/circuloUnitario.png}
{Ubicación de los polos del sistema compensado en el plano $z$. Se verifica estabilidad discreta.}
{fig:PolosSistemaCompensado}{1}

% ============================================================
\subsubsection{Resultados en simulación}
% ============================================================

La respuesta temporal simulada del modelo lineal mostró:

\begin{itemize}
	\item Comportamiento sobreamortiguado.
	\item Ausencia de sobreimpulso.
	\item Tiempo de subida aproximado de $5\,\text{s}$.
\end{itemize}

\insertarfigura{img/Rlocus/rsta.png}
{Respuesta temporal del sistema en lazo cerrado con el compensador diseñado.}
{fig:RespuestaTemporalCompensada}{1}

El esfuerzo de control obtenido en simulación se mantuvo dentro de límites aceptables:

\insertarfigura{img/Rlocus/Esfuerzo.png}
{Esfuerzo de control en lazo cerrado con el compensador diseñado.}
{fig:EsfuerzoControlCompensado}{1}

% ============================================================
\subsubsection{Resultados experimentales}
% ============================================================

En la implementación práctica se obtuvo:

\[
t_r^{\text{exp}} = 1.068\,\text{s} \text{ a } 2.3\,\text{s}
\]

sin presencia de sobreimpulso.

Se observa que el sistema real resulta considerablemente más rápido que el modelo simulado. Esta discrepancia puede atribuirse a:

\begin{itemize}
	\item Simplificaciones del modelo lineal.
	\item Dinámicas no modeladas.
	\item Sensibilidad numérica asociada al uso de un período de muestreo no adecuado.
\end{itemize}

No obstante, el esfuerzo aplicado en la práctica resultó visualmente muy similar al predicho por simulación, validando parcialmente la estructura del compensador adoptado.

\insertarfigura{img/Rlocus/RlocusPractica.png}
{Implementación práctica del controlador diseñado por Lugar de Raíces.}
{fig:Rlocus_practica}{1}
\balance
\newpage

% ============================================================
\subsubsection{Conclusión del método}
% ============================================================

El diseño por Lugar de Raíces permitió estabilizar una planta originalmente inestable en el dominio discreto, garantizando que los polos del lazo cerrado se ubiquen dentro del círculo unitario.
El método demostró ser geométricamente intuitivo y conceptualmente potente, ya que permite visualizar directamente la relación entre ganancia, ubicación de polos y desempeño dinámico.

En esta aplicación particular se evidenció:

\begin{itemize}
	\item Alta sensibilidad paramétrica.
	\item Fuerte dependencia de la frecuencia de muestreo.
	\item Reducción efectiva del espacio geométrico utilizable en el plano Z al emplear una frecuencia de muestreo elevada.
\end{itemize}

No obstante, el controlador diseñado logró un comportamiento sobreamortiguado sin sobreimpulso y con esfuerzo físicamente realizable.

Una ventaja significativa del método es que, mediante los parámetros clásicos de segundo orden (\(\zeta\) y \(\omega_n\)), puede anticiparse de manera intuitiva el comportamiento del lazo cerrado sin necesidad inmediata de herramientas de simulación. La comprensión profunda de la relación entre el plano Z y la respuesta transitoria resultó especialmente enriquecedora en la práctica, permitiendo interpretar directamente los efectos de la ubicación de polos en el desempeño observado.
