
Dado que la planta identificada presenta polos fuera del círculo unitario, el sistema en lazo abierto resulta inestable en el dominio discreto.
\insertarfigura{img/CircUnitarioPlanta.png}
{Ubicación de los polos del sistema sin compensar en el plano $z$ respecto al círculo unitario.}
{fig:CircUnitarioPlanta}{1}

El objetivo del diseño consistió en modificar la ubicación de los polos del lazo cerrado de manera que:

\begin{itemize}
	\item todos los polos del sistema compensado queden dentro del círculo unitario,
	\item se obtenga un compromiso adecuado entre rapidez y amortiguamiento,
	\item el esfuerzo de control permanezca dentro de límites físicamente realizables.
\end{itemize}

\subsubsection{Elección de la estructura del compensador}

Para lograr la estabilización del sistema se adoptó una estructura de tipo \textbf{lag--lead} (atraso--adelanto). Esta elección permitió actuar tanto sobre la estabilidad relativa como sobre el error en régimen permanente.

El término \textit{lead} (adelanto) se utilizó con el fin de aumentar el margen de fase y desplazar los polos dominantes hacia regiones del plano \(z\) con mayor amortiguamiento. Por su parte, el término \textit{lag} (atraso) permitió ajustar la ganancia en régimen permanente sin deteriorar significativamente la estabilidad del sistema.

La incorporación de ceros adecuadamente ubicados permitió atraer las ramas del lugar de raíces hacia la región estable del plano discreto, mientras que los polos adicionales modelaron el compromiso dinámico deseado.

\subsubsection{Determinación de la ganancia \(K\)}

Una vez definida la estructura del compensador, se procedió al análisis del lugar de raíces del sistema compensado. Sin embargo, la determinación manual de la ganancia \(K\) resultó compleja, ya que para ciertos rangos de valores las trayectorias de los polos tendían a salir del círculo unitario antes de cumplir con las especificaciones dinámicas deseadas.

Esta sensibilidad se debe a la naturaleza inestable de la planta y a la influencia significativa de la ganancia sobre la ubicación de los polos del lazo cerrado.

\subsubsection{Ajuste mediante \texttt{Optimization-Based Tuning}}

Con el fin de simplificar el proceso de ajuste y garantizar el cumplimiento simultáneo de múltiples criterios (estabilidad, rapidez y esfuerzo de control), se utilizó la herramienta \texttt{Optimization-Based Tuning} de MATLAB.

Este enfoque permite:

\begin{itemize}
	\item formular especificaciones dinámicas directamente (tiempo de establecimiento, sobreimpulso, etc.),
	\item ajustar automáticamente los parámetros del compensador,
	\item verificar restricciones de estabilidad en el dominio discreto.
\end{itemize}

El resultado final fue un compensador lag--lead cuyos parámetros fueron obtenidos mediante optimización numérica, asegurando que los polos del lazo cerrado permanezcan dentro del círculo unitario y que el desempeño temporal cumpla con los objetivos establecidos para la planta experimental.
