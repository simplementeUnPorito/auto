% ============================================================
\subsection{Control Óptimo con Integrador (LQI)}
% ============================================================

\subsubsection{Motivación}

El regulador cuadrático lineal (LQR) permite optimizar el compromiso
entre desempeño y esfuerzo de control. Sin embargo, no garantiza
error estacionario nulo ante referencias constantes.

Para eliminar el error permanente, se introduce un integrador
del error de seguimiento, ampliando el modelo del sistema.


% ============================================================
\subsubsection{Modelo Aumentado}
% ============================================================

Sistema original:

\[
x_{k+1} = Ax_k + Bu_k
\]

\[
y_k = Cx_k
\]

Se define el estado integrador:

\[
\xi_{k+1} = \xi_k + (r_k - y_k)
\]

Definiendo el estado aumentado:

\[
x_a =
\begin{bmatrix}
	x \\
	\xi
\end{bmatrix}
\]

el sistema aumentado queda:

\[
x_{a,k+1} =
\underbrace{
	\begin{bmatrix}
		A & 0 \\
		-C & 1
	\end{bmatrix}
}_{A_a}
x_{a,k}
+
\underbrace{
	\begin{bmatrix}
		B \\
		0
	\end{bmatrix}
}_{B_a}
u_k
+
\begin{bmatrix}
	0 \\
	1
\end{bmatrix}
r_k
\]


% ============================================================
\subsubsection{Diseño LQR sobre el Sistema Aumentado}
% ============================================================

Se define el funcional de costo:

\[
J=\sum_{k=0}^{\infty}
\left(
x_a^T Q_a x_a + u_k^T R u_k
\right)
\]

Matrices de ponderación:

\[
Q_a = \textbf{[Completar]}
\]

\[
R = \textbf{[Completar]}
\]

La solución se obtiene resolviendo la ecuación de Riccati discreta
para el sistema aumentado.

Ganancia obtenida:

\[
K_a =
\begin{bmatrix}
	K_x & K_i
\end{bmatrix}
\]

donde:

\begin{itemize}
	\item $K_x$ actúa sobre los estados originales.
	\item $K_i$ actúa sobre el estado integrador.
\end{itemize}

Ley de control final:

\[
u_k = -K_x x_k - K_i \xi_k
\]


% ============================================================
\subsubsection{Polos del Sistema Aumentado}
% ============================================================

Los polos del lazo cerrado resultan de:

\[
\lambda(A_a - B_a K_a)
=
\textbf{[Completar]}
\]

Se verifica estabilidad discreta:

\[
|p_i| < 1
\]


% ============================================================
\subsubsection{Resultados}
% ============================================================

\begin{itemize}
	\item Error estacionario: \textbf{[Completar]}
	\item Sobreimpulso: \textbf{[Completar]}
	\item Tiempo de establecimiento: \textbf{[Completar]}
	\item Pico de esfuerzo de control: \textbf{[Completar]}
\end{itemize}

\insertarfigura{img/Estados/respuesta_LQI.png}
{Respuesta temporal con LQI.}
{fig:resp_LQI}{1}

\insertarfigura{img/Estados/esfuerzo_LQI.png}
{Esfuerzo de control con LQI.}
{fig:esfuerzo_LQI}{1}


% ============================================================
\subsubsection{Discusión}
% ============================================================

La incorporación del integrador permite eliminar el error
estacionario frente a referencias constantes.

En comparación con:

\begin{itemize}
	\item Ubicación arbitraria de polos:
	el método LQI introduce un criterio explícito de optimización.
	\item LQR sin integrador:
	se logra seguimiento exacto de referencia.
\end{itemize}

Limitaciones:

\begin{itemize}
	\item Incremento del orden del sistema.
	\item Mayor sensibilidad a saturación si $K_i$ es elevado.
	\item Dependencia del modelo identificado.
\end{itemize}
