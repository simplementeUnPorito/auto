\section{Objetivos}

\subsection{Objetivo general}

Modelar y controlar la altura de una planta experimental basada en propulsión vertical, mediante la aplicación y comparación de técnicas clásicas y modernas de control automático, evaluando su desempeño teórico y experimental sobre el sistema físico real.

\subsection{Objetivos específicos}

\begin{itemize}
	
	\item Obtener un modelo dinámico representativo de la planta a partir del análisis físico y la identificación experimental alrededor del punto de operación.
	
	\item Diseñar e implementar técnicas clásicas de control discreto, incluyendo:
	\begin{itemize}
		\item Control PID.
		\item Diseño mediante lugar de las raíces.
		\item Diseño basado en diagramas de Bode.
		\item Síntesis directa (Truxal--Ragazzini), considerando el efecto de oscilaciones intramuestra.
	\end{itemize}
	
	\item Diseñar e implementar técnicas modernas de control, tales como:
	\begin{itemize}
		\item Ubicación arbitraria de polos en espacio de estados.
		\item Control con acción integral.
		\item Regulador cuadrático lineal (LQR).
		\item Estimación de estados mediante filtro de Kalman.
	\end{itemize}
	
	\item Comparar las distintas estrategias de control en términos de estabilidad, respuesta transitoria, error en régimen permanente, esfuerzo de control y robustez frente a perturbaciones y restricciones físicas del sistema.
	
	\item Analizar las diferencias entre el comportamiento teórico predicho por los modelos y el desempeño experimental observado en la planta real.
	
\end{itemize}
