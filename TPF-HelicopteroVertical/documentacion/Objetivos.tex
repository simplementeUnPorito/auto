\section{Objetivos}

\subsection{Objetivo general}

Desarrollar, analizar e implementar un sistema de control de altura para una planta experimental basada en un motor brushless, aplicando y comparando distintas estrategias de control automático estudiadas en la materia Automatizaciones, a partir del modelado físico y matemático del sistema y su validación experimental sobre la planta real.

\subsection{Objetivos específicos}

\begin{itemize}
	\item Caracterizar físicamente la planta experimental y describir su comportamiento dinámico a partir de sus componentes mecánicos, eléctricos y de sensado.
	
	\item Desarrollar un modelo físico y matemático del sistema que represente adecuadamente la dinámica vertical de la planta y sirva como base para el diseño de controladores.
	
	\item Identificar experimentalmente los parámetros relevantes del modelo, considerando las no idealidades propias del sistema real.
	
	\item Diseñar e implementar controladores clásicos de altura, incluyendo el controlador PID y métodos de diseño basados en el lugar de las raíces, diagramas de Bode, síntesis directa y ubicación arbitraria de polos.
	
	\item Diseñar controladores en el espacio de estados para sistemas de seguimiento, incorporando control integral para la eliminación del error estacionario.
	
	\item Implementar y evaluar estimadores de estado tanto de tipo predictivo como de actualización, analizando su influencia en el desempeño del sistema.
	
	\item Diseñar y aplicar estrategias de control óptimo, incluyendo el uso de integrador y estimadores basados en el filtro de Kalman, culminando en la implementación de un regulador LQG.
	
	\item Comparar el desempeño de las distintas estrategias de control implementadas en términos de estabilidad, respuesta transitoria, error en régimen permanente y robustez frente a perturbaciones y variaciones paramétricas.
	
	\item Validar experimentalmente los resultados obtenidos mediante ensayos sobre la planta real, contrastando el comportamiento observado con las predicciones del modelo.
\end{itemize}
