
\subsubsection{Fundamento}

Se propone la ley de control:

\[
u_k = -Kx_k
\]

lo que conduce a:

\[
x_{k+1} = (A - BK)x_k
\]

El objetivo es imponer:

\[
\lambda(A - BK)=\{p_1,p_2,\dots,p_n\}
\]


\subsubsection{Selección de Polos Deseados}

Polos seleccionados:

\[
\{p_i\} = \textbf{[Completar]}
\]

Criterios:

\begin{itemize}
	\item Rapidez deseada.
	\item Amortiguamiento.
	\item Estabilidad discreta ($|p_i|<1$).
\end{itemize}


\subsubsection{Cálculo de la Ganancia}

La ganancia se obtuvo mediante:

\begin{itemize}
	\item Método de Ackermann.
	\item Función \texttt{place()}.
\end{itemize}

\[
K = \textbf{[Completar]}
\]

Polos resultantes:

\[
\lambda(A-BK)=\textbf{[Completar]}
\]


\subsubsection{Resultados}

\begin{itemize}
	\item Sobreimpulso: \textbf{[Completar]}
	\item Tiempo de establecimiento: \textbf{[Completar]}
	\item Pico de esfuerzo: \textbf{[Completar]}
\end{itemize}

%\insertarfigura{img/Estados/respuesta_poleplacement.png}{Respuesta temporal con ubicación arbitraria de polos.}{fig:resp_poleplacement}{1}
