\subsection{Realimentación de Estados y Estimación}

\subsubsection{Realimentación de Estados}

Considerando el modelo discreto del sistema:

\[
x_{k+1} = A x_k + B u_k
\]
\[
y_k = C x_k
\]

se propone una ley de control por realimentación de estados:

\[
u_k = -K x_k
\]

lo que conduce a la dinámica en lazo cerrado:

\[
x_{k+1} = (A - BK)x_k
\]

El diseño por ubicación arbitraria de polos consiste en determinar la
matriz de ganancia \(K\) tal que:

\[
\lambda(A - BK) = \{p_1, p_2, \dots, p_n\}
\]

siendo \(|p_i| < 1\) condición necesaria para estabilidad discreta.

La determinación de \(K\) puede realizarse mediante el método de
Ackermann o utilizando la función \texttt{place()} de MATLAB, siempre
que el sistema sea completamente controlable.

\vspace{0.3cm}

\subsubsection{Estimador de Estados}

En situaciones donde no todos los estados son medibles, se introduce
un estimador de Luenberger para reconstruir el vector de estados a
partir de la entrada y la salida medida.

El estimador discreto se define como:

\[
\hat{x}_{k+1} = A\hat{x}_k + B u_k + L (y_k - \hat{y}_k)
\]

donde:

\[
\hat{y}_k = C \hat{x}_k
\]

La dinámica del error de estimación:

\[
e_k = x_k - \hat{x}_k
\]

queda gobernada por:

\[
e_{k+1} = (A - LC)e_k
\]

Por lo tanto, la convergencia del estimador depende de la ubicación
de los autovalores de la matriz \(A - LC\), que pueden ser fijados
arbitrariamente siempre que el sistema sea observable:

\[
\lambda(A - LC) = \{p_{obs,1}, \dots, p_{obs,n}\}
\]

\vspace{0.3cm}

\subsubsection{Principio de Separación}

Cuando se combinan realimentación de estados y estimación, la ley de
control adopta la forma:

\[
u_k = -K\hat{x}_k
\]

y la dinámica total del sistema presenta autovalores dados por la
unión de los polos del controlador y los polos del estimador:

\[
\lambda_{\text{total}} =
\lambda(A - BK) \cup \lambda(A - LC)
\]

Este resultado, conocido como principio de separación, permite diseñar
independientemente el controlador y el estimador.
