A partir del modelo discreto identificado (orden $n=3$) y operando en
coordenadas relativas al punto de hover, se adopta la representación:

\[
x_{k+1}=Ax_k+Bu_k,
\qquad
y_k=Cx_k
\]

donde la única salida medida es la altura $y_k$.

El objetivo del diseño por realimentación de estados consiste en
definir una ganancia $K$ tal que la dinámica en lazo cerrado:

\[
x_{k+1}=(A-BK)\,x_k
\]

presente polos ubicados en posiciones deseadas $p_i$ dentro del círculo unitario:

\[
\lambda(A-BK)=\{p_1,p_2,\dots,p_n\},
\qquad |p_i|<1.
\]

La ganancia $K$ se obtuvo mediante la función \texttt{place()} de MATLAB,
aprovechando que el sistema es completamente controlable.

Para la práctica presentada se utilizó:

\[
T_s = 0.02~\text{s},
\qquad
F_s=50~\text{Hz},
\qquad
n=3
\]

con polos deseados del lazo cerrado:

\[
p_{\text{ctrl}}=\{0.95\pm j\,0.15,\;0.95\}
\]

obteniéndose:

\[
K=
\begin{bmatrix}
	0.6169 & -0.4194 & 0.2418
\end{bmatrix}.
\]

\insertarfigura{img/SS/SS_pzmap.png}
{Mapa de polos y ceros en el plano-$z$: polos de la planta discreta, polos del lazo cerrado $(A-BK)$ y polos de los observadores.}
{fig:ss_pzmap}{1}

La Fig.~\ref{fig:ss_pzmap} permite verificar que todos los polos del lazo
cerrado se ubican estrictamente dentro del círculo unitario, garantizando
estabilidad interna discreta.

% ============================================================
\subsubsection{Seguimiento de referencia}
% ============================================================

Como el objetivo del sistema es seguir una referencia de altura (y no
regular a cero), se incorporó un precompensador discreto
$N_{\text{bar}}$ para asegurar ganancia unitaria en régimen permanente
bajo el modelo lineal:

\[
u_k = N_{\text{bar}}\,r_k - K\hat{x}_k.
\]

Para el caso de estudio se obtuvo:

\[
N_{\text{bar}} = 9.1835.
\]

% ============================================================
\subsubsection{Estimación de estados}
% ============================================================

Dado que únicamente se mide la altura, el resto de estados debe estimarse.
Se implementó un observador de orden completo en dos variantes.

% ------------------------------------------------------------
\paragraph{1) Observador predictor}
% ------------------------------------------------------------

\[
\hat{x}_{k+1}
=
A\hat{x}_k + Bu_k + L_{\text{pred}}\big(y_k-\hat{y}_k\big),
\qquad
\hat{y}_k=C\hat{x}_k.
\]

La dinámica del error queda gobernada por:

\[
e_{k+1}=(A-L_{\text{pred}}C)e_k,
\]

por lo que los polos del estimador se fijan mediante:

\[
\lambda(A-L_{\text{pred}}C)=p_{\text{obs}}.
\]

% ------------------------------------------------------------
\paragraph{2) Observador actual}
% ------------------------------------------------------------

\[
z_{k+1}=A\hat{x}_k+Bu_k,
\qquad
\hat{y}_{k+1}^- = C z_{k+1},
\]

\[
\hat{x}_{k+1}=z_{k+1}+L_{\text{act}}\big(y_{k+1}-\hat{y}_{k+1}^-\big).
\]

En este caso la dinámica del error resulta:

\[
e_{k+1}=(A-L_{\text{act}}CA)e_k,
\]

por lo que explícitamente:

\[
\lambda(A-L_{\text{act}}CA)=p_{\text{obs}}.
\]

% ------------------------------------------------------------
\paragraph{Polos seleccionados y criterio numérico}
% ------------------------------------------------------------

En ambos casos se utilizaron:

\[
p_{\text{obs}}=\{0.8\pm j\,0.25,\;0.9\}.
\]

Las ganancias obtenidas fueron:

\[
L_{\text{pred}}=
\begin{bmatrix}
	167.3535\\
	253.6822\\
	86.0310
\end{bmatrix},
\qquad
L_{\text{act}}=
\begin{bmatrix}
	126.8411\\
	172.0621\\
	45.1301
\end{bmatrix}.
\]

La selección de $p_{\text{ctrl}}$ y $p_{\text{obs}}$ no fue arbitraria.
Se buscó que la función \texttt{place()} reportara precisiones numéricas
superiores a 10 dígitos significativos en su segundo argumento
(indicador de condición numérica). En la práctica, elecciones con menor
precisión comportaminetos imprevistos por la simulación.

Asimismo, la elección de polos excesivamente rápidos generaba
ganancias de magnitud muy elevada, resultando en esfuerzos de control
desmedidos y físicamente no implementables en la planta real.

% ============================================================
\subsubsection{Principio de Separación}
% ============================================================

El principio de separación establece que la dinámica del controlador
$(A-BK)$ y la del observador son estructuralmente ortogonales.
Esto significa que el diseño del control y el diseño del estimador
pueden realizarse de manera independiente, ya que la matriz del sistema
aumentado es triangular por bloques.

En consecuencia, los autovalores del sistema completo resultan de la unión:

\[
\lambda_{\text{total}}
=
\lambda(A-BK)
\cup
\begin{cases}
	\lambda(A-L_{\text{pred}}C), & \text{(predictor)} \\
	\lambda(A-L_{\text{act}}CA), & \text{(actual)}.
\end{cases}
\]

Esto permitió ajustar separadamente la rapidez del lazo de control
y la convergencia del observador.

% ============================================================
\subsubsection{Resultados en simulación}
% ============================================================

\insertarfigura{img/SS/SS_step_noSat.png}{Respuesta temporal sin saturación.}{fig:ss_sinSat}{1}

\insertarfigura{img/SS/SS_step_Sat.png}{Respuesta temporal con saturación.}{fig:ss_Sat}{1}

\insertarfigura{img/SS/SS_esfuerzo_Sat.png}{Esfuerzo de control con saturación.}{fig:ss_EsfSat}{1}

\insertarfigura{img/SS/SS_todoJunto.png}{Resumen: respuesta, esfuerzo y círculo unitario.}{fig:ss_TJSat}{1}

En simulación, bajo el modelo lineal ideal, la respuesta resulta
suavemente oscilatoria y con esfuerzo acotado. No se observa
sobreimpulso significativo en el modelo nominal.

% ============================================================
\subsubsection{Resultados experimentales}
% ============================================================

\insertarfigura{img/SS/SS_practico_predictor.png}{Implementación práctica con observador predictor.}{fig:ss_pred_pract}{1}

\insertarfigura{img/SS/SS_actual_practico.png}{Implementación práctica con observador actual.}{fig:ss_act_pract}{1}

Para el observador predictor se obtuvo:

\[
t_r \approx 520~\text{ms},
\qquad
\%OS \approx 13.79\%.
\]

En un segundo levantamiento:

\[
\%OS \approx 11.864\%.
\]

La variación entre ensayos se atribuye principalmente a descarga
de batería y a dinámicas no modeladas.

En la práctica, el predictor mostró un esfuerzo más errático que
en simulación. Esto se interpreta como amplificación del ruido de
medición por parte del observador, generando un comportamiento
aparentemente derivativo y picos cercanos a saturación.

En el caso del observador actual, la respuesta en altura fue similar,
pero los picos de esfuerzo fueron menos frecuentes. La corrección con
$y_{k+1}$ introduce un efecto de filtrado natural debido a la dinámica
de la planta entre $u_k$ y la medición siguiente.

% ============================================================
\subsubsection{Conclusión}
% ============================================================

El control por realimentación de estados permitió seguimiento estable
y rápido de la referencia.

No obstante, este enfoque mostró una dependencia mucho mayor del modelo
que los métodos clásicos. Pequeñas discrepancias entre planta real e
identificada se reflejan directamente en el transitorio.

Comparativamente:

\begin{itemize}
	\item El predictor es más simple computacionalmente, pero más sensible al ruido.
	\item El observador actual mostró mejor comportamiento práctico al filtrar parcialmente el ruido.
\end{itemize}

Siempre que el tiempo de cálculo sea despreciable frente al período
de muestreo, el observador actual resulta preferible para esta planta.


