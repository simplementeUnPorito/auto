\subsubsection{Fundamento}

Se propone una ley de control por realimentación de estados de la forma:

\[
u_k = -Kx_k
\]

lo que conduce a la dinámica en lazo cerrado:

\[
x_{k+1} = (A - BK)x_k
\]

El objetivo del diseño consiste en ubicar arbitrariamente los polos del
sistema en lazo cerrado, imponiendo:

\[
\lambda(A - BK)=\{p_1,p_2,\dots,p_n\}
\]

De esta manera, la dinámica del sistema queda completamente determinada
por la selección de los autovalores deseados.

\vspace{0.3cm}

\subsubsection{Selección de Polos Deseados}

En primera instancia se realiza la simulación considerando un actuador
ideal, es decir, sin restricciones de saturación.

El tiempo de muestreo adoptado es:

\[
T_s = 0.02 \text{ s}
\]

Los polos seleccionados para el diseño fueron:

\[
\{p_i\} =
\left\{
0.95 + 0.15i,\;
0.95 - 0.15i,\;
0.95
\right\}
\]

Los criterios de selección fueron:

\begin{itemize}
	\item Garantizar estabilidad discreta ($|p_i|<1$).
	\item Lograr una dinámica moderadamente rápida sin exigir esfuerzos
	de control excesivos.
	\item Introducir amortiguamiento suficiente para evitar oscilaciones
	pronunciadas.
\end{itemize}

\insertarfigura{img/SS/SS_pzmap.png}
{Mapa de polos del sistema en lazo cerrado con realimentación de estados.}
{fig:ss_pzmap}{1}

\insertarfigura{img/SS/SS_esfuerzo_noSat.png}
{Esfuerzo de control en simulación sin saturación del actuador.}
{fig:ss_esfuerzo_nosat}{1}

%\insertarfigura{img/SS/SS_step_noSat.png}{Respuesta temporal en lazo cerrado sin saturación del actuador.}{fig:ss_step_nosat}{1}

Se observa que los autovalores del sistema en lazo cerrado coinciden con
los polos deseados, verificándose que:

\[
\lambda(A-BK) =
\left\{
0.95 + 0.15i,\;
0.95 - 0.15i,\;
0.95
\right\}
\]

\vspace{0.3cm}

\subsubsection{Cálculo de la Ganancia}

La ganancia de realimentación fue obtenida mediante el método de
Ackermann y verificada utilizando la función \texttt{place()} de MATLAB.

El vector de ganancias resultante es:

\[
K =
\left[
0.6169,\;
-0.4194,\;
0.2418
\right]
\]

Para el caso con estimador de estados (estructura observador actual),
la ganancia del observador obtenida fue:

\[
K_{eA} =
\left[
126.8411,\;
172.0621,\;
45.1301
\right]
\]

\vspace{0.3cm}

\subsubsection{Análisis sin Saturación}

Bajo el supuesto lineal de actuador ideal, el sistema presenta una
respuesta estable y coherente con los polos seleccionados.

Es importante destacar que, en comparación con un conjunto de polos
más agresivo evaluado previamente, el esfuerzo de control disminuye
significativamente. Esto se debe a que la ubicación elegida no exige
ganancias excesivas en la matriz $K$, evitando amplificaciones
innecesarias del estado.

\vspace{0.3cm}

\subsubsection{Análisis con Saturación}

Posteriormente se introduce la saturación del actuador para evaluar la
implementabilidad física del controlador.

\insertarfigura{img/SS/SS_todoJunto.png}
{Comparación de salida y esfuerzo en simulación con saturación del actuador.}
{fig:ss_sim_sat_combined}{1}

\insertarfigura{img/SS/SS_esfuerzo_Sat.png}
{Esfuerzo de control en simulación con saturación del actuador.}
{fig:ss_esfuerzo_sat}{1}

\insertarfigura{img/SS/SS_step_Sat.png}
{Respuesta temporal en lazo cerrado con saturación del actuador.}
{fig:ss_step_sat}{1}

La inclusión de la saturación introduce una no linealidad que modifica
ligeramente la dinámica transitoria, aunque el sistema mantiene la
estabilidad general del lazo cerrado.

\vspace{0.3cm}

\subsubsection{Resultados Experimentales}

\insertarfigura{img/SS/SS_practico.png}
{Respuesta experimental del sistema en lazo cerrado: salida y esfuerzo de control.}
{fig:ss_practico}{1}

Los resultados experimentales muestran coherencia con la simulación
considerando saturación del actuador.

\begin{itemize}
	\item Sobreimpulso: \textbf{[Completar con valor medido]}
	\item Tiempo de establecimiento: \textbf{[Completar con valor medido]}
	\item Pico de esfuerzo: \textbf{[Completar con valor medido]}
\end{itemize}

Se observa una buena correspondencia entre modelo y comportamiento
experimental dentro del rango operativo seguro.
