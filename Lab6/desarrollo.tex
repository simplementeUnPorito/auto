
\section{Desarrollo}


Considerando la planta analógica esquematizada en la Fig.~\ref{fig:planta_planta}, se procedió a diseñar compensadores digitales utilizando en forma conjunta la técnica de \emph{ubicación arbitraria de polos} y el empleo de \emph{estimadores de estado} en tiempo discreto, tal como lo exige la consigna de la Práctica 6. En esta sección se documentan las decisiones de diseño, los cálculos realizados en \texttt{MATLAB} (código completo en el Apéndice, ver Código~\ref{lst:hojadecalculos}) y la comparación entre las dos arquitecturas de estimador: el \textbf{estimador de predicción} y el \textbf{estimador actual}.

\subsection{Selección del tiempo de muestreo}

El primer paso fue fijar un tiempo de muestreo $T_s$ que permita capturar la dinámica más rápida de la planta. Para ello se partió del modelo continuo obtenido a partir de los valores medidos de resistencias y capacitores:
\begin{align*}
	C_1 &= \SI{211.1e-9}{\farad}, & C_2 &= \SI{103.07e-9}{\farad},\\
	R_1 &= \SI{47e3}{\ohm},       & R_2 &= \SI{81.09e3}{\ohm},\\
	R_3 &= \SI{6.8e3}{\ohm},      & R_4 &= \SI{14.760e3}{\ohm}.
\end{align*}

Con estos valores se definieron las constantes de tiempo y las ganancias
\[
\tau_1 = R_2 C_1,\qquad \tau_2 = R_4 C_2,\qquad
k_1 = -\frac{R_2}{R_1},\qquad k_2 = -\frac{R_4}{R_3},
\]
y se armó el modelo continuo en espacio de estados:
\[
F =
\begin{bmatrix}
	-\dfrac{1}{\tau_1} & 0 \\
	\dfrac{k_2}{\tau_2} & -\dfrac{1}{\tau_2}
\end{bmatrix},\quad
G =
\begin{bmatrix}
	\dfrac{k_1}{\tau_1} \\[4pt]
	0
\end{bmatrix},\quad
H = \begin{bmatrix} 0 & 1 \end{bmatrix},\quad
J = 0,
\]
que coincide con la planta analógica real.

El cálculo en \texttt{MATLAB} se hizo con el siguiente fragmento (el código completo está en el Apéndice, Código~\ref{lst:hojadecalculos}):

\begin{lstlisting}[style=matlabstyle,caption={Cálculo de $T_s$ a partir del polo continuo más rápido.},label={lst:seleccion_Ts}]
	clear all; clc; close all;
	
	% --- Planta continua (modelo medido) ---
	C2 = 103.07e-9;  C1 = 211.1e-9;
	R3 = 6.8e3;      R4 = 14.760e3;
	R1 = 47e3;       R2 = 81.09e3;
	
	tau1 = R2*C1;           k1 = -R2/R1;
	tau2 = R4*C2;           k2 = -R4/R3;
	
	F = [ -1/tau1,     0;
	k2/tau2, -1/tau2 ];
	G = [ k1/tau1; 0 ];
	H = [0 1];  J = 0;
	
	sysC = ss(F,G,H,J);
	Gc   = tf(sysC);
	
	% Ts base vía polo más rápido continuo
	fn = max(abs(zpk(Gc).P{1})) / pi;
	Tn = 1/fn;
	Ts = Tn/4;   % 4 muestras por la dinámica más rápida
	
	% Discretización
	sysD = c2d(sysC, Ts, 'zoh');
	[A,B,C,D] = ssdata(sysD);
	
	% Controlabilidad / observabilidad
	Co = ctrb(A,B);
	Ob = obsv(A,C);
	rCo = rank(Co); rOb = rank(Ob);
\end{lstlisting}

La lógica fue:
\begin{enumerate}
	\item obtener los polos continuos de la planta;
	\item tomar el polo de mayor módulo (el más rápido);
	\item definir el período asociado $T_n = 1/f_n$;
	\item fijar el tiempo de muestreo como
	\[
	T_s = \frac{T_n}{4},
	\]
	es decir, cuatro muestras sobre la parte más rápida de la dinámica.
\end{enumerate}
Con ese $T_s$ se discretizó la planta y se verificó que
\[
\operatorname{rango}(\mathrm{ctrb}(A,B)) = \operatorname{rango}(\mathrm{obsv}(A,C)) = 2,
\]
por lo que el diseño por ubicación de polos y el diseño de observadores son viables en discreto.

\insertarfigurawide{img/circuit.png}{Planta analógica discretizada para el laboratorio.}{fig:planta_planta}{0.8}

\subsection{Diseño del controlador por realimentación de estados}

Con el modelo discreto $(A,B,C,D,T_s)$ se diseñó primero el lazo de control. Se eligió la pareja compleja
\[
p_{\text{ctrl}} = 0{,}8 \pm j\,0{,}2
\]
porque:
\begin{itemize}
	\item queda más lenta que un diseño agresivo, por lo que el esfuerzo de control no se dispara;
	\item está dentro del círculo unidad y asegura un régimen aceptable;
	\item está en una zona intermedia respecto de los polos discretizados de la planta.
\end{itemize}
La ganancia de realimentación se obtuvo con Ackermann:

\begin{lstlisting}[style=matlabstyle,caption={Cálculo de $K$ y del prefiltro $N_{\text{bar}}$.},label={lst:control_K}]
	% Polos de control deseados
	p_ctrl = [0.8 + 1j*0.2; 0.8 - 1j*0.2];
	
	% Ganancia de realimentación de estados
	K = acker(A,B,p_ctrl);
	
	% Prefiltro para que y_ss = r (caso SISO y D=0)
	[~,~,Nbar] = refi(A,B,C,K);
\end{lstlisting}

La ley de control que se implementó fue entonces
\[
u[k] = N_{\text{bar}}\,r[k] - K\,\hat{x}[k],
\]
donde \(\hat{x}[k]\) es el \emph{estado estimado}. Esto es así porque en esta planta sólo se mide la salida
\[
y[k] = C\,x[k] = x_2[k],
\]
y con esa única medición se debe reconstruir el vector completo de estado \(x[k]\); por lo tanto, el estimador no es opcional.


\subsection{Diseño de los dos estimadores de estado}


A partir de la única medición de la salida \(y[k]\) se debe llegar a reconstruir el vector de estado completo \(x[k] = [x_1\;x_2]^\top\); por eso el estimador es una parte fundamental del diseño. Se implementaron las dos variantes que se piden en el laboratorio:

\begin{enumerate}
	\item \textbf{Estimador de predicción}:
	\[
	\hat{x}[k+1] = A\,\hat{x}[k] + B\,u[k] + L_{\text{pred}}\,(y[k] - C\,\hat{x}[k]).
	\]
	\item \textbf{Estimador actual} (con la predicción \(z[k+1]\) que usás en tu código):
	\[
	\begin{aligned}
		z[k+1] &= A\,\hat{x}[k] + B\,u[k],\\
		\hat{x}[k+1] &= z[k+1] + L_{\text{act}}\,(y[k+1] - C\,z[k+1]).
	\end{aligned}
	\]
\end{enumerate}

Para comparar el efecto de la dinámica del observador se usaron dos ubicaciones en el plano-$z$:

\begin{itemize}
	\item caso 1 (moderado): \(p_{\text{obs}} = 0{,}4 \pm j\,0{,}4\);
	\item caso 2 (rápido): \(p_{\text{obs}} = 0{,}2 \pm j\,0{,}2\).
\end{itemize}

Las ganancias se obtuvieron por dualidad:

\begin{lstlisting}[style=matlabstyle,caption={Ganancias de los dos estimadores.},label={lst:ganancias_obs}]
	% Polos del observador (cambiar 0.4+0.4j <-> 0.2+0.2j según el caso)
	p_obs = [0.2+1j*0.2, 0.2-1j*0.2];
	
	% Estimador de PREDICCIÓN: corrige con y[k]
	L_pred = acker(A', C', p_obs).';
	
	% Estimador ACTUAL: corrige con y[k+1], usa C*A
	L_act  = acker(A', (C*A)', p_obs).';
\end{lstlisting}

La condición buscada fue que los polos del observador queden más cerca del origen que los polos del lazo cerrado \(A - BK\), de modo que el error de estimación desaparezca en pocas muestras y la dinámica dominante siga siendo la de los polos que se ubicaron para la planta.


\subsection{Simulación comparativa}

Con $K$, $N_{\text{bar}}$, $L_{\text{pred}}$ y $L_{\text{act}}$ se corrieron dos simulaciones sobre el mismo modelo discreto $(A,B,C,D,T_s)$: una con \texttt{sim\_predictor} y otra con \texttt{sim\_current}. En ambas se usó:
\begin{itemize}
	\item la misma referencia alternada;
	\item las mismas condiciones iniciales no nulas para $x[0]$ y $\hat{x}[0]$;
	\item y el mismo horizonte de tiempo.
\end{itemize}

\begin{lstlisting}[style=matlabstyle,caption={Simulación de ambas arquitecturas.},label={lst:simulaciones}]
	simPred = sim_predictor(A,B,C,D, K, L_pred, Nbar, Ts, Tsim, x0, xh0, r);
	simPred.title = "Estimador de prediccion";
	
	simCurr = sim_current(A,B,C,D, K, L_act,  Nbar, Ts, Tsim, x0, xh0, r);
	simCurr.title = "Estimador actual";
\end{lstlisting}


% ====== Comparación para polos del observador en 0.4 +/- 0.4j ======
\compararfigsCwide
{img/predictor04.png}{Predicción, $p_{\mathrm{obs}}=0{,}4\pm j0{,}4$}{fig:pred_04}
{img/actual04.png}{Actual, $p_{\mathrm{obs}}=0{,}4\pm j0{,}4$}{fig:act_04}
{Comparación de respuestas para estimador de predicción y actual con polos de observador moderados.}
{fig:comp_04}
{1}

% ====== Comparación para polos del observador en 0.2 +/- 0.2j ======
\compararfigsCwide
{img/estimador_predictivo.png}{Predicción, $p_{\mathrm{obs}}=0{,}2\pm j0{,}2$}{fig:pred_02}
{img/estimador_actual.png}{Actual, $p_{\mathrm{obs}}=0{,}2\pm j0{,}2$}{fig:act_02}
{Comparación de respuestas para estimador de predicción y actual con polos de observador rápidos.}
{fig:comp_02}
{1}

\subsection{Discusión de resultados}

De las cuatro simulaciones se observa que:
\begin{enumerate}
	\item Para una misma ubicación de polos del observador, el estimador actual converge en menos muestras que el estimador de predicción.
	\item Al acelerar los polos del observador (de $0{,}4\pm j0{,}4$ a $0{,}2\pm j0{,}2$), ambos estimadores se vuelven muy rápidos y la diferencia entre sus trayectorias de estado se reduce.
	\item Como la ley de control es la misma en todos los casos,
	\[
	u[k] = N_{\text{bar}} r[k] - K \hat{x}[k],
	\]
	la salida en lazo cerrado termina siendo prácticamente igual una vez que el estimador alcanza al estado real.
	\item La elección de polos del controlador en $0{,}8\pm j0{,}2$ permitió que las diferencias que se ven en las figuras se deban al observador y no a una saturación del actuador.
\end{enumerate}
\balance
\clearpage