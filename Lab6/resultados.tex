% =========================================================
\newpage
\section{Resultados}

En las figuras anteriores se aprecia que la implementación con un tiempo de muestreo insuficiente difiere notablemente de la respuesta esperada, mientras que tanto el observador \textit{predictivo} como el \textit{actual} logran un desempeño similar al de la simulación, aunque con un sobreimpulso mayor al teórico. 
Las discrepancias se explican principalmente por las tolerancias de los componentes pasivos, pequeñas desviaciones en las ganancias reales del sistema y efectos de cuantización de los conversores DAC/ADC.



\begin{table}[!t]
	\centering
	\caption{Comparación experimental y simulada del desempeño.}
	\label{tab:desempeno_exp}
	\begin{tabular}{
			l
			S
			S}
		\toprule
		\textbf{Caso} & {$t_s$ [ms]} & {OS [\%]} \\
		\midrule
		Predictor (Sim)        & 7.00  & 4.6  \\
		Predictor (clock bajo) & 3.58  & <10 \\
		Predictor (clock alto) & 6.52  & 18.6  \\
		Actual (Sim)           & 7.00  & 4.6  \\
		Actual (clock alto)    & 5.94  & $\approx10$ \\
		\bottomrule
	\end{tabular}
\end{table}

La tabla~\ref{tab:desempeno_exp} resume los valores medidos y simulados. 
Se observa que con el \emph{clock} bajo la respuesta dista mucho de la esperada, evidenciando la importancia del período de que se cumplan los supuestos en el diseño. 
En cambio, con el \emph{clock} alto, ambas variantes del observador mantienen una dinámica coherente con lo diseñado, aunque con un sobreimpulso mayor al simulado. 
El incremento del OS puede atribuirse a las tolerancias de resistencias y capacitores, a la dispersión de los polos reales respecto a los teóricos y a las no idealidades del hardware. 

En conjunto, los resultados confirman que ambos esquemas de estimador son implementables y que \textbf{la elección adecuada del $T_s$ y del reloj interno del PSoC es tan determinante como la ubicación de polos} para lograr un control estable y preciso.
\balance



\onecolumn

\compararfigsC
{img/predictor_feo.JPG}{Estimador de predicción (clock original).}{fig:resp_pred_clk0}
{img/predictor.JPG}{Estimador predicción (clock aumentado).}{fig:resp_act_clk0}
{Respuestas experimentales cambiando el clock.}
{fig:cmp1}



\compararfigsC
{img/predictor.JPG}{Estimador de predicción (clock aumentado).}{fig:resp_pred_clk1}
{img/actual.JPG}{Estimador actual (clock aumentado).}{fig:resp_act_clk1}
{Respuestas experimentales con el clock aumentado.}
{fig:cmp2}

\compararfigsC
{img/predictor.JPG}{Estimador de predicción (clock aumentado).}{fig:resp_pred_clk1}
{img/estimador_predictivo.png}{Estimador de predicción (simulación).}{fig:resp_act_clk1}
{Respuesta experimental con el clock aumentado vs simulación.}
{fig:cmp3}

\compararfigsC
{img/actual.JPG}{Estimador de actual (clock aumentado).}{fig:resp_pred_clk1}
{img/estimador_actual.png}{Estimador de actual (simulación).}{fig:resp_act_clk1}
{Respuesta experimental con el clock aumentado vs simulación.}
{fig:cmp4}