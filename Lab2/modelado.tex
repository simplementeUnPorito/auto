\subsection{Modelado del Sistema}
\subsubsection{Obtención de la función de transferencia en lazo abierto}

La planta puede interpretarse como la conexión en cascada de dos filtros activos de primer orden. Cada uno posee la misma topología: un amplificador operacional en configuración inversora cuya impedancia de realimentación está compuesta por una resistencia en paralelo con un capacitor.

\paragraph{Impedancia de realimentación.}  
Para el paralelo $R_f \parallel C_f$, se obtiene:
\begin{equation}
	Z_f \;=\; R_f \parallel \frac{1}{sC_f}
	\;=\; \frac{R_f}{1 + s R_f C_f}
	\label{eq:Zf}
\end{equation}


El sistema trabaja sobre una tensión de referencia en continua $V_{\text{cc}}/2 = 2.5$ V. En los cálculos posteriores se toma dicho valor como punto de referencia.

\paragraph{Ganancia de una etapa (entrada inversora).}  
Con $V_{\text{ref}}=0$, se cumple el cortocircuito virtual ($V_p = V_n = 0$). Aplicando KCL en el nodo inversor:
\[
\frac{V_i}{R_i} \;=\; \frac{-V_o}{Z_f}
\quad\Rightarrow\quad
\frac{V_o}{V_i} \;=\; -\,\frac{Z_f}{R_i}
\]
y reemplazando \eqref{eq:Zf}:
\begin{equation}
	\frac{V_o}{V_i}\Bigg|_{V_{\text{ref}}=0}
	= -\,\frac{R_f}{R_i}\,\frac{1}{1+sR_f C_f}
	\label{eq:gain_inverting}
\end{equation}

\paragraph{Encadenamiento de etapas.}  
Como ambas etapas AO1 y AO2 responden a la forma \eqref{eq:gain_inverting}, la ganancia total en lazo abierto resulta del producto de sus transferencias:
\[
G_{\text{ol}}(s)
= \Big(-\frac{Z_{f1}(s)}{R_{i1}}\Big)\,
\Big(-\frac{Z_{f2}(s)}{R_{i2}}\Big)
\]

\paragraph{Implementación en Matlab.}  
El siguiente código genera cada etapa de primer orden, calcula la función en lazo abierto y extrae información temporal de la respuesta al escalón:




\paragraph{Resultados numéricos.}  
A partir de la respuesta de la figura~\ref{fig:escalonLazoab} se obtuvieron:
\begin{itemize}
	\item \textbf{Tiempo de subida:} $t_r \approx 0.0332$ s (33.2 ms)
	\item \textbf{Tiempo de establecimiento (2\%):} $t_s \approx 0.0603$ s (60.3 ms)
	\item \textbf{Frecuencia natural estimada:} $\omega_n \approx 54.2$ rad/s
\end{itemize}




