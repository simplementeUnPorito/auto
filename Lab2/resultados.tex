\onecolumn
\subsection{Resultados}
\begin{table}[H]
\centering
\caption{Comparación entre cálculos y mediciones}
\begin{tabular}{|l|l|l|l|}
	\hline
	\textbf{Compensador} & \textbf{Overshoot (\%)} & \textbf{$t_r$ (ms)} & \textbf{$\zeta$} \\ \hline
	Compensador 1 (primer modelo) - Cálculo  & 4.79 & 11.21 & 0.69 \\ \hline
	Compensador 1 (primer modelo) - Medición & NV   & 5     & $>0.6$ \\ \hline
	Compensador 3 (primer modelo) - Cálculo  & 4.62 & 11.21 & 0.69 \\ \hline
	Compensador 3 (primer modelo) - Medición (polo en 1)   & 300  & NV    & NE \\ \hline
	Compensador 3 (primer modelo) - Medición (polo en 0.9) & 25   & 10    & 0.404 \\ \hline
	Compensador 1 (segundo modelo) - Cálculo & 5    & 10.4  & 0.70 \\ \hline
	Compensador 1 (segundo modelo) - Medición& 5    & 7     & 0.69 \\ \hline
	Compensador 2 (segundo modelo) - Cálculo & 5    & 13.6  & 0.70 \\ \hline
	Compensador 2 (segundo modelo) - Medición& 5    & 8     & 0.69 \\ \hline
\end{tabular}
\end{table}

\subsection{Conclusión}

El diseño de los compensadores, si bien logró estabilizar el sistema, no reprodujo exactamente la respuesta esperada según los cálculos iniciales. En particular, los valores de $\zeta$ y del tiempo de subida no coincidieron con lo estimado teóricamente. Esto se debió principalmente a un modelado poco preciso de la planta, sumado al uso de valores nominales de resistencias y capacitores, lo cual introdujo un error significativo en la implementación práctica.  

Además, se observó que al ubicar un polo cercano a $z=1$ la respuesta tiende a desestabilizarse; sin embargo, con un ajuste fino de su posición se logró mejorar el desempeño. Este efecto puede atribuirse también a la incertidumbre en el muestreo y a las pequeñas variaciones en el \textit{clock}, que desplazan el polo fuera del círculo unitario.  

Finalmente, al emplear un modelo más preciso de la planta, sin necesidad de modificar de manera considerable los valores de $K$, polos ni ceros, se obtuvieron respuestas muy cercanas a las calculadas en Matlab. Esto demuestra la importancia de un modelado adecuado como base para una correcta implementación del control en hardware.





\balance