\section{Conclusión}
\begin{table}[H]
	\centering
	
	\resizebox{\linewidth}{!}{
		\begin{tabular}{c|c|c|c}
			\hline
			\textbf{Método} &
			\textbf{$T_{ref}$ [ms]} &
			\textbf{Overshoot [\%]} &
			\textbf{Oscilaciones} \\ 
			\hline
			M1 &
			\begin{tabular}[c]{@{}c@{}}Sim: 4.173 \\ Med: 5.6 \\ $\Delta$: 34.196\%\end{tabular} &
			\begin{tabular}[c]{@{}c@{}}Sim: 25 \\ Med: $\approx$ 20 \\ $\Delta$: $\approx$ 20\%\end{tabular} &
			\begin{tabular}[c]{@{}c@{}}Sim: Sí \\ Med: Sí\end{tabular} \\ 
			\hline
			M2 &
			\begin{tabular}[c]{@{}c@{}}Sim: 8.346 \\ Med: 8.4 \\ $\Delta$: 0.647\%\end{tabular} &
			\begin{tabular}[c]{@{}c@{}}Sim: 0 \\ Med: $<$10\% \\ $\Delta$: $<$10\%\end{tabular} &
			\begin{tabular}[c]{@{}c@{}}Sim: No \\ Med: No\end{tabular} \\ 
			\hline
		\end{tabular}
	}
	\vline
	\caption{Comparación entre simulación y medición para ambos métodos}
	\label{tab:comparacion}
\end{table}
En síntesis, ambos métodos de diseño mostraron un desempeño adecuado tanto en simulación como en la implementación experimental.  

No obstante, este tipo de controladores \textit{analíticos} —tanto el convencional como el \textit{deadbeat}— dependen fuertemente de la precisión del modelo de la planta.  
En la práctica, los parámetros del sistema pueden variar con la temperatura, el envejecimiento de los componentes o pequeñas no linealidades no contempladas, lo que puede afectar la estabilidad y el desempeño del control.  

Si bien los resultados obtenidos validan la teoría en condiciones controladas, su aplicación práctica requiere cierto margen de robustez para adaptarse a variaciones del sistema real.
