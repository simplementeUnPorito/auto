\section{Diseño del controlador.}

\subsection{Controlador con mínimo tiempo de establecimiento}

Se diseñaron los controladores para los distintos tiempos de muestreo seleccionados, con el objetivo de alcanzar un tiempo de establecimiento mínimo. Este diseño parte de la expresión de la función de transferencia en lazo cerrado:

\begin{equation}
	G_{cl}(z) = \frac{C(z) \, G_{ZAS}(z)}{1 + C(z) \, G_{ZAS}(z)}
	\label{eq:closedloop}
\end{equation}

Despejando la ecuación \eqref{eq:closedloop} en función del controlador $C(z)$, se obtiene:

\begin{equation}
	C(z) = \frac{1}{G_{ZAS}(z)} \cdot \frac{G_{cl}(z)}{1 - G_{cl}(z)}
	\label{eq:controladorsynt}
\end{equation}

En particular, para el diseño de un controlador \emph{dead-beat}, se considera que la función de transferencia en lazo cerrado es $G_{cl}(z) = z^{-1}$. Sustituyendo en \eqref{eq:controladorsynt}, la expresión del controlador resulta:

\begin{equation}
	C(z) = \frac{1}{G_{ZAS}(z)} \cdot \frac{1}{z - 1}
	\label{eq:control1}
\end{equation}

Con los tiempos de muestreo seleccionados, se calculan a continuación los controladores correspondientes con el fin de analizar y comparar su desempeño.


\subsection*{Tiempo de muestreo $T_s = 1.25ms$}
\[
G(z) = \frac{45.999  \, (z - 0.9331)(z - 0.4346)}{(z - 1)(z + 0.7411)}
\]

\insertarfigura{./Img/M1/M1_Ts_125.png}{Controlador con tiempo de muestreo $T=1.25\, ms$.}{fig:m1tn125}{1}

\subsection*{Tiempo de muestreo $T_s = 0.0043 \, \text{s}$}
\[
G(z) = \frac{6.7639 \, (z - 0.7859)(z - 0.05515)}{(z - 1)(z + 0.3684)}
\]
\insertarfigura{./Img/M1/M1_Ts_Tn32.png}{Controlador con tiempo de muestreo $T=0.0043\, s$.}{fig:m1tn32}{1}
\subsection*{Tiempo de muestreo $T_s = 0.0087 \, \text{s}$}
\[
G(z) = \frac{3.0614 \, (z - 0.6176)(z - 0.003041)}{(z - 1)(z + 0.1671)}
\]
\insertarfigura{./Img/M1/M1_Ts_Tn16.png}{Controlador con tiempo de muestreo $T=0.0087\, s$.}{fig:m1tn16}{1}
\subsection*{Tiempo de muestreo $T_s = 0.0174 \, \text{s}$}
\[
G(z) = \frac{1.7125 \, (z - 0.3815)(z - 9.249 \times 10^{-6})}{(z - 1)(z + 0.05923)}
\]
\insertarfigura{./Img/M1/M1_Ts_Tn8.png}{Controlador con tiempo de muestreo $T=0.0174\, s$.}{fig:m1tn8}{1}

\subsection*{Tiempo de muestreo $T_s = 0.0348 \, \text{s}$}
\[
G(z) = \frac{1.1886 \, z(z - 0.1455)}{(z - 1)(z + 0.01568)}
\]
\insertarfigura{./Img/M1/M1_Ts_Tn4.png}{Controlador con tiempo de muestreo $T=0.0348\, s$.}{fig:m1tn4}{1}

\subsection*{Tiempo de muestreo $T_s = 0.0695 \, \text{s}$}
\[
G(z) = \frac{0.97691 \, (z + 0.001965)}{z \, (z - 0.02117)}
\]
\insertarfigura{./Img/M1/M1_Ts_Tn2.png}{Controlador con tiempo de muestreo $T=0.0695\, s$.}{fig:m1tn2}{1}

Se puede observar que, a medida que el muestreo es más lento, la respuesta del sistema mejora: la oscilación disminuye y la amplitud del esfuerzo de control se mantiene dentro de un rango aceptable (véase en las Figuras~\ref{fig:m1tn125} y~\ref{fig:m1tn2}).




