
\twocolumn
\section{Implementación y comparación experimental}


% ======= COMPARAR FIGURAS MÉTODO 1 ========
\compararfigsC{Img/M1/exp/ex1_239Hz_ref1_1.5.png}
{Implementación experimental del método 1.}
{fig:exp_m1}
{Img/M1/sim/M1_Ts_Tn4173.png}
{Simulación del método 1 ($T_s = 4.173\,\text{ms}$).}
{fig:sim_m1}
{Comparación entre la simulación y la implementación experimental para el compensador del método 1.}
{fig:comp_m1}

% ======= COMPARAR FIGURAS MÉTODO 2 ========
\compararfigsC{Img/M2/exp/ex2_239Hz_ref1_1.5.png}
{Implementación experimental del método 2.}
{fig:exp_m2}
{Img/M2/sim/M2_Ts_4173.png}
{Simulación del método 2 ($T_s = 4.173\,\text{ms}$).}
{fig:sim_m2}
{Comparación entre la simulación y la implementación experimental para el compensador del método 2.}
{fig:comp_m2}

Para la implementación en el PSoC, se seleccionaron los controladores diseñados que presentaban el menor esfuerzo de control, de modo que su salida fuera físicamente realizable sin saturar el actuador.  
Si bien los controladores con tiempos de muestreo más pequeños ofrecen una respuesta más rápida, el esfuerzo de control resultante continúa siendo elevado, lo que limita su viabilidad práctica.

Por esta razón, durante las pruebas experimentales se utilizó una referencia de pequeña amplitud, lo que incrementó la relación ruido–señal. Aun así, los resultados fueron satisfactorios, mostrando un seguimiento adecuado y tiempos de establecimiento coherentes con los obtenidos en simulación.

En las Figuras~\ref{fig:comp_m1} y~\ref{fig:comp_m2} se presentan las comparaciones entre las simulaciones y los resultados experimentales para ambos métodos de diseño de controladores.  
Las imágenes experimentales corresponden a una referencia de \(1.5\,\text{V}\) a \(239\,\text{Hz}\), cuyos valores se indican también en el nombre de cada archivo.




Se observa que el primer compensador (método~1) alcanza la referencia en aproximadamente un periodo de muestreo, tal como se había previsto teóricamente, aunque con la presencia de oscilaciones y un leve sobreimpulso, comportamientos también esperados.  
En cambio, el compensador del método~2 alcanza la estabilidad en torno a los dos periodos de muestreo, presentando una respuesta más suave y sin oscilaciones notorias, aunque con un esfuerzo de control considerablemente mayor y un pequeño sobrepasamiento no observado en las simulaciones.

Al analizar los cursores mostrados en las imágenes experimentales, se verifica que los tiempos de establecimiento concuerdan con los valores estimados en simulación. En ambos casos, la respuesta del sistema puede considerarse satisfactoria, logrando un equilibrio adecuado entre rapidez, estabilidad y esfuerzo de control.


