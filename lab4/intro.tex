\section{Diseño de Controlador por el Método de Truxal--Ragazzini}

A partir de la planta analógica mostrada en la Figura~\ref{fig:planta}, se desarrollaron las siguientes etapas de diseño:


% ================== INTRODUCCIÓN ==================
\section{Introducción}

 En este laboratorio se aborda el diseño de un controlador digital para una planta analógica, utilizando la técnica analítica de Truxal--Ragazzini. Este método ofrece un enfoque directo para determinar el controlador que garantiza un comportamiento dinámico deseado en el sistema en lazo cerrado.

El trabajo se centra en tres ejes principales: en primer lugar, se discretiza la función de transferencia de la planta para distintos tiempos de muestreo seleccionados bajo criterios de ingeniería; en segundo lugar, se diseña el controlador digital en dos versiones, una que permite oscilaciones entre muestras y otra de tipo \emph{dead-beat}, que elimina dichas oscilaciones; finalmente, se implementa la ecuación en diferencias del controlador en el PSoC y se contrastan los resultados experimentales con los obtenidos en simulación.

De esta manera, se busca cumplir con los objetivos planteados: diseñar un controlador digital por el método de Truxal--Ragazzini empleando Matlab, implementar su ecuación en diferencias en el PSoC, presentar los resultados obtenidos y realizar una discusión crítica de las diferencias y similitudes respecto a las simulaciones.



% ================== OBJETIVOS ==================
\section{Objetivos}
\begin{itemize}
	\item Diseño de un controlador digital para una planta analógica por el método de Truxal--Ragazzini utilizando Matlab.
	\item Implementar la ecuación en diferencias del controlador utilizando el PSoC.
	\item Presentar los resultados obtenidos y compararlos con los obtenidos por simulación.
	\item Discutir los resultados. 
\end{itemize}


% ================== MATERIALES ==================
\section{Materiales}
\begin{itemize}
	\item PC con Matlab.
	\item Planta analógica.
	\item Sistema de adquisición en PSoC.
\end{itemize}

% ================== TEORÍA ==================
\section{Teoría}
Se recomienda consultar la referencia: \\
M. Fadali, A. Visioli, \textit{Digital Control Engineering – Analysis and Design}, Sección 6.6., donde se desarrollan las bases teóricas necesarias para el diseño de compensadores mediante el diagrama de Bode y la evaluación de estabilidad en sistemas discretos.

\begin{enumerate}
	
	\item \textbf{Selección del tiempo de muestreo.}  
	Se eligen al menos dos valores de período de muestreo $T_s$, fundamentados en criterios de ingeniería (relación con frecuencia natural de la planta, tiempo de establecimiento esperado y precisión deseada).
	
	\item \textbf{Discretización de la planta.}  
	Se obtiene la función de transferencia discreta $G(z)$ para cada tiempo de muestreo seleccionado mediante la transformación adecuada.
	
	\item \textbf{Diseño del controlador.}  
	Aplicando el método de Truxal--Ragazzini, se construyen dos versiones del controlador:
	\begin{enumerate}
		\item \emph{Controlador con oscilaciones entre muestras.}  
		Diseño considerando únicamente el tiempo mínimo de establecimiento.
		
		\item \emph{Controlador Dead-Beat.}  
		Diseño aplicando las restricciones necesarias para evitar oscilaciones entre muestras.
	\end{enumerate}
	
	\item \textbf{Simulación y resultados.}  
	Se presentan las respuestas obtenidas para ambos diseños, evaluando métricas de desempeño temporal.
	
	\item \textbf{Discusión comparativa.}  
	Se comparan los resultados obtenidos experimentalmente y por simulación, discutiendo las ventajas, limitaciones y diferencias entre ambos enfoques de control.
\end{enumerate}
