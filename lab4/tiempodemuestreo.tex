\section{Selección de un tiempo de muestreo adecuado}

La selección de los tiempos de muestreo se realizó a partir del criterio de Nyquist. Se determinó que el tiempo crítico es $T_{n} = 139 \, \text{ms}$. A partir de este valor, se estableció un margen que permite definir intervalos de muestreo aceptables para el diseño del controlador.\\
Se tiene la siguiente ecuación, el tiempo de muestreo utilizado para este laboratorio será el mismo que el del laboratorio 2, $T = 1.25ms$, $a = \tfrac{1}{81\,000 \cdot 200 \cdot 10^{-9}}$ y $b = \tfrac{1}{15 \cdot 10^{3} \cdot 100 \cdot 10^{-9}}$, y considerando $K = 1$ se tienen ya calculados los valores de $\alpha$, $\beta$ y $\delta$:
\begin{flalign*}
	\delta &= \frac{K}{ab} \approx 24.3 \times 10^{-6} & \\
	\beta &= \frac{K}{b^2-ab} \approx 2.48 \times 10^{-6} & \\
	\alpha &= \frac{K}{a^2-ab} \approx -26.78 \times 10^{-6}.&
\end{flalign*}
