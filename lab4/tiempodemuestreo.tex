\section{Selección de un tiempo de muestreo adecuado}

La selección de los tiempos de muestreo se realizó a partir del criterio de Nyquist $T_n=\frac{2\pi}{w_n}$. Se determinó que el tiempo crítico es $T_{n} = 139 \, \text{ms}$. A partir de este valor, se estableció un margen que permite definir intervalos de muestreo aceptables para el diseño del controlador.  

Como punto de partida, se consideró el tiempo de muestreo utilizado en otros laboratorios, $T = 1.25 \, \text{ms}$. Con base en ello, se seleccionaron los siguientes valores de muestreo:  
\[
T = 0.0043 \, \text{s}, \quad 0.0087 \, \text{s}, \quad 0.0174 \, \text{s}, \quad 0.0348 \, \text{s}, \quad 0.0695 \, \text{s}.
\]
