% !TeX program = pdflatex
\documentclass[conference]{IEEEtran}
\pagestyle{plain}

% ======= Idioma y codificación =======
\usepackage[spanish, es-nodecimaldot]{babel}
\usepackage[utf8]{inputenc}
\usepackage[T1]{fontenc}

% ======= Bibliografía (IEEE + natbib numérico) =======
\usepackage[numbers]{natbib}
\bibliographystyle{IEEEtranN}

% ======= Matemática y tipografía =======
\usepackage{amsmath, amssymb, amsfonts}
\usepackage{microtype}
\usepackage{float}           % (déjalo una sola vez)

% ======= Números y unidades =======
\usepackage{siunitx}
\sisetup{
	output-decimal-marker = {,}, % coma decimal
	group-separator = {.},
	group-minimum-digits = 4,
	per-mode = symbol,
	round-mode = places,
	round-precision = 3,
	table-number-alignment = center
}

% ======= Gráficos y floats =======
\usepackage{graphicx}
% \usepackage{subcaption}   % <-- QUITAR esto en IEEEtran
\usepackage[caption=false,font=footnotesize]{subfig} % <-- USAR ESTO
\usepackage{booktabs}
\usepackage{cuted}
\usepackage{balance}

% ======= Listas =======
\usepackage{enumitem}
\setlist{leftmargin=*, itemsep=2pt, topsep=4pt}

% ======= Colores (para listings) =======
\usepackage{xcolor}

% ======= LISTINGS =======
\usepackage{listings}
\usepackage{listingsutf8}    % ok con pdfLaTeX
\lstset{
	inputencoding=utf8,
	extendedchars=true,
	basicstyle=\ttfamily\small,
	breaklines=true, columns=fullflexible, keepspaces=true,
	frame=single, numbers=left, numberstyle=\tiny, numbersep=6pt,
	xleftmargin=1em,
	captionpos=b,
	literate=
	{á}{{\'a}}1 {é}{{\'e}}1 {í}{{\'\i}}1 {ó}{{\'o}}1 {ú}{{\'u}}1
	{Á}{{\'A}}1 {É}{{\'E}}1 {Í}{{\'I}}1 {Ó}{{\'O}}1 {Ú}{{\'U}}1
	{ñ}{{\~n}}1 {Ñ}{{\~N}}1
	{ü}{{\"u}}1 {Ü}{{\"U}}1
	{¿}{{\textquestiondown}}1 {¡}{{\textexclamdown}}1
	{º}{{$^\circ$}}1
}

% Lenguajes
\lstdefinelanguage{Matlab}{
	morekeywords={break,case,catch,continue,else,elseif,end,for,function,global,if,otherwise,persistent,return,switch,try,while},
	sensitive=true,
	morecomment=[l]\%,
	morestring=[m]'
}
\lstdefinestyle{matlabstyle}{
	language=Matlab,
	basicstyle=\ttfamily\small,
	keywordstyle=\bfseries\color{blue!70!black},
	commentstyle=\itshape\color{green!40!black},
	stringstyle=\color{red!60!black},
	numbers=left, numberstyle=\tiny, numbersep=6pt,
	frame=single, framerule=0.4pt,
	columns=fullflexible, keepspaces=true,
	showstringspaces=false,
	breaklines=true,
	tabsize=4,
	captionpos=b,
	linewidth=\columnwidth
}

\lstdefinestyle{cstyle}{
	language=C,
	inputencoding=utf8,
	basicstyle=\ttfamily\small,
	keywordstyle=\color{blue}\bfseries,
	stringstyle=\color{red!60!black},
	commentstyle=\color{green!40!black}\itshape,
	numbers=left, numberstyle=\tiny, stepnumber=1, numbersep=8pt,
	showstringspaces=false, breaklines=true, columns=fullflexible, keepspaces=true,
	frame=single, rulecolor=\color{black!20},
	linewidth=\columnwidth, xleftmargin=1em
}

% ======= Macros rápidas =======
\newcommand{\zetaD}{\zeta}
\newcommand{\wn}{\omega_n}
\newcommand{\Ts}{T_s}
\newcommand{\ESS}{\mathrm{ESS}}
\newcommand{\Gz}{G(z)}
\newcommand{\Gs}{G(s)}
\lstset{
	mathescape=false,  % asegura que $ no abre modo matemático dentro del listing
	upquote=true       % comillas rectas en verbatim
}

% ======= COMPARAR FIGURAS =======
\newcommand{\compararfigsC}[8]{%
	\begin{figure}[H]
		\centering
		\subfloat[#2\label{#3}]{%
			\includegraphics[width=0.49\columnwidth]{#1}}
		\hspace{0.3em}
		\subfloat[#5\label{#6}]{%
			\includegraphics[width=0.49\columnwidth]{#4}}
		\caption{#7}
		\label{#8}
	\end{figure}%
}

% Macro para figura simple
\newcommand{\insertarfigura}[4]{%
	\begin{figure}[H]
		\centering
		\includegraphics[width=#4\linewidth]{#1}
		\caption{#2}
		\label{#3}
	\end{figure}
}

% Macro figura a dos columnas
\newcommand{\insertarfigurawide}[5][]{%
	\begin{figure*}[#1]
		\centering
		\includegraphics[width=#5\textwidth]{#2}
		\caption{#3}
		\label{#4}
	\end{figure*}%
}

% ======= Hyperref al final del preámbulo =======
\usepackage[unicode,hidelinks]{hyperref}


% ======= Metadatos =======
\title{Práctica de Laboratorio 4: Diseño de controladores por síntesis directa.}

\author{
	\IEEEauthorblockN{Elías Álvarez}
	\IEEEauthorblockA{Carrera de Ing. Electrónica \\ 
		Universidad Católica Nuestra Señora de la Asunción \\
		Asunción, Paraguay \\
		Email: elias.alvarez@universidadcatolica.edu.py}
	\and
	\IEEEauthorblockN{Tania Romero}
	\IEEEauthorblockA{Carrera de Ing. Electrónica \\
		Universidad Católica Nuestra Señora de la Asunción \\
		Asunción, Paraguay \\
		Email: tania.romero@universidadcatolica.edu.py}
	\and
	\IEEEauthorblockN{\hspace*{2.3em}Docente: Lic. Montserrat González}
	\IEEEauthorblockA{
		\hspace*{2.3em}Facultad de Ingeniería \\
		\hspace*{2.3em}Universidad Católica Nuestra Señora de la Asunción \\
		\hspace*{2.3em}Asunción, Paraguay}
	\and
	\IEEEauthorblockN{\hspace*{2.3em}Docente: PhD. Enrique Vargas}
	\IEEEauthorblockA{\hspace*{2.3em}Facultad de Ingeniería \\  % ← empuja a la derecha
		\hspace*{2.3em}Universidad Católica Nuestra Señora de la Asunción \\
		\hspace*{2.3em}Asunción, Paraguay}
}







\begin{document}
	\maketitle
	
	% ======= Contenido =======
	\section{Introducción}


\section{Objetivos}

\begin{itemize}
	\item Diseñar controladores digitales para una planta analógica utilizando estados estimados.
	\item Aplicar el diseño de realimentación de estados con estimadores de predicción y de actualización.
	\item Presentar y discutir los resultados experimentales en comparación con los obtenidos por simulación, dentro de un informe técnico razonado.
\end{itemize}

	\section{Desarrollo}

\subsection{Modelado del Sistema}

\subsubsection{Obtención de las matrices del sistema $F$, $G$, $H$ y $J$}

Para el modelado del sistema se parte del circuito mostrado en la Figura~\ref{fig:circuit_planta}, a partir del cual se determinan las ecuaciones de estado mediante el análisis de los lazos de realimentación y las relaciones de tensión en los componentes.

\insertarfigura{Otros/circuit.png}{Circuito de la planta.}{fig:circuit_planta}{1}

El sistema se describe mediante las siguientes ecuaciones en espacio de estados:

\begin{equation}
	\dot{x}(t) = F\,x(t) + G\,V_i(t)
	\label{eq:1}
\end{equation}

\begin{equation}
	y(t) = H\,x(t) + J\,V_i(t)
	\label{eq:2}
\end{equation}

Para obtener las expresiones de las variables de estado, se parte del equivalente del paralelo entre un resistor y un capacitor:
\[
R \parallel \frac{1}{sC} = \frac{R}{1 + sRC}
\]

Considerando que ambos amplificadores operacionales se encuentran en configuración no inversora, se obtienen las siguientes relaciones:

\[
V_a = \frac{-R_2}{1 + sR_2C_1}\frac{V_i}{R_1}
\quad \Rightarrow \quad
sV_a = -\frac{1}{R_1C_1}V_i - \frac{1}{R_2C_1}V_a
\]

\[
V_o = -\frac{R_4}{1 + sR_4C_2}\frac{V_a}{R_3}
\quad \Rightarrow \quad
sV_o = -\frac{1}{R_3C_2}V_a - \frac{1}{R_4C_2}V_o
\]

Definiendo como variables de estado $x_1(t) = V_a$ y $x_2(t) = V_o$, las ecuaciones anteriores se expresan en forma matricial como:

\[
\begin{bmatrix}
	\dot{x}_1(t) \\[4pt]
	\dot{x}_2(t)
\end{bmatrix}
=
\begin{bmatrix}
	-\dfrac{1}{R_2C_1} & 0 \\[4pt]
	-\dfrac{1}{R_3C_2} & -\dfrac{1}{R_4C_2}
\end{bmatrix}
\begin{bmatrix}
	x_1(t) \\[4pt]
	x_2(t)
\end{bmatrix}
+
\begin{bmatrix}
	-\dfrac{1}{R_1C_1} \\[4pt]
	0
\end{bmatrix}
V_i(t)
\]

y la ecuación de salida queda definida como:

\[
y(t) =
\begin{bmatrix}
	0 & 1
\end{bmatrix}
\begin{bmatrix}
	x_1(t) \\[4pt]
	x_2(t)
\end{bmatrix}
+ 0\cdot V_i(t)
\]

Sustituyendo los valores de los componentes 
$C_1 = 211.1\times10^{-9}\,\text{F}$, 
$R_1 = 80.55\times10^{3}\,\Omega$, 
$R_2 = 81.09\times10^{3}\,\Omega$, 
$C_2 = 103.07\times10^{-9}\,\text{F}$, 
$R_3 = 14.878\times10^{3}\,\Omega$ y 
$R_4 = 14.76\times10^{3}\,\Omega$, 
se obtienen las siguientes matrices numéricas:

\[
F =
\begin{bmatrix}
	-58.42 & 0 \\[4pt]
	-652.11 & -657.37
\end{bmatrix}, \quad
G =
\begin{bmatrix}
	-58.81 \\[4pt]
	0
\end{bmatrix}, \quad
H =
\begin{bmatrix}
	0 & 1
\end{bmatrix}
\]
\begin{equation}
	y \quad
	J = 0
	\label{eq:J}
\end{equation}
\subsubsection{Mostrar el diagrama de bloques del sistema}


\subsection{Discretización del Sistema}
\insertarfigura{Otros/Diagramas1.png}{Diagrama de bloques del sistema continuo.}{fig:diag_continuo}{1}

\subsubsection{Elección del tiempo de muestreo $T_s = 1~\text{ms}$}

Para la discretización del sistema continuo descrito por las ecuaciones~(\ref{eq:1}) y~(\ref{eq:2}), se busca obtener un modelo equivalente en tiempo discreto que relacione las variables de estado y la señal de entrada en instantes de muestreo definidos.  
Las ecuaciones del sistema discreto se expresan como:

\begin{equation}
	X(k+1) = A\,X(k) + B\,u(k)
	\label{eq:3}
\end{equation}

\begin{equation}
	Y(k) = C\,X(k) + D\,u(k)
	\label{eq:4}
\end{equation}

Usando las matrices continuas $F$, $G$, $H$ y $J$ obtenidas previamente, las matrices discretas se determinan mediante las siguientes expresiones:

\[
A = \mathrm{e}^{F T_s}, \qquad 
B = F^{-1}\!\left(\mathrm{e}^{F T_s} - I\right)G, \qquad 
C = H \qquad 
\]

\[
 \& \quad D = J
\]
\subsubsection{Obtención de las matrices discretas $A$, $B$, $C$ y $D$}

Con un tiempo de muestreo $T_s = 1~\text{ms}$, se obtienen las siguientes matrices discretizadas:

\[
A =
\begin{bmatrix}
	0.943 & 0 \\[4pt]
	-0.462 & 0.518
\end{bmatrix}, \qquad
B =
\begin{bmatrix}
	-0.0571 \\[4pt]
	0.0152
\end{bmatrix}, 
\]
\begin{equation}
	C =
	\begin{bmatrix}
		0 & 1
	\end{bmatrix}, \qquad \& \qquad
	D = 0
	\label{eq:d}
\end{equation}

Estas matrices representan el modelo digital equivalente del sistema continuo, y serán utilizadas posteriormente para el diseño del controlador e implementación en el \texttt{PSoC}.

\subsubsection{Diagrama de bloques del sistema discretizado}

En la Figura~\ref{fig:diag_discreto} se presenta el diagrama de bloques correspondiente al sistema discretizado, donde se observa la relación entre las variables de estado, la entrada $u(k)$ y la salida $Y(k)$.

\insertarfigura{Otros/Diagramas2.png}{Diagrama de bloques del sistema discretizado.}{fig:diag_discreto}{1}


\subsubsection{Verificación de la controlabilidad y observabilidad del sistema}

La matriz de controlabilidad se obtiene a partir de la siguiente relación general:

\[
x(n) - A^{n}x(0) = \sum_{i=0}^{n-1} A^{n-i-1}B\,u(i)
\]

lo que lleva a la siguiente forma matricial:

\[
x(n) - A^{n}x(0) =
\begin{bmatrix}
	A^{n-1}B & A^{n-2}B & \cdots & AB & B
\end{bmatrix}
\begin{bmatrix}
	u(0) \\[2pt]
	u(1) \\[2pt]
	\vdots \\[2pt]
	u(n-1)
\end{bmatrix}
\]

De esta expresión, se define la matriz de controlabilidad como:

\[
\mathcal{C} =
\begin{bmatrix}
	B & AB
\end{bmatrix}
\]

Para el sistema analizado, la matriz resultante es:

\[
\mathcal{C} =
\begin{bmatrix}
	-0.05344 & -0.05713 \\[4pt]
	0.03435 & 0.01527
\end{bmatrix}
\]

El rango de esta matriz es $n = 2$, lo que indica que el sistema es completamente controlable.

\bigskip
La matriz de observabilidad se obtiene a partir de la expresión general:

\[
Y(n-1) = 
\begin{bmatrix}
	C \\[4pt]
	CA \\[4pt]
	\vdots \\[4pt]
	CA^{n-1}
\end{bmatrix}
X(0)
\]

Por lo tanto, la matriz de observabilidad queda definida como:

\[
\mathcal{O} =
\begin{bmatrix}
	C \\[4pt]
	CA
\end{bmatrix}
\]

Sustituyendo los valores del sistema:

\[
\mathcal{O} =
\begin{bmatrix}
	0 & 1 \\[4pt]
	-0.462 & 0.518
\end{bmatrix}
\]

El rango de la matriz de observabilidad también resulta ser $n = 2$.\\  
Por lo tanto, se concluye que el sistema es **completamente controlable y observable**, cumpliendo con las condiciones necesarias para el diseño de control mediante realimentación de estados.

\subsubsection{Comparar los resultados obtenidos con las simulaciones realizadas en \texttt{MATLAB}}
\textbf{Resultados de las matrices continuas:}
\[
F =
\begin{bmatrix}
	-58.42 & 0 \\[4pt]
	-652.10 & -657.30
\end{bmatrix}, \quad
G =
\begin{bmatrix}
	-58.81 \\[4pt]
	0
\end{bmatrix}, \quad
H =
\begin{bmatrix}
	0 & 1
\end{bmatrix}
\]

\[
	\& \quad
	J = [\,0\,]
\]

\textbf{Resultados de las matrices discretas:}
\[
A =
\begin{bmatrix}
	0.9433 & 0 \\[4pt]
	-0.4628 & 0.5182
\end{bmatrix}, \quad
B =
\begin{bmatrix}
	-0.05712 \\[4pt]
	0.01527
\end{bmatrix}, \quad
C =
\begin{bmatrix}
	0 & 1
\end{bmatrix}
\]

\[
	\& \quad
	D = [\,0\,]
\]
Se observa que los resultados de las matrices obtenidas son congruentes con los valores calculados en las ecuaciones~(\ref{eq:J}) y~(\ref{eq:d}), verificando la coherencia entre el modelo teórico y los resultados obtenidos mediante \texttt{MATLAB}.

\insertarfigurawide{matlab/Sim1_noImplementable.png}{Respuesta simulada en \texttt{MATLAB} correspondiente al Caso 2($f_s = 1000 Hz$).}{fig:simulacion_noImplementable}{1}

\insertarfigurawide{matlab/Sim2_noImplementable.png}{Respuesta simulada en \texttt{MATLAB} correspondiente al Caso 2($f_s = 1000 Hz$).}{fig:simulacion_noImplementable2}{1}

En las figuras \ref{fig:simulacion_noImplementable} y \ref{fig:simulacion_noImplementable2}, del \texttt{MATLAB} se pueden observar que no es implementable.

	\newpage	
	
\twocolumn
\section{Implementación y comparación experimental}


% ======= COMPARAR FIGURAS MÉTODO 1 ========
\compararfigsC{Img/M1/exp/ex1_239Hz_ref1_1.5.png}
{Implementación experimental del método 1.}
{fig:exp_m1}
{Img/M1/sim/M1_Ts_Tn4173.png}
{Simulación del método 1 ($T_s = 4.173\,\text{ms}$).}
{fig:sim_m1}
{Comparación entre la simulación y la implementación experimental para el compensador del método 1.}
{fig:comp_m1}

% ======= COMPARAR FIGURAS MÉTODO 2 ========
\compararfigsC{Img/M2/exp/ex2_239Hz_ref1_1.5.png}
{Implementación experimental del método 2.}
{fig:exp_m2}
{Img/M2/sim/M2_Ts_4173.png}
{Simulación del método 2 ($T_s = 4.173\,\text{ms}$).}
{fig:sim_m2}
{Comparación entre la simulación y la implementación experimental para el compensador del método 2.}
{fig:comp_m2}

Para la implementación en el PSoC, se seleccionaron los controladores diseñados que presentaban el menor esfuerzo de control, de modo que su salida fuera físicamente realizable sin saturar el actuador.  
Si bien los controladores con tiempos de muestreo más pequeños ofrecen una respuesta más rápida, el esfuerzo de control resultante continúa siendo elevado, lo que limita su viabilidad práctica.

Por esta razón, durante las pruebas experimentales se utilizó una referencia de pequeña amplitud, lo que incrementó la relación ruido–señal. Aun así, los resultados fueron satisfactorios, mostrando un seguimiento adecuado y tiempos de establecimiento coherentes con los obtenidos en simulación.

En las Figuras~\ref{fig:comp_m1} y~\ref{fig:comp_m2} se presentan las comparaciones entre las simulaciones y los resultados experimentales para ambos métodos de diseño de controladores.  
Las imágenes experimentales corresponden a una referencia de \(1.5\,\text{V}\) a \(239\,\text{Hz}\), cuyos valores se indican también en el nombre de cada archivo.




Se observa que el primer compensador (método~1) alcanza la referencia en aproximadamente un periodo de muestreo, tal como se había previsto teóricamente, aunque con la presencia de oscilaciones y un leve sobreimpulso, comportamientos también esperados.  
En cambio, el compensador del método~2 alcanza la estabilidad en torno a los dos periodos de muestreo, presentando una respuesta más suave y sin oscilaciones notorias, aunque con un esfuerzo de control considerablemente mayor y un pequeño sobrepasamiento no observado en las simulaciones.

Al analizar los cursores mostrados en las imágenes experimentales, se verifica que los tiempos de establecimiento concuerdan con los valores estimados en simulación. En ambos casos, la respuesta del sistema puede considerarse satisfactoria, logrando un equilibrio adecuado entre rapidez, estabilidad y esfuerzo de control.


	
	
	\section{Conclusión}

En síntesis, ambos métodos de diseño mostraron un desempeño adecuado tanto en simulación como en la implementación experimental.  

No obstante, este tipo de controladores \textit{analíticos} —tanto el convencional como el \textit{deadbeat}— dependen fuertemente de la precisión del modelo de la planta.  
En la práctica, los parámetros del sistema pueden variar con la temperatura, el envejecimiento de los componentes o pequeñas no linealidades no contempladas, lo que puede afectar la estabilidad y el desempeño del control.  

Si bien los resultados obtenidos validan la teoría en condiciones controladas, su aplicación práctica requiere cierto margen de robustez para adaptarse a variaciones del sistema real.

	
	\newpage
	\onecolumn
\section{Anexo.}

	\bibliographystyle{IEEEtranN}
	\bibliography{referencias}
	
\end{document}
