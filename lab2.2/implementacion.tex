\newpage
\onecolumn

\subsection{Implementación en PSoC}
\begin{enumerate}[label=4.\arabic*.]
	\item Implementar las ecuaciones en diferencias:
	
	\begin{enumerate}[label*=4.\arabic*.]
			\item Con el compensador de la sección 3.1.\\
			Se realizó la implementación en \textit{PSoC Creator} empleando los parámetros obtenidos en Matlab para un amortiguamiento objetivo de $\zeta = 0.7$. La Figura~\ref{fig:CLOSCCOMP1} muestra el primer compensador y la respuesta medida en placa.

			\compararfigsC{./Img/CLOSCcomp1}{Primer compensador diseñado para $\zeta = 0.7$.}{fig:CLOSCCOMP1}{./Img/step_y_esfuerzoCompensador1}{Step y esfuerzo del compensador diseñado para $\zeta = 0.7$.}{fig:stepysfuerzcomp1}{}{fig:comparacion1}
			
			En la Figura \ref{fig:CLOSCCOMP1} se observa un pico pronunciado en el esfuerzo de control (traza amarilla), de manera consistente con lo obtenido en la simulación. La salida (traza magenta) presenta un comportamiento muy similar al simulado (Figura \ref{fig:stepysfuerzcomp1}), manteniéndose dentro del rango establecido y logrando un adecuado seguimiento de la referencia.
			
			\insertarfigurawide{./Img/step_y_esfuerzoCompensador2}{Segundo compensador diseñado para $\zeta = 0.7$.}{fig:stepysfuerzcomp2}{1}
			
			\compararfigsC{./Img/CLOSCcomp2_1}{Primer compensador diseñado para $\zeta = 0.7$ con $ESS = 0$ y un polo en el circulo unitario.}{fig:CLOSCCOMP2}{./Img/CLOSCcomp2_09}{Segundo compensador diseñado para $\zeta = 0.7$ con $ESS=0$ y un polo cercano al circulo unitario.}{fig:CLOSCCOMP209}{}{fig:comparacion2}
			
			En la Figura~\ref{fig:CLOSCCOMP2} se observa que el caso con el polo ubicado cerca del círculo unitario presenta picos en la respuesta, ya que este posicionamiento acelera la dinámica del sistema. En cambio, el sistema con el polo en el círculo unitario (Figura~\ref{fig:CLOSCCOMP209}) muestra una respuesta menos enérgica y más lenta, debido a su mayor amortiguamiento.
			\item Con el compensador de la sección 3.2.
			
			
			\insertarfigurawide{./Img/step_y_esfuerzoCompensador3}{Tercer compensador diseñado en Matlab.}{fig:stepysfuerzcomp3}{1}
			
			\compararfigsC{./Img/CLOSCcomp3_1}{Tercer compensador diseñado para $\zeta = 0.7$ con $ESS=0$ y un polo cercano en el circulo unitario.}{fig:CLOSCCOMP31}{./Img/CLOSCcomp3_09}{Tercer compensador diseñado para $\zeta = 0.7$ con $ESS=0$ y un polo cercano en el circulo unitario.}{fig:CLOSCCOMP209}{}{fig:comparacion3}
			
			

			
			\subsection{Nuevo Modelado de la Planta}

Con el objetivo de mejorar las prestaciones y evitar las limitaciones derivadas de las tolerancias de los componentes pasivos (resistencias y capacitores de baja precisión), se decidió realizar un modelado alternativo de la planta utilizando \textsc{Matlab}.

Para ello, se empleó el osciloscopio a fin de muestrear la respuesta del sistema ante una entrada escalón. Posteriormente, mediante las funciones de identificación de sistemas en \textsc{Matlab}, se llevó a cabo una estimación de segundo orden que aproxima de manera adecuada el comportamiento dinámico de la planta.  

\insertarfigura{./Img/circuito.png}
{Respuesta al escalón de la planta estimada mediante identificación en \textsc{Matlab}.}
{fig:PlantaEstimada}{0.8}

De esta estimación se obtuvo la siguiente función de transferencia característica del sistema identificado:
\begin{equation}
	G(s) \;=\; \frac{K}{s^2 + a s + b},
\end{equation}
donde los parámetros $K$, $a$ y $b$ fueron ajustados automáticamente por el algoritmo de estimación en función de los datos experimentales.  

A partir de este nuevo modelo, y siguiendo procedimientos análogos a los presentados en secciones anteriores, se diseñaron dos compensadores que fueron implementados de manera similar al caso inicial.

(FIGURAS)



\begin{table}[H]
	\centering
	\caption{Parámetros del Compensador 1 (con la nueva planta)}
	\begin{tabular}{|c|c|}
		\hline
		$z_0$ & 0.380927627742363 \\ \hline
		$p_0$ & 0.836970379456654 \\ \hline
		$K$   & 0.226961597782889 \\ \hline
	\end{tabular}
\end{table}

\begin{table}[H]
	\centering
	\caption{Parámetros del Compensador 2 (con la nueva planta)}
	\begin{tabular}{|c|c|}
		\hline
		$z_{0,0}$ & 0.380927627742363 \\ \hline
		$z_{0,1}$ & 0.821584532561100 \\ \hline
		$p_{0,0}$ & 0.753973536568200 \\ \hline
		$p_{0,1}$ & 1 \\ \hline
		$K$       & 0.287529112579611 \\ \hline
	\end{tabular}
\end{table}
\onecolumn
\compararfigsC{./Img/comp1_v2}{Version dos del primer compensador diseñado en Matlab para $\zeta = 0.7$.}{fig:comp1v2}{./Img/CLOSCcomp1_v2}{Implementación del compensador en el PSoC.}{fig:CLOSCCOMP1v2}{}{fig:comparacion4}



\compararfigsC{./Img/comp2_v2}{Version dos del segundo compensador diseñado en Matlab para $\zeta = 0.7$ y $ESS = 0$.}{fig:comp2v2}{./Img/CLOSCcomp2_v2}{Implementación del compensador en el PSoC.}{fig:CLOSCCOMP2v2}{}{fig:comparacion5}

		\end{enumerate}
\end{enumerate}
