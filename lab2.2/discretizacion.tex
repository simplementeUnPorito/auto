\subsection{Discretización de la Planta}
\begin{enumerate}[label=2.\arabic*.]
	\item Elegir el tiempo de muestreo $\Ts$ para que existan 8 muestras dentro del tiempo de subida. Para este caso se considera un factor de amortiguamiento $\zeta = 0.7$. \\
	
	La planta a discretizar está dada por:
	\begin{equation}
		G(s) = \frac{K}{(s+a)(s+b)}
	\end{equation}
	Este sistema corresponde a una planta de segundo orden, representada como el producto de dos polos reales.
	
	Para discretizar, partimos del método de la \emph{transformación de la respuesta al impulso}. La relación general es:
	\begin{equation}
		G(z)=(1 - z^{-1})\,\mathcal{Z}\left[\mathcal{L}^{-1}\left(\frac{G(s)}{s}\right)^*\right]
		\label{eq:DiscGz}
	\end{equation}
	Es decir, primero se obtiene la respuesta al escalón unitario en el tiempo continuo, luego se transforma con $\mathcal{Z}$ y finalmente se multiplica por $(1-z^{-1})$ para obtener la función de transferencia discreta.
	
	\vspace{0.5em}
	\textbf{Desarrollo de fracciones parciales:}  
	Se descompone $\tfrac{G(s)}{s}$ en términos más simples para poder aplicar transformadas inversas de Laplace:
	\begin{equation}
		\frac{G(s)}{s} =
		\underset{\delta}{\frac{K}{ab}} \cdot \frac{1}{s}
		+ \underset{\beta}{\frac{K}{\,b^2-ab\,}} \cdot \frac{1}{s+b}
		+ \underset{\alpha}{\frac{K}{\,a^2-ab\,}} \cdot \frac{1}{s+a}
	\end{equation}
	Aquí aparecen tres coeficientes, que llamamos $\delta$, $\beta$ y $\alpha$, cada uno asociado a un término simple de la expansión.
	
	\vspace{0.5em}
	\textbf{Respuesta en el tiempo continuo:}  
	Aplicando la transformada de Laplace inversa, cada término corresponde a una exponencial decreciente o a una constante:
	\begin{equation}
		\mathcal{L}^{-1} \left\{\frac{G(s)}{s}\right\}=\delta + \beta e^{-bTk}+\alpha e^{-aTk}, \quad k \geq 0
	\end{equation}
	Este resultado representa la respuesta al escalón unitario en tiempo discreto, evaluada en múltiplos de $T$.
	
	\vspace{0.5em}
	\textbf{Transformada $Z$:}  
	Al transformar esta secuencia al dominio-$z$, se obtiene:
	\begin{equation}
		\mathcal{Z}\left[\frac{G(s)}{s}\right] = \delta \frac{z}{z-1}+\beta \frac{z}{z-e^{-bT}}+\alpha \frac{z}{z-e^{-aT}}
	\end{equation}
	Cada término se convierte en una fracción simple con un polo en $z=1$, $z=e^{-bT}$ y $z=e^{-aT}$ respectivamente.
	
	\vspace{0.5em}
	\textbf{Función de transferencia discreta:}  
	Finalmente, al multiplicar por $(1-z^{-1})$, se obtiene la expresión general de la planta discretizada, G(z), vista enteriormente en la ecuación \ref{eq:DiscGz}.
	
	\vspace{0.5em}
	\textbf{Cálculo de coeficientes:}  
	Con los valores $a = \tfrac{1}{81\,000 \cdot 200 \cdot 10^{-9}}$ y $b = \tfrac{1}{15 \cdot 10^{3} \cdot 100 \cdot 10^{-9}}$, y considerando $K = 1$ y $T = 1.2452$, se obtienen los siguientes coeficientes:
	\[
	\delta = \frac{K}{ab} \approx 24.3 \times 10^{6}, \qquad
	\beta = \frac{K}{b^2-ab} \approx -2.4 \times 10^{6},
	\]
	
	\[
	\alpha = \frac{K}{a^2-ab} \approx -26.78 \times 10^{6}
	\]
	\vspace{0.5em}
	\textbf{Planta discretizada:}  
	Reemplazando los valores en la expresión final, se obtiene aproximadamente:
	\begin{equation}
		G(z) \approx \frac{0.022(z+0.74)}{(z-0.93)(z-0.44)}
	\end{equation}
	
	\item Comparar este resultado con el obtenido en Matlab utilizando el comando \texttt{c2d}.
	VER DESPUES
\end{enumerate}
