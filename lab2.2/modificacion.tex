\subsection{Modificaci\'on de la Din\'amica}

Una vez digitalizada la planta, es posible identificar la ubicaci\'on de polos y ceros en el plano-$z$ (v\'ease la Fig.~\ref{img:rlocusOP}).  

\insertarfigura{./Img/lugarDeRaicesPlantaDigitalizada.png}
{Lugar de las ra\'ices de la planta digitalizada con el tiempo de muestreo $T$ calculado.}
{img:rlocusOP}{1}

De acuerdo con las especificaciones solicitadas, se debe determinar un punto en el plano-$z$ que satisfaga las condiciones de desempe\~no requeridas.  

Recordando que 
\[
z = e^{sT}, \quad \text{con } s = \sigma + j\omega_d,
\]
donde $\sigma = \zeta \cdot \omega_n$ y $\omega_d = \omega_n \sqrt{1-\zeta^2}$,
se obtiene un punto candidato $p_{\text{target}}$, cuyas coordenadas en el plano-$z$ son
\[
x = 0.843273, \qquad y = 0.136677.
\]

Este punto representa la ubicaci\'on de los polos deseados que garantizar\'ian la aproximaci\'on de segundo orden y, por ende, las caracter\'isticas din\'amicas solicitadas.  

Para verificar si el punto pertenece al lugar de las ra\'ices de la planta, se aplica la \emph{condici\'on de fase}, la cual establece que la suma de \'angulos debe satisfacer la Ec.~\eqref{eq:condicionFase}:

\begin{equation}
	\begin{aligned}
		\angle (p_{\text{target}}-z_0) 
		&- \angle (p_{\text{target}}-p_0) \\
		&- \angle (p_{\text{target}}-p_1) 
		&=& \phi \\&&\approx& -136.42^\circ
	\end{aligned}
	\label{eq:condicionFase}
\end{equation}

De este an\'alisis se deduce que es necesario introducir un compensador que aporte un desfase adicional de $-43.042^\circ$ para que el punto deseado forme parte del lugar de ra\'ices y el sistema cumpla con las especificaciones de dise\~no (v\'ease la Fig.~\ref{img:rlocusTarget}).  

(CAMBIAR IMAGEN)
\insertarfigura{./Img/lugarDeRaicesPlantaDigitalizada.png}
{Lugar de las ra\'ices de la planta digitalizada con el polo candidato.}
{img:rlocusTarget}{1}

\begin{enumerate}[label=3.\arabic*.]
	\item Diseñar compensador para $\zeta = 0.7$: 
	Una vez digitalizada la planta, es posible identificar la ubicaci\'on de polos y ceros en el plano-$z$.  

\insertarfigura{./Img/lugarDeRaicesPlantaDigitalizada.png}
{Lugar de las ra\'ices de la planta digitalizada con el tiempo de muestreo $T$ calculado.}
{ref:rlocusOP}{1}

De acuerdo con las especificaciones solicitadas, se debe determinar un punto en el plano-$z$ que satisfaga las condiciones de desempe\~no requeridas.  

Recordando que 
\[
z = e^{sT}, \quad \text{con } s = \sigma + j\omega_d,
\]
donde $\sigma = \zeta \cdot \omega_n$ y $\omega_d = \omega_n \sqrt{1-\zeta^2},
\]
se obtiene un punto candidato cuyas coordenadas en el plano-$z$ son
\[
x = 0.843273, \qquad y = 0.136677.
\]

Este punto representa la ubicaci\'on de los polos deseados que garantizar\'ian la aproximaci\'on de segundo orden y, por ende, las caracter\'isticas din\'amicas solicitadas.  

Sin embargo, para verificar si el punto pertenece al lugar de las ra\'ices de la planta, es necesario aplicar la \emph{condici\'on de fase}, la cual establece que la suma de \'angulos debe satisfacer:
\[
\angle (z-z_0) - \angle (z-p_0) - \bigl(180^\circ - \angle (z-p_1)\bigr) \approx -136.42^\circ.
\]

De este an\'alisis se deduce que es preciso introducir un compensador que aporte un desfase adicional de $-43.042^\circ$ para que el punto deseado forme parte del lugar de ra\'ices y el sistema cumpla con las especificaciones de dise\~no.  

(Poner imagen de plano-$z$ con el punto candidato marcado)

	\item Diseñar compensador para $\zeta = 0.7$ y $\ESS = 0$: 
	

\input{integrador1.tex}
\input{integrador2.tex}
\end{enumerate}