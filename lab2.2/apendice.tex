% ================== APÉNDICE OPCIONAL ==================
\onecolumn
\appendices
\section{Códigos de Matlab}

\begin{lstlisting}[style=matlabstyle,caption={Primera hoja de cálculos utilizada.},label={matlab:calculo1}]
	close all
	clear all
	
	addpath('..\Lab1\')
	
	%% Definicion de parametros
	R_1 = 15e3;
	R_3 = 15e3;
	C_2 = 100e-9;
	
	R_2 = 82e3;
	R_4 = 82e3;
	C_1 = 0.22e-6;
	
	%% Generar funcion de transferencia d
	numStage = [-R_3/R_1 -R_4/R_2];
	denStage = { [C_2*R_3 1], [C_1*R_4 1] };
	
	% Usamos celdas para guardar los tf de cada stage
	Gstage = cell(1,2);
	G = 1;
	for i = 1:2
	Gstage{i} = tf(numStage(i), denStage{i});
	G = G*Gstage{i};
	end
	
	%% Analizamos en el tiempo
	[tr, ts, wn] = plot_step_info(G);
	disp('Planta continua G(s):')
	G
	zpk(G)
	
	
	
	%% Paso 1: Definir periodo de muestreo
	% Se busca una mejora de 4 en el tiempo de rising , o sea tr = 10 ms
	N = 4; 
	T = tr/(8*N);
	%Gcl = feedback(G,1);
	%figure;
	%step(Gcl)
	%T = stepinfo(Gcl).SettlingTime/8;
	
	%% Paso 2: Digitalizar con ZOH
	Gd = c2d(G, T, 'zoh');
	disp('Planta digital G(z):')
	Gd
	zpk(Gd)
	figure;
	pzmap(Gd);
	title('Lugar de raíces de G(z).')
	
	zgrid;
	z = tf([1 0],1,T);
	
	%% Paso 3: Análisis en tiempo discreto lazo abierto
	figure;
	
	% Obtenemos salida y tiempo de la función step
	[y, t] = step(Gd);
	
	% Graficamos con stairs (propio de señales discretas)
	stairs(t*1000, y, 'LineWidth',1.4);
	
	title('Respuesta al escalón de la planta digitalizad')
	xlabel('Tiempo [ms]');
	ylabel('Salida');
	grid on;
	ax = gca;
	ax.XMinorTick = 'on';  % activamos minor ticks
	ax.YMinorTick = 'on';
	grid minor;
	
	%% Paso 4: Definimimos los valores de la simulación
	umin = 0;
	umax =  4.08;
	refmin = 1.5;
	refmax = 2.5;
	n_per_seg = 500;
	%% Paso 5: Lugar de raíces para zita = 0.7, compensador P
	%figure
	
	%rlocus(Gd)
	%zgrid     % agrega la grilla en el plano-z
	z0=0.436;
	p0=0.769221;%0.7705;
	C_1 = (z-z0)/(z-p0);
	figure;
	Gc1 = Gd*C_1;
	rlocus(Gc1);
	title('Lugar de raíces de G(z) compensado con un lag-filter de 1º orden, anulando el polo más rápido de la planta.')
	
	zgrid     % agrega la grilla en el plano-z
	%pause;
	%[K, ~] = rlocfind(Gc1);   % hacés click donde querés los polos
	%K = 0.73139;
	%K = 0.82543;
	K = 0.743;
	Gc1f = feedback(K*Gc1,1);
	%figure;
	%step(Gc1f)
	info = stepinfo(Gc1f);
	info
	
	info_ext = plot_step_annot(Gc1f, 'la planta compensada con un lag-filter de 1º orden, anulando el polo más rápido de la planta.');
	%step(Gc3f);
	info = stepinfo(Gc1f);
	info
	info_ext
	
	[td, refd, yd, ud, ed, coefs] = sim_compensador_first_order( ...
	Gd, z0, p0, K, T, umin, umax, refmin, refmax, n_per_seg);
	
	Nini = 100;              % descartar al inicio
	Nfin = 100;              % descartar al final
	idx0 = Nini + 1;         % índice inicial válido
	idx1 = length(td) - Nfin; % índice final válido
	
	td   = td(idx0:idx1)-td(idx0);
	refd = refd(idx0:idx1);
	yd   = yd(idx0:idx1);
	ud   = ud(idx0:idx1);
	ed   = ed(idx0:idx1);
	
	% Si querés que el tiempo arranque en 0
	
	
	
	
	
	figure;
	
	% ---------- Subplot 1 ----------
	axAbs = subplot(2,1,1);  % eje izquierdo (absoluto)
	hold(axAbs,'on'); grid(axAbs,'on');
	
	% y[k] en rojo (abs)
	hY = stairs(axAbs, td, yd, 'r-', 'LineWidth', 1.4);
	
	% Armamos eje derecho transparente
	axPct = axes('Position', get(axAbs,'Position'), ...
	'Color','none', 'YAxisLocation','right', ...
	'XLim', get(axAbs,'XLim'), 'XTick',[], 'Box','off');
	hold(axPct,'on');
	
	% r[k] en negro, graficado en %
	ref_pct = 100*(refd - refmin)/(refmax - refmin);
	hR = stairs(axPct, td, refd, 'k--', 'LineWidth', 1.2);
	
	% ======= Cálculo de límites con margen =======
	yd_pct  = 100*(yd - refmin)/(refmax - refmin);
	pctAll  = [yd_pct; ref_pct];
	
	% Valores extremos en % con margen del 5 %
	rawMin = min(pctAll);
	rawMax = max(pctAll);
	span   = rawMax - rawMin;
	pctMin = rawMin - 0.05*span;
	pctMax = rawMax + 0.05*span;
	
	% Redondeamos a múltiplos de 20 para ticks
	stepPct = 20;
	tickMin = stepPct*floor(pctMin/stepPct);
	tickMax = stepPct*ceil (pctMax/stepPct);
	pctTicks = tickMin:stepPct:tickMax;
	
	% Convertimos a absolutos
	valTicks = refmin + (pctTicks/100)*(refmax - refmin);
	
	% Aplicamos a ambos ejes
	set(axAbs,'YLim',[valTicks(1) valTicks(end)],'YTick',valTicks);
	set(axPct,'YLim',[valTicks(1) valTicks(end)],'YTick',valTicks,...
	'YTickLabel',compose('%.0f %%',pctTicks));
	
	% Etiquetas
	xlabel(axAbs,'Tiempo [s]');
	ylabel(axAbs,'Respuesta [valor absoluto]');
	ylabel(axPct,'Escala relativa a ref [%]');
	title(axAbs,'Respuesta discreta con compensador 1º orden, anulando el polo más rápido de la planta.');
	legend(axAbs, [hY, hR], {'y[k]', 'r[k]'}, 'Location', 'best');
	
	% ---------- Subplot 2: esfuerzo de control ----------
	axU = subplot(2,1,2);
	stairs(axU, td, ud, 'LineWidth',1.2); grid(axU,'on');
	yline(axU, umax,'r:'); yline(axU, umin,'r:');
	xlabel(axU,'Tiempo [s]'); ylabel(axU,'u[k]');
	title(axU,'Esfuerzo de control con saturación');
	
	% --- Margen de 5% en eje Y ---
	uMin = min(ud);
	uMax = max(ud);
	span = uMax - uMin;
	ylim(axU, [uMin - 0.05*span, uMax + 0.05*span]);
	
	
	grid(axAbs,'on');    % grilla principal
	grid(axAbs,'minor'); % grilla secundaria
	
	grid(axPct,'on');
	grid(axPct,'minor');
	
	grid(axU,'on');
	grid(axU,'minor');
	
	%% Paso 6: Lugar de raices para zita = 0.7 y y ESS = 0, compensador PI
	z0 = 0.857419558001;
	p0 = 1;
	C_2 = (z-z0)/(z-p0); %z=0.8433 %8433
	Gc2 =Gd*C_2;
	figure;
	rlocus(Gc2);
	title('Lugar de raíces de G(z) compensado con un lag-filter de 1º orden , fijando un polo en z = 1.')
	
	zgrid
	%[K, ~] = rlocfind(Gc2);   % hacés click donde querés los polos
	%K = 3.2779;
	K2 = 3.09;
	%figure;
	Gc2f= feedback(K2*Gc2,1);
	%step(Gc2f);
	info = stepinfo(Gc2f);
	info
	
	info_ext = plot_step_annot(Gc2f, 'la planta compensada con un lag-filter de 1º orden , fijando un polo en z = 1.');
	%step(Gc3f);
	info = stepinfo(Gc2f);
	info
	info_ext
	
	
	[td, refd, yd, ud, ed, coefs] = sim_compensador_first_order( ...
	Gd, z0, p0, K, T, umin, umax, refmin, refmax, n_per_seg);
	
	
	Nini = 100;              % descartar al inicio
	Nfin = 100;              % descartar al final
	idx0 = Nini + 1;         % índice inicial válido
	idx1 = length(td) - Nfin; % índice final válido
	
	td   = td(idx0:idx1)-td(idx0);
	refd = refd(idx0:idx1);
	yd   = yd(idx0:idx1);
	ud   = ud(idx0:idx1);
	ed   = ed(idx0:idx1);
	
	
	
	
	figure;
	
	% ---------- Subplot 1 ----------
	axAbs = subplot(2,1,1);  % eje izquierdo (absoluto)
	hold(axAbs,'on'); grid(axAbs,'on');
	
	% y[k] en rojo (abs)
	hY = stairs(axAbs, td, yd, 'r-', 'LineWidth', 1.4);
	
	% Armamos eje derecho transparente
	axPct = axes('Position', get(axAbs,'Position'), ...
	'Color','none', 'YAxisLocation','right', ...
	'XLim', get(axAbs,'XLim'), 'XTick',[], 'Box','off');
	hold(axPct,'on');
	
	% r[k] en negro, graficado en %
	ref_pct = 100*(refd - refmin)/(refmax - refmin);
	hR = stairs(axPct, td, refd, 'k--', 'LineWidth', 1.2);
	
	% ======= Cálculo de límites con margen =======
	yd_pct  = 100*(yd - refmin)/(refmax - refmin);
	pctAll  = [yd_pct; ref_pct];
	
	% Valores extremos en % con margen del 5 %
	rawMin = min(pctAll);
	rawMax = max(pctAll);
	span   = rawMax - rawMin;
	pctMin = rawMin - 0.05*span;
	pctMax = rawMax + 0.05*span;
	
	% Redondeamos a múltiplos de 20 para ticks
	stepPct = 20;
	tickMin = stepPct*floor(pctMin/stepPct);
	tickMax = stepPct*ceil (pctMax/stepPct);
	pctTicks = tickMin:stepPct:tickMax;
	
	% Convertimos a absolutos
	valTicks = refmin + (pctTicks/100)*(refmax - refmin);
	
	% Aplicamos a ambos ejes
	set(axAbs,'YLim',[valTicks(1) valTicks(end)],'YTick',valTicks);
	set(axPct,'YLim',[valTicks(1) valTicks(end)],'YTick',valTicks,...
	'YTickLabel',compose('%.0f %%',pctTicks));
	
	% Etiquetas
	xlabel(axAbs,'Tiempo [s]');
	ylabel(axAbs,'Respuesta [valor absoluto]');
	ylabel(axPct,'Escala relativa a ref [%]');
	title(axAbs,'Respuesta discreta con compensador 1º orden, fijando un polo en z = 1.');
	legend(axAbs, [hY, hR], {'y[k]', 'r[k]'}, 'Location', 'best');
	
	% ---------- Subplot 2: esfuerzo de control ----------
	axU = subplot(2,1,2);
	stairs(axU, td, ud, 'LineWidth',1.2); grid(axU,'on');
	yline(axU, umax,'r:'); yline(axU, umin,'r:');
	xlabel(axU,'Tiempo [s]'); ylabel(axU,'u[k]');
	title(axU,'Esfuerzo de control con saturación');
	
	% --- Margen de 5% en eje Y ---
	uMin = min(ud);
	uMax = max(ud);
	span = uMax - uMin;
	ylim(axU, [uMin - 0.05*span, uMax + 0.05*span]);
	
	grid(axAbs,'on');    % grilla principal
	grid(axAbs,'minor'); % grilla secundaria
	
	grid(axPct,'on');
	grid(axPct,'minor');
	
	grid(axU,'on');
	grid(axU,'minor');
	
	
	
	%% Paso 7: Lugar de raices para zita = 0.7 y y ESS = 0, compensador PI de segundo orden
	z0_0 = 0.9333;
	p0_0 = 1;
	z0_1 = 0.436;
	p0_1 = 0.71135;
	C_3 = ((z-z0_0)*(z-z0_1))/((z-p0_0)*(z-p0_1));
	Gc3 =Gd*C_3; %z=0.8433 %8433
	figure;
	rlocus(Gc3)
	title('Lugar de raíces de G(z) compensado con un lag-filter de 2º orden, fijando un polo en z = 1 y anulando ambos polos de la planta.');
	
	zgrid
	%pause;
	%[K, ~] = rlocfind(Gc3);   % hacés click donde querés los polos
	K = 1.14;
	%figure;
	Gc3f= feedback(K*Gc3,1);
	
	info_ext = plot_step_annot(Gc3f, 'la planta compensada con un lag-filter de 2º orden, fijando un polo en z = 1 y anulando ambos polos de la planta.');
	%step(Gc3f);
	info = stepinfo(Gc3f);
	info
	info_ext
	
	
	% --- Llamada a tu simulador (misma interfaz que tu first_order) ---
	[td, refd, yd, ud, ed, coefs] = sim_compensador_second_order( ...
	Gd, z0_0, z0_1, p0_0, p0_1, K, T, umin, umax, refmin, refmax, n_per_seg);
	
	
	Nini = 100;              % descartar al inicio
	Nfin = 100;              % descartar al final
	idx0 = Nini + 1;         % índice inicial válido
	idx1 = length(td) - Nfin; % índice final válido
	
	td   = td(idx0:idx1)-td(idx0);
	refd = refd(idx0:idx1);
	yd   = yd(idx0:idx1);
	ud   = ud(idx0:idx1);
	ed   = ed(idx0:idx1);
	
	
	
	
	figure;
	
	% ---------- Subplot 1 ----------
	axAbs = subplot(2,1,1);  % eje izquierdo (absoluto)
	hold(axAbs,'on'); grid(axAbs,'on');
	
	% y[k] en rojo (abs)
	hY = stairs(axAbs, td, yd, 'r-', 'LineWidth', 1.4);
	
	% Armamos eje derecho transparente
	axPct = axes('Position', get(axAbs,'Position'), ...
	'Color','none', 'YAxisLocation','right', ...
	'XLim', get(axAbs,'XLim'), 'XTick',[], 'Box','off');
	hold(axPct,'on');
	
	% r[k] en negro, graficado en %
	ref_pct = 100*(refd - refmin)/(refmax - refmin);
	hR = stairs(axPct, td, refd, 'k--', 'LineWidth', 1.2);
	
	% ======= Cálculo de límites con margen =======
	yd_pct  = 100*(yd - refmin)/(refmax - refmin);
	pctAll  = [yd_pct; ref_pct];
	
	% Valores extremos en % con margen del 5 %
	rawMin = min(pctAll);
	rawMax = max(pctAll);
	span   = rawMax - rawMin;
	pctMin = rawMin - 0.05*span;
	pctMax = rawMax + 0.05*span;
	
	% Redondeamos a múltiplos de 20 para ticks
	stepPct = 20;
	tickMin = stepPct*floor(pctMin/stepPct);
	tickMax = stepPct*ceil (pctMax/stepPct);
	pctTicks = tickMin:stepPct:tickMax;
	
	% Convertimos a absolutos
	valTicks = refmin + (pctTicks/100)*(refmax - refmin);
	
	% Aplicamos a ambos ejes
	set(axAbs,'YLim',[valTicks(1) valTicks(end)],'YTick',valTicks);
	set(axPct,'YLim',[valTicks(1) valTicks(end)],'YTick',valTicks,...
	'YTickLabel',compose('%.0f %%',pctTicks));
	
	% Etiquetas
	xlabel(axAbs,'Tiempo [s]');
	ylabel(axAbs,'Respuesta [valor absoluto]');
	ylabel(axPct,'Escala relativa a ref [%]');
	title(axAbs,'Respuesta discreta con compensador 2º orden, fijando un polo en z = 1 y anulando ambos polos de la planta');
	legend(axAbs, [hY, hR], {'y[k]', 'r[k]'}, 'Location', 'best');
	
	% ---------- Subplot 2: esfuerzo de control ----------
	axU = subplot(2,1,2);
	stairs(axU, td, ud, 'LineWidth',1.2); grid(axU,'on');
	yline(axU, umax,'r:'); yline(axU, umin,'r:');
	xlabel(axU,'Tiempo [s]'); ylabel(axU,'u[k]');
	title(axU,'Esfuerzo de control con saturación');
	
	% --- Margen de 5% en eje Y ---
	uMin = min(ud);
	uMax = max(ud);
	span = uMax - uMin;
	ylim(axU, [uMin - 0.05*span, uMax + 0.05*span]);
	
	
	
	
	
	
	
	grid(axAbs,'on');    % grilla principal
	grid(axAbs,'minor'); % grilla secundaria
	
	grid(axPct,'on');
	grid(axPct,'minor');
	
	grid(axU,'on');
	grid(axU,'minor');
	
	
	%% impresion antigua
	% % --- Gráficas estilo informe ---
	% figure; 
	% subplot(2,1,1);
	% stairs(td, refd, 'k--','LineWidth',1.0); hold on;
	% stairs(td, yd,   'LineWidth',1.4);
	% grid on; xlabel('Tiempo [s]'); ylabel('Respuesta de la planta/Referencia');
	% title('Respuesta discreta con compensador 2º orden'); legend('r[k]','y[k]','Location','best');
	% 
	% subplot(2,1,2);
	% stairs(td, ud, 'LineWidth',1.2);
	% yline(umax,'r:'); yline(umin,'r:');
	% grid on; xlabel('Tiempo [s]'); ylabel('u[k]');
	% title('Esfuerzo de control con saturación');
\end{lstlisting}

\begin{lstlisting}[style=matlabstyle,caption={Funciones desarrolladas para la práctica.},label={matlab:func}]
	function info_ext = plot_step_annot(sys, nameStr, Ts_override)
	% STEP discreto con STAIRS; anota 10-90% (en s y en T), OS, zeta y omega_n.
	% Grafica en milisegundos y marca el OS en el pico.
	
	if nargin < 2 || isempty(nameStr), nameStr = inputname(1); end
	if isempty(nameStr), nameStr = 'sys'; end
	
	% ----- Respuesta y métricas básicas -----
	[y, t] = step(sys); y = y(:); t = t(:);
	t_ms = t * 1000;   % graficamos en milisegundos
	
	% Estimar yss por promedio en cola
	Ntail = max(10, round(0.05*length(y)));
	yss = mean(y(end-Ntail+1:end));
	
	% Límites 10-90
	if abs(yss) < 1e-12
	t10 = NaN; t90 = NaN; tr_10_90 = NaN; y10 = 0; y90 = 0;
	else
	y10 = 0.10*yss; y90 = 0.90*yss;
	sgn = sign(yss);
	idx10 = find(sgn*y >= sgn*y10, 1, 'first');
	idx90 = find(sgn*y >= sgn*y90, 1, 'first');
	t10 = tern(~isempty(idx10), t(idx10), NaN);
	t90 = tern(~isempty(idx90), t(idx90), NaN);
	tr_10_90 = tern(~isnan(t10)&&~isnan(t90)&&t90>=t10, t90 - t10, NaN);
	end
	t10_ms = t10 * 1000; t90_ms = t90 * 1000;
	
	% Pico y %OS (para zeta y omega_n)
	[ypeak_raw, idxpk] = max(sign(yss).*y);
	ypeak = sign(yss)*ypeak_raw;
	tpeak = t(idxpk); tpeak_ms = tpeak * 1000;
	OS_percent = tern(yss ~= 0, max(0, (ypeak - yss)/abs(yss)*100), NaN);
	
	% zeta desde %OS:  OS% = 100*exp(-zeta*pi/sqrt(1-zeta^2))
	if isnan(OS_percent) || OS_percent <= 0
	zeta_est = NaN;
	else
	logterm = log(OS_percent/100);
	zeta_est = -logterm / sqrt(pi^2 + logterm^2);
	end
	
	% ----- Ts y rise en T -----
	Ts = NaN;
	try
	if isprop(sys,'Ts') && ~isempty(sys.Ts) && sys.Ts > 0, Ts = sys.Ts; end
	catch, end
	if nargin >= 3 && ~isempty(Ts_override) && Ts_override > 0, Ts = Ts_override; end
	tr_10_90_T = tern(~isnan(Ts) && ~isnan(tr_10_90), tr_10_90/Ts, NaN);
	
	% ----- Estimacion de omega_n -----
	omega_n = NaN;
	if ~isnan(zeta_est) && zeta_est < 1 && OS_percent > 0 && tpeak > 0
	omega_n = pi / ( tpeak * sqrt(1 - zeta_est^2) );
	end
	if isnan(omega_n)
	S = stepinfo(y, t);   % Ts(2%) ~ 4/(zeta*omega_n)
	if isfield(S,'SettlingTime') && ~isempty(S.SettlingTime) && S.SettlingTime > 0 && ~isnan(zeta_est) && zeta_est > 0
	omega_n = 4 / ( zeta_est * S.SettlingTime );
	end
	end
	if isnan(omega_n) && ~isnan(tr_10_90) && tr_10_90 > 0
	k_rise = 1.4;  % aprox para 10-90% en 2do orden subamortiguado
	omega_n = k_rise / tr_10_90;
	end
	
	% ----- Grafico -----
	figure;
	stairs(t_ms, y, 'LineWidth', 1.5); grid on;grid minor; hold on;
	xlabel('Tiempo [ms]'); ylabel('Salida');
	title(sprintf('Respuesta al escalon de %s', nameStr));
	
	% Líneas guía 10% y 90% y marcadores (en ms)
	if ~isnan(tr_10_90)
	yline(y10, '--', '10% yss', 'LabelHorizontalAlignment','left');
	yline(y90, '--', '90% yss', 'LabelHorizontalAlignment','left');
	xline(t10_ms, ':', 't10%');
	xline(t90_ms, ':', 't90%');
	plot([t10_ms t90_ms], [y10 y90], 'o', 'MarkerSize', 6, 'LineWidth', 1.2);
	end
	% Línea de yss
	yline(yss, '-', sprintf('yss=%.3g', yss), 'LabelHorizontalAlignment','left');
	
	% ----- Marca del OS en el pico -----
	if ~isnan(OS_percent) && OS_percent > 0
	% línea vertical desde yss a ypeak en tpeak
	plot([tpeak_ms tpeak_ms], [yss ypeak], '-', 'LineWidth', 1.2);
	% texto al costado derecho de la línea
	dx = 0.02 * (t_ms(end) - t_ms(1));
	text(tpeak_ms + dx, yss + 0.5*(ypeak - yss), ...
	sprintf('OS=%.3g%%', OS_percent), ...
	'Interpreter','none', 'Margin',2);
	% marcar el pico
	plot(tpeak_ms, ypeak, 's', 'MarkerSize', 6, 'LineWidth', 1.2);
	end
	
	% ----- Cuadro de metricas (abajo-derecha; chico) -----
	w = 0.15; h = 0.09; rmargin = 0.1; x = 1 - rmargin - w; y0 = 0.30;
	txtLines = {
		sprintf('t_r(10%%-90%%) = %.4g ms  (~%.4g Ts)', safeNum(tr_10_90*1000), safeNum(tr_10_90_T))
		sprintf('OS = %.4g%%',          safeNum(OS_percent))
		sprintf('zeta = %.4g',          safeNum(zeta_est))
		sprintf('omega_n = %.4g rad/s', safeNum(omega_n))
	};
	annotation('textbox', [x y0 w h], ...
	'String', txtLines, 'Interpreter','none', ...
	'FontName','Consolas', 'FontSize',9, ...
	'EdgeColor',[0.5 0.5 0.5], 'FaceAlpha',0.88, 'BackgroundColor','w');
	
	hold off;
	
	% ----- Salida -----
	info_ext = struct( ...
	'yss', yss, ...
	't10', t10, ...
	't90', t90, ...
	'tr_10_90', tr_10_90, ...
	'tr_10_90_T', tr_10_90_T, ...
	'OS_percent', OS_percent, ...
	'zeta_est', zeta_est, ...
	'omega_n', omega_n, ...
	'Ts', Ts);
	end
	
	% --------- helpers ---------
	function y = tern(cond, a, b)
	if cond, y = a; else, y = b; end
	end
	
	function v = safeNum(x)
	if isnan(x) || isinf(x), v = NaN; else, v = x; end
	end
	function [td, refd, yd, ud, ed, coefs] = sim_compensador_first_order( ...
	Gd, z0, p0, Kc, T, umin, umax, refmin, refmax, n_per_seg)
	
	% ===== Igual que tu PID: y(k+1)=b0*u(k)+b1*u(k-1)-a1*y(k)-a2*y(k-1) =====
	% Controlador (posición): u(k) = p0*u(k-1) + Kc*ed(k) - Kc*z0*ed(k-1)
	% ed(k) = refd(k) - yd(k)
	
	if nargin < 10 || isempty(n_per_seg), n_per_seg = 600; end
	if ~isa(Gd,'tf'), error('Gd debe ser tf discreto.'); end
	
	% --- Coefs planta exactamente como en tu ejemplo ---
	[numD, denD] = tfdata(Gd, 'v');
	b0 = numD(2);
	b1 = numD(3);
	a1 = denD(2);
	a2 = denD(3);
	coefs = struct('b0',b0,'b1',b1,'a1',a1,'a2',a2);
	
	% --- Tiempo y referencia (refmin -> refmax -> refmin) ---
	N  = 3*n_per_seg;
	td = (0:N-1)' * T;
	refd = [refmin*ones(n_per_seg,1);
	refmax*ones(n_per_seg,1);
	refmin*ones(n_per_seg,1)];
	
	% --- Inicialización idéntica a tu estilo ---
	yd = zeros(N,1);
	ed = zeros(N,1);
	ud = zeros(N,1);
	
	% --- Loop (misma estructura y orden que tu PID) ---
	for k = 3:N-1
	ed(k) = refd(k) - yd(k);
	
	% Compensador: u(k) = p0*u(k-1) + Kc*ed(k) - Kc*z0*ed(k-1)
	u_k = p0*ud(k-1) + Kc*ed(k) - Kc*z0*ed(k-1);
	
	
	if u_k>umax
	ud(k) = umax;
	elseif u_k<umin
	ud(k) = umin;
	else 
	ud(k) = u_k;
	end
	
	
	% Planta discreta (dividido por a0 implícito como en tu ejemplo)
	yd(k+1) = b0*ud(k) + b1*ud(k-1) - a1*yd(k) - a2*yd(k-1);
	end
	end
	
	function [td, refd, yd, ud, ed, coefs] = sim_compensador_second_order( ...
	Gd, z0_0, z0_1, p0_0, p0_1, Kc, T, umin, umax, refmin, refmax, n_per_seg)
	
	% ===== Planta: y(k+1)=b0*u(k)+b1*u(k-1)+b2*u(k-2)-a1*y(k)-a2*y(k-1) =====
	% Compensador (posición, 2º orden):
	% u(k) = (p0_0+p0_1)*u(k-1) - (p0_0*p0_1)*u(k-2) ...
	%      + Kc*[ e(k) - (z0_0+z0_1)*e(k-1) + (z0_0*z0_1)*e(k-2) ]
	% e(k) = refd(k) - yd(k)
	
	if nargin < 12 || isempty(n_per_seg), n_per_seg = 600; end
	if ~isa(Gd,'tf'), error('Gd debe ser tf discreto.'); end
	
	% --- Coefs planta (normalizados a a0=1) ---
	[numD, denD] = tfdata(Gd, 'v');
	numD = numD(:).'; denD = denD(:).';
	if abs(denD(1) - 1) > 1e-12
	numD = numD/denD(1);
	denD = denD/denD(1);
	end
	
	% Asegurar longitud mínima del denominador [1 a1 a2]
	if numel(denD) < 3, denD(end+1:3) = 0; end
	a1 = denD(2);
	a2 = denD(3);
	
	% b0,b1,b2 sin romper si faltan términos en el numerador
	b0 = 0; b1 = 0; b2 = 0;
	if numel(numD) >= 2, b0 = numD(2); end
	if numel(numD) >= 3, b1 = numD(3); end
	if numel(numD) >= 4, b2 = numD(4); end
	% Nota: esto respeta tu estilo previo (b0=numD(2), b1=numD(3), ...)
	
	coefs = struct('b0',b0,'b1',b1,'b2',b2,'a1',a1,'a2',a2, ...
	'z0_0',z0_0,'z0_1',z0_1,'p0_0',p0_0,'p0_1',p0_1,'Kc',Kc);
	
	% --- Tiempo y referencia (refmin -> refmax -> refmin) ---
	N  = 3*n_per_seg;
	td = (0:N-1)' * T;
	refd = [refmin*ones(n_per_seg,1);
	refmax*ones(n_per_seg,1);
	refmin*ones(n_per_seg,1)];
	
	% --- Inicialización ---
	yd = zeros(N,1);
	ed = zeros(N,1);
	ud = zeros(N,1);
	
	% --- Precalculos del compensador ---
	P1 = (p0_0 + p0_1);
	P2 = (p0_0 * p0_1);
	Z1 = (z0_0 + z0_1);
	Z2 = (z0_0 * z0_1);
	
	% --- Loop ---
	for k = 3:N-1
	ed(k) = refd(k) - yd(k);
	
	% Compensador 2º orden
	u_k = P1*ud(k-1) - P2*ud(k-2) + Kc*( ed(k) - Z1*ed(k-1) + Z2*ed(k-2) );
	
	% Saturación
	if u_k > umax
	ud(k) = umax;
	elseif u_k < umin
	ud(k) = umin;
	else
	ud(k) = u_k;
	end
	
	% Planta discreta
	yd(k+1) = b0*ud(k) + b1*ud(k-1) + b2*ud(k-2) - a1*yd(k) - a2*yd(k-1);
	end
	end
	
\end{lstlisting}

\begin{lstlisting}[style=matlabstyle,
	caption={Modelo mejorado de la planta mediante los datos recolectados del osciloscopio},
	label={matlab:modelo}
	]

	close all
	clear all
	
	% Cargar CSV
	data = readmatrix('opltab.csv');
	
	% Ignorar cabeceras, quedarte solo con datos numéricos
	u = data(3:end, 5); % columna E = salida
	y = data(3:end, 11); % columna K = entrada
	Ts = 0.000099999997474; % tiempo de muestreo de B3
	
	% Crear objeto de identificación
	data_id = iddata(y, u, Ts);
	
	% Estimar una función de transferencia discreta (ejemplo: 2 polos, 1 cero)
	sys0 = tfest(data_id, 2, 0);
	
	% % Ver resultado
	%  step(sys);
	% % Crear datos de identificación
	% data_id = iddata(y, u, Ts);
	% 
	% % Probar con distintos órdenes
	% sys1 = tfest(data_id, 1, 0); % 1 polo
	% sys2 = tfest(data_id, 2, 1); % 2 polos, 1 cero
	% 
	% % Comparar
	figure;
	compare(data_id,sys0);
	
	G = tf(sys0);
	T = 1.25e-3;
	Gd = c2d(G, T, 'zoh');
	disp('Planta digital G(z):')
	zpk(Gd)
	controlSystemDesigner(Gd)
	
	save('planta.mat', 'G');
\end{lstlisting}


\begin{lstlisting}[style=matlabstyle,caption={Segunda hoja de cálculos utilizada.},label={matlab:calculo2}]
	close all
	clear all
	
	addpath('..\Lab1\')
	
	
	load('planta.mat');
	%% Analizamos en el tiempo
	[tr, ts, wn] = plot_step_info(G);
	disp('Planta continua G(s):')
	G
	zpk(G)
	
	%% Paso 1: Definir periodo de muestreo
	% Se busca una mejora de 4 en el tiempo de rising , o sea tr = 10 ms
	
	T = 1.25e-3;
	%Gcl = feedback(G,1);
	%figure;
	%step(Gcl)
	%T = stepinfo(Gcl).SettlingTime/8;
	
	%% Paso 2: Digitalizar con ZOH
	Gd = c2d(G, T, 'zoh');
	disp('Planta digital G(z):')
	Gd
	zpk(Gd)
	figure;
	pzmap(Gd);
	title('Lugar de raíces de G(z).')
	
	zgrid;
	z = tf([1 0],1,T);
	
	%% Paso 3: Análisis en tiempo discreto lazo abierto
	figure;
	
	% Obtenemos salida y tiempo de la función step
	[y, t] = step(Gd);
	
	% Graficamos con stairs (propio de señales discretas)
	stairs(t*1000, y, 'LineWidth',1.4);
	
	title('Respuesta al escalón de la planta digitalizad')
	xlabel('Tiempo [ms]');
	ylabel('Salida');
	grid on;
	ax = gca;
	ax.XMinorTick = 'on';  % activamos minor ticks
	ax.YMinorTick = 'on';
	grid minor;
	
	%% Paso 4: Definimimos los valores de la simulación
	umin = 0;
	umax =  4.08;
	refmin = 1.5;
	refmax = 2.5;
	n_per_seg = 500;
	%% Paso 5: Lugar de raíces para zita = 0.7, compensador P
	%figure
	
	%rlocus(Gd)
	%zgrid     % agrega la grilla en el plano-z
	z0 = 0.380927627742363;
	p0 = 0.836970379456654;
	K = 0.226961597782889;
	C_1 = (z-z0)/(z-p0);
	figure;
	Gc1 = Gd*C_1;
	rlocus(Gc1);
	title('Lugar de raíces de G(z) compensado con un lag-filter de 1º orden, anulando el polo más rápido de la planta.')
	
	zgrid     % agrega la grilla en el plano-z
	%pause;
	%[K, ~] = rlocfind(Gc1);   % hacés click donde querés los polos
	%K = 0.73139;
	%K = 0.82543;
	%K = 1.796463473892606;
	Gc1f = feedback(K*Gc1,1);
	%figure;
	%step(Gc1f)
	info = stepinfo(Gc1f);
	info
	
	info_ext = plot_step_annot(Gc1f, 'la planta compensada con un lag-filter de 1º orden, anulando el polo más rápido de la planta.');
	%step(Gc3f);
	info = stepinfo(Gc1f);
	info
	info_ext
	
	[td, refd, yd, ud, ed, coefs] = sim_compensador_first_order( ...
	Gd, z0, p0, K, T, umin, umax, refmin, refmax, n_per_seg);
	
	Nini = 100;              % descartar al inicio
	Nfin = 100;              % descartar al final
	idx0 = Nini + 1;         % índice inicial válido
	idx1 = length(td) - Nfin; % índice final válido
	
	td   = td(idx0:idx1)-td(idx0);
	refd = refd(idx0:idx1);
	yd   = yd(idx0:idx1);
	ud   = ud(idx0:idx1);
	ed   = ed(idx0:idx1);
	
	% Si querés que el tiempo arranque en 0
	
	
	
	
	
	figure;
	
	% ---------- Subplot 1 ----------
	axAbs = subplot(2,1,1);  % eje izquierdo (absoluto)
	hold(axAbs,'on'); grid(axAbs,'on');
	
	% y[k] en rojo (abs)
	hY = stairs(axAbs, td, yd, 'r-', 'LineWidth', 1.4);
	
	% Armamos eje derecho transparente
	axPct = axes('Position', get(axAbs,'Position'), ...
	'Color','none', 'YAxisLocation','right', ...
	'XLim', get(axAbs,'XLim'), 'XTick',[], 'Box','off');
	hold(axPct,'on');
	
	% r[k] en negro, graficado en %
	ref_pct = 100*(refd - refmin)/(refmax - refmin);
	hR = stairs(axPct, td, refd, 'k--', 'LineWidth', 1.2);
	
	% ======= Cálculo de límites con margen =======
	yd_pct  = 100*(yd - refmin)/(refmax - refmin);
	pctAll  = [yd_pct; ref_pct];
	
	% Valores extremos en % con margen del 5 %
	rawMin = min(pctAll);
	rawMax = max(pctAll);
	span   = rawMax - rawMin;
	pctMin = rawMin - 0.05*span;
	pctMax = rawMax + 0.05*span;
	
	% Redondeamos a múltiplos de 20 para ticks
	stepPct = 20;
	tickMin = stepPct*floor(pctMin/stepPct);
	tickMax = stepPct*ceil (pctMax/stepPct);
	pctTicks = tickMin:stepPct:tickMax;
	
	% Convertimos a absolutos
	valTicks = refmin + (pctTicks/100)*(refmax - refmin);
	
	% Aplicamos a ambos ejes
	set(axAbs,'YLim',[valTicks(1) valTicks(end)],'YTick',valTicks);
	set(axPct,'YLim',[valTicks(1) valTicks(end)],'YTick',valTicks,...
	'YTickLabel',compose('%.0f %%',pctTicks));
	
	% Etiquetas
	xlabel(axAbs,'Tiempo [s]');
	ylabel(axAbs,'Respuesta [valor absoluto]');
	ylabel(axPct,'Escala relativa a ref [%]');
	title(axAbs,'Respuesta discreta con compensador 1º orden, anulando el polo más rápido de la planta.');
	legend(axAbs, [hY, hR], {'y[k]', 'r[k]'}, 'Location', 'best');
	
	% ---------- Subplot 2: esfuerzo de control ----------
	axU = subplot(2,1,2);
	stairs(axU, td, ud, 'LineWidth',1.2); grid(axU,'on');
	yline(axU, umax,'r:'); yline(axU, umin,'r:');
	xlabel(axU,'Tiempo [s]'); ylabel(axU,'u[k]');
	title(axU,'Esfuerzo de control con saturación');
	
	% --- Margen de 5% en eje Y ---
	uMin = min(ud);
	uMax = max(ud);
	span = uMax - uMin;
	ylim(axU, [uMin - 0.05*span, uMax + 0.05*span]);
	
	
	grid(axAbs,'on');    % grilla principal
	grid(axAbs,'minor'); % grilla secundaria
	
	grid(axPct,'on');
	grid(axPct,'minor');
	
	grid(axU,'on');
	grid(axU,'minor');
	
	
	
	%% Paso 7: Lugar de raices para zita = 0.7 y y ESS = 0, compensador PI de segundo orden
	z0_0 = 0.380927627742363;
	z0_1 = 0.821584532561100;
	
	
	p0_0 = 0.753973536568200;
	p0_1 = 1;
	K = 0.287529112579611;
	C_3 = ((z-z0_0)*(z-z0_1))/((z-p0_0)*(z-p0_1));
	Gc3 =Gd*C_3; %z=0.8433 %8433
	figure;
	rlocus(Gc3)
	title('Lugar de raíces de G(z) compensado con un lag-filter de 2º orden, fijando un polo en z = 1 y anulando ambos polos de la planta.');
	
	zgrid
	%pause;
	%[K, ~] = rlocfind(Gc3);   % hacés click donde querés los polos
	%K = 1.14;
	%figure;
	Gc3f= feedback(K*Gc3,1);
	
	info_ext = plot_step_annot(Gc3f, 'la planta compensada con un lag-filter de 2º orden, fijando un polo en z = 1 y anulando ambos polos de la planta.');
	%step(Gc3f);
	info = stepinfo(Gc3f);
	info
	info_ext
	
	
	% --- Llamada a tu simulador (misma interfaz que tu first_order) ---
	[td, refd, yd, ud, ed, coefs] = sim_compensador_second_order( ...
	Gd, z0_0, z0_1, p0_0, p0_1, K, T, umin, umax, refmin, refmax, n_per_seg);
	
	
	Nini = 100;              % descartar al inicio
	Nfin = 100;              % descartar al final
	idx0 = Nini + 1;         % índice inicial válido
	idx1 = length(td) - Nfin; % índice final válido
	
	td   = td(idx0:idx1)-td(idx0);
	refd = refd(idx0:idx1);
	yd   = yd(idx0:idx1);
	ud   = ud(idx0:idx1);
	ed   = ed(idx0:idx1);
	
	
	
	
	figure;
	
	% ---------- Subplot 1 ----------
	axAbs = subplot(2,1,1);  % eje izquierdo (absoluto)
	hold(axAbs,'on'); grid(axAbs,'on');
	
	% y[k] en rojo (abs)
	hY = stairs(axAbs, td, yd, 'r-', 'LineWidth', 1.4);
	
	% Armamos eje derecho transparente
	axPct = axes('Position', get(axAbs,'Position'), ...
	'Color','none', 'YAxisLocation','right', ...
	'XLim', get(axAbs,'XLim'), 'XTick',[], 'Box','off');
	hold(axPct,'on');
	
	% r[k] en negro, graficado en %
	ref_pct = 100*(refd - refmin)/(refmax - refmin);
	hR = stairs(axPct, td, refd, 'k--', 'LineWidth', 1.2);
	
	% ======= Cálculo de límites con margen =======
	yd_pct  = 100*(yd - refmin)/(refmax - refmin);
	pctAll  = [yd_pct; ref_pct];
	
	% Valores extremos en % con margen del 5 %
	rawMin = min(pctAll);
	rawMax = max(pctAll);
	span   = rawMax - rawMin;
	pctMin = rawMin - 0.05*span;
	pctMax = rawMax + 0.05*span;
	
	% Redondeamos a múltiplos de 20 para ticks
	stepPct = 20;
	tickMin = stepPct*floor(pctMin/stepPct);
	tickMax = stepPct*ceil (pctMax/stepPct);
	pctTicks = tickMin:stepPct:tickMax;
	
	% Convertimos a absolutos
	valTicks = refmin + (pctTicks/100)*(refmax - refmin);
	
	% Aplicamos a ambos ejes
	set(axAbs,'YLim',[valTicks(1) valTicks(end)],'YTick',valTicks);
	set(axPct,'YLim',[valTicks(1) valTicks(end)],'YTick',valTicks,...
	'YTickLabel',compose('%.0f %%',pctTicks));
	
	% Etiquetas
	xlabel(axAbs,'Tiempo [s]');
	ylabel(axAbs,'Respuesta [valor absoluto]');
	ylabel(axPct,'Escala relativa a ref [%]');
	title(axAbs,'Respuesta discreta con compensador 2º orden, fijando un polo en z = 1 y anulando ambos polos de la planta');
	legend(axAbs, [hY, hR], {'y[k]', 'r[k]'}, 'Location', 'best');
	
	% ---------- Subplot 2: esfuerzo de control ----------
	axU = subplot(2,1,2);
	stairs(axU, td, ud, 'LineWidth',1.2); grid(axU,'on');
	yline(axU, umax,'r:'); yline(axU, umin,'r:');
	xlabel(axU,'Tiempo [s]'); ylabel(axU,'u[k]');
	title(axU,'Esfuerzo de control con saturación');
	
	% --- Margen de 5% en eje Y ---
	uMin = min(ud);
	uMax = max(ud);
	span = uMax - uMin;
	ylim(axU, [uMin - 0.05*span, uMax + 0.05*span]);
	
	
	
	
	
	
	
	grid(axAbs,'on');    % grilla principal
	grid(axAbs,'minor'); % grilla secundaria
	
	grid(axPct,'on');
	grid(axPct,'minor');
	
	grid(axU,'on');
	grid(axU,'minor');
	
	
	%% impresion antigua
	% % --- Gráficas estilo informe ---
	% figure; 
	% subplot(2,1,1);
	% stairs(td, refd, 'k--','LineWidth',1.0); hold on;
	% stairs(td, yd,   'LineWidth',1.4);
	% grid on; xlabel('Tiempo [s]'); ylabel('Respuesta de la planta/Referencia');
	% title('Respuesta discreta con compensador 2º orden'); legend('r[k]','y[k]','Location','best');
	% 
	% subplot(2,1,2);
	% stairs(td, ud, 'LineWidth',1.2);
	% yline(umax,'r:'); yline(umin,'r:');
	% grid on; xlabel('Tiempo [s]'); ylabel('u[k]');
	% title('Esfuerzo de control con saturación');
\end{lstlisting}
\section{Código desarrollado para el PSoC }
\begin{lstlisting}[style=cstyle,
	caption={Código desarrollado para la implementación de los compensadores con el PSoC en lenguaje C.},
	label={c:psoc}
	]
	
	\\XDXDXD
\end{lstlisting}
