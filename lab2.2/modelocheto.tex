\subsection{Nuevo Modelado de la Planta}

Con el objetivo de mejorar las prestaciones y evitar las limitaciones derivadas de las tolerancias de los componentes pasivos (resistencias y capacitores de baja precisión), se decidió realizar un modelado alternativo de la planta utilizando \textsc{Matlab}.

Para ello, se empleó el osciloscopio a fin de muestrear la respuesta del sistema ante una entrada escalón. Posteriormente, mediante las funciones de identificación de sistemas en \textsc{Matlab}, se llevó a cabo una estimación de segundo orden que aproxima de manera adecuada el comportamiento dinámico de la planta.  

\insertarfigura{./Img/estimado.png}
{Respuesta al escalón de la planta estimada mediante identificación en \textsc{Matlab}.}
{fig:PlantaEstimada}{0.8}

De esta estimación se obtuvo la siguiente función de transferencia característica del sistema identificado:
\begin{equation}
	G(s) \;=\; \frac{1.2472e05}{(s+771.3)(s+157.3)},
\end{equation}
donde los parámetros $K$, $a$ y $b$ fueron ajustados automáticamente por el algoritmo de estimación en función de los datos experimentales.  

A partir de este nuevo modelo, y siguiendo procedimientos análogos a los presentados en secciones anteriores, se diseñaron dos compensadores que fueron implementados de manera similar al caso inicial.

