% ================== BORRADOR PUERCO EN FORMATO IEEE ==================
\documentclass[conference]{IEEEtran}

% ======= Preambulo mínimo pero útil para escribir rápido =======
\usepackage[spanish, es-nodecimaldot]{babel}
\usepackage[utf8]{inputenc}
\usepackage[T1]{fontenc}
\usepackage{amsmath, amssymb, amsfonts}
\usepackage{siunitx}

\usepackage{siunitx}
\sisetup{
	output-decimal-marker = {,}, % coma decimal en vez de punto
	group-separator = {.},       % separador de miles con punto
	group-minimum-digits = 4,    % agrupa desde 4 dígitos
	per-mode = symbol            % para usar "/" en unidades
}

\usepackage{graphicx}
\usepackage{caption}
\usepackage{subcaption}
\usepackage{booktabs}
\usepackage{hyperref}
\usepackage{enumitem}
\setlist{leftmargin=*, itemsep=2pt, topsep=4pt}
\usepackage{xcolor}

% ======= Macros rápidas =======
\newcommand{\zetaD}{\zeta}
\newcommand{\wn}{\omega_n}
\newcommand{\Ts}{T_s}
\newcommand{\ESS}{\mathrm{ESS}}
\newcommand{\Gz}{G(z)}
\newcommand{\Gs}{G(s)}

% ======= Metadatos =======
\title{Práctica de Laboratorio 2: Diseño de controladores por el método de lugar de raíces.}

\author{
	\IEEEauthorblockN{Elías Álvarez}
	\IEEEauthorblockA{Carrera de Ing. Electrónica \\ 
		Universidad Católica Nuestra Señora de la Asunción \\
		Asunción, Paraguay \\
		Email: elias.alvarez@universidadcatolica.edu.py}
	\and
	\IEEEauthorblockN{Tania Romero}
	\IEEEauthorblockA{Carrera de Ing. Electrónica \\
		Universidad Católica Nuestra Señora de la Asunción \\
		Asunción, Paraguay \\
		Email: tania.romero@universidadcatolica.edu.py}
		\and
	\IEEEauthorblockN{\hspace*{3.5em}Docente: Lic. Montserrat González}
	\IEEEauthorblockA{
		\hspace*{3.5em}Facultad de Ingeniería \\
		\hspace*{3.5em}Universidad Católica Nuestra Señora de la Asunción \\
		\hspace*{3.5em}Asunción, Paraguay}
	\and
	\IEEEauthorblockN{\hspace*{3.5em}Docente: PhD. Enrique Vargas}
	\IEEEauthorblockA{%
		\hspace*{3.5em}Facultad de Ingeniería \\  % ← empuja a la derecha
		\hspace*{3.5em}Universidad Católica Nuestra Señora de la Asunción \\
		\hspace*{3.5em}Asunción, Paraguay}
}

\begin{document}
	\maketitle
	
	\section{Introducción}

En esta práctica se aborda el diseño de controladores mediante la técnica de \textbf{ubicación arbitraria de polos}, un método fundamental en el análisis y síntesis de sistemas de control en el espacio de estados. El propósito principal es modificar la dinámica de una planta determinada para que el sistema en lazo cerrado cumpla con \textit{especificaciones deseadas de respuesta transitoria}, tales como el tiempo de establecimiento y el coeficiente de amortiguamiento.

A través del cálculo de las matrices del sistema, su discretización e implementación en un \textbf{controlador digital basado en el PSoC}, se busca comprender cómo la realimentación de estados permite alterar el comportamiento dinámico y mejorar el desempeño del sistema. 

Finalmente, se comparan los resultados obtenidos mediante simulación en \texttt{MATLAB} con los resultados experimentales, analizando la efectividad del diseño y las posibles diferencias entre el modelo teórico y la práctica.

\section*{Objetivos}

\begin{itemize}
	\item Diseñar un controlador que modifique la dinámica de la planta para satisfacer condiciones específicas de la respuesta transitoria del sistema de control en lazo cerrado.
	\item Asegurar que el sistema regulado sea estable.
	\item Observar y analizar los efectos del controlador en el comportamiento dinámico del sistema.
	\item Considerar distintos métodos para el ajuste de los parámetros del controlador y analizar los resultados obtenidos.
	\item Diseñar el sistema de control en \texttt{MATLAB} e implementar la ecuación en diferencia correspondiente en el \texttt{PSoC}.
\end{itemize}

	
	\section{Desarrollo}
\subsection{Modelado}
\subsubsection{Obtener la función de transferencia de lazo abierto}

\paragraph{Impedancia de realimentación (paralelo $R_f \parallel C_f$).}

\insertarfigura{./img/circuito.png}{Circuito esquemático.}{fig:circuitoini}{1}


\begin{equation}
	Z_f \;=\; R_f \parallel \frac{1}{sC_f}
	\;=\; \frac{R_f\cdot \frac{1}{sC_f}}{R_f + \frac{1}{sC_f}}
	\;=\; \frac{R_f}{1 + s R_f C_f}
	\label{eq:Zf}
\end{equation}

\insertarfigura{./img/CircRealim.png}{Circuito Modo inversor.}{fig:impedanciaReal}{0.5}

\paragraph{Esquema general del AO (entrada en $R_i$ al nodo inversor.}
Sea $V_{\text{ref}}$ la tensión aplicada a la entrada no inversora (por ejemplo, $V_{\text{ref}}=0$). La salida es:
\[
V_o \;=\;  V_o\big|_{V_i=0}
\]

\textbf{(A) Aporte de $V_i$ con $V_{\text{ref}}=0$ (no inversora a masa AC).}
Por cortocircuito virtual, $V_n=V_p=0$. Corrientes:
\[
i_1=\frac{V_i-0}{R_i}
\]
KCL en el nodo inversor ($i_1=i_2$):
\[
\frac{V_i}{R_i}=\frac{-V_o}{Z_f}
\;\;\Rightarrow\;\;
\frac{V_o}{V_i}=-\frac{Z_f}{R_i}
\]
Con \eqref{eq:Zf}:
\begin{equation}
	\frac{V_o}{V_i}\Bigg|_{V_{\text{ref}}=0}
	= -\,\frac{Z_f}{R_i}
	= -\,\frac{R_f}{R_i}\,\frac{1}{1+sR_f C_f}
	\label{eq:gain_inverting}
\end{equation}


\paragraph{Encadenamiento de etapas (lazo abierto total).}
Si el AO3 es buffer ($G_3(s)=1$) y AO1, AO2 tienen cada uno la forma \eqref{eq:gain_inverting}, entonces la función de lazo abierto (desde la \emph{señal de entrada} hasta la \emph{salida del último AO}), es el producto de las ganancias de las etapas activadas por esa señal, finalmente la ganancia será:
\[
G_{\text{ol}}(s)
= \Big(-\frac{Z_{f1}(s)}{R_{i1}}\Big)\,
\Big(-\frac{Z_{f2}(s)}{R_{i2}}\Big)\
\]
\paragraph{Matlab} Se muestra el siguiente código de Matlab, el cual arma las dos etapas de 1er orden, realiza la multiplicación en cascada para el lazo abierto y mide los tiempos de respuesta al escalón.

\onecolumn

\begin{lstlisting}[style=matlabstyle,caption={Script en Matlab},label={lst:mat}]
	
	%% Definicion de parametros
	R_1 = 15e3;
	R_3 = 15e3;
	C_2 = 100e-9;
	R_2 = 75e3;
	R_4 = 75e3;
	C_1 = 0.2e-6;
	%% Generar funcion de transferencia d
	numStage = [-R_3/R_1 -R_4/R_2];
	denStage = { [C_2*R_3 1], [C_1*R_4 1] };
	% Usamos celdas para guardar los tf de cada stage
	Gstage = cell(1,2);
	G = 1;
	for i = 1:2
	Gstage{i} = tf(numStage(i), denStage{i});
	G = G*Gstage{i};
	end
	%% Analizamos en el tiempo
	[tr, ts, wn] = plot_step_info(G);
	
	function [tr, ts, wn] = plot_step_info(G)
		info = stepinfo(G);
		tr = info.RiseTime;
		ts = info.SettlingTime;
		wn = 1.8 / tr;
		t_end = 1.1 * ts;
		if ~isfinite(t_end) || t_end <= 0, 
			t_end = 5 * max(tr, 1e-3); 
		end
		t = linspace(0, t_end, 2*pi/wn);
		[y, tout] = step(G, t);
		figure; plot(tout, y, 'LineWidth', 1.4); grid on; hold on;
		xlabel('Tiempo [s]'); ylabel('Salida');
		title(sprintf('Escalón: tr=%.4gs, ts=%.4gs, \\omega_n=%.4g rad/s', tr, ts, wn));
		xline(tr, '--', sprintf('  t_r=%.3g s', tr), 'LabelOrientation','horizontal');
		xline(ts,  ':', sprintf('  t_s=%.3g s', ts), 'LabelOrientation','horizontal');
		legend('G(t)', 'Location', 'best');
	end
	
\end{lstlisting}
\twocolumn
\insertarfigurawide{./img/EscalonLazoAb.png}{Escalón del lazo Abierto.}{fig:escalonLazoab}{1}
\paragraph{Resultados numéricos(respuesta al escalón)} Se sacan los coeficientes de la figura~\ref{fig:escalonLazoab}:
\begin{itemize}
	\item \textbf{Rise time $t_r \approx 0.0332 s$ (33.2ms)}
	\item \textbf{Settling time $t_s(2\%) \approx 0.0603 s$ (60.3ms)}
	\item \textbf{Estimación $\omega_n \approx \frac{1.8}{t_r} s$ (54.17rad/s)}
\end{itemize}





\subsection{Modelado del sistema}
\subsubsection{Ajustar empíricamente el PID buscando sobrepaso máximo del $10\%$}

\subsubsection{Analizar el efecto del control}

\subsection{Implementación del sistema}

\subsubsection{Discretización del sistema}
\subsubsection{Discretización del compensador}
\twocolumn
\subsubsection{ Implementación con un tiempo de muestreo de 10 veces el ancho de banda del sistema
}


Si muestreamos 10 veces el ancho de banda:
\[
f_s = 10 f_{\mathrm{bw}}, \qquad T = \frac{1}{f_s}.
\]

La simulación se realiza con:
\[
\texttt{sim\_pid\_euler\_ideal(G,\,T,\,Kp,\,Ti,\,Td,\,1);}
\]

En la Figura~\ref{fig:fs10} se observa que la respuesta discreta no reproduce bien a la continua: se introduce más retardo y el seguimiento al escalón empeora.
Un muestreo de 10 veces el ancho de banda apenas alcanza para representar la dinámica dominante. El efecto combinado del ZOH y la discretización generan un retardo en la fase, reduciendo el desempeño.  

\begin{figure}[!t]
	\centering
	\includegraphics[width=\columnwidth]{img/fs10.png}
	\caption{Respuesta con $f_s=10 f_{\mathrm{bw}}$: la discretización no es suficiente y la respuesta discreta se aparta del continuo.}
	\label{fig:fs10}
\end{figure}



\subsubsection{ Implementación con un tiempo de muestreo de 30 veces el ancho de banda del sistema
}

Al triplicar la frecuencia de muestreo:
\[
f_s = 30 f_{\mathrm{bw}}, \qquad T = \frac{1}{f_s},
\]
Podemos ver en la Figura~\ref{fig:fs30} muestra una mejora clara: la respuesta discreta se asemeja mucho más a la continua, con menor error con respecto a la continua.

\begin{figure}[!t]
	\centering
	\includegraphics[width=\columnwidth]{img/fs30.png}
	\caption{Respuesta con $f_s=30 f_{\mathrm{bw}}$: el lazo discreto reproduce con mayor fidelidad al continuo.}
	\label{fig:fs30}
\end{figure}

\paragraph{Limitación con Ziegler--Nichols.}  
Al usar parámetros obtenidos con ZN, el control logra buen seguimiento, pero el esfuerzo de control $u_d[k]$ supera en más de dos órdenes de magnitud la capacidad de los DAC del PSoC (Figura~\ref{fig:zn_effort}). Esto vuelve la solución impracticable.

\begin{figure}[!t]
	\centering
	\includegraphics[width=\columnwidth]{img/zn_effort.png}
	\caption{Sintonía ZN: buena respuesta, pero esfuerzo $u_d[k]$ excesivo (no implementable).}
	\label{fig:zn_effort}
\end{figure}

Para evidenciarlo, añadimos ruido y saturación al esfuerzo en la simulación. El resultado (Figura~\ref{fig:sat_noise}) confirma que el sistema no puede seguir al escalón bajo esas condiciones.

\begin{figure}[!t]
	\centering
	\includegraphics[width=\columnwidth]{img/sat_noise.png}
	\caption{PID discreto con ruido y saturación: se observa windup y pérdida total del seguimiento.}
	\label{fig:sat_noise}
\end{figure}


\onecolumn
\subsubsection*{Simulación con saturación y ruido}

La función utilizada en la sección anterior es la siguiente:

\begin{lstlisting}[language=Matlab,style=matlabstyle, caption={PID discreto con saturación y ruido}, label={lst:pid_sat_noise}]
	function sim_pid_euler(G, T, Kp, Ti, Td, stepSize, sigma, umin, umax)
	% --- Control continuo (referencia) ---
	numC = Kp*[Ti*Td, Ti, 1]; denC = [Ti, 0];
	controlador = tf(numC, denC);
	cloop_c = feedback(controlador*G, 1);
	info  = stepinfo(cloop_c);
	tend  = 3*info.SettlingTime;
	wn    = 1.8/info.RiseTime;
	Tstep = (2*pi/wn)/1000;
	t  = 0:Tstep:tend;
	td = 0:T:tend;
	
	% Referencia con ruido
	refc = stepSize*ones(size(t)) + sigma*randn(size(t));
	yc   = lsim(cloop_c, refc, t);
	refd = interp1(t, refc, td, 'linear', 'extrap');
	
	% --- Planta discreta (ZOH) ---
	Gd = c2d(G, T, 'zoh');
	[numD, denD] = tfdata(Gd, 'v');
	b0 = numD(2); b1 = numD(3);
	a1 = denD(2); a2 = denD(3);
	
	% Inicialización
	yd = zeros(size(td));
	ed = zeros(size(td));
	ud = zeros(size(td));
	
	% --- PID incremental con saturación ---
	for k = 3:numel(td)-1
	ed(k) = refd(k) - yd(k);
	u_inc = ud(k-1) + Kp*((1+T/Ti+Td/T)*ed(k) ...
	-(1+2*Td/T)*ed(k-1) ...
	+(Td/T)*ed(k-2));
	% Saturación
	if u_inc > umax
	ud(k) = umax;
	elseif u_inc < umin
	ud(k) = umin;
	else
	ud(k) = u_inc;
	end
	% Planta discreta
	yd(k+1) = b0*ud(k) + b1*ud(k-1) - a1*yd(k) - a2*yd(k-1);
	end
	
	% --- Gráficos ---
	yshift = circshift(yd, -2);
	figure; hold on; grid on;
	yyaxis left
	plot(t*1000, refc, ':','Color',[0.7 0.7 0.7]);
	plot(t*1000, yc,  '--','LineWidth',1.3);
	stairs(td*1000, yshift,'-','LineWidth',1.2);
	ylabel('Salida');
	yyaxis right
	plot(td*1000, ud, 'r--','LineWidth',1.2);
	ylabel('u_d[k]');
	xlabel('Tiempo [ms]');
	title(sprintf('PID discreto: Kp=%.3g, Ti=%.3g, Td=%.3g', Kp, Ti, Td));
	legend('Referencia','Continuo ideal','Discreto sat+ruido','u_d');
	end
\end{lstlisting}

\twocolumn

\subsubsection{Implementación del modelo alternativo y análisis con ruido y saturación}

Dado que la sintonía inicial de Ziegler--Nichols exige un esfuerzo de control incompatible con el actuador disponible, nos centramos en implementar el \emph{modelo alternativo} (PID ajustado por prueba y error con discretización incremental).  

\paragraph{Efectos del ruido (escenario amplificado).}
Para visualizar con claridad la sensibilidad al ruido (especialmente por el término derivativo), forzamos un caso más severo: desviación estándar del \(10\%\) de la amplitud de referencia. En la simulación:


\begin{lstlisting}[language=Matlab,breaklines=true,basicstyle=\ttfamily\footnotesize]
	sim_pid_euler(G,T/30,Kp,Ti,Td,2.5,0.25,0,5);
\end{lstlisting}

La Figura~\ref{fig:ruido_10pct} evidencia el efecto del ruido sobre \(u_d[k]\) y la salida discreta, pero sigue dando un resultado aceptable en el seguimiendo de la referencia aun en este caso extremo.

\begin{figure}[!t]
	\centering
	\includegraphics[width=\columnwidth]{img/ruido_10pct.png}% <-- exportá esta imagen
	\caption{Caso ruidoso (sigma $=10\%$ de la amplitud): ondulación amplificada por sensibilidad del término derivativo.}
	\label{fig:ruido_10pct}
\end{figure}

\paragraph{Efectos de la saturación (windup) sin ruido.}
Para aislar el fenómeno de \emph{windup} quitamos el ruido y exigimos amplitud elevada, manteniendo límites del actuador:
\begin{lstlisting}[language=Matlab,breaklines=true,basicstyle=\ttfamily\footnotesize]
	sim_pid_euler(G,T/30,Kp,Ti,Td,5,0,0,5);
\end{lstlisting}

La Figura~\ref{fig:windup} muestra la saturación de \(u_d[k]\) y la consecuente degradación del seguimiento: lento despegue y respuesta sobreamortiguada.

\begin{figure}[!t]
	\centering
	\includegraphics[width=\columnwidth]{img/windup.png}% <-- exportá esta imagen
	\caption{Caso con saturación y sin ruido: evidencia de windup e imposibilidad de seguimiento para amplitudes elevadas.}
	\label{fig:windup}
\end{figure}

\subsubsection{Implementar con un tiempo de muestreo de 10 veces el ancho de banda del sistema y evaluar el comportamiento}

\insertarfigurawide{./img/Astrom1.png}{Respuesta al escalón con $f_s=30 f_{\mathrm{bw}}$ (curva azul), su versión continua (curva azul puntuada) y el esfuerzo generado por el compensador (curva naranja puntuada) saturable, utilizando el primer compensador diseñado, añadiendo ruido a la entrada (curva gris clara) con \(\sigma = 10\%\), utilizando las modificaciones propuestas en el capitulo 6 del Astr\"om.}{fig:astrom1}{0.8}

\subsubsection{Implementación del PID de Astr\"om}

\paragraph{Controlador Continuo}
Para esta implementación, se procedió a añadir un polo al término derivativo, con el objetivo de disminuir su efecto amplificador sobre el ruido de alta frecuencia. Asimismo, la ecuación de transferencia del compensador queda de la siguiente forma:
\begin{equation}
	C(s)=K_p\left(1+\frac{1}{T_is}+\frac{T_ds}{1+\frac{T_d}{N}s}\right)
\end{equation}

\paragraph{Forma incremental (discreta) según Astr\"om}
Aplicando la aproximación de Euler hacia atrás para las derivadas, previo a derivar ambas partes de la ecuación (buscando reducir la parte integral), se obtiene una ecuación en diferencia de segundo orden entre el esfuerzo \(u[k]\) y el error \(e[k]\):


\begin{equation}
	\begin{aligned}
			&U_0u[k]+U_1u[k-1]+U_2u[k-2] =\\
			&E_0e[k]+E_1e[k-1]+E_2e[k-2]
	\end{aligned}
\end{equation}
Despejando \(u[k]\):
\begin{equation}
	\begin{aligned}
	u[k]=\frac{-U_1}{U_0}u[k-1]+\frac{-U_2}{U_0}u[k-2]\\
	+\frac{E_0}{U_0}e[k]+\frac{E_1}{U_0}e[k-1]+\frac{E_2}{U_0}e[k-2]
	\end{aligned}
\end{equation}
Dado:
\[
	\begin{aligned}
		\alpha_2 &= \frac{T_d T_i}{T^2}, 
		&\qquad \alpha_1 &= \frac{N T_i}{T}, \\[6pt]
		\beta_2  &= \frac{K_p\,(N T_d T_i + T_d T_i)}{T^2}, 
		&\qquad \beta_1 &= \frac{K_p\,(N T_i + T_d)}{T}, \\[6pt]
		\beta_0  &= K_p N, 
		&\qquad U_0 &= \alpha_2 + \alpha_1, \\[6pt]
		U_1 &= -(2\alpha_2 + \alpha_1), 
		&\qquad U_2 &= \alpha_2, \\[6pt]
		E_0 &= \beta_2 + \beta_1 + \beta_0, 
		&\qquad E_1 &= -(2\beta_2 + \beta_1), \\[6pt]
		E_2 &= \beta_2
	\end{aligned}
\]

Utilizando lo calculado, se desarrolló la función \ref{lst:mat6} de Matlab para la simulación del nuevo compensador.
Se observa en las figura \ref{fig:astrom1} que la respuesta ante el ruido mejora un poco, esto es gracias al polo que se ingresó en el termino derivativo $T_d$, cuya motivo de mejora se denotó anteriormente. Además, en la figura \ref{fig:astrom2} se puede ver como el sistema reacciona ante la saturación.


\onecolumn
\begin{lstlisting}[language=Matlab,style=matlabstyle, caption={Función para la simulación del PID de  Astr\"om discreto con saturación y ruido.}, label={lst:mat6}]
	function sim_pid_euler_astrom(G, T, Kp, Ti, Td,N,stepSize,sigma,umin,umax)
	
	% ===== Coeficientes PID incremental (Euler + filtro derivativo N) =====
	alpha2 = Td*Ti/(T^2);alpha1 = N*Ti/T; beta2  = Kp*(N*Td*Ti + Td*Ti)/(T^2);
	beta1  = Kp*(N*Ti + Td)/T;beta0  = Kp*N;U0 = alpha2 + alpha1;U1 = -(2*alpha2 + alpha1);
	U2 = alpha2;E0 = beta2 + beta1 + beta0;E1 = -(2*beta2 + beta1);E2 = beta2;
	% --- Control analógico ---
	numC = Kp*[Ti*Td, Ti, 1];
	denC = [Ti, 0];
	controlador = tf(numC, denC);
	cloop_c = feedback(controlador*G, 1);
	info = stepinfo(cloop_c);
	tend = info.SettlingTime*3;
	wn = (1.8/ info.RiseTime);
	Tstep = (2*pi/wn)/3000;
	t = 0:Tstep:tend;
	td = 0:T:tend;
	refc  = stepSize*ones(size(t)) + sigma*randn(size(t));   % ref continua
	% Salida continua con la MISMA referencia (ZOH)
	yc = lsim(cloop_c, refc, t);
	% "Digitalización" por muestreo ideal en kT:
	refd  = interp1(t, refc, td, 'linear', 'extrap');        % r[k] = r_c(kT)
	% --- Discretización y lazo discreto ---
	td = 0:T:tend;
	Gd = c2d(G, T, 'zoh');
	[numD, denD] = tfdata(Gd, 'v');
	b0 = numD(2);b1 = numD(3);a1 = denD(2);a2 = denD(3);
	yd   = zeros(size(td));ed   = zeros(size(td));ud   = zeros(size(td));
	for k = 3:numel(td)-1
	ed(k) = refd(k) - yd(k);
	ud_sinsaturar = (-U1*ud(k-1) + -U2*ud(k-2) + E0*ed(k) + E1*ed(k-1) + E2*ed(k-2))/U0;
	if ud_sinsaturar >umax
	ud(k)=umax;
	elseif ud_sinsaturar<umin
	ud(k) = umin;
	else
	ud(k) = ud_sinsaturar;
	end
	yd(k+1) = b0*ud(k) + b1*ud(k-1) - a1*yd(k) - a2*yd(k-1);
	end
	% === Graficar ===
	yshift = circshift(yd, -2);
	figure;
	yyaxis left
	plot(t*1000, refc, ':','Color','#B0B0B0'); hold on;
	plot(t*1000, yc, '--','LineWidth',1.3);
	stairs(td*1000, yshift,'-','LineWidth',1.2);
	ylabel('Salida');
	yyaxis right
	plot(td*1000, ud, 'r--','LineWidth',1.2);
	ylabel('u_d(k)');
	xlabel('Tiempo [ms]');
	title(sprintf('PID Astrom: Kp=%.3g, Ti=%.3g, Td=%.3g, N=%.3g', Kp, Ti, Td,N));
	legend(sprintf('Escalon Ruidoso (A=%.3g, $\\sigma$=%.3g)', stepSize, sigma), ...
	'Respuesta al Escalon Continua sin Wind-Up', ...
	'Respuesta al Escalon Discreta con Wind-Up', ...
	sprintf('Esfuerzo de Control (Rango:%.3g-%.3g)',umin,umax), ...
	'Location', 'best', 'Interpreter','latex');
	grid on;
	end
\end{lstlisting}



%\insertarfigura{./img/Astrom1.png}{Circuito Modo inversor.}{fig:impedanciaReal}{0.5}
%\insertarfigura{./img/Astrom2.png}{Circuito Modo inversor.}{fig:impedanciaReal}{0.5}


%Dando los siguientes resultados:
%sim_pid_euler_astrom(G, T/30, Kp, Ti, Td,8,2.5,0.25,0,5);
%sim_pid_euler_astrom(G, T/30, Kp, Ti, Td,8,5,0,0,5);


\clearpage
\twocolumn

\insertarfigurawide{./img/Astrom2.png}{Respuesta al escalón con $f_s=30 f_{\mathrm{bw}}$ (curva azul), su versión continua (curva azul puntuada) y el esfuerzo generado por el compensador (curva naranja puntuada) saturable, utilizando el primer compensador diseñado, con un escalón de 5V, utilizando las modificaciones propuestas en el capitulo 6 del Astr\"om.}{fig:astrom2}{1}

\subsubsection{Analizar efectos del filtro y cambios propuestos}
Para la implementación del circuito se utilizó un \textbf{Timer} que genera la señal de referencia de forma periódica (figura~\ref{fig:esqPsoc}), almacenando su estado en un registro. De esta manera, también es posible observar el estado lógico de la referencia a través del pin digital \textbf{ref}.  

Durante la práctica no se disponía de resistencias de \SI{75}{k\ohm}, por lo que se reemplazaron por un valor cercano de \SI{82}{k\ohm}. Si bien el polo generado no presenta una variación significativa, los parámetros de sintonía $K_p$, $T_i$ y $T_d$ sí se ven afectados de manera considerable, como se mostrará más adelante.


\insertarfigura{./img/Circuito1.png}{Circuito Esquemático utilizado en el Psoc}{fig:esqPsoc}{1}
\onecolumn

\paragraph{Código implementado en el PSoC}\hfill \break

\begin{lstlisting}[style=cstyle,caption={Código completo del PSoC},label={lst:c}]
#include "project.h"
#include <stdio.h>

#define ref_max 3.0f
#define ref_min 2.0f
double ref = ref_min;
double e = 0.0;
double e_1 = 0.0;
double e_2 = 0.0;
double u = 0.0;
double u_unsat = 0.0;
double u_sat = 0.0;
double u_1 = 0.0;
double u_2 = 0.0;
double offset = 2.5f; 
uint8 dac_value=0;

volatile uint8 flag_5seg = 0;  // Bandera para 5 segundos
static uint16 counter = 0;
CY_ISR_PROTO(Adquisicion);

// ISR del Timer (5 segundos)
uint8 flag = '0';
float muestra;
#define T (double)(1.0/100.0)
#define Ts T
#define Kp (double) 0.5//4.1
#define Ti (double)0.015//0.01
#define Td (double)0.005//0.003

CY_ISR_PROTO(Adquisicion){
	//static uint16_t contador = 0;
	
	flag = '1';
	ref = (Status_Reg_1_Read()) ? ref_max : ref_min;
	isr_1_ClearPending();

}

int main(void) {
	uint32 num = counter;
	char buffer[10];
	//uint8 i = 0;
	
	CyGlobalIntEnable;
	
	// Inicializar componentes
	
	//UART_Start();           // Iniciar UART
	Opamp_1_Start();
	Opamp_2_Start();
	Opamp_3_Start();
	ADC_SAR_1_Start();
	VDAC8_1_Start();
	Clock_1_Start();
	isr_1_StartEx(Adquisicion);
	
	isr_1_Enable();
	Timer_1_Start();
	
	//float Kp = 0.5;//4.1;
	
	//float Td = 0.0003;//0.0001;
	//float Ti = 0.01;//0.0115;
	int16 aux=0;
	VDAC8_1_SetValue(128u);
	for(;;) {
		
		if (flag == '1'){
			
			//Cuando se recibe una "I" por el puerto serial
			//Se inicializa una nueva adquisición de datos
			flag = '0'; 
			//Se habilita la digitalización y la INT 1   
			
			muestra =ADC_SAR_1_CountsTo_Volts(ADC_SAR_1_GetResult16());
			// muestra=muestra-2.0;
			e = ref - muestra; //se calcula el error
			
			//se calcula el esfuerzo de control con la ecuacion en diferencia del compensador
			
			
			u =  u_1 + Kp*(((1+(Ts/Ti) + (Td/Ts))*e)  - (1 +(2*Td/Ts))*e_1 + (Td/Ts)*e_2)  ; 
			
			dac_value = (uint8)((u / 4.0f) * 255.0f);
			
			VDAC8_1_SetValue(dac_value);           
			
			u_1 = u;
			u_2 = u_1;
			e_2 = e_1;
			e_1 = e;
			
			isr_1_Enable();
		}
		
	}
}
	
\end{lstlisting}
\twocolumn
\paragraph{Resultados Osciloscopio}
En primer lugar, se simuló el \textbf{textbook} utilizando dos frecuencias diferentes: \SI{100}{\hertz} y \SI{300}{\hertz}.

\insertarfigura{./img/textbook100.jpg}{Implementación a \SI{100}{\hertz}}{fig:imp100}{1}
\insertarfigura{./img/textbook300.jpg}{Implementación a \SI{300}{\hertz}}{fig:imp300}{1}

\insertarfigura{./img/sinAstrom.jpg}{Implementación sin Astr\"om}{fig:impsas}{1}
\insertarfigura{./img/sinAstrom1.jpg}{Implementación sin Astr\"om}{fig:impsas1}{1}

\onecolumn
\paragraph{Modificación del código para la implementación del método de Astr\"om}
Se añadieron las siguientes constantes \texttt{define}: \hfill \break

\begin{lstlisting}[style=cstyle,caption={Definiciones adicionales para implementar Astr\"om},label={lst:c1}]
	#define ud_max (double)4.08*0.9
	#define ud_min (double)4.08*0.1
	#define int_max (double)255
	#define int_min (double)0
	#define m (double)(int_max-int_min)/(ud_max-ud_min)
	
	
	#define N 8
	#define alpha2  (double)Td*Ti/(T*T)
	#define alpha1  (double)N*Ti/T
	#define beta2   (double)Kp*(N*Td*Ti + Td*Ti)/(T*T)
	#define beta1   (double)Kp*(N*Ti + Td)/T
	#define beta0   (double)Kp*N
	
	#define U0  (double)(alpha2 + alpha1)
	#define U1  (double)(-(2*alpha2 + alpha1))
	#define U2  (double)(alpha2)
	
	#define E0  (double)(beta2 + beta1 + beta0)
	#define E1  (double)(-1*(2*beta2 + beta1))
	#define E2  (double)(beta2)
	
\end{lstlisting}

Posteriormente, se modificó el \texttt{main} del código~\ref{lst:c} para implementar el siguiente algoritmo:

\begin{lstlisting}[style=cstyle,caption={Implementación del algoritmo de Aström con saturación},label={lst:c1}]
u_unsat = (-U1*u_1 + -U2*u_2 + E0*e + E1*e_1 + E2*e_2)/U0;
if (u_unsat > ud_max) u = ud_max;
else if (u_unsat < ud_min) u = ud_min;
else u = u_unsat;

\end{lstlisting}

\twocolumn
\insertarfigura{./img/sinsat300.jpg}{Implementación sin Saturar a \SI{300}{\hertz}, con Astr\"om}{fig:sinsat300}{1}
\insertarfigura{./img/sat300.jpg}{Implementación Saturarando a \SI{300}{\hertz}, con Astr\"om}{fig:sat300}{1}

	\subsection{Resultados}
	\begin{enumerate}[label=5.\arabic*.]
		\item Gráficas del sistema con los compensadores.
		\item Comparar Matlab vs.\ resultados experimentales.
	\end{enumerate}
	
	% ================== APÉNDICE OPCIONAL ==================
	\appendices
	\section{Snippets Matlab (borrador)}
	\begin{verbatim}
		G = tf(num, den);
		Ts = 0.001; % placeholder
		Gz = c2d(G, Ts, 'zoh');
		step(Gz); grid on;
	\end{verbatim}
	
	\section{Notas PSoC (borrador)}
	\begin{verbatim}
		// ISR @ Ts:
		// 1) Leer ADC -> yk
		// 2) Calcular error ek = rk - yk
		// 3) Calcular uk con ecuaciones en diferencias
		// 4) Escribir DAC -> uk
	\end{verbatim}
	
\end{document}
