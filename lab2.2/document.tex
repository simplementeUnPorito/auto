% ================== BORRADOR PUERCO EN FORMATO IEEE ==================
\documentclass[conference]{IEEEtran}

% ======= Preambulo mínimo pero útil para escribir rápido =======
\usepackage[spanish, es-nodecimaldot]{babel}
\usepackage[utf8]{inputenc}
\usepackage[T1]{fontenc}
\usepackage{amsmath, amssymb, amsfonts}
\usepackage{siunitx}
\usepackage{balance}
\usepackage{siunitx}
\sisetup{
	output-decimal-marker = {,}, % coma decimal en vez de punto
	group-separator = {.},       % separador de miles con punto
	group-minimum-digits = 4,    % agrupa desde 4 dígitos
	per-mode = symbol            % para usar "/" en unidades
}

\usepackage{graphicx}
\usepackage{caption}
\usepackage{subcaption}
\usepackage{booktabs}
\usepackage{hyperref}
\usepackage{enumitem}
\setlist{leftmargin=*, itemsep=2pt, topsep=4pt}
\usepackage{xcolor}

% ======= Macros rápidas =======
\newcommand{\zetaD}{\zeta}
\newcommand{\wn}{\omega_n}
\newcommand{\Ts}{T_s}
\newcommand{\ESS}{\mathrm{ESS}}
\newcommand{\Gz}{G(z)}
\newcommand{\Gs}{G(s)}

\usepackage{float}

% Macro para insertar figuras centradas con ancho ajustable
% Ejemplo de uso \insertarfigura{ruta}{leyenda}{etiqueta}{escala entre 0 y 1}
\newcommand{\insertarfigura}[4]{
	\begin{figure}[H]
		\centerline{\includegraphics[width=#4\linewidth]{#1}}
		\caption{#2}
		\label{#3}
	\end{figure}
	% Referencia cruzada si se desea en el texto: véase la Fig.~\ref{#4}
}
% Macro: \insertarfigurawide{ruta}{caption}{label}{scale}
\usepackage{cuted} % en el preámbulo

% 
\newcommand{\insertarfigurawide}[4]{%
	\begin{figure*}[ht]
		\centering
		\includegraphics[width=#4\textwidth]{#1}
		\caption{#2}
		\label{#3}
	\end{figure*}
}

\usepackage{subcaption} % en el preámbulo

% ======= Metadatos =======
\title{Práctica de Laboratorio 2: Diseño de controladores por el método de lugar de raíces.}

\author{
	\IEEEauthorblockN{Elías Álvarez}
	\IEEEauthorblockA{Carrera de Ing. Electrónica \\ 
		Universidad Católica Nuestra Señora de la Asunción \\
		Asunción, Paraguay \\
		Email: elias.alvarez@universidadcatolica.edu.py}
	\and
	\IEEEauthorblockN{Tania Romero}
	\IEEEauthorblockA{Carrera de Ing. Electrónica \\
		Universidad Católica Nuestra Señora de la Asunción \\
		Asunción, Paraguay \\
		Email: tania.romero@universidadcatolica.edu.py}
		\and
	\IEEEauthorblockN{\hspace*{3.5em}Docente: Lic. Montserrat González}
	\IEEEauthorblockA{
		\hspace*{3.5em}Facultad de Ingeniería \\
		\hspace*{3.5em}Universidad Católica Nuestra Señora de la Asunción \\
		\hspace*{3.5em}Asunción, Paraguay}
	\and
	\IEEEauthorblockN{\hspace*{3.5em}Docente: PhD. Enrique Vargas}
	\IEEEauthorblockA{%
		\hspace*{3.5em}Facultad de Ingeniería \\  % ← empuja a la derecha
		\hspace*{3.5em}Universidad Católica Nuestra Señora de la Asunción \\
		\hspace*{3.5em}Asunción, Paraguay}
}

\begin{document}
	\maketitle
	
	\section{Introducción}


\section{Objetivos}

\begin{itemize}
	\item Diseñar controladores digitales para una planta analógica utilizando estados estimados.
	\item Aplicar el diseño de realimentación de estados con estimadores de predicción y de actualización.
	\item Presentar y discutir los resultados experimentales en comparación con los obtenidos por simulación, dentro de un informe técnico razonado.
\end{itemize}

	
	\section{Desarrollo}

\subsection{Modelado del Sistema}

\subsubsection{Obtención de las matrices del sistema $F$, $G$, $H$ y $J$}

Para el modelado del sistema se parte del circuito mostrado en la Figura~\ref{fig:circuit_planta}, a partir del cual se determinan las ecuaciones de estado mediante el análisis de los lazos de realimentación y las relaciones de tensión en los componentes.

\insertarfigura{Otros/circuit.png}{Circuito de la planta.}{fig:circuit_planta}{1}

El sistema se describe mediante las siguientes ecuaciones en espacio de estados:

\begin{equation}
	\dot{x}(t) = F\,x(t) + G\,V_i(t)
	\label{eq:1}
\end{equation}

\begin{equation}
	y(t) = H\,x(t) + J\,V_i(t)
	\label{eq:2}
\end{equation}

Para obtener las expresiones de las variables de estado, se parte del equivalente del paralelo entre un resistor y un capacitor:
\[
R \parallel \frac{1}{sC} = \frac{R}{1 + sRC}
\]

Considerando que ambos amplificadores operacionales se encuentran en configuración no inversora, se obtienen las siguientes relaciones:

\[
V_a = \frac{-R_2}{1 + sR_2C_1}\frac{V_i}{R_1}
\quad \Rightarrow \quad
sV_a = -\frac{1}{R_1C_1}V_i - \frac{1}{R_2C_1}V_a
\]

\[
V_o = -\frac{R_4}{1 + sR_4C_2}\frac{V_a}{R_3}
\quad \Rightarrow \quad
sV_o = -\frac{1}{R_3C_2}V_a - \frac{1}{R_4C_2}V_o
\]

Definiendo como variables de estado $x_1(t) = V_a$ y $x_2(t) = V_o$, las ecuaciones anteriores se expresan en forma matricial como:

\[
\begin{bmatrix}
	\dot{x}_1(t) \\[4pt]
	\dot{x}_2(t)
\end{bmatrix}
=
\begin{bmatrix}
	-\dfrac{1}{R_2C_1} & 0 \\[4pt]
	-\dfrac{1}{R_3C_2} & -\dfrac{1}{R_4C_2}
\end{bmatrix}
\begin{bmatrix}
	x_1(t) \\[4pt]
	x_2(t)
\end{bmatrix}
+
\begin{bmatrix}
	-\dfrac{1}{R_1C_1} \\[4pt]
	0
\end{bmatrix}
V_i(t)
\]

y la ecuación de salida queda definida como:

\[
y(t) =
\begin{bmatrix}
	0 & 1
\end{bmatrix}
\begin{bmatrix}
	x_1(t) \\[4pt]
	x_2(t)
\end{bmatrix}
+ 0\cdot V_i(t)
\]

Sustituyendo los valores de los componentes 
$C_1 = 211.1\times10^{-9}\,\text{F}$, 
$R_1 = 80.55\times10^{3}\,\Omega$, 
$R_2 = 81.09\times10^{3}\,\Omega$, 
$C_2 = 103.07\times10^{-9}\,\text{F}$, 
$R_3 = 14.878\times10^{3}\,\Omega$ y 
$R_4 = 14.76\times10^{3}\,\Omega$, 
se obtienen las siguientes matrices numéricas:

\[
F =
\begin{bmatrix}
	-58.42 & 0 \\[4pt]
	-652.11 & -657.37
\end{bmatrix}, \quad
G =
\begin{bmatrix}
	-58.81 \\[4pt]
	0
\end{bmatrix}, \quad
H =
\begin{bmatrix}
	0 & 1
\end{bmatrix}
\]
\begin{equation}
	y \quad
	J = 0
	\label{eq:J}
\end{equation}
\subsubsection{Mostrar el diagrama de bloques del sistema}


\subsection{Discretización del Sistema}
\insertarfigura{Otros/Diagramas1.png}{Diagrama de bloques del sistema continuo.}{fig:diag_continuo}{1}

\subsubsection{Elección del tiempo de muestreo $T_s = 1~\text{ms}$}

Para la discretización del sistema continuo descrito por las ecuaciones~(\ref{eq:1}) y~(\ref{eq:2}), se busca obtener un modelo equivalente en tiempo discreto que relacione las variables de estado y la señal de entrada en instantes de muestreo definidos.  
Las ecuaciones del sistema discreto se expresan como:

\begin{equation}
	X(k+1) = A\,X(k) + B\,u(k)
	\label{eq:3}
\end{equation}

\begin{equation}
	Y(k) = C\,X(k) + D\,u(k)
	\label{eq:4}
\end{equation}

Usando las matrices continuas $F$, $G$, $H$ y $J$ obtenidas previamente, las matrices discretas se determinan mediante las siguientes expresiones:

\[
A = \mathrm{e}^{F T_s}, \qquad 
B = F^{-1}\!\left(\mathrm{e}^{F T_s} - I\right)G, \qquad 
C = H \qquad 
\]

\[
 \& \quad D = J
\]
\subsubsection{Obtención de las matrices discretas $A$, $B$, $C$ y $D$}

Con un tiempo de muestreo $T_s = 1~\text{ms}$, se obtienen las siguientes matrices discretizadas:

\[
A =
\begin{bmatrix}
	0.943 & 0 \\[4pt]
	-0.462 & 0.518
\end{bmatrix}, \qquad
B =
\begin{bmatrix}
	-0.0571 \\[4pt]
	0.0152
\end{bmatrix}, 
\]
\begin{equation}
	C =
	\begin{bmatrix}
		0 & 1
	\end{bmatrix}, \qquad \& \qquad
	D = 0
	\label{eq:d}
\end{equation}

Estas matrices representan el modelo digital equivalente del sistema continuo, y serán utilizadas posteriormente para el diseño del controlador e implementación en el \texttt{PSoC}.

\subsubsection{Diagrama de bloques del sistema discretizado}

En la Figura~\ref{fig:diag_discreto} se presenta el diagrama de bloques correspondiente al sistema discretizado, donde se observa la relación entre las variables de estado, la entrada $u(k)$ y la salida $Y(k)$.

\insertarfigura{Otros/Diagramas2.png}{Diagrama de bloques del sistema discretizado.}{fig:diag_discreto}{1}


\subsubsection{Verificación de la controlabilidad y observabilidad del sistema}

La matriz de controlabilidad se obtiene a partir de la siguiente relación general:

\[
x(n) - A^{n}x(0) = \sum_{i=0}^{n-1} A^{n-i-1}B\,u(i)
\]

lo que lleva a la siguiente forma matricial:

\[
x(n) - A^{n}x(0) =
\begin{bmatrix}
	A^{n-1}B & A^{n-2}B & \cdots & AB & B
\end{bmatrix}
\begin{bmatrix}
	u(0) \\[2pt]
	u(1) \\[2pt]
	\vdots \\[2pt]
	u(n-1)
\end{bmatrix}
\]

De esta expresión, se define la matriz de controlabilidad como:

\[
\mathcal{C} =
\begin{bmatrix}
	B & AB
\end{bmatrix}
\]

Para el sistema analizado, la matriz resultante es:

\[
\mathcal{C} =
\begin{bmatrix}
	-0.05344 & -0.05713 \\[4pt]
	0.03435 & 0.01527
\end{bmatrix}
\]

El rango de esta matriz es $n = 2$, lo que indica que el sistema es completamente controlable.

\bigskip
La matriz de observabilidad se obtiene a partir de la expresión general:

\[
Y(n-1) = 
\begin{bmatrix}
	C \\[4pt]
	CA \\[4pt]
	\vdots \\[4pt]
	CA^{n-1}
\end{bmatrix}
X(0)
\]

Por lo tanto, la matriz de observabilidad queda definida como:

\[
\mathcal{O} =
\begin{bmatrix}
	C \\[4pt]
	CA
\end{bmatrix}
\]

Sustituyendo los valores del sistema:

\[
\mathcal{O} =
\begin{bmatrix}
	0 & 1 \\[4pt]
	-0.462 & 0.518
\end{bmatrix}
\]

El rango de la matriz de observabilidad también resulta ser $n = 2$.\\  
Por lo tanto, se concluye que el sistema es **completamente controlable y observable**, cumpliendo con las condiciones necesarias para el diseño de control mediante realimentación de estados.

\subsubsection{Comparar los resultados obtenidos con las simulaciones realizadas en \texttt{MATLAB}}
\textbf{Resultados de las matrices continuas:}
\[
F =
\begin{bmatrix}
	-58.42 & 0 \\[4pt]
	-652.10 & -657.30
\end{bmatrix}, \quad
G =
\begin{bmatrix}
	-58.81 \\[4pt]
	0
\end{bmatrix}, \quad
H =
\begin{bmatrix}
	0 & 1
\end{bmatrix}
\]

\[
	\& \quad
	J = [\,0\,]
\]

\textbf{Resultados de las matrices discretas:}
\[
A =
\begin{bmatrix}
	0.9433 & 0 \\[4pt]
	-0.4628 & 0.5182
\end{bmatrix}, \quad
B =
\begin{bmatrix}
	-0.05712 \\[4pt]
	0.01527
\end{bmatrix}, \quad
C =
\begin{bmatrix}
	0 & 1
\end{bmatrix}
\]

\[
	\& \quad
	D = [\,0\,]
\]
Se observa que los resultados de las matrices obtenidas son congruentes con los valores calculados en las ecuaciones~(\ref{eq:J}) y~(\ref{eq:d}), verificando la coherencia entre el modelo teórico y los resultados obtenidos mediante \texttt{MATLAB}.

\insertarfigurawide{matlab/Sim1_noImplementable.png}{Respuesta simulada en \texttt{MATLAB} correspondiente al Caso 2($f_s = 1000 Hz$).}{fig:simulacion_noImplementable}{1}

\insertarfigurawide{matlab/Sim2_noImplementable.png}{Respuesta simulada en \texttt{MATLAB} correspondiente al Caso 2($f_s = 1000 Hz$).}{fig:simulacion_noImplementable2}{1}

En las figuras \ref{fig:simulacion_noImplementable} y \ref{fig:simulacion_noImplementable2}, del \texttt{MATLAB} se pueden observar que no es implementable.
	
	\section{Resultados}

\subsection{Respuestas temporales (simulado vs.\ experimental)}
En esta sección se presentan comparaciones \emph{en el dominio del tiempo} entre simulación y experimento para: planta sin compensar, C1 (proporcional), C2 (lead), C3 (integrador+lead). En cada figura se incluyen las métricas \emph{RMSE}, \emph{NRMSE} y \(e_{\max}\) (definidas en las ecuaciones \eqref{eq:rmse_temporal}–\eqref{eq:emax_temporal}). Se muestran versiones \emph{con} y \emph{sin} remoción de offset para evidenciar el descalce de referencia señalado en la implementación.

% ===== AJUSTAR RUTAS =====
\insertarfigura{img/OpenLoop/comparacionLazoAbierto.png}
{Planta sin compensación en lazo abierto: respuesta al escalón (simulado vs.\ experimental, con bandas de tolerancia para \(\sigma = \frac{tolerance}{3}\)).}
{fig:step_openLoop}{1}



\insertarfigurawide{img/C1_Lead/conOffset}
{Compensador de adelanto: respuesta al escalón y esfuerzo (simulado vs.\ experimental, \emph{con} offset).}
{fig:step_c1_con_offset}{1}

\insertarfigurawide{img/C1_Lead/sinOffset}
{Compensador de adelanto: respuesta al escalón y esfuerzo (simulado vs.\ experimental, \emph{sin} offset).}
{fig:step_c1_sin_offset}{1}

\insertarfigurawide{img/C1_K/compConOffset}
{Compensador Proporcional: respuesta al escalón y esfuerzo (simulado vs.\ experimental, \emph{con} offset).}
{fig:step_prop_con_offset}{1}

\insertarfigurawide{img/C1_K/compSinOffset}
{Compensador Proporcional: respuesta al escalón y esfuerzo (simulado vs.\ experimental, \emph{sin} offset).}
{fig:step_prop_sin_offset}{1}

\insertarfigurawide{img/C2/conOffset}
{Compensador de adelanto + integrador: respuesta al escalón y esfuerzo (simulado vs.\ experimental, \emph{con} offset).}
{fig:step_c2_con_offset}{1}

\insertarfigurawide{img/C2/sinOFFset}
{Compensador de adelanto + integrador: respuesta al escalón y esfuerzo (simulado vs.\ experimental, \emph{sin} offset).}
{fig:step_c2_sin_offset}{0.92}



\subsection{Métricas de desempeño temporal}
Las métricas usadas en cada gráfica (y en la tabla resumen) son:
\begin{equation}
	\label{eq:rmse_temporal}
	\mathrm{RMSE}=\sqrt{\frac{1}{N}\sum_{k=1}^{N}\big(y[k]-\hat{y}[k]\big)^2},
\end{equation}
\begin{equation}
	\label{eq:rmse_porct}
	\mathrm{NRMSE}=\frac{\mathrm{RMSE}}{y_{\max}-y_{\min}},
\end{equation}
\begin{equation}
	\label{eq:emax_temporal}
	e_{\max}=\max_k\,\lvert y[k]-\hat{y}[k]\rvert.
\end{equation}
\balance
% ===== Resultados (solo tiempo) =====
En la tabla \ref{tab:comparativa_temporal_min}, los índices de \emph{tiempo de subida} \(t_r\) y \emph{sobreimpulso} \(M_p\) se obtienen de las curvas de esta sección y anteriores. El error en estado estacionario \(e_{ss}\) se reporta cuando aplica (lazo cerrado), y el \emph{error cuadratico medio normalizado} entre las mediciones y simulaciones se especifica tanto \emph{con} como \emph{sin} el offset de las señales.

% EN EL TEXTO (sin \onecolumn ni \balance aquí)
\begin{table}[t]
	\centering
	\caption{Comparativa temporal (simulado vs.\ experimental).}
	\label{tab:comparativa_temporal_min}
	\small
	\setlength{\tabcolsep}{4pt}
	\renewcommand{\arraystretch}{1.1}
	\begin{adjustbox}{max width=\columnwidth}
		\begin{tabular}{lcccc}
			\toprule
			\textbf{Sistema} &
			\makecell{\(\mathbf{t_r}\) [ms]\\(sim/exp)} &
			\makecell{\(\mathbf{M_p}\) [\%]\\(sim/exp)} &
			\makecell{\(\mathbf{e_{ss}}\) [\%]\\(sim/exp)} &
			\makecell{\(\mathbf{NRMSE}\) [\%]\\(con/sin off)}\\
			\midrule
			Lazo abierto       & 40.0 / 34.2   & 0 / 0                 & 0 / 1.8182             & 19.14 / 18.43 \\
			C1 (lead)          & 21.17 / 19.8  & 0 / 0                 & 45.77 / 42--46.79      & 38.02 / 7.31 \\
			Proporcional       & 3.736 / 3.950 & 15.4 / 10--20         & 12.5 / \(\approx 15.825\) & 19.24 / 6.57 \\
			Lead + integrador  & 13.7 / 17.10  & 7.433 / \(\approx 5\) & 0 / 0                  & 13.54 / 5.61 \\
			\bottomrule
		\end{tabular}
	\end{adjustbox}
\end{table}

\subsection{Error de velocidad (seguimiento de rampa)}
Para el compensador C2 (integrador+lead) se ensayó seguimiento a rampa \(r[k]=kT\) (pendiente \(1\ \mathrm{u}/\mathrm{s}\)). El error en régimen para sistemas tipo~1 en discreto es
\begin{equation}
	\label{eq:ess_rampa_discreto}
	e_{ss}=\frac{T}{K_{v,z}},\qquad K_{v,z}=\lim_{z\to1}(z-1)\,L(z).
\end{equation}
Con el \(L(z)\) obtenido, se midió/estimó \(K_v\simeq 78.43\Rightarrow e_{ss}^{(\mathrm{teo})}\approx 1/78.43\approx 1.28\%\), en concordancia con el valor observado tras un transitorio breve.

% ===== AJUSTAR RUTA =====
\insertarfigura{img/C2/rampaC2linda}
{Seguimiento de rampa con Compensador integrador + lead.}
{fig:rampa_c2}{0.92}

\subsection{Discusión de discrepancias}
Las diferencias entre curvas simuladas y experimentales se explican principalmente por: (i) \textbf{limitación/saturación} en la señal de esfuerzo debido a la carga en el DAC (la excitación queda “achatada”), y (ii) \textbf{descalce de referencia} (offset). A ello se suman tolerancias de componentes pasivos, lo que desplaza levemente parámetros característicos de la planta. Aun así, la \emph{dinámica global} buscada (forma de la respuesta y tiempos) se mantuvo acorde a la simulación.




\section{Conclusiones}

El conjunto de ensayos y comparaciones en el \emph{dominio del tiempo} demuestra que, a pesar de las limitaciones de implementación y de la dispersión de la planta real, el desempeño experimental se mantiene coherente con el diseño y dentro de las bandas esperadas. A continuación se sintetizan los hallazgos principales.

\subsection*{Ajuste simulación–experimento}
Al simular con la \textbf{misma entrada} medida y remover el \textbf{offset estático} entre DAC y ADC, el ajuste mejora de forma notable (véase la tabla~\ref{tab:comparativa_temporal_min}). Lo que persiste se explica por \emph{efectos dinámicos} no ideales: como saturaciones suaves del actuador (no linealidad) y  atenuaciones por carga o discrepancias entre resistencias que definen la ganancia.

\subsection*{Limitaciones de hardware observadas}
Se verificó que la \textbf{corriente de salida del DAC} no es suficiente para excitar directamente la planta ---aunque no logramos conseguir los rangos exactos de corriente que puede suministrar en el datasheet---, lo que produce \emph{achatamiento} y \emph{recorte} del esfuerzo de control y, en consecuencia, discrepancias de forma en las respuestas. Además, la referencia $\mathrm{VDDA}/2$ y las líneas de alimentación presentan \emph{desacople insuficiente}, favoreciendo derivas de nivel que se manifiestan como offset. Estas condiciones explican parte del desajuste restante aun tras corregir el offset en posprocesado.

\subsection*{Planta real vs.\ modelo nominal}
La planta implementada difiere levemente del modelo continuo asumido por \emph{tolerancias} de pasivos, ordenamiento de etapas e impedancia de entrada vista por la fuente del esfuerzo. Aun así, las respuestas medidas se ubican mayormente \textbf{dentro de la banda de tolerancias} obtenida por análisis estadístico, lo que respalda la \emph{validez del modelo} para propósitos de diseño y la \emph{robustez} del procedimiento seguido.



\subsection*{Acciones recomendadas}
\begin{itemize}
	\item Añadir un \textbf{buffer} a la salida del DAC para eliminar el error de carga en el DAC y evitar recortes del esfuerzo.
	\item \textbf{Biaspassear} la referencia $\mathrm{VDDA}/2$ y las líneas de alimentación, mejorando estabilidad de nivel y rechazo de ruido.
\end{itemize}

\subsection*{Conclusión general}
En conjunto, los resultados muestran que el \textbf{método de diseño es robusto}: aun bajo carga del DAC, offsets y dispersión de componentes, las respuestas experimentales se mantienen \emph{razonablemente cercanas} a las simuladas, especialmente con el compensador  con integrador + adelanto, que satisface los objetivos de seguimiento y estabilidad con un compromiso adecuado entre rapidez y exactitud.

	% ================== APÉNDICE OPCIONAL ==================
\onecolumn
\appendices
\section{Códigos de Matlab}

\begin{lstlisting}[style=matlabstyle,caption={Primera hoja de cálculos utilizada.},label={matlab:calculo1}]
	close all
	clear all
	
	addpath('..\Lab1\')
	
	%% Definicion de parametros
	R_1 = 15e3;
	R_3 = 15e3;
	C_2 = 100e-9;
	
	R_2 = 82e3;
	R_4 = 82e3;
	C_1 = 0.22e-6;
	
	%% Generar funcion de transferencia d
	numStage = [-R_3/R_1 -R_4/R_2];
	denStage = { [C_2*R_3 1], [C_1*R_4 1] };
	
	% Usamos celdas para guardar los tf de cada stage
	Gstage = cell(1,2);
	G = 1;
	for i = 1:2
	Gstage{i} = tf(numStage(i), denStage{i});
	G = G*Gstage{i};
	end
	
	%% Analizamos en el tiempo
	[tr, ts, wn] = plot_step_info(G);
	disp('Planta continua G(s):')
	G
	zpk(G)
	
	
	
	%% Paso 1: Definir periodo de muestreo
	% Se busca una mejora de 4 en el tiempo de rising , o sea tr = 10 ms
	N = 4; 
	T = tr/(8*N);
	%Gcl = feedback(G,1);
	%figure;
	%step(Gcl)
	%T = stepinfo(Gcl).SettlingTime/8;
	
	%% Paso 2: Digitalizar con ZOH
	Gd = c2d(G, T, 'zoh');
	disp('Planta digital G(z):')
	Gd
	zpk(Gd)
	figure;
	pzmap(Gd);
	title('Lugar de raíces de G(z).')
	
	zgrid;
	z = tf([1 0],1,T);
	
	%% Paso 3: Análisis en tiempo discreto lazo abierto
	figure;
	
	% Obtenemos salida y tiempo de la función step
	[y, t] = step(Gd);
	
	% Graficamos con stairs (propio de señales discretas)
	stairs(t*1000, y, 'LineWidth',1.4);
	
	title('Respuesta al escalón de la planta digitalizad')
	xlabel('Tiempo [ms]');
	ylabel('Salida');
	grid on;
	ax = gca;
	ax.XMinorTick = 'on';  % activamos minor ticks
	ax.YMinorTick = 'on';
	grid minor;
	
	%% Paso 4: Definimimos los valores de la simulación
	umin = 0;
	umax =  4.08;
	refmin = 1.5;
	refmax = 2.5;
	n_per_seg = 500;
	%% Paso 5: Lugar de raíces para zita = 0.7, compensador P
	%figure
	
	%rlocus(Gd)
	%zgrid     % agrega la grilla en el plano-z
	z0=0.436;
	p0=0.769221;%0.7705;
	C_1 = (z-z0)/(z-p0);
	figure;
	Gc1 = Gd*C_1;
	rlocus(Gc1);
	title('Lugar de raíces de G(z) compensado con un lag-filter de 1º orden, anulando el polo más rápido de la planta.')
	
	zgrid     % agrega la grilla en el plano-z
	%pause;
	%[K, ~] = rlocfind(Gc1);   % hacés click donde querés los polos
	%K = 0.73139;
	%K = 0.82543;
	K = 0.743;
	Gc1f = feedback(K*Gc1,1);
	%figure;
	%step(Gc1f)
	info = stepinfo(Gc1f);
	info
	
	info_ext = plot_step_annot(Gc1f, 'la planta compensada con un lag-filter de 1º orden, anulando el polo más rápido de la planta.');
	%step(Gc3f);
	info = stepinfo(Gc1f);
	info
	info_ext
	
	[td, refd, yd, ud, ed, coefs] = sim_compensador_first_order( ...
	Gd, z0, p0, K, T, umin, umax, refmin, refmax, n_per_seg);
	
	Nini = 100;              % descartar al inicio
	Nfin = 100;              % descartar al final
	idx0 = Nini + 1;         % índice inicial válido
	idx1 = length(td) - Nfin; % índice final válido
	
	td   = td(idx0:idx1)-td(idx0);
	refd = refd(idx0:idx1);
	yd   = yd(idx0:idx1);
	ud   = ud(idx0:idx1);
	ed   = ed(idx0:idx1);
	
	% Si querés que el tiempo arranque en 0
	
	
	
	
	
	figure;
	
	% ---------- Subplot 1 ----------
	axAbs = subplot(2,1,1);  % eje izquierdo (absoluto)
	hold(axAbs,'on'); grid(axAbs,'on');
	
	% y[k] en rojo (abs)
	hY = stairs(axAbs, td, yd, 'r-', 'LineWidth', 1.4);
	
	% Armamos eje derecho transparente
	axPct = axes('Position', get(axAbs,'Position'), ...
	'Color','none', 'YAxisLocation','right', ...
	'XLim', get(axAbs,'XLim'), 'XTick',[], 'Box','off');
	hold(axPct,'on');
	
	% r[k] en negro, graficado en %
	ref_pct = 100*(refd - refmin)/(refmax - refmin);
	hR = stairs(axPct, td, refd, 'k--', 'LineWidth', 1.2);
	
	% ======= Cálculo de límites con margen =======
	yd_pct  = 100*(yd - refmin)/(refmax - refmin);
	pctAll  = [yd_pct; ref_pct];
	
	% Valores extremos en % con margen del 5 %
	rawMin = min(pctAll);
	rawMax = max(pctAll);
	span   = rawMax - rawMin;
	pctMin = rawMin - 0.05*span;
	pctMax = rawMax + 0.05*span;
	
	% Redondeamos a múltiplos de 20 para ticks
	stepPct = 20;
	tickMin = stepPct*floor(pctMin/stepPct);
	tickMax = stepPct*ceil (pctMax/stepPct);
	pctTicks = tickMin:stepPct:tickMax;
	
	% Convertimos a absolutos
	valTicks = refmin + (pctTicks/100)*(refmax - refmin);
	
	% Aplicamos a ambos ejes
	set(axAbs,'YLim',[valTicks(1) valTicks(end)],'YTick',valTicks);
	set(axPct,'YLim',[valTicks(1) valTicks(end)],'YTick',valTicks,...
	'YTickLabel',compose('%.0f %%',pctTicks));
	
	% Etiquetas
	xlabel(axAbs,'Tiempo [s]');
	ylabel(axAbs,'Respuesta [valor absoluto]');
	ylabel(axPct,'Escala relativa a ref [%]');
	title(axAbs,'Respuesta discreta con compensador 1º orden, anulando el polo más rápido de la planta.');
	legend(axAbs, [hY, hR], {'y[k]', 'r[k]'}, 'Location', 'best');
	
	% ---------- Subplot 2: esfuerzo de control ----------
	axU = subplot(2,1,2);
	stairs(axU, td, ud, 'LineWidth',1.2); grid(axU,'on');
	yline(axU, umax,'r:'); yline(axU, umin,'r:');
	xlabel(axU,'Tiempo [s]'); ylabel(axU,'u[k]');
	title(axU,'Esfuerzo de control con saturación');
	
	% --- Margen de 5% en eje Y ---
	uMin = min(ud);
	uMax = max(ud);
	span = uMax - uMin;
	ylim(axU, [uMin - 0.05*span, uMax + 0.05*span]);
	
	
	grid(axAbs,'on');    % grilla principal
	grid(axAbs,'minor'); % grilla secundaria
	
	grid(axPct,'on');
	grid(axPct,'minor');
	
	grid(axU,'on');
	grid(axU,'minor');
	
	%% Paso 6: Lugar de raices para zita = 0.7 y y ESS = 0, compensador PI
	z0 = 0.857419558001;
	p0 = 1;
	C_2 = (z-z0)/(z-p0); %z=0.8433 %8433
	Gc2 =Gd*C_2;
	figure;
	rlocus(Gc2);
	title('Lugar de raíces de G(z) compensado con un lag-filter de 1º orden , fijando un polo en z = 1.')
	
	zgrid
	%[K, ~] = rlocfind(Gc2);   % hacés click donde querés los polos
	%K = 3.2779;
	K2 = 3.09;
	%figure;
	Gc2f= feedback(K2*Gc2,1);
	%step(Gc2f);
	info = stepinfo(Gc2f);
	info
	
	info_ext = plot_step_annot(Gc2f, 'la planta compensada con un lag-filter de 1º orden , fijando un polo en z = 1.');
	%step(Gc3f);
	info = stepinfo(Gc2f);
	info
	info_ext
	
	
	[td, refd, yd, ud, ed, coefs] = sim_compensador_first_order( ...
	Gd, z0, p0, K, T, umin, umax, refmin, refmax, n_per_seg);
	
	
	Nini = 100;              % descartar al inicio
	Nfin = 100;              % descartar al final
	idx0 = Nini + 1;         % índice inicial válido
	idx1 = length(td) - Nfin; % índice final válido
	
	td   = td(idx0:idx1)-td(idx0);
	refd = refd(idx0:idx1);
	yd   = yd(idx0:idx1);
	ud   = ud(idx0:idx1);
	ed   = ed(idx0:idx1);
	
	
	
	
	figure;
	
	% ---------- Subplot 1 ----------
	axAbs = subplot(2,1,1);  % eje izquierdo (absoluto)
	hold(axAbs,'on'); grid(axAbs,'on');
	
	% y[k] en rojo (abs)
	hY = stairs(axAbs, td, yd, 'r-', 'LineWidth', 1.4);
	
	% Armamos eje derecho transparente
	axPct = axes('Position', get(axAbs,'Position'), ...
	'Color','none', 'YAxisLocation','right', ...
	'XLim', get(axAbs,'XLim'), 'XTick',[], 'Box','off');
	hold(axPct,'on');
	
	% r[k] en negro, graficado en %
	ref_pct = 100*(refd - refmin)/(refmax - refmin);
	hR = stairs(axPct, td, refd, 'k--', 'LineWidth', 1.2);
	
	% ======= Cálculo de límites con margen =======
	yd_pct  = 100*(yd - refmin)/(refmax - refmin);
	pctAll  = [yd_pct; ref_pct];
	
	% Valores extremos en % con margen del 5 %
	rawMin = min(pctAll);
	rawMax = max(pctAll);
	span   = rawMax - rawMin;
	pctMin = rawMin - 0.05*span;
	pctMax = rawMax + 0.05*span;
	
	% Redondeamos a múltiplos de 20 para ticks
	stepPct = 20;
	tickMin = stepPct*floor(pctMin/stepPct);
	tickMax = stepPct*ceil (pctMax/stepPct);
	pctTicks = tickMin:stepPct:tickMax;
	
	% Convertimos a absolutos
	valTicks = refmin + (pctTicks/100)*(refmax - refmin);
	
	% Aplicamos a ambos ejes
	set(axAbs,'YLim',[valTicks(1) valTicks(end)],'YTick',valTicks);
	set(axPct,'YLim',[valTicks(1) valTicks(end)],'YTick',valTicks,...
	'YTickLabel',compose('%.0f %%',pctTicks));
	
	% Etiquetas
	xlabel(axAbs,'Tiempo [s]');
	ylabel(axAbs,'Respuesta [valor absoluto]');
	ylabel(axPct,'Escala relativa a ref [%]');
	title(axAbs,'Respuesta discreta con compensador 1º orden, fijando un polo en z = 1.');
	legend(axAbs, [hY, hR], {'y[k]', 'r[k]'}, 'Location', 'best');
	
	% ---------- Subplot 2: esfuerzo de control ----------
	axU = subplot(2,1,2);
	stairs(axU, td, ud, 'LineWidth',1.2); grid(axU,'on');
	yline(axU, umax,'r:'); yline(axU, umin,'r:');
	xlabel(axU,'Tiempo [s]'); ylabel(axU,'u[k]');
	title(axU,'Esfuerzo de control con saturación');
	
	% --- Margen de 5% en eje Y ---
	uMin = min(ud);
	uMax = max(ud);
	span = uMax - uMin;
	ylim(axU, [uMin - 0.05*span, uMax + 0.05*span]);
	
	grid(axAbs,'on');    % grilla principal
	grid(axAbs,'minor'); % grilla secundaria
	
	grid(axPct,'on');
	grid(axPct,'minor');
	
	grid(axU,'on');
	grid(axU,'minor');
	
	
	
	%% Paso 7: Lugar de raices para zita = 0.7 y y ESS = 0, compensador PI de segundo orden
	z0_0 = 0.9333;
	p0_0 = 1;
	z0_1 = 0.436;
	p0_1 = 0.71135;
	C_3 = ((z-z0_0)*(z-z0_1))/((z-p0_0)*(z-p0_1));
	Gc3 =Gd*C_3; %z=0.8433 %8433
	figure;
	rlocus(Gc3)
	title('Lugar de raíces de G(z) compensado con un lag-filter de 2º orden, fijando un polo en z = 1 y anulando ambos polos de la planta.');
	
	zgrid
	%pause;
	%[K, ~] = rlocfind(Gc3);   % hacés click donde querés los polos
	K = 1.14;
	%figure;
	Gc3f= feedback(K*Gc3,1);
	
	info_ext = plot_step_annot(Gc3f, 'la planta compensada con un lag-filter de 2º orden, fijando un polo en z = 1 y anulando ambos polos de la planta.');
	%step(Gc3f);
	info = stepinfo(Gc3f);
	info
	info_ext
	
	
	% --- Llamada a tu simulador (misma interfaz que tu first_order) ---
	[td, refd, yd, ud, ed, coefs] = sim_compensador_second_order( ...
	Gd, z0_0, z0_1, p0_0, p0_1, K, T, umin, umax, refmin, refmax, n_per_seg);
	
	
	Nini = 100;              % descartar al inicio
	Nfin = 100;              % descartar al final
	idx0 = Nini + 1;         % índice inicial válido
	idx1 = length(td) - Nfin; % índice final válido
	
	td   = td(idx0:idx1)-td(idx0);
	refd = refd(idx0:idx1);
	yd   = yd(idx0:idx1);
	ud   = ud(idx0:idx1);
	ed   = ed(idx0:idx1);
	
	
	
	
	figure;
	
	% ---------- Subplot 1 ----------
	axAbs = subplot(2,1,1);  % eje izquierdo (absoluto)
	hold(axAbs,'on'); grid(axAbs,'on');
	
	% y[k] en rojo (abs)
	hY = stairs(axAbs, td, yd, 'r-', 'LineWidth', 1.4);
	
	% Armamos eje derecho transparente
	axPct = axes('Position', get(axAbs,'Position'), ...
	'Color','none', 'YAxisLocation','right', ...
	'XLim', get(axAbs,'XLim'), 'XTick',[], 'Box','off');
	hold(axPct,'on');
	
	% r[k] en negro, graficado en %
	ref_pct = 100*(refd - refmin)/(refmax - refmin);
	hR = stairs(axPct, td, refd, 'k--', 'LineWidth', 1.2);
	
	% ======= Cálculo de límites con margen =======
	yd_pct  = 100*(yd - refmin)/(refmax - refmin);
	pctAll  = [yd_pct; ref_pct];
	
	% Valores extremos en % con margen del 5 %
	rawMin = min(pctAll);
	rawMax = max(pctAll);
	span   = rawMax - rawMin;
	pctMin = rawMin - 0.05*span;
	pctMax = rawMax + 0.05*span;
	
	% Redondeamos a múltiplos de 20 para ticks
	stepPct = 20;
	tickMin = stepPct*floor(pctMin/stepPct);
	tickMax = stepPct*ceil (pctMax/stepPct);
	pctTicks = tickMin:stepPct:tickMax;
	
	% Convertimos a absolutos
	valTicks = refmin + (pctTicks/100)*(refmax - refmin);
	
	% Aplicamos a ambos ejes
	set(axAbs,'YLim',[valTicks(1) valTicks(end)],'YTick',valTicks);
	set(axPct,'YLim',[valTicks(1) valTicks(end)],'YTick',valTicks,...
	'YTickLabel',compose('%.0f %%',pctTicks));
	
	% Etiquetas
	xlabel(axAbs,'Tiempo [s]');
	ylabel(axAbs,'Respuesta [valor absoluto]');
	ylabel(axPct,'Escala relativa a ref [%]');
	title(axAbs,'Respuesta discreta con compensador 2º orden, fijando un polo en z = 1 y anulando ambos polos de la planta');
	legend(axAbs, [hY, hR], {'y[k]', 'r[k]'}, 'Location', 'best');
	
	% ---------- Subplot 2: esfuerzo de control ----------
	axU = subplot(2,1,2);
	stairs(axU, td, ud, 'LineWidth',1.2); grid(axU,'on');
	yline(axU, umax,'r:'); yline(axU, umin,'r:');
	xlabel(axU,'Tiempo [s]'); ylabel(axU,'u[k]');
	title(axU,'Esfuerzo de control con saturación');
	
	% --- Margen de 5% en eje Y ---
	uMin = min(ud);
	uMax = max(ud);
	span = uMax - uMin;
	ylim(axU, [uMin - 0.05*span, uMax + 0.05*span]);
	
	
	
	
	
	
	
	grid(axAbs,'on');    % grilla principal
	grid(axAbs,'minor'); % grilla secundaria
	
	grid(axPct,'on');
	grid(axPct,'minor');
	
	grid(axU,'on');
	grid(axU,'minor');
	
	
	%% impresion antigua
	% % --- Gráficas estilo informe ---
	% figure; 
	% subplot(2,1,1);
	% stairs(td, refd, 'k--','LineWidth',1.0); hold on;
	% stairs(td, yd,   'LineWidth',1.4);
	% grid on; xlabel('Tiempo [s]'); ylabel('Respuesta de la planta/Referencia');
	% title('Respuesta discreta con compensador 2º orden'); legend('r[k]','y[k]','Location','best');
	% 
	% subplot(2,1,2);
	% stairs(td, ud, 'LineWidth',1.2);
	% yline(umax,'r:'); yline(umin,'r:');
	% grid on; xlabel('Tiempo [s]'); ylabel('u[k]');
	% title('Esfuerzo de control con saturación');
\end{lstlisting}

\begin{lstlisting}[style=matlabstyle,caption={Funciones desarrolladas para la práctica.},label={matlab:func}]
	function info_ext = plot_step_annot(sys, nameStr, Ts_override)
	% STEP discreto con STAIRS; anota 10-90% (en s y en T), OS, zeta y omega_n.
	% Grafica en milisegundos y marca el OS en el pico.
	
	if nargin < 2 || isempty(nameStr), nameStr = inputname(1); end
	if isempty(nameStr), nameStr = 'sys'; end
	
	% ----- Respuesta y métricas básicas -----
	[y, t] = step(sys); y = y(:); t = t(:);
	t_ms = t * 1000;   % graficamos en milisegundos
	
	% Estimar yss por promedio en cola
	Ntail = max(10, round(0.05*length(y)));
	yss = mean(y(end-Ntail+1:end));
	
	% Límites 10-90
	if abs(yss) < 1e-12
	t10 = NaN; t90 = NaN; tr_10_90 = NaN; y10 = 0; y90 = 0;
	else
	y10 = 0.10*yss; y90 = 0.90*yss;
	sgn = sign(yss);
	idx10 = find(sgn*y >= sgn*y10, 1, 'first');
	idx90 = find(sgn*y >= sgn*y90, 1, 'first');
	t10 = tern(~isempty(idx10), t(idx10), NaN);
	t90 = tern(~isempty(idx90), t(idx90), NaN);
	tr_10_90 = tern(~isnan(t10)&&~isnan(t90)&&t90>=t10, t90 - t10, NaN);
	end
	t10_ms = t10 * 1000; t90_ms = t90 * 1000;
	
	% Pico y %OS (para zeta y omega_n)
	[ypeak_raw, idxpk] = max(sign(yss).*y);
	ypeak = sign(yss)*ypeak_raw;
	tpeak = t(idxpk); tpeak_ms = tpeak * 1000;
	OS_percent = tern(yss ~= 0, max(0, (ypeak - yss)/abs(yss)*100), NaN);
	
	% zeta desde %OS:  OS% = 100*exp(-zeta*pi/sqrt(1-zeta^2))
	if isnan(OS_percent) || OS_percent <= 0
	zeta_est = NaN;
	else
	logterm = log(OS_percent/100);
	zeta_est = -logterm / sqrt(pi^2 + logterm^2);
	end
	
	% ----- Ts y rise en T -----
	Ts = NaN;
	try
	if isprop(sys,'Ts') && ~isempty(sys.Ts) && sys.Ts > 0, Ts = sys.Ts; end
	catch, end
	if nargin >= 3 && ~isempty(Ts_override) && Ts_override > 0, Ts = Ts_override; end
	tr_10_90_T = tern(~isnan(Ts) && ~isnan(tr_10_90), tr_10_90/Ts, NaN);
	
	% ----- Estimacion de omega_n -----
	omega_n = NaN;
	if ~isnan(zeta_est) && zeta_est < 1 && OS_percent > 0 && tpeak > 0
	omega_n = pi / ( tpeak * sqrt(1 - zeta_est^2) );
	end
	if isnan(omega_n)
	S = stepinfo(y, t);   % Ts(2%) ~ 4/(zeta*omega_n)
	if isfield(S,'SettlingTime') && ~isempty(S.SettlingTime) && S.SettlingTime > 0 && ~isnan(zeta_est) && zeta_est > 0
	omega_n = 4 / ( zeta_est * S.SettlingTime );
	end
	end
	if isnan(omega_n) && ~isnan(tr_10_90) && tr_10_90 > 0
	k_rise = 1.4;  % aprox para 10-90% en 2do orden subamortiguado
	omega_n = k_rise / tr_10_90;
	end
	
	% ----- Grafico -----
	figure;
	stairs(t_ms, y, 'LineWidth', 1.5); grid on;grid minor; hold on;
	xlabel('Tiempo [ms]'); ylabel('Salida');
	title(sprintf('Respuesta al escalon de %s', nameStr));
	
	% Líneas guía 10% y 90% y marcadores (en ms)
	if ~isnan(tr_10_90)
	yline(y10, '--', '10% yss', 'LabelHorizontalAlignment','left');
	yline(y90, '--', '90% yss', 'LabelHorizontalAlignment','left');
	xline(t10_ms, ':', 't10%');
	xline(t90_ms, ':', 't90%');
	plot([t10_ms t90_ms], [y10 y90], 'o', 'MarkerSize', 6, 'LineWidth', 1.2);
	end
	% Línea de yss
	yline(yss, '-', sprintf('yss=%.3g', yss), 'LabelHorizontalAlignment','left');
	
	% ----- Marca del OS en el pico -----
	if ~isnan(OS_percent) && OS_percent > 0
	% línea vertical desde yss a ypeak en tpeak
	plot([tpeak_ms tpeak_ms], [yss ypeak], '-', 'LineWidth', 1.2);
	% texto al costado derecho de la línea
	dx = 0.02 * (t_ms(end) - t_ms(1));
	text(tpeak_ms + dx, yss + 0.5*(ypeak - yss), ...
	sprintf('OS=%.3g%%', OS_percent), ...
	'Interpreter','none', 'Margin',2);
	% marcar el pico
	plot(tpeak_ms, ypeak, 's', 'MarkerSize', 6, 'LineWidth', 1.2);
	end
	
	% ----- Cuadro de metricas (abajo-derecha; chico) -----
	w = 0.15; h = 0.09; rmargin = 0.1; x = 1 - rmargin - w; y0 = 0.30;
	txtLines = {
		sprintf('t_r(10%%-90%%) = %.4g ms  (~%.4g Ts)', safeNum(tr_10_90*1000), safeNum(tr_10_90_T))
		sprintf('OS = %.4g%%',          safeNum(OS_percent))
		sprintf('zeta = %.4g',          safeNum(zeta_est))
		sprintf('omega_n = %.4g rad/s', safeNum(omega_n))
	};
	annotation('textbox', [x y0 w h], ...
	'String', txtLines, 'Interpreter','none', ...
	'FontName','Consolas', 'FontSize',9, ...
	'EdgeColor',[0.5 0.5 0.5], 'FaceAlpha',0.88, 'BackgroundColor','w');
	
	hold off;
	
	% ----- Salida -----
	info_ext = struct( ...
	'yss', yss, ...
	't10', t10, ...
	't90', t90, ...
	'tr_10_90', tr_10_90, ...
	'tr_10_90_T', tr_10_90_T, ...
	'OS_percent', OS_percent, ...
	'zeta_est', zeta_est, ...
	'omega_n', omega_n, ...
	'Ts', Ts);
	end
	
	% --------- helpers ---------
	function y = tern(cond, a, b)
	if cond, y = a; else, y = b; end
	end
	
	function v = safeNum(x)
	if isnan(x) || isinf(x), v = NaN; else, v = x; end
	end
	function [td, refd, yd, ud, ed, coefs] = sim_compensador_first_order( ...
	Gd, z0, p0, Kc, T, umin, umax, refmin, refmax, n_per_seg)
	
	% ===== Igual que tu PID: y(k+1)=b0*u(k)+b1*u(k-1)-a1*y(k)-a2*y(k-1) =====
	% Controlador (posición): u(k) = p0*u(k-1) + Kc*ed(k) - Kc*z0*ed(k-1)
	% ed(k) = refd(k) - yd(k)
	
	if nargin < 10 || isempty(n_per_seg), n_per_seg = 600; end
	if ~isa(Gd,'tf'), error('Gd debe ser tf discreto.'); end
	
	% --- Coefs planta exactamente como en tu ejemplo ---
	[numD, denD] = tfdata(Gd, 'v');
	b0 = numD(2);
	b1 = numD(3);
	a1 = denD(2);
	a2 = denD(3);
	coefs = struct('b0',b0,'b1',b1,'a1',a1,'a2',a2);
	
	% --- Tiempo y referencia (refmin -> refmax -> refmin) ---
	N  = 3*n_per_seg;
	td = (0:N-1)' * T;
	refd = [refmin*ones(n_per_seg,1);
	refmax*ones(n_per_seg,1);
	refmin*ones(n_per_seg,1)];
	
	% --- Inicialización idéntica a tu estilo ---
	yd = zeros(N,1);
	ed = zeros(N,1);
	ud = zeros(N,1);
	
	% --- Loop (misma estructura y orden que tu PID) ---
	for k = 3:N-1
	ed(k) = refd(k) - yd(k);
	
	% Compensador: u(k) = p0*u(k-1) + Kc*ed(k) - Kc*z0*ed(k-1)
	u_k = p0*ud(k-1) + Kc*ed(k) - Kc*z0*ed(k-1);
	
	
	if u_k>umax
	ud(k) = umax;
	elseif u_k<umin
	ud(k) = umin;
	else 
	ud(k) = u_k;
	end
	
	
	% Planta discreta (dividido por a0 implícito como en tu ejemplo)
	yd(k+1) = b0*ud(k) + b1*ud(k-1) - a1*yd(k) - a2*yd(k-1);
	end
	end
	
	function [td, refd, yd, ud, ed, coefs] = sim_compensador_second_order( ...
	Gd, z0_0, z0_1, p0_0, p0_1, Kc, T, umin, umax, refmin, refmax, n_per_seg)
	
	% ===== Planta: y(k+1)=b0*u(k)+b1*u(k-1)+b2*u(k-2)-a1*y(k)-a2*y(k-1) =====
	% Compensador (posición, 2º orden):
	% u(k) = (p0_0+p0_1)*u(k-1) - (p0_0*p0_1)*u(k-2) ...
	%      + Kc*[ e(k) - (z0_0+z0_1)*e(k-1) + (z0_0*z0_1)*e(k-2) ]
	% e(k) = refd(k) - yd(k)
	
	if nargin < 12 || isempty(n_per_seg), n_per_seg = 600; end
	if ~isa(Gd,'tf'), error('Gd debe ser tf discreto.'); end
	
	% --- Coefs planta (normalizados a a0=1) ---
	[numD, denD] = tfdata(Gd, 'v');
	numD = numD(:).'; denD = denD(:).';
	if abs(denD(1) - 1) > 1e-12
	numD = numD/denD(1);
	denD = denD/denD(1);
	end
	
	% Asegurar longitud mínima del denominador [1 a1 a2]
	if numel(denD) < 3, denD(end+1:3) = 0; end
	a1 = denD(2);
	a2 = denD(3);
	
	% b0,b1,b2 sin romper si faltan términos en el numerador
	b0 = 0; b1 = 0; b2 = 0;
	if numel(numD) >= 2, b0 = numD(2); end
	if numel(numD) >= 3, b1 = numD(3); end
	if numel(numD) >= 4, b2 = numD(4); end
	% Nota: esto respeta tu estilo previo (b0=numD(2), b1=numD(3), ...)
	
	coefs = struct('b0',b0,'b1',b1,'b2',b2,'a1',a1,'a2',a2, ...
	'z0_0',z0_0,'z0_1',z0_1,'p0_0',p0_0,'p0_1',p0_1,'Kc',Kc);
	
	% --- Tiempo y referencia (refmin -> refmax -> refmin) ---
	N  = 3*n_per_seg;
	td = (0:N-1)' * T;
	refd = [refmin*ones(n_per_seg,1);
	refmax*ones(n_per_seg,1);
	refmin*ones(n_per_seg,1)];
	
	% --- Inicialización ---
	yd = zeros(N,1);
	ed = zeros(N,1);
	ud = zeros(N,1);
	
	% --- Precalculos del compensador ---
	P1 = (p0_0 + p0_1);
	P2 = (p0_0 * p0_1);
	Z1 = (z0_0 + z0_1);
	Z2 = (z0_0 * z0_1);
	
	% --- Loop ---
	for k = 3:N-1
	ed(k) = refd(k) - yd(k);
	
	% Compensador 2º orden
	u_k = P1*ud(k-1) - P2*ud(k-2) + Kc*( ed(k) - Z1*ed(k-1) + Z2*ed(k-2) );
	
	% Saturación
	if u_k > umax
	ud(k) = umax;
	elseif u_k < umin
	ud(k) = umin;
	else
	ud(k) = u_k;
	end
	
	% Planta discreta
	yd(k+1) = b0*ud(k) + b1*ud(k-1) + b2*ud(k-2) - a1*yd(k) - a2*yd(k-1);
	end
	end
	
\end{lstlisting}

\begin{lstlisting}[style=matlabstyle,
	caption={Modelo mejorado de la planta mediante los datos recolectados del osciloscopio},
	label={matlab:modelo}
	]

	close all
	clear all
	
	% Cargar CSV
	data = readmatrix('opltab.csv');
	
	% Ignorar cabeceras, quedarte solo con datos numéricos
	u = data(3:end, 5); % columna E = salida
	y = data(3:end, 11); % columna K = entrada
	Ts = 0.000099999997474; % tiempo de muestreo de B3
	
	% Crear objeto de identificación
	data_id = iddata(y, u, Ts);
	
	% Estimar una función de transferencia discreta (ejemplo: 2 polos, 1 cero)
	sys0 = tfest(data_id, 2, 0);
	
	% % Ver resultado
	%  step(sys);
	% % Crear datos de identificación
	% data_id = iddata(y, u, Ts);
	% 
	% % Probar con distintos órdenes
	% sys1 = tfest(data_id, 1, 0); % 1 polo
	% sys2 = tfest(data_id, 2, 1); % 2 polos, 1 cero
	% 
	% % Comparar
	figure;
	compare(data_id,sys0);
	
	G = tf(sys0);
	T = 1.25e-3;
	Gd = c2d(G, T, 'zoh');
	disp('Planta digital G(z):')
	zpk(Gd)
	controlSystemDesigner(Gd)
	
	save('planta.mat', 'G');
\end{lstlisting}


\begin{lstlisting}[style=matlabstyle,caption={Segunda hoja de cálculos utilizada.},label={matlab:calculo2}]
	close all
	clear all
	
	addpath('..\Lab1\')
	
	
	load('planta.mat');
	%% Analizamos en el tiempo
	[tr, ts, wn] = plot_step_info(G);
	disp('Planta continua G(s):')
	G
	zpk(G)
	
	%% Paso 1: Definir periodo de muestreo
	% Se busca una mejora de 4 en el tiempo de rising , o sea tr = 10 ms
	
	T = 1.25e-3;
	%Gcl = feedback(G,1);
	%figure;
	%step(Gcl)
	%T = stepinfo(Gcl).SettlingTime/8;
	
	%% Paso 2: Digitalizar con ZOH
	Gd = c2d(G, T, 'zoh');
	disp('Planta digital G(z):')
	Gd
	zpk(Gd)
	figure;
	pzmap(Gd);
	title('Lugar de raíces de G(z).')
	
	zgrid;
	z = tf([1 0],1,T);
	
	%% Paso 3: Análisis en tiempo discreto lazo abierto
	figure;
	
	% Obtenemos salida y tiempo de la función step
	[y, t] = step(Gd);
	
	% Graficamos con stairs (propio de señales discretas)
	stairs(t*1000, y, 'LineWidth',1.4);
	
	title('Respuesta al escalón de la planta digitalizad')
	xlabel('Tiempo [ms]');
	ylabel('Salida');
	grid on;
	ax = gca;
	ax.XMinorTick = 'on';  % activamos minor ticks
	ax.YMinorTick = 'on';
	grid minor;
	
	%% Paso 4: Definimimos los valores de la simulación
	umin = 0;
	umax =  4.08;
	refmin = 1.5;
	refmax = 2.5;
	n_per_seg = 500;
	%% Paso 5: Lugar de raíces para zita = 0.7, compensador P
	%figure
	
	%rlocus(Gd)
	%zgrid     % agrega la grilla en el plano-z
	z0 = 0.380927627742363;
	p0 = 0.836970379456654;
	K = 0.226961597782889;
	C_1 = (z-z0)/(z-p0);
	figure;
	Gc1 = Gd*C_1;
	rlocus(Gc1);
	title('Lugar de raíces de G(z) compensado con un lag-filter de 1º orden, anulando el polo más rápido de la planta.')
	
	zgrid     % agrega la grilla en el plano-z
	%pause;
	%[K, ~] = rlocfind(Gc1);   % hacés click donde querés los polos
	%K = 0.73139;
	%K = 0.82543;
	%K = 1.796463473892606;
	Gc1f = feedback(K*Gc1,1);
	%figure;
	%step(Gc1f)
	info = stepinfo(Gc1f);
	info
	
	info_ext = plot_step_annot(Gc1f, 'la planta compensada con un lag-filter de 1º orden, anulando el polo más rápido de la planta.');
	%step(Gc3f);
	info = stepinfo(Gc1f);
	info
	info_ext
	
	[td, refd, yd, ud, ed, coefs] = sim_compensador_first_order( ...
	Gd, z0, p0, K, T, umin, umax, refmin, refmax, n_per_seg);
	
	Nini = 100;              % descartar al inicio
	Nfin = 100;              % descartar al final
	idx0 = Nini + 1;         % índice inicial válido
	idx1 = length(td) - Nfin; % índice final válido
	
	td   = td(idx0:idx1)-td(idx0);
	refd = refd(idx0:idx1);
	yd   = yd(idx0:idx1);
	ud   = ud(idx0:idx1);
	ed   = ed(idx0:idx1);
	
	% Si querés que el tiempo arranque en 0
	
	
	
	
	
	figure;
	
	% ---------- Subplot 1 ----------
	axAbs = subplot(2,1,1);  % eje izquierdo (absoluto)
	hold(axAbs,'on'); grid(axAbs,'on');
	
	% y[k] en rojo (abs)
	hY = stairs(axAbs, td, yd, 'r-', 'LineWidth', 1.4);
	
	% Armamos eje derecho transparente
	axPct = axes('Position', get(axAbs,'Position'), ...
	'Color','none', 'YAxisLocation','right', ...
	'XLim', get(axAbs,'XLim'), 'XTick',[], 'Box','off');
	hold(axPct,'on');
	
	% r[k] en negro, graficado en %
	ref_pct = 100*(refd - refmin)/(refmax - refmin);
	hR = stairs(axPct, td, refd, 'k--', 'LineWidth', 1.2);
	
	% ======= Cálculo de límites con margen =======
	yd_pct  = 100*(yd - refmin)/(refmax - refmin);
	pctAll  = [yd_pct; ref_pct];
	
	% Valores extremos en % con margen del 5 %
	rawMin = min(pctAll);
	rawMax = max(pctAll);
	span   = rawMax - rawMin;
	pctMin = rawMin - 0.05*span;
	pctMax = rawMax + 0.05*span;
	
	% Redondeamos a múltiplos de 20 para ticks
	stepPct = 20;
	tickMin = stepPct*floor(pctMin/stepPct);
	tickMax = stepPct*ceil (pctMax/stepPct);
	pctTicks = tickMin:stepPct:tickMax;
	
	% Convertimos a absolutos
	valTicks = refmin + (pctTicks/100)*(refmax - refmin);
	
	% Aplicamos a ambos ejes
	set(axAbs,'YLim',[valTicks(1) valTicks(end)],'YTick',valTicks);
	set(axPct,'YLim',[valTicks(1) valTicks(end)],'YTick',valTicks,...
	'YTickLabel',compose('%.0f %%',pctTicks));
	
	% Etiquetas
	xlabel(axAbs,'Tiempo [s]');
	ylabel(axAbs,'Respuesta [valor absoluto]');
	ylabel(axPct,'Escala relativa a ref [%]');
	title(axAbs,'Respuesta discreta con compensador 1º orden, anulando el polo más rápido de la planta.');
	legend(axAbs, [hY, hR], {'y[k]', 'r[k]'}, 'Location', 'best');
	
	% ---------- Subplot 2: esfuerzo de control ----------
	axU = subplot(2,1,2);
	stairs(axU, td, ud, 'LineWidth',1.2); grid(axU,'on');
	yline(axU, umax,'r:'); yline(axU, umin,'r:');
	xlabel(axU,'Tiempo [s]'); ylabel(axU,'u[k]');
	title(axU,'Esfuerzo de control con saturación');
	
	% --- Margen de 5% en eje Y ---
	uMin = min(ud);
	uMax = max(ud);
	span = uMax - uMin;
	ylim(axU, [uMin - 0.05*span, uMax + 0.05*span]);
	
	
	grid(axAbs,'on');    % grilla principal
	grid(axAbs,'minor'); % grilla secundaria
	
	grid(axPct,'on');
	grid(axPct,'minor');
	
	grid(axU,'on');
	grid(axU,'minor');
	
	
	
	%% Paso 7: Lugar de raices para zita = 0.7 y y ESS = 0, compensador PI de segundo orden
	z0_0 = 0.380927627742363;
	z0_1 = 0.821584532561100;
	
	
	p0_0 = 0.753973536568200;
	p0_1 = 1;
	K = 0.287529112579611;
	C_3 = ((z-z0_0)*(z-z0_1))/((z-p0_0)*(z-p0_1));
	Gc3 =Gd*C_3; %z=0.8433 %8433
	figure;
	rlocus(Gc3)
	title('Lugar de raíces de G(z) compensado con un lag-filter de 2º orden, fijando un polo en z = 1 y anulando ambos polos de la planta.');
	
	zgrid
	%pause;
	%[K, ~] = rlocfind(Gc3);   % hacés click donde querés los polos
	%K = 1.14;
	%figure;
	Gc3f= feedback(K*Gc3,1);
	
	info_ext = plot_step_annot(Gc3f, 'la planta compensada con un lag-filter de 2º orden, fijando un polo en z = 1 y anulando ambos polos de la planta.');
	%step(Gc3f);
	info = stepinfo(Gc3f);
	info
	info_ext
	
	
	% --- Llamada a tu simulador (misma interfaz que tu first_order) ---
	[td, refd, yd, ud, ed, coefs] = sim_compensador_second_order( ...
	Gd, z0_0, z0_1, p0_0, p0_1, K, T, umin, umax, refmin, refmax, n_per_seg);
	
	
	Nini = 100;              % descartar al inicio
	Nfin = 100;              % descartar al final
	idx0 = Nini + 1;         % índice inicial válido
	idx1 = length(td) - Nfin; % índice final válido
	
	td   = td(idx0:idx1)-td(idx0);
	refd = refd(idx0:idx1);
	yd   = yd(idx0:idx1);
	ud   = ud(idx0:idx1);
	ed   = ed(idx0:idx1);
	
	
	
	
	figure;
	
	% ---------- Subplot 1 ----------
	axAbs = subplot(2,1,1);  % eje izquierdo (absoluto)
	hold(axAbs,'on'); grid(axAbs,'on');
	
	% y[k] en rojo (abs)
	hY = stairs(axAbs, td, yd, 'r-', 'LineWidth', 1.4);
	
	% Armamos eje derecho transparente
	axPct = axes('Position', get(axAbs,'Position'), ...
	'Color','none', 'YAxisLocation','right', ...
	'XLim', get(axAbs,'XLim'), 'XTick',[], 'Box','off');
	hold(axPct,'on');
	
	% r[k] en negro, graficado en %
	ref_pct = 100*(refd - refmin)/(refmax - refmin);
	hR = stairs(axPct, td, refd, 'k--', 'LineWidth', 1.2);
	
	% ======= Cálculo de límites con margen =======
	yd_pct  = 100*(yd - refmin)/(refmax - refmin);
	pctAll  = [yd_pct; ref_pct];
	
	% Valores extremos en % con margen del 5 %
	rawMin = min(pctAll);
	rawMax = max(pctAll);
	span   = rawMax - rawMin;
	pctMin = rawMin - 0.05*span;
	pctMax = rawMax + 0.05*span;
	
	% Redondeamos a múltiplos de 20 para ticks
	stepPct = 20;
	tickMin = stepPct*floor(pctMin/stepPct);
	tickMax = stepPct*ceil (pctMax/stepPct);
	pctTicks = tickMin:stepPct:tickMax;
	
	% Convertimos a absolutos
	valTicks = refmin + (pctTicks/100)*(refmax - refmin);
	
	% Aplicamos a ambos ejes
	set(axAbs,'YLim',[valTicks(1) valTicks(end)],'YTick',valTicks);
	set(axPct,'YLim',[valTicks(1) valTicks(end)],'YTick',valTicks,...
	'YTickLabel',compose('%.0f %%',pctTicks));
	
	% Etiquetas
	xlabel(axAbs,'Tiempo [s]');
	ylabel(axAbs,'Respuesta [valor absoluto]');
	ylabel(axPct,'Escala relativa a ref [%]');
	title(axAbs,'Respuesta discreta con compensador 2º orden, fijando un polo en z = 1 y anulando ambos polos de la planta');
	legend(axAbs, [hY, hR], {'y[k]', 'r[k]'}, 'Location', 'best');
	
	% ---------- Subplot 2: esfuerzo de control ----------
	axU = subplot(2,1,2);
	stairs(axU, td, ud, 'LineWidth',1.2); grid(axU,'on');
	yline(axU, umax,'r:'); yline(axU, umin,'r:');
	xlabel(axU,'Tiempo [s]'); ylabel(axU,'u[k]');
	title(axU,'Esfuerzo de control con saturación');
	
	% --- Margen de 5% en eje Y ---
	uMin = min(ud);
	uMax = max(ud);
	span = uMax - uMin;
	ylim(axU, [uMin - 0.05*span, uMax + 0.05*span]);
	
	
	
	
	
	
	
	grid(axAbs,'on');    % grilla principal
	grid(axAbs,'minor'); % grilla secundaria
	
	grid(axPct,'on');
	grid(axPct,'minor');
	
	grid(axU,'on');
	grid(axU,'minor');
	
	
	%% impresion antigua
	% % --- Gráficas estilo informe ---
	% figure; 
	% subplot(2,1,1);
	% stairs(td, refd, 'k--','LineWidth',1.0); hold on;
	% stairs(td, yd,   'LineWidth',1.4);
	% grid on; xlabel('Tiempo [s]'); ylabel('Respuesta de la planta/Referencia');
	% title('Respuesta discreta con compensador 2º orden'); legend('r[k]','y[k]','Location','best');
	% 
	% subplot(2,1,2);
	% stairs(td, ud, 'LineWidth',1.2);
	% yline(umax,'r:'); yline(umin,'r:');
	% grid on; xlabel('Tiempo [s]'); ylabel('u[k]');
	% title('Esfuerzo de control con saturación');
\end{lstlisting}
\section{Código desarrollado para el PSoC }
\begin{lstlisting}[style=cstyle,
	caption={Código desarrollado para la implementación de los compensadores con el PSoC en lenguaje C.},
	label={c:psoc}
	]
	
	\\XDXDXD
\end{lstlisting}

	
\end{document}
