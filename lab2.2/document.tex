% ================== BORRADOR PUERCO EN FORMATO IEEE ==================
\documentclass[conference]{IEEEtran}

% ======= Preambulo mínimo pero útil para escribir rápido =======
\usepackage[spanish, es-nodecimaldot]{babel}
\usepackage[utf8]{inputenc}
\usepackage[T1]{fontenc}
\usepackage{amsmath, amssymb, amsfonts}
\usepackage{siunitx}

\usepackage{siunitx}
\sisetup{
	output-decimal-marker = {,}, % coma decimal en vez de punto
	group-separator = {.},       % separador de miles con punto
	group-minimum-digits = 4,    % agrupa desde 4 dígitos
	per-mode = symbol            % para usar "/" en unidades
}

\usepackage{graphicx}
\usepackage{caption}
\usepackage{subcaption}
\usepackage{booktabs}
\usepackage{hyperref}
\usepackage{enumitem}
\setlist{leftmargin=*, itemsep=2pt, topsep=4pt}
\usepackage{xcolor}

% ======= Macros rápidas =======
\newcommand{\zetaD}{\zeta}
\newcommand{\wn}{\omega_n}
\newcommand{\Ts}{T_s}
\newcommand{\ESS}{\mathrm{ESS}}
\newcommand{\Gz}{G(z)}
\newcommand{\Gs}{G(s)}

\usepackage{float}

% Macro para insertar figuras centradas con ancho ajustable
% Ejemplo de uso \insertarfigura{ruta}{leyenda}{etiqueta}{escala entre 0 y 1}
\newcommand{\insertarfigura}[4]{
	\begin{figure}[H]
		\centerline{\includegraphics[width=#4\linewidth]{#1}}
		\caption{#2}
		\label{#3}
	\end{figure}
	% Referencia cruzada si se desea en el texto: véase la Fig.~\ref{#4}
}
% Macro: \insertarfigurawide{ruta}{caption}{label}{scale}
\usepackage{cuted} % en el preámbulo

% 
\newcommand{\insertarfigurawide}[4]{%
	\begin{figure*}[ht]
		\centering
		\includegraphics[width=#4\textwidth]{#1}
		\caption{#2}
		\label{#3}
	\end{figure*}
}

\usepackage{subcaption} % en el preámbulo

% ======= Metadatos =======
\title{Práctica de Laboratorio 2: Diseño de controladores por el método de lugar de raíces.}

\author{
	\IEEEauthorblockN{Elías Álvarez}
	\IEEEauthorblockA{Carrera de Ing. Electrónica \\ 
		Universidad Católica Nuestra Señora de la Asunción \\
		Asunción, Paraguay \\
		Email: elias.alvarez@universidadcatolica.edu.py}
	\and
	\IEEEauthorblockN{Tania Romero}
	\IEEEauthorblockA{Carrera de Ing. Electrónica \\
		Universidad Católica Nuestra Señora de la Asunción \\
		Asunción, Paraguay \\
		Email: tania.romero@universidadcatolica.edu.py}
		\and
	\IEEEauthorblockN{\hspace*{3.5em}Docente: Lic. Montserrat González}
	\IEEEauthorblockA{
		\hspace*{3.5em}Facultad de Ingeniería \\
		\hspace*{3.5em}Universidad Católica Nuestra Señora de la Asunción \\
		\hspace*{3.5em}Asunción, Paraguay}
	\and
	\IEEEauthorblockN{\hspace*{3.5em}Docente: PhD. Enrique Vargas}
	\IEEEauthorblockA{%
		\hspace*{3.5em}Facultad de Ingeniería \\  % ← empuja a la derecha
		\hspace*{3.5em}Universidad Católica Nuestra Señora de la Asunción \\
		\hspace*{3.5em}Asunción, Paraguay}
}

\begin{document}
	\maketitle
	
	\section{Introducción}


\section{Objetivos}

\begin{itemize}
	\item Diseñar controladores digitales para una planta analógica utilizando estados estimados.
	\item Aplicar el diseño de realimentación de estados con estimadores de predicción y de actualización.
	\item Presentar y discutir los resultados experimentales en comparación con los obtenidos por simulación, dentro de un informe técnico razonado.
\end{itemize}

	
	\section{Desarrollo}

\subsection{Modelado del Sistema}

\subsubsection{Obtención de las matrices del sistema $F$, $G$, $H$ y $J$}

Para el modelado del sistema se parte del circuito mostrado en la Figura~\ref{fig:circuit_planta}, a partir del cual se determinan las ecuaciones de estado mediante el análisis de los lazos de realimentación y las relaciones de tensión en los componentes.

\insertarfigura{Otros/circuit.png}{Circuito de la planta.}{fig:circuit_planta}{1}

El sistema se describe mediante las siguientes ecuaciones en espacio de estados:

\begin{equation}
	\dot{x}(t) = F\,x(t) + G\,V_i(t)
	\label{eq:1}
\end{equation}

\begin{equation}
	y(t) = H\,x(t) + J\,V_i(t)
	\label{eq:2}
\end{equation}

Para obtener las expresiones de las variables de estado, se parte del equivalente del paralelo entre un resistor y un capacitor:
\[
R \parallel \frac{1}{sC} = \frac{R}{1 + sRC}
\]

Considerando que ambos amplificadores operacionales se encuentran en configuración no inversora, se obtienen las siguientes relaciones:

\[
V_a = \frac{-R_2}{1 + sR_2C_1}\frac{V_i}{R_1}
\quad \Rightarrow \quad
sV_a = -\frac{1}{R_1C_1}V_i - \frac{1}{R_2C_1}V_a
\]

\[
V_o = -\frac{R_4}{1 + sR_4C_2}\frac{V_a}{R_3}
\quad \Rightarrow \quad
sV_o = -\frac{1}{R_3C_2}V_a - \frac{1}{R_4C_2}V_o
\]

Definiendo como variables de estado $x_1(t) = V_a$ y $x_2(t) = V_o$, las ecuaciones anteriores se expresan en forma matricial como:

\[
\begin{bmatrix}
	\dot{x}_1(t) \\[4pt]
	\dot{x}_2(t)
\end{bmatrix}
=
\begin{bmatrix}
	-\dfrac{1}{R_2C_1} & 0 \\[4pt]
	-\dfrac{1}{R_3C_2} & -\dfrac{1}{R_4C_2}
\end{bmatrix}
\begin{bmatrix}
	x_1(t) \\[4pt]
	x_2(t)
\end{bmatrix}
+
\begin{bmatrix}
	-\dfrac{1}{R_1C_1} \\[4pt]
	0
\end{bmatrix}
V_i(t)
\]

y la ecuación de salida queda definida como:

\[
y(t) =
\begin{bmatrix}
	0 & 1
\end{bmatrix}
\begin{bmatrix}
	x_1(t) \\[4pt]
	x_2(t)
\end{bmatrix}
+ 0\cdot V_i(t)
\]

Sustituyendo los valores de los componentes 
$C_1 = 211.1\times10^{-9}\,\text{F}$, 
$R_1 = 80.55\times10^{3}\,\Omega$, 
$R_2 = 81.09\times10^{3}\,\Omega$, 
$C_2 = 103.07\times10^{-9}\,\text{F}$, 
$R_3 = 14.878\times10^{3}\,\Omega$ y 
$R_4 = 14.76\times10^{3}\,\Omega$, 
se obtienen las siguientes matrices numéricas:

\[
F =
\begin{bmatrix}
	-58.42 & 0 \\[4pt]
	-652.11 & -657.37
\end{bmatrix}, \quad
G =
\begin{bmatrix}
	-58.81 \\[4pt]
	0
\end{bmatrix}, \quad
H =
\begin{bmatrix}
	0 & 1
\end{bmatrix}
\]
\begin{equation}
	y \quad
	J = 0
	\label{eq:J}
\end{equation}
\subsubsection{Mostrar el diagrama de bloques del sistema}


\subsection{Discretización del Sistema}
\insertarfigura{Otros/Diagramas1.png}{Diagrama de bloques del sistema continuo.}{fig:diag_continuo}{1}

\subsubsection{Elección del tiempo de muestreo $T_s = 1~\text{ms}$}

Para la discretización del sistema continuo descrito por las ecuaciones~(\ref{eq:1}) y~(\ref{eq:2}), se busca obtener un modelo equivalente en tiempo discreto que relacione las variables de estado y la señal de entrada en instantes de muestreo definidos.  
Las ecuaciones del sistema discreto se expresan como:

\begin{equation}
	X(k+1) = A\,X(k) + B\,u(k)
	\label{eq:3}
\end{equation}

\begin{equation}
	Y(k) = C\,X(k) + D\,u(k)
	\label{eq:4}
\end{equation}

Usando las matrices continuas $F$, $G$, $H$ y $J$ obtenidas previamente, las matrices discretas se determinan mediante las siguientes expresiones:

\[
A = \mathrm{e}^{F T_s}, \qquad 
B = F^{-1}\!\left(\mathrm{e}^{F T_s} - I\right)G, \qquad 
C = H \qquad 
\]

\[
 \& \quad D = J
\]
\subsubsection{Obtención de las matrices discretas $A$, $B$, $C$ y $D$}

Con un tiempo de muestreo $T_s = 1~\text{ms}$, se obtienen las siguientes matrices discretizadas:

\[
A =
\begin{bmatrix}
	0.943 & 0 \\[4pt]
	-0.462 & 0.518
\end{bmatrix}, \qquad
B =
\begin{bmatrix}
	-0.0571 \\[4pt]
	0.0152
\end{bmatrix}, 
\]
\begin{equation}
	C =
	\begin{bmatrix}
		0 & 1
	\end{bmatrix}, \qquad \& \qquad
	D = 0
	\label{eq:d}
\end{equation}

Estas matrices representan el modelo digital equivalente del sistema continuo, y serán utilizadas posteriormente para el diseño del controlador e implementación en el \texttt{PSoC}.

\subsubsection{Diagrama de bloques del sistema discretizado}

En la Figura~\ref{fig:diag_discreto} se presenta el diagrama de bloques correspondiente al sistema discretizado, donde se observa la relación entre las variables de estado, la entrada $u(k)$ y la salida $Y(k)$.

\insertarfigura{Otros/Diagramas2.png}{Diagrama de bloques del sistema discretizado.}{fig:diag_discreto}{1}


\subsubsection{Verificación de la controlabilidad y observabilidad del sistema}

La matriz de controlabilidad se obtiene a partir de la siguiente relación general:

\[
x(n) - A^{n}x(0) = \sum_{i=0}^{n-1} A^{n-i-1}B\,u(i)
\]

lo que lleva a la siguiente forma matricial:

\[
x(n) - A^{n}x(0) =
\begin{bmatrix}
	A^{n-1}B & A^{n-2}B & \cdots & AB & B
\end{bmatrix}
\begin{bmatrix}
	u(0) \\[2pt]
	u(1) \\[2pt]
	\vdots \\[2pt]
	u(n-1)
\end{bmatrix}
\]

De esta expresión, se define la matriz de controlabilidad como:

\[
\mathcal{C} =
\begin{bmatrix}
	B & AB
\end{bmatrix}
\]

Para el sistema analizado, la matriz resultante es:

\[
\mathcal{C} =
\begin{bmatrix}
	-0.05344 & -0.05713 \\[4pt]
	0.03435 & 0.01527
\end{bmatrix}
\]

El rango de esta matriz es $n = 2$, lo que indica que el sistema es completamente controlable.

\bigskip
La matriz de observabilidad se obtiene a partir de la expresión general:

\[
Y(n-1) = 
\begin{bmatrix}
	C \\[4pt]
	CA \\[4pt]
	\vdots \\[4pt]
	CA^{n-1}
\end{bmatrix}
X(0)
\]

Por lo tanto, la matriz de observabilidad queda definida como:

\[
\mathcal{O} =
\begin{bmatrix}
	C \\[4pt]
	CA
\end{bmatrix}
\]

Sustituyendo los valores del sistema:

\[
\mathcal{O} =
\begin{bmatrix}
	0 & 1 \\[4pt]
	-0.462 & 0.518
\end{bmatrix}
\]

El rango de la matriz de observabilidad también resulta ser $n = 2$.\\  
Por lo tanto, se concluye que el sistema es **completamente controlable y observable**, cumpliendo con las condiciones necesarias para el diseño de control mediante realimentación de estados.

\subsubsection{Comparar los resultados obtenidos con las simulaciones realizadas en \texttt{MATLAB}}
\textbf{Resultados de las matrices continuas:}
\[
F =
\begin{bmatrix}
	-58.42 & 0 \\[4pt]
	-652.10 & -657.30
\end{bmatrix}, \quad
G =
\begin{bmatrix}
	-58.81 \\[4pt]
	0
\end{bmatrix}, \quad
H =
\begin{bmatrix}
	0 & 1
\end{bmatrix}
\]

\[
	\& \quad
	J = [\,0\,]
\]

\textbf{Resultados de las matrices discretas:}
\[
A =
\begin{bmatrix}
	0.9433 & 0 \\[4pt]
	-0.4628 & 0.5182
\end{bmatrix}, \quad
B =
\begin{bmatrix}
	-0.05712 \\[4pt]
	0.01527
\end{bmatrix}, \quad
C =
\begin{bmatrix}
	0 & 1
\end{bmatrix}
\]

\[
	\& \quad
	D = [\,0\,]
\]
Se observa que los resultados de las matrices obtenidas son congruentes con los valores calculados en las ecuaciones~(\ref{eq:J}) y~(\ref{eq:d}), verificando la coherencia entre el modelo teórico y los resultados obtenidos mediante \texttt{MATLAB}.

\insertarfigurawide{matlab/Sim1_noImplementable.png}{Respuesta simulada en \texttt{MATLAB} correspondiente al Caso 2($f_s = 1000 Hz$).}{fig:simulacion_noImplementable}{1}

\insertarfigurawide{matlab/Sim2_noImplementable.png}{Respuesta simulada en \texttt{MATLAB} correspondiente al Caso 2($f_s = 1000 Hz$).}{fig:simulacion_noImplementable2}{1}

En las figuras \ref{fig:simulacion_noImplementable} y \ref{fig:simulacion_noImplementable2}, del \texttt{MATLAB} se pueden observar que no es implementable.
	\subsection{Resultados}
	\begin{enumerate}[label=5.\arabic*.]
		\item Gráficas del sistema con los compensadores.
		\item Comparar Matlab vs.\ resultados experimentales.
	\end{enumerate}
	
	% ================== APÉNDICE OPCIONAL ==================
	\appendices
	\section{Snippets Matlab (borrador)}
	\begin{verbatim}
		G = tf(num, den);
		Ts = 0.001; % placeholder
		Gz = c2d(G, Ts, 'zoh');
		step(Gz); grid on;
	\end{verbatim}
	
	\section{Notas PSoC (borrador)}
	\begin{verbatim}
		// ISR @ Ts:
		// 1) Leer ADC -> yk
		// 2) Calcular error ek = rk - yk
		// 3) Calcular uk con ecuaciones en diferencias
		// 4) Escribir DAC -> uk
	\end{verbatim}
	
\end{document}
