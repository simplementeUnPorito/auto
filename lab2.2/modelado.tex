\subsection{Modelado del Sistema}
\subsubsection{Obtención de la función de transferencia en lazo abierto}

La planta puede interpretarse como la conexión en cascada de dos filtros activos de primer orden. Cada uno posee la misma topología: un amplificador operacional en configuración inversora cuya impedancia de realimentación está compuesta por una resistencia en paralelo con un capacitor.

\paragraph{Impedancia de realimentación.}  
Para el paralelo $R_f \parallel C_f$, se obtiene:
\begin{equation}
	Z_f \;=\; R_f \parallel \frac{1}{sC_f}
	\;=\; \frac{R_f}{1 + s R_f C_f}
	\label{eq:Zf}
\end{equation}

\insertarfigura{./img/circuito.png}{Circuito esquemático de la planta, que emula el comportamiento dinámico de un motor de CC.}{fig:circuitoini}{1}

\insertarfigura{./img/CircRealim.png}{Configuración inversora con un polo simple en la realimentación.}{fig:impedanciaReal}{0.5}

El sistema trabaja sobre una tensión de referencia en continua $V_{\text{cc}}/2 = 2.5$ V. En los cálculos posteriores se toma dicho valor como punto de referencia.

\paragraph{Ganancia de una etapa (entrada inversora).}  
Con $V_{\text{ref}}=0$, se cumple el cortocircuito virtual ($V_p = V_n = 0$). Aplicando KCL en el nodo inversor:
\[
\frac{V_i}{R_i} \;=\; \frac{-V_o}{Z_f}
\quad\Rightarrow\quad
\frac{V_o}{V_i} \;=\; -\,\frac{Z_f}{R_i}
\]
y reemplazando \eqref{eq:Zf}:
\begin{equation}
	\frac{V_o}{V_i}\Bigg|_{V_{\text{ref}}=0}
	= -\,\frac{R_f}{R_i}\,\frac{1}{1+sR_f C_f}
	\label{eq:gain_inverting}
\end{equation}

\paragraph{Encadenamiento de etapas.}  
Como ambas etapas AO1 y AO2 responden a la forma \eqref{eq:gain_inverting}, la ganancia total en lazo abierto resulta del producto de sus transferencias:
\[
G_{\text{ol}}(s)
= \Big(-\frac{Z_{f1}(s)}{R_{i1}}\Big)\,
\Big(-\frac{Z_{f2}(s)}{R_{i2}}\Big)
\]

\paragraph{Implementación en Matlab.}  
El siguiente código genera cada etapa de primer orden, calcula la función en lazo abierto y extrae información temporal de la respuesta al escalón:



\begin{strip}
	\vspace{-\baselineskip} % ajusta si ves demasiado aire arriba
	\noindent
	\begin{minipage}{0.98\textwidth}
		\begin{lstlisting}[style=matlabstyle,caption={Script en Matlab},label={lst:mat}]
			
			%% Definicion de parametros
			R_1 = 15e3;
			R_3 = 15e3;
			C_2 = 100e-9;
			R_2 = 75e3;
			R_4 = 75e3;
			C_1 = 0.2e-6;
			%% Generar funcion de transferencia d
			numStage = [-R_3/R_1 -R_4/R_2];
			denStage = { [C_2*R_3 1], [C_1*R_4 1] };
			% Usamos celdas para guardar los tf de cada stage
			Gstage = cell(1,2);
			G = 1;
			for i = 1:2
			Gstage{i} = tf(numStage(i), denStage{i});
			G = G*Gstage{i};
			end
			%% Analizamos en el tiempo
			[tr, ts, wn] = plot_step_info(G);
			
			function [tr, ts, wn] = plot_step_info(G)
			info = stepinfo(G);
			tr = info.RiseTime;
			ts = info.SettlingTime;
			wn = 1.8 / tr;
			t_end = 1.1 * ts;
			if ~isfinite(t_end) || t_end <= 0, 
			t_end = 5 * max(tr, 1e-3); 
			end
			t = linspace(0, t_end, 2*pi/wn);
			[y, tout] = step(G, t);
			figure; plot(tout, y, 'LineWidth', 1.4); grid on; hold on;
			xlabel('Tiempo [s]'); ylabel('Salida');
			title(sprintf('Escalón: tr=%.4gs, ts=%.4gs, \\omega_n=%.4g rad/s', tr, ts, wn));
			xline(tr, '--', sprintf('  t_r=%.3g s', tr), 'LabelOrientation','horizontal');
			xline(ts,  ':', sprintf('  t_s=%.3g s', ts), 'LabelOrientation','horizontal');
			legend('G(t)', 'Location', 'best');
			end
			
		\end{lstlisting}
	\end{minipage}
	\vspace{-\baselineskip} % ajusta si ves aire abajo
\end{strip}

\insertarfigurawide{./img/EscalonLazoAb.png}{Respuesta al escalón del sistema en lazo abierto.}{fig:escalonLazoab}{1}

\paragraph{Resultados numéricos.}  
A partir de la respuesta de la figura~\ref{fig:escalonLazoab} se obtuvieron:
\begin{itemize}
	\item \textbf{Tiempo de subida:} $t_r \approx 0.0332$ s (33.2 ms)
	\item \textbf{Tiempo de establecimiento (2\%):} $t_s \approx 0.0603$ s (60.3 ms)
	\item \textbf{Frecuencia natural estimada:} $\omega_n \approx 54.2$ rad/s
\end{itemize}




