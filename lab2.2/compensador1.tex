Para el primer compensador, a partir del an\'alisis del \emph{ángulo faltante} $\phi$ para cumplir con la condici\'on de fase (Ec.~\eqref{eq:condicionFase}), la alternativa m\'as sencilla consiste en dise\~nar un filtro \emph{lag} en cascada de primer orden.  

Una opci\'on es cancelar uno de los polos originales de la planta mediante la ubicaci\'on de un cero en la misma posici\'on, y colocar a continuaci\'on un nuevo polo de manera que el aporte de fase conjunto de estos dos compense el desfase requerido. De esta forma, el punto seleccionado pasar\'a a pertenecer al lugar de las ra\'ices.  

En este caso, se procede a cancelar el polo m\'as lento del sistema. Para ello, se analiza el tri\'angulo formado por el cero, el nuevo polo y el punto deseado en el plano-$z$.  

(Poner esquema ilustrativo del tri\'angulo formado por cero, polo y punto deseado)  

Ecuacionando, por la ley de senos se obtienen las siguientes 2 relaciones:  
\begin{equation}
	\theta = \arcsin\!\left(\frac{y}{\sqrt{(x-z_c)^2+y^2}}\right)
	\label{eq:theta_c1}
\end{equation}
\begin{equation}
	\frac{p_c - z_c}{\sin(180^\circ+\phi)} 
	= \frac{\sqrt{(x-z_c)^2+y^2}}{\sin(-\phi+\theta)}
	\label{eq:pz_c1}
\end{equation}

Usando las Ec.~\eqref{eq:theta_c1} y \eqref{eq:pz_c1}, se concluye que:  
\begin{equation}
	p_c = 
	\frac{\sin(180^\circ+\phi)\,\sqrt{(x-z_c)^2+y^2}}
	{\sin\!\left(-\arcsin\!\left(\tfrac{y}{\sqrt{(x-z_c)^2+y^2}}\right)-\phi\right)}
	+ z_c
	\label{eq:p_c1}
\end{equation}

Con la Ec.~\eqref{eq:p_c1}, se obtiene un filtro \emph{lag} que asegura que el punto deseado pertenezca al lugar de las ra\'ices. Restar\'a determinar el valor de la ganancia $K$ que garantice que los polos en lazo cerrado se ubiquen efectivamente en dicha posici\'on.  

Esto puede realizarse aplicando el \emph{criterio de magnitud}, expresado en la Ec.~\eqref{eq:criterioMagnitud}:  
\begin{equation}
	K = \frac{\prod_i \sqrt{(p_i-x)^2+y^2}}
	{\prod_j \sqrt{(z_j-x)^2+y^2}}
	\label{eq:criterioMagnitud}
\end{equation}

No obstante, tambi\'en es posible emplear directamente la funci\'on \texttt{rlocus} de \textsc{Matlab}, lo cual facilita el c\'alculo.  

\insertarfigura{./Img/lugarDeRaicesCompensador1_completo.png}
{Lugar de las ra\'ices de la planta digitalizada y compensada con un filtro lag de primer orden.}
{ref:rlocusC1_v1}{1}

\insertarfigura{./Img/lugarDeRaicesCompensador1_zoom.png}
{Lugar de las ra\'ices de la planta digitalizada y compensada con un filtro lag de primer orden (detalle ampliado).}
{ref:rlocusC1_zoom_v1}{1}

Una vez identificado el compensador y la ganancia, se procede a cerrar el lazo y analizar la respuesta en el tiempo. La Fig.~\ref{ref:stepC1_v1} muestra la respuesta al escal\'on con el primer compensador dise\~nado.  

\insertarfigura{./Img/stepCompensador1.png}
{Respuesta al escal\'on del sistema en lazo cerrado con el primer compensador dise\~nado.}
{ref:stepC1_v1}{1}

Se observa que se cumplen las condiciones de desempe\~no solicitadas; sin embargo, aparece un error en estado estacionario considerable. Esto se debe a que el sistema en lazo abierto $K \cdot C(z) \cdot G_d(z)$ es de tipo 0, por lo cual no posee la capacidad de seguir correctamente a una entrada escal\'on.  

Asimismo, se eval\'ua el esfuerzo de control aplicado (Fig.~\ref{ref:stepYEsfuerzoC1_v1}). Si bien resulta implementable, se aprecia que presenta una magnitud inicial elevada con picos pronunciados.  

\insertarfigura{./Img/step_y_esfuerzoCompensador1.png}
{Respuesta al escal\'on del sistema en lazo cerrado con el primer compensador dise\~nado, junto con el esfuerzo de control aplicado por el compensador.}
{ref:stepYEsfuerzoC1_v1}{1}
\balance
\clearpage
