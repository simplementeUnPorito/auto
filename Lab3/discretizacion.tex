\subsection{Discretización de la Planta}
\subsubsection{Se debe elegir un tiempo de muestreo adecuado}
La planta a discretizar está dada por:
\[
G(s) = \frac{K}{(s+a)(s+b)}.
\]

Para discretizarla, se utiliza el método de la \emph{transformación de la respuesta al impulso}. La relación general es:
\begin{equation}
G(z) = (1 - z^{-1}) \, \mathcal{Z}\left\{\mathcal{L}^{-1}\left(\frac{G(s)}{s}\right)\right\}.
\label{eq:RelacionGen}
\end{equation}

\noindent A partir de la descomposición en fracciones parciales, se obtiene:
\begin{equation}
	\frac{G(s)}{s} =
	\underset{\delta}{\frac{K}{ab}} \cdot \frac{1}{s}
	+ \underset{\beta}{\frac{K}{\,b^2-ab\,}} \cdot \frac{1}{s+b}
	+ \underset{\alpha}{\frac{K}{\,a^2-ab\,}} \cdot \frac{1}{s+a},
	\label{eq:Gs}
\end{equation}
donde los coeficientes $\alpha$, $\beta$ y $\delta$ corresponden a los términos de la expansión.

\textbf{Transformada-$Z$:}  
Aplicando la transformada-$Z$ a cada término se obtiene:
\[
\mathcal{Z}\left[\frac{G(s)}{s}\right] =
\delta \frac{z}{z-1} +
\beta \frac{z}{z-e^{-bT}} +
\alpha \frac{z}{z-e^{-aT}}.
\]

\noindent Finalmente, la planta discretizada puede escribirse como:
\begin{equation}
	G(z) = \frac{\gamma z^2 - \theta z + \psi}{(z-e^{-aT})(z-e^{-bT})},
	\label{eq:DiscZ}
\end{equation}
donde los parámetros $\gamma$, $\theta$ y $\psi$ se expresan en función de los coeficientes de la fracción parcial \eqref{eq:Gs}:
\begin{flalign*}
	\gamma &= \alpha + \beta + \delta & \\
	\theta &= \alpha + \beta + \delta + \alpha e^{-bT} + \beta e^{-aT} + \delta \big(e^{-aT}+e^{-bT}\big) & \\
	\psi   &= \alpha e^{-bT} + \beta e^{-aT} + \delta e^{-aT} e^{-bT}\big. &
\end{flalign*}

Se tiene la siguiente ecuación, el tiempo de muestreo utilizado para este laboratorio será el mismo que el del laboratorio 2, $T = 1.25ms$, $a = \tfrac{1}{81\,000 \cdot 200 \cdot 10^{-9}}$ y $b = \tfrac{1}{15 \cdot 10^{3} \cdot 100 \cdot 10^{-9}}$, y considerando $K = 1$ se tienen ya calculados los valores de $\alpha$, $\beta$ y $\delta$:
\begin{flalign*}
	\delta &= \frac{K}{ab} \approx 24.3 \times 10^{-6} & \\
	\beta &= \frac{K}{b^2-ab} \approx 2.48 \times 10^{-6} & \\
	\alpha &= \frac{K}{a^2-ab} \approx -26.78 \times 10^{-6}.&
\end{flalign*}

con los siguientes valores se tiene G(z):
\begin{equation}
	G(z) = \frac{0.22(z+0.74)}{(z-0.933)(z-0.44)}
	\label{eq:DiscZfin}
\end{equation}
Puesto a que los métodos convencionales de la \emph{respuesta en frecuencia} no se aplican en el plano $\mathcal{Z}$, volvemos al plano $\omega$ usando la siguiente ecuación:
\begin{equation}
	Z = \frac{1+(\frac{T}{2})\omega}{1-(\frac{T}{2})\omega}
	\label{eq:planoZ}
\end{equation}
Sustituyendo la z de la ecuación \ref{eq:DiscZfin} con la ecuación \ref{eq:planoZ}:
\begin{equation}
	G(w) = \frac{-2.41 \times 10^{-9} (w +10.675 \times 10^{3})(w - 1.6 \times 10^{3})}{1.089 \times 10^{-6} (w + 55.372)(w + 622.22)}
	\label{eq:planoW}
\end{equation}

Con esto obtenemos la función de lazo abierto en el plano $\mathcal{\omega}$.  
A continuación, trazamos las curvas de Bode. Para ello, es necesario recordar que al hacer tender $\omega \rightarrow jv$, donde $v$ representa la \emph{frecuencia ficticia} y $\omega$ la \emph{frecuencia real}, estando ambas relacionadas por:

\begin{equation}
	w \big|_{w = jv} = jv = \frac{2(z-1)}{T(z+1)}\bigg|_{z = e^{j\omega T}}
	\label{Relavw}
\end{equation}

De esta forma se obtiene la relación:

\begin{equation}
	v = \frac{2}{T}\,\tan\!\left(\frac{\omega T}{2}\right)
\end{equation}

\subsubsection{Comparar los resultados obtenidos con la simulación en Matlab}

La planta discretizada obtenida en \textsc{Matlab} es:
\[
H(z) = \frac{21.596 \times 10^{-3}\,(z + 0.7419)}{(z - 0.9333)(z - 0.436)} 
\]

Como puede observarse, la expresión de la ecuación~\ref{eq:DiscZfin}:
\[
G(z) = \frac{22 \times 10^{-3}\,(z + 0.74)}{(z - 0.933)(z - 0.44)} 
\]
presenta una buena aproximación a los valores calculados.

Por otro lado, la planta en el plano $\mathcal{\omega}$ obtenida a partir de \textsc{Matlab} es:
\[
H(w) = \frac{-2.0075 \times 10^{-3}\,(s + 10.84 \times 10^{3})(s - 1606)}{(s + 630.8)(s + 55.41)} 
\]
mientras que la expresión calculada resulta (\ref{eq:planoW}):
\[
G(w) = \frac{-2.21 \times 10^{-3}\,(w + 10.675 \times 10^{3})(w - 1.6 \times 10^{3})}{(w + 55.372)(w + 622.22)} 
\]
la cual también presenta valores aproximados.
