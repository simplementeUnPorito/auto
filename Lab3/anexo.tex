\onecolumn
\appendices
\section{Códigos de Matlab}

\begin{lstlisting}[style=matlabstyle,caption={Hoja cálculos utilizada para el diseño de los compensadores.},label={matlab:hojaDeCalculos}]
close all
clear all
close all
clear all

addpath('..\Lab1\')
addpath('..\Lab2\')

%% Definicion de parametros
R_1 = 15e3;
R_3 = 15e3;
C_2 = 100e-9;

R_2 = 82e3;
R_4 = 82e3;
C_1 = 0.22e-6;

%% Generar funcion de transferencia d
numStage = [-R_3/R_1 -R_4/R_2];
denStage = { [C_2*R_3 1], [C_1*R_4 1] };

% Usamos celdas para guardar los tf de cada stage
Gstage = cell(1,2);
G = 1;
for i = 1:2
Gstage{i} = tf(numStage(i), denStage{i});
G = G*Gstage{i};
end

%% Analizamos en el tiempo
[tr, ts, wn] = plot_step_info(G);
disp('Planta continua G(s):')
G
zpk(G)



%% Paso 1: Definir periodo de muestreo
% Se busca una mejora de 4 en el tiempo de rising , o sea tr = 10 ms
N = 4; 
T = tr/(8*N);
%Gcl = feedback(G,1);
%figure;
%step(Gcl)
%T = stepinfo(Gcl).SettlingTime/8;

%% Paso 2: Digitalizar con ZOH
Gd = c2d(G, T, 'zoh');
disp('Planta digital G(z):')
Gd
zpk(Gd)

%% === K para PM deseada ===
Gw = d2c(Gd,'tustin');        % continuo equivalente por Tustin
figure;
margin(Gw);grid on;
%% === LOOP INTERACTIVO: ajustar K, agregar LEAD/LAG, o auto-ajustar specs ===
% Requiere tener Gw (continuo por Tustin) ya definido.
s = tf('s');
C_total = tf(1);
K = 1;

fprintf('\n=== Estado inicial (K=1, C=1) ===\n');
L_now = K*C_total*Gw;
[GM, PM, Wcg, Wcp] = margin(L_now);
fprintf('GM = %.2f dB, PM = %.2f°, Wcg = %.3g rad/s, Wcp = %.3g rad/s\n', 20*log10(GM), PM, Wcg, Wcp);
figure; margin(L_now); grid on; title('L(s) = K*C*G_w  (inicio)');

seguir = true;
while seguir
fprintf('\nOpciones:\n');
fprintf('  [1] Ajustar K a un GM deseado (dB)\n');
fprintf('  [2] Agregar etapa LEAD/LAG (normalizada en Wcp)\n');
fprintf('  [3] Auto-ajustar a (GM_dB, PM)  — o usar K fijo y PM\n');
fprintf('  [4] Fijar K explícito (sin cálculos)\n');
fprintf('  [0] Terminar\n');
op = str2double(input('Elegí opción: ','s'));

switch op
case 1
GM_des_dB = str2double(input(' GM deseado (dB): ','s'));
L_now = K*C_total*Gw;
[GM_now, ~, ~, ~] = margin(L_now);
GM_now_dB = 20*log10(GM_now);
if ~isfinite(GM_now_dB)
error('GM no definido; agregá una etapa o cambiá el rango antes de ajustar K.');
end
% K_new saca GM a GM_des_dB (aprox directo en dB)
K = K * 10^((GM_now_dB - GM_des_dB)/20);
L_now = K*C_total*Gw;
[GM, PM, Wcg, Wcp] = margin(L_now);
fprintf('  -> K = %.6g; GM = %.2f dB, PM = %.2f°, Wcg = %.3g, Wcp = %.3g\n', ...
K, 20*log10(GM), PM, Wcg, Wcp);
figure; margin(L_now); grid on; title('Tras ajustar K');

case 2
tipo = lower(strtrim(input(' Tipo ("lead" o "lag"): ','s')));
phi  = str2double(input(' Fase a compensar (grados, positivo): ','s'));
[C_total, rep] = add_stage_at_Wcp(Gw, C_total, K, tipo, phi, s);
fprintf('  -> Etapa %s %+g° aplicada en wc=%.3g rad/s (normalizada)\n', rep.type, rep.phi_deg, rep.wc_used);
L_now = K*C_total*Gw;
[GM, PM, Wcg, Wcp] = margin(L_now);
fprintf('  -> GM = %.2f dB, PM = %.2f°, Wcg = %.3g, Wcp = %.3g\n', 20*log10(GM), PM, Wcg, Wcp);
figure; margin(L_now); grid on; title('Tras agregar etapa');

case 3
usa_k_fijo = lower(strtrim(input(' ¿Usar K fijo? [y/n]: ','s')));
if strcmpi(usa_k_fijo,'y')
K_fixed = str2double(input(' Valor de K fijo: ','s'));
PM_des  = str2double(input(' PM deseada (grados): ','s'));
[K, C_total, rep] = auto_hit_specs_fixedK(Gw, C_total, K_fixed, PM_des, s);
fprintf('  -> AUTO (K fijo=%.6g) listo: PM=%.2f°, GM=%.2f dB, Wcg=%.3g, Wcp=%.3g\n', ...
K, rep.PM, rep.GMdB, rep.Wcg, rep.Wcp);
else
GM_des_dB = str2double(input(' GM deseado (dB): ','s'));
PM_des    = str2double(input(' PM deseada (grados): ','s'));
[K, C_total, rep] = auto_hit_specs(Gw, C_total, K, GM_des_dB, PM_des, s);
fprintf('  -> AUTO listo (iters=%d): GM=%.2f dB, PM=%.2f°, Wcg=%.3g, Wcp=%.3g\n', ...
rep.iters, rep.GMdB, rep.PM, rep.Wcg, rep.Wcp);
end
L_now = K*C_total*Gw;
figure; margin(L_now); grid on; title('Tras auto-ajuste');

case 4
K = str2double(input(' K = ','s'));
L_now = K*C_total*Gw;
[GM, PM, Wcg, Wcp] = margin(L_now);
fprintf('  -> K fijado. GM = %.2f dB, PM = %.2f°, Wcg = %.3g, Wcp = %.3g\n', 20*log10(GM), PM, Wcg, Wcp);
figure; margin(L_now); grid on; title('Tras fijar K');

case 0
seguir = false;
continue;

otherwise
fprintf('Opción inválida.\n');
end
end


%% Resultado final
K = 7;
C_total = K;
% C_total = 11.7057*d2c(C1,'tustin',T);
L_final = C_total*Gw;
[GMf, PMf, Wcgf, Wcpf] = margin(L_final);
fprintf('Márgenes: GM = %.2f dB, PM = %.2f°, Wcg = %.3g, Wcp = %.3g\n', 20*log10(GMf), PMf, Wcgf, Wcpf);
% % figure; bode(C_total); grid on;
% figure; margin(L_final); grid on; title('L_{final}(s) = K · C_{total}(s) · G_w(s)');
% % C1 = c2d(C_total,T,'tustin');
%  figure;step(feedback(C1*Gd,1));grid on;title('Respuesta al escalon');
% figure;step(feedback(C1,Gd));grid on;title('Esfuerzo al escalon');



umin = 0;
umax =  4.08;
refmin = 1.75;
refmax = 2.25;
n_per_seg = 500;

Gc1f = feedback(Gd*C1,1);
%figure;
%step(Gc1f)
info = stepinfo(Gc1f);


info_ext = plot_step_annot(Gc1f, 'la planta compensada con un lag-filter de 1º orden, anulando el polo más rápido de la planta.');
compensador = zpk(C1);

% [td, refd, yd, ud, ed, coefs] = sim_compensador_first_order( ...
%     Gd, compensador.Z{1}, compensador.P{1}, compensador.K, T, umin, umax, refmin, refmax, n_per_seg);

[td, refd, yd, ud, ed, coefs] = sim_compensador_first_order( ...
Gd, 0, 0, K, T, umin, umax, refmin, refmax, n_per_seg);

% [td, refd, yd, ud, ed, coefs] = sim_compensador_second_order( ...
%     Gd, compensador.Z{1}(1), ...
%     compensador.Z{1}(2), ...
%     compensador.P{1}(1), ...
%     compensador.P{1}(2), compensador.K, ...
%     T, umin, umax, refmin, refmax, n_per_seg);

Nini = 100;              % descartar al inicio
Nfin = 100;              % descartar al final
idx0 = Nini + 1;         % índice inicial válido
idx1 = length(td) - Nfin; % índice final válido

td   = td(idx0:idx1)-td(idx0);
refd = refd(idx0:idx1);
yd   = yd(idx0:idx1);
ud   = ud(idx0:idx1);
ed   = ed(idx0:idx1);

% Si querés que el tiempo arranque en 0





figure;

% ---------- Subplot 1 ----------
axAbs = subplot(2,1,1);  % eje izquierdo (absoluto)
hold(axAbs,'on'); grid(axAbs,'on');

% y[k] en rojo (abs)
hY = stairs(axAbs, td, yd, 'r-', 'LineWidth', 1.4);

% Armamos eje derecho transparente
axPct = axes('Position', get(axAbs,'Position'), ...
'Color','none', 'YAxisLocation','right', ...
'XLim', get(axAbs,'XLim'), 'XTick',[], 'Box','off');
hold(axPct,'on');

% r[k] en negro, graficado en %
ref_pct = 100*(refd - refmin)/(refmax - refmin);
hR = stairs(axPct, td, refd, 'k--', 'LineWidth', 1.2);

% ======= Cálculo de límites con margen =======
yd_pct  = 100*(yd - refmin)/(refmax - refmin);
pctAll  = [yd_pct; ref_pct];

% Valores extremos en % con margen del 5 %
rawMin = min(pctAll);
rawMax = max(pctAll);
span   = rawMax - rawMin;
pctMin = rawMin - 0.05*span;
pctMax = rawMax + 0.05*span;

% Redondeamos a múltiplos de 20 para ticks
stepPct = 20;
tickMin = stepPct*floor(pctMin/stepPct);
tickMax = stepPct*ceil (pctMax/stepPct);
pctTicks = tickMin:stepPct:tickMax;

% Convertimos a absolutos
valTicks = refmin + (pctTicks/100)*(refmax - refmin);

% Aplicamos a ambos ejes
set(axAbs,'YLim',[valTicks(1) valTicks(end)],'YTick',valTicks);
set(axPct,'YLim',[valTicks(1) valTicks(end)],'YTick',valTicks,...
'YTickLabel',compose('%.0f %%',pctTicks));

% Etiquetas
xlabel(axAbs,'Tiempo [s]');
ylabel(axAbs,'Respuesta [valor absoluto]');
ylabel(axPct,'Escala relativa a ref [%]');
title(axAbs,'Respuesta discreta con compensador 1º orden, anulando el polo más rápido de la planta.');
legend(axAbs, [hY, hR], {'y[k]', 'r[k]'}, 'Location', 'best');

% ---------- Subplot 2: esfuerzo de control ----------
axU = subplot(2,1,2);
stairs(axU, td, ud, 'LineWidth',1.2); grid(axU,'on');
yline(axU, umax,'r:'); yline(axU, umin,'r:');
xlabel(axU,'Tiempo [s]'); ylabel(axU,'u[k]');
title(axU,'Esfuerzo de control con saturación');

% --- Margen de 5% en eje Y ---
uMin = min(ud);
uMax = max(ud);
span = uMax - uMin;
ylim(axU, [uMin - 0.05*span, uMax + 0.05*span]);


grid(axAbs,'on');    % grilla principal
grid(axAbs,'minor'); % grilla secundaria

grid(axPct,'on');
grid(axPct,'minor');

grid(axU,'on');
grid(axU,'minor');
%% ================== SETUP INICIAL ==================
% Asumimos que ya construiste G (continuo), calculaste T, y obtuviste:
%   Gd = c2d(G, T, 'zoh');
%   Gw = d2c(Gd, 'tustin');   % "plano w" para diseñar
%
% Si no, descomenta y ajustá a tu caso:
% R_1 = 15e3; R_3 = 15e3; C_2 = 100e-9; R_2 = 82e3; R_4 = 82e3; C_1 = 0.22e-6;
% numStage = [-R_3/R_1 -R_4/R_2];
% denStage = { [C_2*R_3 1], [C_1*R_4 1] };
% Gstage = cell(1,2); G = 1;
% for i = 1:2, Gstage{i} = tf(numStage(i), denStage{i}); G = G*Gstage{i}; end
% [tr, ts, wn] = plot_step_info(G); N = 4; T = tr/(8*N);
% Gd = c2d(G, T, 'zoh'); Gw = d2c(Gd, 'tustin');

assert(exist('Gw','var')==1 && exist('Gd','var')==1 && exist('T','var')==1, ...
'Definí primero Gw, Gd y T antes de correr este script.');

s = tf('s');

%% ================== COMPENSADOR INICIAL ==================
% Tu elección: integrador por defecto (podés cambiar a tf(1) si no querés I)
C_total = 1/s;     % <-- SI NO QUERÉS integrador, poné: C_total = tf(1);
K = 1;

%% ================== ESTADO INICIAL (CONTINUO, PLANO w) ==================
fprintf('\n=== Estado inicial (K=1, C inicial) ===\n');
L_now = K*C_total*Gw;
[GM, PM, Wcg, Wcp] = margin(L_now);
fprintf('GM = %.2f dB, PM = %.2f°, Wcg = %.3g rad/s, Wcp = %.3g rad/s\n', 20*log10(GM), PM, Wcg, Wcp);
figure; margin(L_now); grid on; title('L(s) = K*C(s)*G_w(s)  (inicio)');

%% ================== LOOP INTERACTIVO ==================
seguir = true;
while seguir
fprintf('\nOpciones:\n');
fprintf('  [1] Ajustar K a un GM deseado (dB)\n');
fprintf('  [2] Agregar etapa LEAD/LAG (normalizada en Wcp)\n');
fprintf('  [3] Auto-ajustar a (GM_dB, PM)  — o usar K fijo y PM\n');
fprintf('  [4] Fijar K explícito (sin cálculos)\n');
fprintf('  [5] Ir a DISCRETO (c2d con prewarp) y simular\n');
fprintf('  [0] Terminar\n');
op = str2double(input('Elegí opción: ','s'));

switch op
case 1  % Ajustar K a GM deseado
GM_des_dB = str2double(input(' GM deseado (dB): ','s'));
L_now = K*C_total*Gw;
[GM_now, ~, ~, ~] = margin(L_now);
GM_now_dB = 20*log10(GM_now);
if ~isfinite(GM_now_dB)
error('GM no definido; agregá una etapa o cambiá el rango antes de ajustar K.');
end
K = K * 10^((GM_now_dB - GM_des_dB)/20);
L_now = K*C_total*Gw;
[GM, PM, Wcg, Wcp] = margin(L_now);
fprintf('  -> K = %.6g; GM = %.2f dB, PM = %.2f°, Wcg = %.3g, Wcp = %.3g\n', ...
K, 20*log10(GM), PM, Wcg, Wcp);
figure; margin(L_now); grid on; title('Tras ajustar K');

case 2  % Agregar etapa lead/lag normalizada en Wcp
tipo = lower(strtrim(input(' Tipo ("lead" o "lag"): ','s')));
phi  = str2double(input(' Fase a compensar (grados, positivo): ','s'));
[C_total, rep] = add_stage_at_Wcp(Gw, C_total, K, tipo, phi, s);
fprintf('  -> Etapa %s %+g° aplicada en wc=%.3g rad/s (normalizada)\n', rep.type, rep.phi_deg, rep.wc_used);
L_now = K*C_total*Gw;
[GM, PM, Wcg, Wcp] = margin(L_now);
fprintf('  -> GM = %.2f dB, PM = %.2f°, Wcg = %.3g, Wcp = %.3g\n', 20*log10(GM), PM, Wcg, Wcp);
figure; margin(L_now); grid on; title('Tras agregar etapa');

case 3  % Auto-ajuste
usa_k_fijo = lower(strtrim(input(' ¿Usar K fijo? [y/n]: ','s')));
if strcmpi(usa_k_fijo,'y')
K_fixed = str2double(input(' Valor de K fijo: ','s'));
PM_des  = str2double(input(' PM deseada (grados): ','s'));
[K, C_total, rep] = auto_hit_specs_fixedK(Gw, C_total, K_fixed, PM_des, s);
fprintf('  -> AUTO (K fijo=%.6g) listo: PM=%.2f°, GM=%.2f dB, Wcg=%.3g, Wcp=%.3g\n', ...
K, rep.PM, rep.GMdB, rep.Wcg, rep.Wcp);
else
GM_des_dB = str2double(input(' GM deseado (dB): ','s'));
PM_des    = str2double(input(' PM deseada (grados): ','s'));
[K, C_total, rep] = auto_hit_specs(Gw, C_total, K, GM_des_dB, PM_des, s);
fprintf('  -> AUTO listo (iters=%d): GM=%.2f dB, PM=%.2f°, Wcg=%.3g, Wcp=%.3g\n', ...
rep.iters, rep.GMdB, rep.PM, rep.Wcg, rep.Wcp);
end
L_now = K*C_total*Gw;
figure; margin(L_now); grid on; title('Tras auto-ajuste');

case 4  % Fijar K
K = str2double(input(' K = ','s'));
L_now = K*C_total*Gw;
[GM, PM, Wcg, Wcp] = margin(L_now);
fprintf('  -> K fijado. GM = %.2f dB, PM = %.2f°, Wcg = %.3g, Wcp = %.3g\n', 20*log10(GM), PM, Wcg, Wcp);
figure; margin(L_now); grid on; title('Tras fijar K');

case 5  % Discretizar C(s) -> C(z) con prewarp y simular
Lc = K*C_total*Gw;
[~, ~, ~, Wcp_c] = margin(Lc);
if ~(isfinite(Wcp_c) && Wcp_c>0)
warning('No hay Wcp continuo definido; uso prewarp=0 (Tustin simple).');
opt = c2dOptions('tustin');
else
opt = c2dOptions('tustin','PrewarpFrequency', Wcp_c);
end
C2 = c2d(C_total, T, opt);

% Renormalizar en la frecuencia digital equivalente
if isfinite(Wcp_c) && Wcp_c>0
Omega_c = 2*atan(Wcp_c*T/2);                 % Ωc
Cz_ejO  = evalfr(C2, exp(1j*Omega_c));
gfix    = 1/abs(Cz_ejO);
C2      = gfix * C2;
end

% Lazo discreto y márgenes
Lz = K * C2 * Gd;
figure; margin(Lz); grid on; title('Discreto: Lz = K*C(z)*Gd(z)');
[GMz, PMz, Wcgz, Wcpz] = margin(Lz);
fprintf('DISCRETO: GM=%.2f dB, PM=%.2f°, Wcg=%.3g rad/s, Wcp=%.3g rad/s\n', 20*log10(GMz), PMz, Wcgz, Wcpz);

% Respuesta al escalón (y[k])
Ts = Gd.Ts; if Ts<=0, Ts = T; end
N  = 2000; tvec = (0:N-1)'*Ts;
CLz_y = feedback(Lz, 1);
[y, tstep] = step(CLz_y, tvec);
figure; stairs(tstep, y); grid on; title('Discreto: y[k] ante escalón'); xlabel('t [s]');

% Esfuerzo u[k] = C(z)/(1+C(z)Gd(z)) * r[k]  (para r=step)
T_u_r = feedback(C2, Gd*C2);   % C2 / (1 + C2*Gd)
[u, tstep2] = step(T_u_r, tvec);
figure; stairs(tstep2, u); grid on; title('Discreto: esfuerzo u[k] ante escalón'); xlabel('t [s]');

case 0
seguir = false;
continue;

otherwise
fprintf('Opción inválida.\n');
end
end

%% ================== RESUMEN FINAL (CONTINUO Y DISCRETO, TUSTIN SIMPLE) ==================
% --- Continuo (diseño sobre Gw) ---
L_final = K*C_total*Gw;
[GMf, PMf, Wcgf, Wcpf] = margin(L_final);

fprintf('\n=== RESULTADO FINAL (CONTINUO, DISEÑO) ===\n');
fprintf('K = %.6g\n', K);
disp('C_total(s) ='); disp(zpk(C_total));
fprintf('Márgenes: GM = %.2f dB, PM = %.2f°, Wcg = %.3g, Wcp = %.3g\n', 20*log10(GMf), PMf, Wcgf, Wcpf);

figure; bode(C_total); grid on; title('Compensador total C_{total}(s)');
figure; margin(L_final); grid on; title('L_{final}(s) = K · C_{total}(s) · G_w(s)');

% --- Discretizar el compensador (Tustin “normal”, sin prewarp) ---
C1  = K*c2d(C_total, T, 'tustin');

% --- Lazo discreto y márgenes ---
Lz = C1*Gd;
figure; margin(Lz); grid on; title('Final discreto: Lz = K·C(z)·G_d(z)');
[GMz, PMz, Wcgz, Wcpz] = margin(Lz);
fprintf('=== DISCRETO FINAL (Tustin simple) ===\nGM=%.2f dB, PM=%.2f°, Wcg=%.3g, Wcp=%.3g\n', ...
20*log10(GMz), PMz, Wcgz, Wcpz);


% Salida y[k] ante escalón (CL: (K*C2*Gd)/(1+K*C2*Gd))
CLz_y = feedback(Lz, 1);
[y, tstep] = step(CLz_y);
figure; stairs(tstep, y); grid on; xlabel('t [s]'); ylabel('y[k]');
title(' y[k] ante escalón');



% Esfuerzo u[k] ante escalón: T_{u<-r}(z) = (K*C2)/(1+K*C2*Gd)
T_u_r = feedback(C1, Gd);   % equivalente a (K*C2) / (1 + K*C2*Gd)
[u, tstep2] = step(T_u_r);
figure; stairs(tstep2, u); grid on; xlabel('t [s]'); ylabel('u[k]');
title('Final discreto: esfuerzo u[k] ante escalón');


%% ================== COMPENSADOR INICIAL ==================
% Tu elección: integrador por defecto (podés cambiar a tf(1) si no querés I)
C_total = 1/s;     % <-- SI NO QUERÉS integrador, poné: C_total = tf(1);
K = 1;

fprintf('\n=== Estado inicial (K=1, C inicial) ===\n');
L_now = K*C_total*Gw;
[GM, PM, Wcg, Wcp] = margin(L_now);
fprintf('GM = %.2f dB, PM = %.2f°, Wcg = %.3g rad/s, Wcp = %.3g rad/s\n', 20*log10(GM), PM, Wcg, Wcp);
figure; margin(L_now); grid on; title('L(s) = K*C(s)*G_w(s)  (inicio)');

seguir = true;
while seguir
fprintf('\nOpciones:\n');
fprintf('  [1] Ajustar K a un GM deseado (dB)\n');
fprintf('  [2] Agregar etapa LEAD/LAG (normalizada en Wcp)\n');
fprintf('  [3] Auto-ajustar a (GM_dB, PM)  — o usar K fijo y PM\n');
fprintf('  [4] Fijar K explícito (sin cálculos)\n');
fprintf('  [5] Ir a DISCRETO (c2d con prewarp) y simular\n');
fprintf('  [0] Terminar\n');
op = str2double(input('Elegí opción: ','s'));

switch op
case 1  % Ajustar K a GM deseado
GM_des_dB = str2double(input(' GM deseado (dB): ','s'));
L_now = K*C_total*Gw;
[GM_now, ~, ~, ~] = margin(L_now);
GM_now_dB = 20*log10(GM_now);
if ~isfinite(GM_now_dB)
error('GM no definido; agregá una etapa o cambiá el rango antes de ajustar K.');
end
K = K * 10^((GM_now_dB - GM_des_dB)/20);
L_now = K*C_total*Gw;
[GM, PM, Wcg, Wcp] = margin(L_now);
fprintf('  -> K = %.6g; GM = %.2f dB, PM = %.2f°, Wcg = %.3g, Wcp = %.3g\n', ...
K, 20*log10(GM), PM, Wcg, Wcp);
figure; margin(L_now); grid on; title('Tras ajustar K');

case 2  % Agregar etapa lead/lag normalizada en Wcp
tipo = lower(strtrim(input(' Tipo ("lead" o "lag"): ','s')));
phi  = str2double(input(' Fase a compensar (grados, positivo): ','s'));
[C_total, rep] = add_stage_at_Wcp(Gw, C_total, K, tipo, phi, s);
fprintf('  -> Etapa %s %+g° aplicada en wc=%.3g rad/s (normalizada)\n', rep.type, rep.phi_deg, rep.wc_used);
L_now = K*C_total*Gw;
[GM, PM, Wcg, Wcp] = margin(L_now);
fprintf('  -> GM = %.2f dB, PM = %.2f°, Wcg = %.3g, Wcp = %.3g\n', 20*log10(GM), PM, Wcg, Wcp);
figure; margin(L_now); grid on; title('Tras agregar etapa');

case 3  % Auto-ajuste
usa_k_fijo = lower(strtrim(input(' ¿Usar K fijo? [y/n]: ','s')));
if strcmpi(usa_k_fijo,'y')
K_fixed = str2double(input(' Valor de K fijo: ','s'));
PM_des  = str2double(input(' PM deseada (grados): ','s'));
[K, C_total, rep] = auto_hit_specs_fixedK(Gw, C_total, K_fixed, PM_des, s);
fprintf('  -> AUTO (K fijo=%.6g) listo: PM=%.2f°, GM=%.2f dB, Wcg=%.3g, Wcp=%.3g\n', ...
K, rep.PM, rep.GMdB, rep.Wcg, rep.Wcp);
else
GM_des_dB = str2double(input(' GM deseado (dB): ','s'));
PM_des    = str2double(input(' PM deseada (grados): ','s'));
[K, C_total, rep] = auto_hit_specs(Gw, C_total, K, GM_des_dB, PM_des, s);
fprintf('  -> AUTO listo (iters=%d): GM=%.2f dB, PM=%.2f°, Wcg=%.3g, Wcp=%.3g\n', ...
rep.iters, rep.GMdB, rep.PM, rep.Wcg, rep.Wcp);
end
L_now = K*C_total*Gw;
figure; margin(L_now); grid on; title('Tras auto-ajuste');

case 4  % Fijar K
K = str2double(input(' K = ','s'));
L_now = K*C_total*Gw;
[GM, PM, Wcg, Wcp] = margin(L_now);
fprintf('  -> K fijado. GM = %.2f dB, PM = %.2f°, Wcg = %.3g, Wcp = %.3g\n', 20*log10(GM), PM, Wcg, Wcp);
figure; margin(L_now); grid on; title('Tras fijar K');

case 5  % Discretizar C(s) -> C(z) con prewarp y simular
Lc = K*C_total*Gw;
[~, ~, ~, Wcp_c] = margin(Lc);
if ~(isfinite(Wcp_c) && Wcp_c>0)
warning('No hay Wcp continuo definido; uso prewarp=0 (Tustin simple).');
opt = c2dOptions('tustin');
else
opt = c2dOptions('tustin','PrewarpFrequency', Wcp_c);
end
C2 = c2d(C_total, T, opt);

% Renormalizar en la frecuencia digital equivalente
if isfinite(Wcp_c) && Wcp_c>0
Omega_c = 2*atan(Wcp_c*T/2);                 % Ωc
Cz_ejO  = evalfr(C2, exp(1j*Omega_c));
gfix    = 1/abs(Cz_ejO);
C2      = gfix * C2;
end

% Lazo discreto y márgenes
Lz = K * C2 * Gd;
figure; margin(Lz); grid on; title('Discreto: Lz = K*C(z)*Gd(z)');
[GMz, PMz, Wcgz, Wcpz] = margin(Lz);
fprintf('DISCRETO: GM=%.2f dB, PM=%.2f°, Wcg=%.3g rad/s, Wcp=%.3g rad/s\n', 20*log10(GMz), PMz, Wcgz, Wcpz);

% Respuesta al escalón (y[k])
Ts = Gd.Ts; if Ts<=0, Ts = T; end
N  = 2000; tvec = (0:N-1)'*Ts;
CLz_y = feedback(Lz, 1);
[y, tstep] = step(CLz_y, tvec);
figure; stairs(tstep, y); grid on; title('Discreto: y[k] ante escalón'); xlabel('t [s]');

% Esfuerzo u[k] = C(z)/(1+C(z)Gd(z)) * r[k]  (para r=step)
T_u_r = feedback(C2, Gd*C2);   % C2 / (1 + C2*Gd)
[u, tstep2] = step(T_u_r, tvec);
figure; stairs(tstep2, u); grid on; title('Discreto: esfuerzo u[k] ante escalón'); xlabel('t [s]');

case 0
seguir = false;
continue;

otherwise
fprintf('Opción inválida.\n');
end
end


%% Resultado final
C_total = K*C_total;
L_final = C_total*Gw;
[GMf, PMf, Wcgf, Wcpf] = margin(L_final);
fprintf('Márgenes: GM = %.2f dB, PM = %.2f°, Wcg = %.3g, Wcp = %.3g\n', 20*log10(GMf), PMf, Wcgf, Wcpf);
figure; bode(C_total); grid on;
figure; margin(L_final); grid on; title('L_{final}(s) = K · C_{total}(s) · G_w(s)');
C2 = c2d(C_total,T,'tustin');
figure;step(feedback(C2*Gd,1));grid on;title('Respuesta al escalon');
figure;step(feedback(C2,Gd));grid on;title('Esfuerzo al escalon');


Lz = Gd*C2;
% Salida y[k] ante la rampa (CL: (K*C2*Gd)/(1+K*C2*Gd))
CLz_y = feedback(Lz, 1);
N = 100*T;            % duración total
t = 0:T:N;            % vector de tiempo
rampa = t;            % rampa pendiente 1

[y, tstep] = lsim(CLz_y, rampa, t);

figure;
stairs(tstep, y, 'LineWidth',1.5); hold on;
plot(tstep, rampa, 'r--','LineWidth',1.2);

xlabel('t [s]');
ylabel('y[k]');
title('Respuesta y[k] ante rampa');
grid on; grid minor;

% Forzar relación 1:1 en ejes
axis equal




umin = 0;
umax =  4.08;
refmin = 1.75;
refmax = 2.25;
n_per_seg = 500;

Gc1f = feedback(Gd*C2,1);
%figure;
%step(Gc1f)
info = stepinfo(Gc1f);


info_ext = plot_step_annot(Gc1f, 'la planta compensada con un lag-filter de 1º orden, anulando el polo más rápido de la planta.');
compensador = zpk(C2);


[td, refd, yd, ud, ed, coefs] = sim_compensador_second_order( ...
Gd, compensador.Z{1}(1), ...
compensador.Z{1}(2), ...
compensador.P{1}(1), ...
compensador.P{1}(2), compensador.K, ...
T, umin, umax, refmin, refmax, n_per_seg);

figure;
[y,td]=lsim(Gc1f,td);
stairs(td,y);
hold on;
grid on;
plot(td,td);


Nini = 100;              % descartar al inicio
Nfin = 100;              % descartar al final
idx0 = Nini + 1;         % índice inicial válido
idx1 = length(td) - Nfin; % índice final válido

td   = td(idx0:idx1)-td(idx0);
refd = refd(idx0:idx1);
yd   = yd(idx0:idx1);
ud   = ud(idx0:idx1);
ed   = ed(idx0:idx1);

% Si querés que el tiempo arranque en 0





figure;

% ---------- Subplot 1 ----------
axAbs = subplot(2,1,1);  % eje izquierdo (absoluto)
hold(axAbs,'on'); grid(axAbs,'on');

% y[k] en rojo (abs)
hY = stairs(axAbs, td, yd, 'r-', 'LineWidth', 1.4);

% Armamos eje derecho transparente
axPct = axes('Position', get(axAbs,'Position'), ...
'Color','none', 'YAxisLocation','right', ...
'XLim', get(axAbs,'XLim'), 'XTick',[], 'Box','off');
hold(axPct,'on');

% r[k] en negro, graficado en %
ref_pct = 100*(refd - refmin)/(refmax - refmin);
hR = stairs(axPct, td, refd, 'k--', 'LineWidth', 1.2);

% ======= Cálculo de límites con margen =======
yd_pct  = 100*(yd - refmin)/(refmax - refmin);
pctAll  = [yd_pct; ref_pct];

% Valores extremos en % con margen del 5 %
rawMin = min(pctAll);
rawMax = max(pctAll);
span   = rawMax - rawMin;
pctMin = rawMin - 0.05*span;
pctMax = rawMax + 0.05*span;

% Redondeamos a múltiplos de 20 para ticks
stepPct = 20;
tickMin = stepPct*floor(pctMin/stepPct);
tickMax = stepPct*ceil (pctMax/stepPct);
pctTicks = tickMin:stepPct:tickMax;

% Convertimos a absolutos
valTicks = refmin + (pctTicks/100)*(refmax - refmin);

% Aplicamos a ambos ejes
set(axAbs,'YLim',[valTicks(1) valTicks(end)],'YTick',valTicks);
set(axPct,'YLim',[valTicks(1) valTicks(end)],'YTick',valTicks,...
'YTickLabel',compose('%.0f %%',pctTicks));

% Etiquetas
xlabel(axAbs,'Tiempo [s]');
ylabel(axAbs,'Respuesta [valor absoluto]');
ylabel(axPct,'Escala relativa a ref [%]');
title(axAbs,'Respuesta discreta con compensador 1º orden, anulando el polo más rápido de la planta.');
legend(axAbs, [hY, hR], {'y[k]', 'r[k]'}, 'Location', 'best');

% ---------- Subplot 2: esfuerzo de control ----------
axU = subplot(2,1,2);
stairs(axU, td, ud, 'LineWidth',1.2); grid(axU,'on');
yline(axU, umax,'r:'); yline(axU, umin,'r:');
xlabel(axU,'Tiempo [s]'); ylabel(axU,'u[k]');
title(axU,'Esfuerzo de control con saturación');

% --- Margen de 5% en eje Y ---
uMin = min(ud);
uMax = max(ud);
span = uMax - uMin;
ylim(axU, [uMin - 0.05*span, uMax + 0.05*span]);


grid(axAbs,'on');    % grilla principal
grid(axAbs,'minor'); % grilla secundaria

grid(axPct,'on');
grid(axPct,'minor');

grid(axU,'on');
grid(axU,'minor');
\end{lstlisting}

\begin{lstlisting}[style=matlabstyle,caption={Funciones creadas para este laboratorio.},label={matlab:newfunc}]
function C = lag_norm_at_wc(phi_deg, wc, s)
phi_deg = max(1, min(60, phi_deg));
sphi = sind(phi_deg);
beta = (1 + sphi) / (1 - sphi);       % >1
T = 1 / (wc * sqrt(beta));
C0 = (1 + T*s) / (1 + (T/beta)*s);
g = 1 / abs(freqresp(C0, 1j*wc));     % |C(jwc)|=1
C = g * C0;
end


function C = lag_gain_at_wc_db(drop_dB, wc, s, kfactor)
% Lag que REDUCE el módulo ~drop_dB en torno a wc sin tocar K.
% NO normalizado (justamente para cambiar GM). Ubica esquinas a wc/kfactor.
% Aproxima |C(jω)| ~ 1/β para ω >> 1/T (efecto lag clásico).
if nargin<4 || isempty(kfactor), kfactor = 10; end
beta = max(1.05, 10^(drop_dB/20));     % objetivo de caída de |C|≈1/beta (dB ~ -drop_dB)
% Colocar esquinas bien por debajo: ωz=1/T, ωp=1/(βT); con β>1 => polo más bajo.
% Elegimos ωz = wc/kfactor => T = kfactor/wc
T = kfactor / wc;
C = (1 + T*s) / (1 + beta*T*s);        % lag correcto (módulo ~1/β a altas frecuencias)
end
function [K_out, C_out, report] = auto_hit_specs_fixedK(Gw, C_in, K_fixed, PM_des, s)
% No toca K. Ajusta PM usando:
%  (1) Lead normalizado en el Wcp actual para sumar fase hasta PM_des
%  (2) Si por el lead se movió GM y querés recuperar un poquito de fase,
%      mete un lead suavito extra (<=20°). (Lag normalizado opcional si querés bajar fase)
tolPM = 1.0;    % deg
maxit = 4;

K = K_fixed;
C = C_in;

for it = 1:maxit
L = K*C*Gw;
[~, PM, ~, Wcp] = margin(L);
if ~(isfinite(Wcp) && Wcp > 0)
error('No hay Wcp con K fijo; probá agregar una etapa o cambiar K.');
end

phi_need = PM_des - PM;
if phi_need > 0.5
phi_apply = min(max(phi_need, 1), 60);
C = lead_norm_at_wc(phi_apply, Wcp, s) * C;
elseif phi_need < -0.5
% Si te pasaste de PM, podés usar lag NORMALIZADO suave para restar fase
phi_lag = min(max(abs(phi_need), 1), 30);
C = lag_norm_at_wc(phi_lag, Wcp, s) * C;
else
break;  % dentro de tolerancia
end

% ¿ya estamos dentro de tolerancia?
L2 = K*C*Gw;
[~, PM2] = margin(L2);
if isfinite(PM2) && abs(PM2 - PM_des) <= tolPM
break;
end
end

K_out = K;
C_out = C;

Lf = K_out*C_out*Gw;
[GMf, PMf, Wcgf, Wcpf] = margin(Lf);
report = struct('GMdB',20*log10(GMf), 'PM',PMf, 'Wcg',Wcgf, 'Wcp',Wcpf, 'iters',it);
end

function [K_out, C_out, report] = auto_hit_specs(Gw, C_in, K_in, GM_des_dB, PM_des, s)
% AUTO que prioriza: (1) ajustar K ~ GM, (2) sumar LEAD (normalizado) para PM,
% (3) si GM < objetivo, sumar LAG "de ganancia" (NO normalizado) para subir GM,
% (4) si el LAG te bajó PM, agrega un LEAD chiquito para recuperar.
%
% Devuelve K_out, C_out y reporte (GM, PM, Wcg, Wcp, iters).

tolGM = 0.5;    % dB
tolPM = 1.0;    % deg
maxit = 4;

K = K_in;
C = C_in;

for it = 1:maxit
% --- (A) Ajustar K para GM deseado (aprox directo en dB)
L = K*C*Gw;
[GM, PM, ~, ~] = margin(L); GMdB = 20*log10(GM);
if ~isfinite(GMdB), error('GM no definido; añadí una etapa antes.'); end
K = K * 10^((GMdB - GM_des_dB)/20);

% --- (B) LEAD normalizado para alcanzar PM deseada en Wcp actual
L = K*C*Gw;
[~, PM, ~, Wcp] = margin(L);
if ~(isfinite(Wcp) && Wcp>0), error('No hay Wcp tras ajustar K.'); end

phi_need = PM_des - PM;            % fase a sumar
if phi_need > 0.5
phi_apply = min(max(phi_need, 1), 60);   % limitar 1..60°
C = lead_norm_at_wc(phi_apply, Wcp, s) * C;
end

% --- Recalculamos márgenes
L = K*C*Gw;
[GM, PM, ~, Wcp] = margin(L); GMdB = 20*log10(GM);

% --- (C) Si GM quedó por debajo, metemos un LAG de ganancia (NO normalizado)
if isfinite(GMdB) && (GMdB < GM_des_dB - tolGM)
dB_needed = GM_des_dB - GMdB;                 % cuánto GM falta
C = lag_gain_at_wc_db(dB_needed, Wcp, s, 10) * C;  % kfactor=10 => esquinas bien por debajo

% recuperar PM si el lag bajó demasiado
L2 = K*C*Gw;
[~, PM2, ~, Wcp2] = margin(L2);
if isfinite(PM2) && PM2 < PM_des - tolPM
phi_fix = min(max(PM_des - PM2, 1), 30);         % LEAD suave de corrección
C = lead_norm_at_wc(phi_fix, Wcp2, s) * C;
end
end

% --- (D) Chequeo de convergencia
Lf = K*C*Gw;
[GMf, PMf, ~, ~] = margin(Lf); GMf_dB = 20*log10(GMf);
doneGM = isfinite(GMf_dB) && abs(GMf_dB - GM_des_dB) <= tolGM;
donePM = isfinite(PMf)    && abs(PMf    - PM_des)    <= tolPM;
if doneGM && donePM, break; end
end

K_out = K; C_out = C;
Lf = K_out*C_out*Gw; [GMf, PMf, Wcgf, Wcpf] = margin(Lf);
report = struct('GMdB',20*log10(GMf), 'PM',PMf, 'Wcg',Wcgf, 'Wcp',Wcpf, 'iters',it);
end

function [C_out, report] = add_stage_at_Wcp(Gw, C_in, K, comp_type, phi_deg, s)
% Agrega UNA etapa (lead o lag) normalizada en el Wcp del lazo actual L = K*C*G.
% No modifica K. Devuelve C_out y un pequeño reporte con los nuevos márgenes.

% 1) Wcp del lazo actual
L_now = K * C_in * Gw;
[GM_now, PM_now, Wcg_now, Wcp_now] = margin(L_now);
if ~(isfinite(Wcp_now) && Wcp_now > 0)
error('No hay cruce de ganancia (Wcp) con el lazo actual. Ajustá K o añadí otra etapa antes.');
end
wc = Wcp_now;

% 2) Construir la etapa normalizada en wc
comp_type = lower(strtrim(comp_type));
if strcmpi(comp_type,'lead')
C_stage = lead_norm_at_wc(phi_deg, wc, s);     % |C(j wc)| = 1
elseif strcmpi(comp_type,'lag')
C_stage = lag_norm_at_wc(phi_deg, wc, s);      % |C(j wc)| = 1
else
error('comp_type debe ser "lead" o "lag".');
end

% 3) Encadenar etapa (K NO cambia)
C_out = C_stage * C_in;

% 4) Reporte (con K fijo)
L_new = K * C_out * Gw;
[GMf, PMf, Wcgf, Wcpf] = margin(L_new);
report = struct('GMdB', 20*log10(GMf), 'PM', PMf, 'Wcg', Wcgf, 'Wcp', Wcpf, ...
'wc_used', wc, 'type', comp_type, 'phi_deg', phi_deg);
end

function C = lead_norm_at_wc(phi_deg, wc, s)
phi_deg = max(1, min(60, phi_deg));
sphi = sind(phi_deg);
alpha = (1 - sphi) / (1 + sphi);      % 0<alpha<1
T = 1 / (wc * sqrt(alpha));
C0 = (1 + T*s) / (1 + alpha*T*s);
g = 1 / abs(freqresp(C0, 1j*wc));     % |C(jwc)|=1
C = g * C0;
end

\end{lstlisting}

\begin{lstlisting}[style=matlabstyle,caption={Funciones reutilizadas de laboratorios pasados.},label={matlab:oldfunc}]
function info_ext = plot_step_annot(sys, nameStr, Ts_override)
% STEP discreto con STAIRS; anota 10-90% (en s y en T), OS, zeta y omega_n.
% Grafica en milisegundos y marca el OS en el pico.

if nargin < 2 || isempty(nameStr), nameStr = inputname(1); end
if isempty(nameStr), nameStr = 'sys'; end

% ----- Respuesta y métricas básicas -----
[y, t] = step(sys); y = y(:); t = t(:);
t_ms = t * 1000;   % graficamos en milisegundos

% Estimar yss por promedio en cola
Ntail = max(10, round(0.05*length(y)));
yss = mean(y(end-Ntail+1:end));

% Límites 10-90
if abs(yss) < 1e-12
t10 = NaN; t90 = NaN; tr_10_90 = NaN; y10 = 0; y90 = 0;
else
y10 = 0.10*yss; y90 = 0.90*yss;
sgn = sign(yss);
idx10 = find(sgn*y >= sgn*y10, 1, 'first');
idx90 = find(sgn*y >= sgn*y90, 1, 'first');
t10 = tern(~isempty(idx10), t(idx10), NaN);
t90 = tern(~isempty(idx90), t(idx90), NaN);
tr_10_90 = tern(~isnan(t10)&&~isnan(t90)&&t90>=t10, t90 - t10, NaN);
end
t10_ms = t10 * 1000; t90_ms = t90 * 1000;

% Pico y %OS (para zeta y omega_n)
[ypeak_raw, idxpk] = max(sign(yss).*y);
ypeak = sign(yss)*ypeak_raw;
tpeak = t(idxpk); tpeak_ms = tpeak * 1000;
OS_percent = tern(yss ~= 0, max(0, (ypeak - yss)/abs(yss)*100), NaN);

% zeta desde %OS:  OS% = 100*exp(-zeta*pi/sqrt(1-zeta^2))
if isnan(OS_percent) || OS_percent <= 0
zeta_est = NaN;
else
logterm = log(OS_percent/100);
zeta_est = -logterm / sqrt(pi^2 + logterm^2);
end

% ----- Ts y rise en T -----
Ts = NaN;
try
if isprop(sys,'Ts') && ~isempty(sys.Ts) && sys.Ts > 0, Ts = sys.Ts; end
catch, end
if nargin >= 3 && ~isempty(Ts_override) && Ts_override > 0, Ts = Ts_override; end
tr_10_90_T = tern(~isnan(Ts) && ~isnan(tr_10_90), tr_10_90/Ts, NaN);

% ----- Estimacion de omega_n -----
omega_n = NaN;
if ~isnan(zeta_est) && zeta_est < 1 && OS_percent > 0 && tpeak > 0
omega_n = pi / ( tpeak * sqrt(1 - zeta_est^2) );
end
if isnan(omega_n)
S = stepinfo(y, t);   % Ts(2%) ~ 4/(zeta*omega_n)
if isfield(S,'SettlingTime') && ~isempty(S.SettlingTime) && S.SettlingTime > 0 && ~isnan(zeta_est) && zeta_est > 0
omega_n = 4 / ( zeta_est * S.SettlingTime );
end
end
if isnan(omega_n) && ~isnan(tr_10_90) && tr_10_90 > 0
k_rise = 1.4;  % aprox para 10-90% en 2do orden subamortiguado
omega_n = k_rise / tr_10_90;
end

% ----- Grafico -----
figure;
stairs(t_ms, y, 'LineWidth', 1.5); grid on;grid minor; hold on;
xlabel('Tiempo [ms]'); ylabel('Salida');
title(sprintf('Respuesta al escalon de %s', nameStr));

% Líneas guía 10% y 90% y marcadores (en ms)
if ~isnan(tr_10_90)
yline(y10, '--', '10% yss', 'LabelHorizontalAlignment','left');
yline(y90, '--', '90% yss', 'LabelHorizontalAlignment','left');
xline(t10_ms, ':', 't10%');
xline(t90_ms, ':', 't90%');
plot([t10_ms t90_ms], [y10 y90], 'o', 'MarkerSize', 6, 'LineWidth', 1.2);
end
% Línea de yss
yline(yss, '-', sprintf('yss=%.3g', yss), 'LabelHorizontalAlignment','left');

% ----- Marca del OS en el pico -----
if ~isnan(OS_percent) && OS_percent > 0
% línea vertical desde yss a ypeak en tpeak
plot([tpeak_ms tpeak_ms], [yss ypeak], '-', 'LineWidth', 1.2);
% texto al costado derecho de la línea
dx = 0.02 * (t_ms(end) - t_ms(1));
text(tpeak_ms + dx, yss + 0.5*(ypeak - yss), ...
sprintf('OS=%.3g%%', OS_percent), ...
'Interpreter','none', 'Margin',2);
% marcar el pico
plot(tpeak_ms, ypeak, 's', 'MarkerSize', 6, 'LineWidth', 1.2);
end

% ----- Cuadro de metricas (abajo-derecha; chico) -----
w = 0.15; h = 0.12; rmargin = 0.1; x = 1 - rmargin - w; y0 = 0.30;
txtLines = {
	sprintf('t_r(10%%-90%%) = %.4g ms  (~%.4g Ts)', safeNum(tr_10_90*1000), safeNum(tr_10_90_T))
	sprintf('OS = %.4g%%',          safeNum(OS_percent))
	sprintf('zeta = %.4g',          safeNum(zeta_est))
	sprintf('omega_n = %.4g rad/s', safeNum(omega_n))
};
annotation('textbox', [x y0 w h], ...
'String', txtLines, 'Interpreter','none', ...
'FontName','Consolas', 'FontSize',9, ...
'EdgeColor',[0.5 0.5 0.5], 'FaceAlpha',0.88, 'BackgroundColor','w');

hold off;

% ----- Salida -----
info_ext = struct( ...
'yss', yss, ...
't10', t10, ...
't90', t90, ...
'tr_10_90', tr_10_90, ...
'tr_10_90_T', tr_10_90_T, ...
'OS_percent', OS_percent, ...
'zeta_est', zeta_est, ...
'omega_n', omega_n, ...
'Ts', Ts);
end

% --------- helpers ---------
function y = tern(cond, a, b)
if cond, y = a; else, y = b; end
end

function v = safeNum(x)
if isnan(x) || isinf(x), v = NaN; else, v = x; end
end

function [td, refd, yd, ud, ed, coefs] = sim_compensador_second_order( ...
Gd, z0_0, z0_1, p0_0, p0_1, Kc, T, umin, umax, refmin, refmax, n_per_seg)

% ===== Planta: y(k+1)=b0*u(k)+b1*u(k-1)+b2*u(k-2)-a1*y(k)-a2*y(k-1) =====
% Compensador (posición, 2º orden):
% u(k) = (p0_0+p0_1)*u(k-1) - (p0_0*p0_1)*u(k-2) ...
%      + Kc*[ e(k) - (z0_0+z0_1)*e(k-1) + (z0_0*z0_1)*e(k-2) ]
% e(k) = refd(k) - yd(k)

if nargin < 12 || isempty(n_per_seg), n_per_seg = 600; end
if ~isa(Gd,'tf'), error('Gd debe ser tf discreto.'); end

% --- Coefs planta (normalizados a a0=1) ---
[numD, denD] = tfdata(Gd, 'v');
numD = numD(:).'; denD = denD(:).';
if abs(denD(1) - 1) > 1e-12
numD = numD/denD(1);
denD = denD/denD(1);
end

% Asegurar longitud mínima del denominador [1 a1 a2]
if numel(denD) < 3, denD(end+1:3) = 0; end
a1 = denD(2);
a2 = denD(3);

% b0,b1,b2 sin romper si faltan términos en el numerador
b0 = 0; b1 = 0; b2 = 0;
if numel(numD) >= 2, b0 = numD(2); end
if numel(numD) >= 3, b1 = numD(3); end
if numel(numD) >= 4, b2 = numD(4); end
% Nota: esto respeta tu estilo previo (b0=numD(2), b1=numD(3), ...)

coefs = struct('b0',b0,'b1',b1,'b2',b2,'a1',a1,'a2',a2, ...
'z0_0',z0_0,'z0_1',z0_1,'p0_0',p0_0,'p0_1',p0_1,'Kc',Kc);

% --- Tiempo y referencia (refmin -> refmax -> refmin) ---
N  = 3*n_per_seg;
td = (0:N-1)' * T;
refd = [refmin*ones(n_per_seg,1);
refmax*ones(n_per_seg,1);
refmin*ones(n_per_seg,1)];

% --- Inicialización ---
yd = zeros(N,1);
ed = zeros(N,1);
ud = zeros(N,1);

% --- Precalculos del compensador ---
P1 = (p0_0 + p0_1);
P2 = (p0_0 * p0_1);
Z1 = (z0_0 + z0_1);
Z2 = (z0_0 * z0_1);

% --- Loop ---
for k = 3:N-1
ed(k) = refd(k) - yd(k);

% Compensador 2º orden
u_k = P1*ud(k-1) - P2*ud(k-2) + Kc*( ed(k) - Z1*ed(k-1) + Z2*ed(k-2) );

% Saturación
if u_k > umax
ud(k) = umax;
elseif u_k < umin
ud(k) = umin;
else
ud(k) = u_k;
end

% Planta discreta
yd(k+1) = b0*ud(k) + b1*ud(k-1) + b2*ud(k-2) - a1*yd(k) - a2*yd(k-1);
end
end

function [td, refd, yd, ud, ed, coefs] = sim_compensador_first_order( ...
Gd, z0, p0, Kc, T, umin, umax, refmin, refmax, n_per_seg)

% ===== Igual que tu PID: y(k+1)=b0*u(k)+b1*u(k-1)-a1*y(k)-a2*y(k-1) =====
% Controlador (posición): u(k) = p0*u(k-1) + Kc*ed(k) - Kc*z0*ed(k-1)
% ed(k) = refd(k) - yd(k)

if nargin < 10 || isempty(n_per_seg), n_per_seg = 600; end
if ~isa(Gd,'tf'), error('Gd debe ser tf discreto.'); end

% --- Coefs planta exactamente como en tu ejemplo ---
[numD, denD] = tfdata(Gd, 'v');
b0 = numD(2);
b1 = numD(3);
a1 = denD(2);
a2 = denD(3);
coefs = struct('b0',b0,'b1',b1,'a1',a1,'a2',a2);

% --- Tiempo y referencia (refmin -> refmax -> refmin) ---
N  = 3*n_per_seg;
td = (0:N-1)' * T;
refd = [refmin*ones(n_per_seg,1);
refmax*ones(n_per_seg,1);
refmin*ones(n_per_seg,1)];

% --- Inicialización idéntica a tu estilo ---
yd = zeros(N,1);
ed = zeros(N,1);
ud = zeros(N,1);

% --- Loop (misma estructura y orden que tu PID) ---
for k = 3:N-1
ed(k) = refd(k) - yd(k);

% Compensador: u(k) = p0*u(k-1) + Kc*ed(k) - Kc*z0*ed(k-1)
u_k = p0*ud(k-1) + Kc*ed(k) - Kc*z0*ed(k-1);


if u_k>umax
ud(k) = umax;
elseif u_k<umin
ud(k) = umin;
else 
ud(k) = u_k;
end


% Planta discreta (dividido por a0 implícito como en tu ejemplo)
yd(k+1) = b0*ud(k) + b1*ud(k-1) - a1*yd(k) - a2*yd(k-1);
end
end

function [tr, ts, wn] = plot_step_info(G)
info = stepinfo(G);
tr = info.RiseTime; 
ts = info.SettlingTime; 
wn = 1.8 / tr;

t_end = 1.1 * ts;
if ~isfinite(t_end) || t_end <= 0, t_end = 5 * max(tr, 1e-3); end
t = linspace(0, t_end, 2*pi/wn);

[y, tout] = step(G, t);
figure; plot(tout*1000, y, 'LineWidth', 1.4); grid on;grid minor; hold on;
xlabel('Tiempo [ms]'); ylabel('Salida');
title(sprintf('Escalón: tr=%.4gms, ts=%.4gms, \\omega_n=%.4g rad/s', tr*1000, ts*1000, wn));
xline(tr*1000, '--', sprintf('  t_r=%.3g ms', tr*1000), 'LabelOrientation','horizontal');
xline(ts*1000,  '--', sprintf('  t_s=%.3g ms', ts*1000), 'LabelOrientation','horizontal');
legend('G(t)', 'Location', 'best');
end
\end{lstlisting}

\begin{lstlisting}[style=matlabstyle,caption={Script para la comparación de la planta en lazo abierto nominal vs mediciones.},label={matlab:openLoop}]
close all; clear; clc;

%% ====================== Parámetros nominales ======================
R_1 = 15e3;  R_2 = 82e3;  R_3 = 15e3;  R_4 = 82e3;
C_1 = 0.22e-6;  C_2 = 100e-9;

% Tolerancias de catálogo (especificación)
tolR = 0.05;   % ±5% resistencias
tolC = 0.20;   % ±20% capacitores

% Para Monte Carlo (3σ ≈ tolerancia) -> σ = tol/3
sigR = tolR/3;
sigC = tolC/3;

% Correlación entre resistores "pareja" del mismo valor (más realista)
rhoR = 0;  % entre R1-R3 (15k/15k) y R2-R4 (82k/82k)

% Nominales (alias)
R1n = R_1; R2n = R_2; R3n = R_3; R4n = R_4;
C1n = C_1; C2n = C_2;

%% ====================== Helper de planta continua ======================
% G(s) = [(R3/R1)*(R4/R2)] / [(R3*C2 s + 1)(R4*C1 s + 1)]
makeG = @(R1,R2,R3,R4,C1,C2) tf( (R3/R1)*(R4/R2), conv([R3*C2 1],[R4*C1 1]) );

%% ====================== Cargar datos medidos ======================
data = readmatrix('./imagenes/OpenLoop/mediciones.csv');

% u = ENTRADA medida; y = SALIDA medida
u = data(:, 11);   % ajustá si cambia columna
y = data(:, 17);

Ts = 0.000099999997474;  % tiempo de muestreo (≈100 µs)

% Vector de tiempo alineado a la medición
Ndata = min(length(y), length(u));
u = u(1:Ndata); y = y(1:Ndata);
u = u(:); y = y(:);
tdata = (0:Ndata-1).' * Ts;

%% ====================== Modelo estimado (opcional, referencia) ======================
try
data_id = iddata(y, u, Ts);
sys0 = tfest(data_id, 2,0);              % 2 polos por defecto
if sys0.Ts==0, sys0d = c2d(sys0, Ts, 'zoh'); else, sys0d = d2d(sys0, Ts); end
catch
sys0d = [];  % si no está el toolbox, seguimos sin el estimado
end

%% ====================== Nominal (discreto) ======================
Gc_nom = makeG(R1n,R2n,R3n,R4n,C1n,C2n);
Gd_nom = c2d(Gc_nom, Ts, 'zoh');

%% ====================== MONTE CARLO realista ======================
Nmc = 1000;           % # de muestras (subí si querés más suavidad)
rng(123);             % reproducible

% Ruido normal correlacionado para (R1,R3) y (R2,R4)
Zc13 = randn(Nmc,1);  % común par (R1,R3)
Z1   = randn(Nmc,1);  % propio R1
Z3   = randn(Nmc,1);  % propio R3

Zc24 = randn(Nmc,1);  % común par (R2,R4)
Z2   = randn(Nmc,1);  % propio R2
Z4   = randn(Nmc,1);  % propio R4

% Construir deltas relativos con correlación rhoR
a = rhoR; b = sqrt(1 - rhoR^2);
dR1 = sigR*(a*Zc13 + b*Z1);
dR3 = sigR*(a*Zc13 + b*Z3);
dR2 = sigR*(a*Zc24 + b*Z2);
dR4 = sigR*(a*Zc24 + b*Z4);

% Capacitores: independientes (distinto valor/naturaleza)
dC1 = sigC*randn(Nmc,1);
dC2 = sigC*randn(Nmc,1);

% Clip a las tolerancias físicas (no exceder ±tol)
clip = @(x,tol) min(max(x, -tol), tol);
dR1 = clip(dR1, tolR); dR3 = clip(dR3, tolR);
dR2 = clip(dR2, tolR); dR4 = clip(dR4, tolR);
dC1 = clip(dC1, tolC); dC2 = clip(dC2, tolC);

% Vectores absolutos
R1s = R1n*(1 + dR1);  R3s = R3n*(1 + dR3);
R2s = R2n*(1 + dR2);  R4s = R4n*(1 + dR4);
C1s = C1n*(1 + dC1);  C2s = C2n*(1 + dC2);

% Simular todas las muestras con la MISMA entrada u
Ymc = zeros(Ndata, Nmc);
for k = 1:Nmc
Gc  = makeG(R1s(k),R2s(k),R3s(k),R4s(k),C1s(k),C2s(k));
Gdz = c2d(Gc, Ts, 'zoh');
yk  = lsim(Gdz, u, tdata);
if isrow(yk), yk = yk.'; end
Ymc(:,k) = yk;
end

% Envolvente percentil (más realista) y también min/max por si querés ver extremos
yP05 = prctile(Ymc, 5,  2);
yP95 = prctile(Ymc, 95, 2);
yMin = min(Ymc, [], 2);
yMax = max(Ymc, [], 2);

%% ====================== (Opcional) WORST-CASE 64 corners ======================
plot_worstcase = false;   % ponelo true si querés superponerlo
if plot_worstcase
R1_lo = (1 - tolR)*R1n;   R1_hi = (1 + tolR)*R1n;
R2_lo = (1 - tolR)*R2n;   R2_hi = (1 + tolR)*R2n;
R3_lo = (1 - tolR)*R3n;   R3_hi = (1 + tolR)*R3n;
R4_lo = (1 - tolR)*R4n;   R4_hi = (1 + tolR)*R4n;
C1_lo = (1 - tolC)*C1n;   C1_hi = (1 + tolC)*C1n;
C2_lo = (1 - tolC)*C2n;   C2_hi = (1 + tolC)*C2n;

pairs = [ R1_lo R1_hi; R2_lo R2_hi; R3_lo R3_hi; R4_lo R4_hi; C1_lo C1_hi; C2_lo C2_hi ];
comb = dec2bin(0:(2^6-1)) - '0';   % 64x6

Ywc = zeros(Ndata, size(comb,1));
for k = 1:size(comb,1)
sel = comb(k,:)+1;
R1 = pairs(1,sel(1)); R2 = pairs(2,sel(2));
R3 = pairs(3,sel(3)); R4 = pairs(4,sel(4));
C1 = pairs(5,sel(5)); C2 = pairs(6,sel(6));
Gc  = makeG(R1,R2,R3,R4,C1,C2);
Gdz = c2d(Gc, Ts, 'zoh');
Ywc(:,k) = lsim(Gdz, u, tdata);
end
yWmin = min(Ywc, [], 2);
yWmax = max(Ywc, [], 2);
else
yWmin = []; yWmax = [];
end

%% ====================== Simulación nominal y estimado con misma u ======================
y_nom = lsim(Gd_nom, u, tdata);
if ~isempty(sys0d), y_est = lsim(sys0d, u, tdata); else, y_est = []; end

%% ====================== Plot comparativo ======================
figure; hold on; box on;

% Banda percentil (más realista)
fill([tdata; flipud(tdata)], [yP05; flipud(yP95)], [0.80 0.92 1.00], ...
'EdgeColor','none','FaceAlpha',0.55);

% (opcional) extremos worst-case en líneas finas
if ~isempty(yWmin)
plot(tdata, yWmin, 'Color',[0.2 0.4 0.8], 'LineStyle','--', 'LineWidth',0.8);
plot(tdata, yWmax, 'Color',[0.2 0.4 0.8], 'LineStyle','--', 'LineWidth',0.8);
end

% Medición y modelos
stairs(tdata, y, 'k', 'LineWidth', 1.3);          % medición
plot(tdata, y_nom, 'LineWidth', 1.2);             % nominal
if ~isempty(y_est), plot(tdata, y_est, 'LineWidth', 1.2); end  % estimado

xlabel('t [s]'); ylabel('y');
leg = {'Banda MC (p5–p95)'};
if ~isempty(yWmin), leg{end+1} = 'Worst-case min/max'; end
leg = [leg, {'Medición','Nominal'}];
if ~isempty(y_est), leg{end+1} = 'Estimado (tfest)'; end
legend(leg, 'Location','best');
title('Comparación temporal con la MISMA entrada u[k] — banda de tolerancia realista');
grid on; grid minor;

%% ====================== (Opcional) corner MC más parecido por RMSE ======================
RMSE = sqrt(mean( (Ymc - y).^2, 1 ));
[rmseBest, kBest] = min(RMSE);
fprintf('MC con menor RMSE: #%d  RMSE = %.4g\n', kBest, rmseBest);
best = struct('R1',R1s(kBest),'R2',R2s(kBest),'R3',R3s(kBest),'R4',R4s(kBest),'C1',C1s(kBest),'C2',C2s(kBest));
disp(best);
\end{lstlisting}


\begin{lstlisting}[style=matlabstyle,caption={Script utilizado para la comparación de las simulaciones con las mediciones de los compensadores.},label={matlab:calculo1}]
close all
clear all

addpath('..\Lab1\')

%% Definicion de parametros
R_1 = 15e3;
R_3 = 15e3;
C_2 = 100e-9;

R_2 = 82e3;
R_4 = 82e3;
C_1 = 0.22e-6;

%% Generar funcion de transferencia d
numStage = [-R_3/R_1 -R_4/R_2];
denStage = { [C_2*R_3 1], [C_1*R_4 1] };

% Usamos celdas para guardar los tf de cada stage
Gstage = cell(1,2);
G = 1;
for i = 1:2
Gstage{i} = tf(numStage(i), denStage{i});
G = G*Gstage{i};
end

%% Analizamos en el tiempo
[tr, ts, wn] = plot_step_info(G);

disp('Planta continua G(s):');
%% Paso 1: Definir periodo de muestreo
% Se busca una mejora de 2 en el tiempo de rising =, o sea tr = 10 ms
N = 4; 
T = tr/(8*N);
%Gcl = feedback(G,1);
%figure;
%step(Gcl)
%T = stepinfo(Gcl).SettlingTime/8;

%% Paso 2: Digitalizar con ZOH
Gd = c2d(G, T, 'zoh');
disp('Planta digital G(z):')
Gd





%% C1_K

data = readmatrix('.\C1_K\tablas.csv');  % lee todo el CSV en una matriz

t_meas   = data(:,4);   % columna 1 = tiempo
ref_meas = data(:,5);   % columna 2 = referencia
u_meas   = data(:,11);   % columna 3 = esfuerzo
y_meas   = data(:,17);   % columna 4 = salida

% Suponiendo que tenés vectores del osciloscopio:
% t_meas, ref_meas, u_meas, y_meas  (todos Nx1, en segundos y unidades reales)
opts = struct('Nini', 1, 'Nfin', 1, 'plot', true,'demean_errors',false);

S = sim_compensador_first_order_meas( ...
Gd, 0, 0, 7, T, 0, 4.08, ...
t_meas, ref_meas, u_meas, y_meas, ...
opts);

% Accesos:
S.errors.u    % métricas esfuerzo sim vs real
S.errors.y    % métricas salida  sim vs real

opts = struct('Nini', 1, 'Nfin', 1, 'plot', true);

S = sim_compensador_first_order_meas( ...
Gd, 0, 0, 7, T, 0, 4.08, ...
t_meas, ref_meas, u_meas, y_meas, ...
opts);

% Accesos:
S.errors.u    % métricas esfuerzo sim vs real
S.errors.y    % métricas salida  sim vs real


%% C1_Lead

data = readmatrix('.\C1_lead\tablas.csv');  % lee todo el CSV en una matriz

t_meas   = data(:,4);   % columna 1 = tiempo
ref_meas = data(:,5);   % columna 2 = referencia
u_meas   = data(:,11);   % columna 3 = esfuerzo
y_meas   = data(:,17);   % columna 4 = salida

% Suponiendo que tenés vectores del osciloscopio:
% t_meas, ref_meas, u_meas, y_meas  (todos Nx1, en segundos y unidades reales)
opts = struct('Nini', 1, 'Nfin', 1, 'plot', true);

S = sim_compensador_first_order_meas( ...
Gd, 0.622387144541200, 0.374311732736374,  1.184734400491065, T, 0, 4.08, ...
t_meas, ref_meas, u_meas, y_meas, ...
opts);

% Accesos:
S.errors.u    % métricas esfuerzo sim vs real
S.errors.y    % métricas salida  sim vs real

opts = struct('Nini', 1, 'Nfin', 1, 'plot', true,'demean_errors',true);

S = sim_compensador_first_order_meas( ...
Gd, 0.622387144541200, 0.374311732736374,  1.184734400491065, T, 0, 4.08, ...
t_meas, ref_meas, u_meas, y_meas, ...
opts);

% Accesos:
S.errors.u    % métricas esfuerzo sim vs real
S.errors.y    % métricas salida  sim vs real

%% C2

data = readmatrix('.\C2\tablas.csv');  % lee todo el CSV en una matriz

t_meas   = data(:,4);   % columna 1 = tiempo
ref_meas = data(:,5);   % columna 2 = referencia
u_meas   = data(:,11);   % columna 3 = esfuerzo
y_meas   = data(:,17);   % columna 4 = salida

% Suponiendo que tenés vectores del osciloscopio:
% t_meas, ref_meas, u_meas, y_meas  (todos Nx1, en segundos y unidades reales)
opts = struct('Nini', 1, 'Nfin', 1, 'plot', true,'demean_errors',false);

S = sim_compensador_second_order_meas( ...
Gd,-1,0.948680356607820,...
1,0.805044751405714,...
0.185620357009816,...
T, 0, 4.08, ...
t_meas, ref_meas, u_meas, y_meas, ...
opts);


opts = struct('Nini', 1, 'Nfin', 1, 'plot', true,'demean_errors',true);

S = sim_compensador_second_order_meas( ...
Gd,-1,0.948680356607820,...
1,0.805044751405714,...
0.185620357009816,...
T, 0, 4.08, ...
t_meas, ref_meas, u_meas, y_meas, ...
opts);

% Accesos:
S.errors.u    % métricas esfuerzo sim vs real
S.errors.y    % métricas salida  sim vs real
\end{lstlisting}

\section{Código desarrollado para el PSoC }
\begin{lstlisting}[style=cstyle,
	caption={Código desarrollado para la implementación de los compensadores con el PSoC en lenguaje C.},
	label={c:psoc}
	]

#include "project.h"
#include <stdio.h>
#include <string.h>
#include <stdlib.h>
#include <ctype.h>

/* ==== Prototipos ==== */
CY_ISR_PROTO(adquirirMuestra);
static void pc_process_rx(void);
static void handle_command(const char *line);
static void set_ref_value(float *target, float val, const char *name);
static void set_ref_period(uint16_t period);
static void ref_status(void);

/* ==== Config general / límites ==== */
#define SAT_MIN        (0.0f)
#define SAT_MAX        (4.08f)
#define MIDDLE_VOLTAGE ((SAT_MAX - SAT_MIN)/2.0f)

#define RX_BUF_SZ 64

/* ==== Estado global ==== */
volatile char  flag    = '0';
volatile float ref_max = (float)(MIDDLE_VOLTAGE + 0.5f);
volatile float ref_min = (float)(MIDDLE_VOLTAGE - 0.5f);
volatile float ref     = 0.0f;
volatile float muestra = 0.0f;
volatile float e       = 0.0f;

/* Modo de operación: 0 = closed loop (PID), 1 = open loop (DAC = ref) */
enum { MODE_CLOSED = 0, MODE_OPEN = 1 };
volatile uint8 mode = MODE_CLOSED;

/* ==== UART RX line buffer ==== */
static char   rx_buf[RX_BUF_SZ];
static uint8  rx_len = 0;

/* ==== Helpers sin libm ==== */
static long my_lroundf(float x) { return (x >= 0.0f) ? (long)(x + 0.5f) : (long)(x - 0.5f); }

/* Mapear voltaje a 0..255 con clamps a [SAT_MIN, SAT_MAX] */
static inline uint8 volt_to_dac(float v)
{
	if (v < SAT_MIN) v = SAT_MIN;
	if (v > SAT_MAX) v = SAT_MAX;
	float norm = v / SAT_MAX;               /* 0..1 */
	if (norm < 0.0f) norm = 0.0f;
	if (norm > 1.0f) norm = 1.0f;
	return (uint8)(norm * 255.0f);
}

/* ==== ISR de muestra ==== */
CY_ISR(adquirirMuestra)
{
	/* Limpiar fuente de interrupción de “muestra disponible” */
	muestra_disponible_ClearPending();
	
	/* Conmutar referencia según pin/switch (1 -> max, 0 -> min) */
	ref = (ref_state_Read()) ? ref_max : ref_min;
	
	VDAC8_ref_SetValue(volt_to_dac(ref));
	
	/* Leer ADC y calcular error */
	muestra = VADC_CountsTo_Volts(VADC_GetResult16());
	e = ref - muestra;
	
	flag = '1';
	
	/* Según tu diseño, deshabilitás y re-habilitás en el main */
	muestra_disponible_Disable();
}

/* ==== Controlador discreto (forma recursiva estándar) ==== */
/* Parámetros (ajustá a gusto) */
#define T   (double)(0.001245213061220)
#define Ts  T
//
//
//#define U0  -0.737363460561365/1.18473440049107
//#define U1  1
//#define U2  0
//
//#define E0  -0.737363460561365
//#define E1  1.18473440049107
//#define E2  0


#define z0  -1//0//0.6224
#define z1  0.948680356607820//0
#define p0  1.00000000000000//0//0.3743
#define p1  0.805044751405714//0
#define K  0.185620357009816 //7//1.1847


#define U0  1
#define U1  -(p0+p1)
#define U2  p0*p1


#define E0  1
#define E1  -(z0+z1)
#define E2 z0*z1



/* Hook del esfuerzo: respeta el modo (open/closed) */
static inline void actualizarEsfuerzo(void)
{
	if (mode == MODE_OPEN) {
		/* === LAZO ABIERTO: el DAC sigue la referencia (ref_min / ref_max) === */
		VDAC8_SetValue(volt_to_dac(ref));
		return;
	}
	
	/* === LAZO CERRADO: controlador PID discreto === */
	static float u = 0.0f, u_1 = 0.0f, u_2 = 0.0f;
	static float e_1 = 0.0f, e_2 = 0.0f;
	
	/* Salida no saturada */
	float u_unsat = (float)((-U1*u_1 + -U2*u_2 + K*(E0*e + E1*e_1 + E2*e_2))/U0);
	
	/* Saturación correcta (sobre u_unsat) */
	float u_sat;
	if (u_unsat > SAT_MAX)      u_sat = SAT_MAX;
	else if (u_unsat < SAT_MIN) u_sat = SAT_MIN;
	else                        u_sat = u_unsat;
	
	/* DAC: escribir esfuerzo saturado */
	VDAC8_SetValue(volt_to_dac(u_sat));
	
	/* Correr estados (orden correcto) */
	u_2 = u_1; u_1 = u_sat;
	e_2 = e_1; e_1 = e;
}

/* ==== Helpers de parsing ==== */

/* Recorta espacios a la derecha (CR/LF/espacios) */
static void rstrip(char *s)
{
	size_t n = strlen(s);
	while (n && (s[n-1]=='\r' || s[n-1]=='\n' || isspace((unsigned char)s[n-1])))
	s[--n] = '\0';
}

static void to_lower_str(char *s)
{
	for (; *s; ++s) *s = (char)tolower((unsigned char)*s);
}

/* Procesa bytes RX por líneas (eco por línea) */
static void pc_process_rx(void)
{
	int ch;
	while ((ch = PC_GetChar()) != 0) {
		if (ch == '\n' || ch == '\r') {
			if (rx_len > 0) {
				rx_buf[rx_len] = '\0';
				PC_PutString(rx_buf); PC_PutString("\r\n"); /* eco de línea */
				handle_command(rx_buf);
				rx_len = 0;
			}
		} else {
			if (rx_len < (RX_BUF_SZ-1)) {
				rx_buf[rx_len++] = (char)ch;
			} else {
				rx_len = 0;
				PC_PutString("ERR: line too long\r\n");
			}
		}
	}
}

/* ==== Status command (sin floats en printf; usa mV) ==== */
static void ref_status(void)
{
	long ref_min_mV = my_lroundf(ref_min * 1000.0f);
	long ref_max_mV = my_lroundf(ref_max * 1000.0f);
	long ref_mV     = my_lroundf(ref * 1000.0f);
	unsigned per    = (unsigned)timer_ref_ReadPeriod();
	
	char msg[160];
	snprintf(msg, sizeof(msg),
	"STATUS:\r\n"
	"  mode=%s\r\n"
	"  ref_min=%ld mV\r\n"
	"  ref_max=%ld mV\r\n"
	"  ref=%ld mV\r\n"
	"  ref_period=%u\r\n",
	(mode == MODE_OPEN) ? "open" : "closed",
	ref_min_mV, ref_max_mV, ref_mV, per);
	PC_PutString(msg);
}

/* ==== Parser de comandos ==== */
static void handle_command(const char *line_in)
{
	char line[RX_BUF_SZ];
	strncpy(line, line_in, sizeof(line));
	line[sizeof(line)-1] = '\0';
	rstrip(line);
	to_lower_str(line);
	
	/* Comandos sin valor */
	if (strcmp(line, "ref_status") == 0 || strcmp(line, "status") == 0) {
		ref_status(); return;
	}
	if (strcmp(line, "help") == 0 || strcmp(line, "?") == 0) {
		PC_PutString("Commands:\r\n"
		"  ref_max:<float 0..4.08>\r\n"
		"  ref_min:<float 0..4.08>\r\n"
		"  ref_period:<uint16 0..65535>\r\n"
		"  mode:<open|closed|1|0>\r\n"
		"  ref_status\r\n");
		return;
	}
	
	/* Clave:valor */
	char *colon = strchr(line, ':');
	if (!colon) { PC_PutString("ERR: expected key:value or ref_status\r\n"); return; }
	*colon = '\0';
	const char *key = line;
	const char *val_str = colon + 1;
	while (*val_str && isspace((unsigned char)*val_str)) val_str++;
	
	if (strcmp(key, "ref_max") == 0) {
		float v = (float)atof(val_str);
		set_ref_value((float*)&ref_max, v, "ref_max");
	} else if (strcmp(key, "ref_min") == 0) {
		float v = (float)atof(val_str);
		set_ref_value((float*)&ref_min, v, "ref_min");
	} else if (strcmp(key, "ref_period") == 0) {
		long v = strtol(val_str, NULL, 0);
		if (v < 0 || v > 65535) PC_PutString("ERR: ref_period out of range (0..65535)\r\n");
		else set_ref_period((uint16_t)v);
	} else if (strcmp(key, "mode") == 0) {
		if (strcmp(val_str, "open") == 0 || strcmp(val_str, "1") == 0) {
			mode = MODE_OPEN;  PC_PutString("OK: mode=open\r\n");
		} else if (strcmp(val_str, "closed") == 0 || strcmp(val_str, "0") == 0) {
			mode = MODE_CLOSED; PC_PutString("OK: mode=closed\r\n");
		} else {
			PC_PutString("ERR: mode must be open|closed|1|0\r\n");
		}
	} else {
		PC_PutString("ERR: unknown key. Use ref_max, ref_min, ref_period, mode, ref_status\r\n");
	}
}

/* Set de ref_* con validación y sección crítica */
static void set_ref_value(float *target, float val, const char *name)
{
	if (val < SAT_MIN || val > SAT_MAX) {
		char msg[64];
		long lo = my_lroundf(SAT_MIN*1000.0f), hi = my_lroundf(SAT_MAX*1000.0f);
		snprintf(msg, sizeof(msg), "ERR: %s out of range (%ld..%ld mV)\r\n", name, lo, hi);
		PC_PutString(msg);
		return;
	}
	
	uint8 intr = CyEnterCriticalSection();
	*target = val;
	/* Si está activa esa referencia, actualizá ref inmediatamente */
	ref = (ref_state_Read()) ? ref_max : ref_min;
	CyExitCriticalSection(intr);
	
	char msg[64];
	long v_mV = my_lroundf(val*1000.0f);
	snprintf(msg, sizeof(msg), "OK: %s=%ld mV\r\n", name, v_mV);
	PC_PutString(msg);
}

/* Cambia el periodo del timer de referencia de forma segura */
static void set_ref_period(uint16_t period)
{
	timer_ref_Stop();
	timer_ref_WritePeriod(period);
	timer_ref_WriteCounter(period);
	timer_ref_Start();
	
	char msg[64];
	snprintf(msg, sizeof(msg), "OK: ref_period=%u\r\n", (unsigned)period);
	PC_PutString(msg);
}

/* ==== main ==== */
int main(void)
{
	CyGlobalIntEnable;
	
	/* HW init */
	Opa_stage1_Start();
	Opa_vdda_2_Start();
	Opa_stage2_Start();
	VADC_Start();
	VDAC8_Start();
	VDAC8_ref_Start();
	
	/* UART PC */
	PC_Start();
	PC_PutString("\r\nReady. Commands:\r\n");
	PC_PutString("  ref_max:<float 0..4.08>\r\n");
	PC_PutString("  ref_min:<float 0..4.08>\r\n");
	PC_PutString("  ref_period:<uint16 0..65535>\r\n");
	PC_PutString("  mode:<open|closed|1|0>\r\n");
	PC_PutString("  ref_status\r\n");
	
	/* Timer / ISR de muestra */
	muestra_disponible_StartEx(adquirirMuestra);
	muestra_disponible_Enable();
	timer_ref_Start();
	
	flag = '0';
	
	for (;;)
	{
		/* Procesar UART sin bloquear */
		pc_process_rx();
		
		/* Procesar muestra si disponible */
		if (flag == '1') {
			flag = '0';
			actualizarEsfuerzo();           /* respeta el modo */
			muestra_disponible_Enable();    /* re-habilitar ISR */
		}
	}
}

\end{lstlisting}