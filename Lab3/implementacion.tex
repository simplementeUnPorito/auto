\subsection{Implementación del Sistema}
La implementación se realizó en plataforma PSoC, ejecutando el controlador discreto a período fijo de muestreo \(T=\SI{1.2452}{ms}\) (promediado). Se programó un firmware que permite: (i) abrir/cerrar el lazo para ensayos en lazo abierto, (ii) imponer distintos valores mínimos y máximos para el escalón y (iii) cambiar el período del escalón desde el puerto serie. El esfuerzo de control se genera con un \texttt{VDAC8} hacia la planta (referenciada a \(\mathrm{VDDA}/2\)) y la salida de la planta se adquiere con un ADC-SAR para cerrar el lazo. La referencia es generada mediante software.

\subsubsection{Ecuación en diferencias y saturación}
Cada compensador discreto se implementa a partir de la forma:
\begin{equation}
	\label{eq:Cz_form}
	\begin{split}
		C(z)
		= \frac{K\big(E_0 + E_1 z^{-1} + E_2 z^{-2}\big)}
		{U_0 + U_1 z^{-1} + U_2 z^{-2}}.
	\end{split}
\end{equation}

Sabiendo entonces que \(\mathcal{Z}^{-1}\!\big\{C_{den}(z)\,U(z)\big\}=\mathcal{Z}^{-1}\!\big\{KC_{num}(z)\,E(z)\big\}\), se llega a la ecuación en diferencias:
\begin{equation}
	\label{eq:diff_balance}
	\begin{aligned}
		&U_0\,u[k] + U_1\,u[k-1] + U_2\,u[k-2]
		=\\
		 & K\Big(E_0\,e[k] + E_1\,e[k-1] + E_2\,e[k-2]\Big),
	\end{aligned}
\end{equation}
y despejando \(u[k]\):
\begin{equation}
	\label{eq:diff_u}
	\begin{split}
		u[k] \;=\;\frac{1}{U0}[
		-\,U_1\,u[k-1] \;-\; U_2\,u[k-2]
			\;+\; \\K\big(E_0\,e[k] + E_1\,e[k-1] + E_2\,e[k-2]\big)]
	\end{split}
\end{equation}


y luego se aplica saturación (montaje: \(\texttt{SAT\_MIN}=\SI{0.0}{V}\), \(\texttt{SAT\_MAX}=\SI{4.08}{V}\)):
\begin{equation}
	\label{eq:u_sat}
	\texttt{u\_sat}=
	\begin{cases}
		\texttt{SAT\_MAX}, & \text{si } \texttt{u\_unsat}>\texttt{SAT\_MAX},\\
		\texttt{SAT\_MIN}, & \text{si } \texttt{u\_unsat}<\texttt{SAT\_MIN},\\
		\texttt{u\_unsat}, & \text{en otro caso.}
	\end{cases}
\end{equation}
Finalmente se escribe al DAC: \(\texttt{VDAC8\_SetValue(volt\_to\_dac(u\_sat));}\)
en firmware, las ecuaciones \eqref{eq:diff_u} y \eqref{eq:u_sat} se calculan en la función \texttt{actualizarEsfuerzo()}.


% Figuras de implementación (usar tu macro)
\insertarfigura{./img/OpenLoop/LazoAbiertoSinCompensador.png}
{Comparación de un  VDAC sin carga (traza cian) vs.\ VDAC cargado (traza violeta) con la planta y respuesta de la planta (traza verde) a la excitación en lazo abierto.}
{fig:openloop_dac}{1}

\subsubsection{Modo lazo abierto y parametrización}
Para verificar la planta y \emph{debuguear} la cadena de adquisición/actuación, se incorporó un modo de \textbf{lazo abierto} que \emph{bypassea} el compensador y permite inyectar una referencia (escalón) directamente al DAC. La amplitud, los límites mínimo/máximo y el período del escalón se ajustan \emph{on-line} por comandos serie.

\subsubsection{Observaciones de hardware (carga del VDAC)}
En ensayos de lazo abierto se observó \textbf{error de carga} en el VDAC al excitar directamente la planta (Fig.~\ref{fig:openloop_dac}), por lo que se propuso colocar primero la etapa con resistencia de entrada \(\SI{82}{k\ohm}\) y luego la de \(\SI{15}{k\ohm}\) mitigó parcialmente el problema (Fig.~\ref{fig:openloop_dac}). Con la de \(\SI{15}{k\ohm}\) al inicio, la excursión quedó limitada a \(\sim\SI{250}{mV}\)–\(\SI{300}{mV}\) pico a pico, frente a \(\sim\SI{500}{mV}\) p–p del DAC en vacío. Tras reordenar, la excursión aumentó hasta \(\sim\SI{440}{mV}\) p–p (Fig.~\ref{fig:openloop_dac}).
\textbf{Solución propuesta para siguientes laboratorios:} añadir un \emph{buffer} a la salida del VDAC; desacoplar \(\mathrm{VDDA}/2\) y alimentación con capacitores de \SIrange{1}{10}{\micro\farad} (y cerámicos \SIrange{100}{470}{nF} en paralelo); utilizar el el modo \emph{bypass}/\emph{buffered} de los DAC/ADC, manteniendo el cambio hecho con el orden de las etapas de la planta para aumentar la impendancia vista desde la entrada,



\subsubsection{Desempeño medido vs.\ esperado}
El tiempo de subida medido fue \(\SI{34.2}{ms}\) frente a \(\sim\SI{40}{ms}\) estimados por simulación. La diferencia concuerda con tolerancias de componentes (\(\pm\SI{10}{\percent}\) en resistencias y hasta \(\pm\SI{20}{\percent}\) en capacitores) y con la atenuación observada en la cadena de actuación (de \(\SI{440}{mV}\) a \(\SI{432}{mV}\)). Aun con dichas limitaciones, la dinámica global (parámetros y forma de la respuesta) se mantuvo alineada con el diseño en Matlab.

\subsubsection{Compensadores implementados}
Se cargaron en firmware tres compensadores: (i) proporcional, (ii) proporcional con adelanto de primer orden y (iii) adelanto con integrador para seguimiento de rampa y ESS = 0 para el escalón . Las respuestas temporales cada uno se muestran en las Figs.~\ref{fig:c1_k}, \ref{fig:c1_lead} y \ref{fig:c2} respectivamente.
                                                                        \insertarfigura{img/C1_K/esfuerzo.png}
                                                                        {Compensador proporcional: referencia tipo escalon (traza cian), esfuerzo (traza violeta) y salida de la planta (traza verde).}
                                                                        {fig:c1_k}{1}
                                                                        
                                                                        \insertarfigura{img/C1_Lead/esfuerzo_step.png}
                                                                        {Compensador adelanto: referencia tipo escalon (traza cian), esfuerzo (traza violeta) y salida de la planta (traza verde).}
                                                                        {fig:c1_lead}{1}
                                                                        
                                                                        \insertarfigura{img/C2/esfuerzo.png}
                                                                        {Compensador de adelanto + integrador: referencia tipo escalon (traza cian), esfuerzo (traza violeta) y salida de la planta (traza verde).}
                                                                        {fig:c2}{1}                                                                                                                              ―
                                                                                                                                                                                                      ―
