\section{Resultados}

\subsection{Respuestas temporales (simulado vs.\ experimental)}
En esta sección se presentan comparaciones \emph{en el dominio del tiempo} entre simulación y experimento para: planta sin compensar, C1 (proporcional), C2 (lead), C3 (integrador+lead). En cada figura se incluyen las métricas \emph{RMSE}, \emph{NRMSE} y \(e_{\max}\) (definidas en las ecuaciones \eqref{eq:rmse_temporal}–\eqref{eq:emax_temporal}). Se muestran versiones \emph{con} y \emph{sin} remoción de offset para evidenciar el descalce de referencia señalado en la implementación.

% ===== AJUSTAR RUTAS =====
\insertarfigura{img/OpenLoop/comparacionLazoAbierto.png}
{Planta sin compensación en lazo abierto: respuesta al escalón (simulado vs.\ experimental, con bandas de tolerancia para \(\sigma = \frac{tolerance}{3}\)).}
{fig:step_openLoop}{1}



\insertarfigura{img/C1_Lead/conOffset}
{Compensador de adelanto: respuesta al escalón y esfuerzo (simulado vs.\ experimental, \emph{con} offset).}
{fig:step_c1_con_offset}{1}

\insertarfigura{img/C1_Lead/sinOffset}
{Compensador de adelanto: respuesta al escalón y esfuerzo (simulado vs.\ experimental, \emph{sin} offset).}
{fig:step_c1_sin_offset}{1}

\insertarfigura{img/C1_K/compConOffset}
{Compensador Proporcional: respuesta al escalón y esfuerzo (simulado vs.\ experimental, \emph{con} offset).}
{fig:step_prop_con_offset}{1}

\insertarfigura{img/C1_K/compSinOffset}
{Compensador Proporcional: respuesta al escalón y esfuerzo (simulado vs.\ experimental, \emph{sin} offset).}
{fig:step_prop_sin_offset}{1}

\insertarfigura{img/C2/conOffset}
{Compensador de adelanto + integrador: respuesta al escalón y esfuerzo (simulado vs.\ experimental, \emph{con} offset).}
{fig:step_c2_con_offset}{1}

\insertarfigura{img/C2/sinOFFset}
{Compensador de adelanto + integrador: respuesta al escalón y esfuerzo (simulado vs.\ experimental, \emph{sin} offset).}
{fig:step_c2_sin_offset}{0.92}



\subsection{Métricas de ajuste y desempeño temporal}
Las métricas usadas en cada gráfica (y en la tabla resumen) son:
\begin{equation}
	\label{eq:rmse_temporal}
	\mathrm{RMSE}=\sqrt{\frac{1}{N}\sum_{k=1}^{N}\big(y[k]-\hat{y}[k]\big)^2},
\end{equation}
\begin{equation}
	\label{eq:rmse_porct}
	\mathrm{NRMSE}=\frac{\mathrm{RMSE}}{y_{\max}-y_{\min}},
\end{equation}
\begin{equation}
	\label{eq:emax_temporal}
	e_{\max}=\max_k\,\lvert y[k]-\hat{y}[k]\rvert.
\end{equation}
\balance
% ===== Resultados (solo tiempo) =====
En la tabla \ref{tab:comparativa_temporal_min}, los índices de \emph{tiempo de subida} \(t_r\) y \emph{sobreimpulso} \(M_p\) se obtienen de las curvas de esta sección. El error en estado estacionario \(e_{ss}\) se reporta cuando aplica (lazo cerrado), y el \emph{error cuadratico medio normalizado} entre las mediciones y simulaciones se especifica tanto \emph{con} como \emph{sin} el offset de las señales.

% EN EL TEXTO (sin \onecolumn ni \balance aquí)
\begin{table}[t]
	\centering
	\caption{Comparativa temporal (simulado vs.\ experimental) y NRMSE (con/sin offset).}
	\label{tab:comparativa_temporal_min}
	\small
	\setlength{\tabcolsep}{4pt}
	\renewcommand{\arraystretch}{1.1}
	\begin{adjustbox}{max width=\columnwidth}
		\begin{tabular}{lcccc}
			\toprule
			\textbf{Sistema} &
			\makecell{\(\mathbf{t_r}\) [ms]\\(sim/exp)} &
			\makecell{\(\mathbf{M_p}\) [\%]\\(sim/exp)} &
			\makecell{\(\mathbf{e_{ss}}\) [\%]\\(sim/exp)} &
			\makecell{\(\mathbf{NRMSE}\) [\%]\\(con/sin off)}\\
			\midrule
			Lazo abierto       & 40.0 / 34.2   & 0 / 0                 & 0 / 1.8182             & 19.14 / 18.43 \\
			C1 (lead)          & 21.17 / 19.8  & 0 / 0                 & 45.77 / 42--46.79      & 38.02 / 7.31 \\
			Proporcional       & 3.736 / 3.950 & 15.4 / 10--20         & 12.5 / \(\approx 15.825\) & 19.24 / 6.57 \\
			Lead + integrador  & 13.7 / 17.10  & 7.433 / \(\approx 5\) & 0 / 0                  & 13.54 / 5.61 \\
			\bottomrule
		\end{tabular}
	\end{adjustbox}
\end{table}

\subsection{Error de velocidad (seguimiento de rampa)}
Para el compensador C2 (integrador+lead) se ensayó seguimiento a rampa \(r[k]=kT\) (pendiente \(1\ \mathrm{u}/\mathrm{s}\)). El error en régimen para sistemas tipo~1 en discreto es
\begin{equation}
	\label{eq:ess_rampa_discreto}
	e_{ss}=\frac{T}{K_{v,z}},\qquad K_{v,z}=\lim_{z\to1}(z-1)\,L(z).
\end{equation}
Con el \(L(z)\) obtenido, se midió/estimó \(K_v\simeq 78.43\Rightarrow e_{ss}^{(\mathrm{teo})}\approx 1/78.43\approx 1.28\%\), en concordancia con el valor observado tras un transitorio breve.

% ===== AJUSTAR RUTA =====
\insertarfigura{img/C2/rampaC2linda}
{Seguimiento de rampa con Compensador integrador + lead.}
{fig:rampa_c2}{0.92}

\subsection{Discusión de discrepancias}
Las diferencias entre curvas simuladas y experimentales se explican principalmente por: (i) \textbf{limitación/saturación} en la señal de esfuerzo debido a la carga del DAC (la excitación queda “achatada”), y (ii) \textbf{descalce de referencia} (offset). A ello se suman tolerancias de componentes pasivos, lo que desplaza levemente parámetros característicos de la planta. Aun así, la \emph{dinámica global} buscada (forma de la respuesta y tiempos) se mantuvo acorde a la simulación.




\section{Conclusiones}

El conjunto de ensayos y comparaciones \emph{dominio del tiempo} demuestra que, a pesar de las limitaciones de implementación y de la dispersión de la planta real, el desempeño experimental se mantiene coherente con el diseño y dentro de las bandas esperadas. A continuación se sintetizan los hallazgos principales.

\subsection*{Ajuste simulación–experimento}
Al simular con la \textbf{misma entrada} medida y remover el \textbf{offset estático} entre DAC y ADC, el ajuste mejora de forma notable (véase la tabla~\ref{tab:comparativa_temporal_min}). Lo que persiste se explica por \emph{efectos dinámicos} no ideales: como saturaciones suaves del actuador y  atenuaciones por carga.

\subsection*{Limitaciones de hardware observadas}
Se verificó que la \textbf{corriente de salida del DAC} no es suficiente para excitar directamente la planta ---no logramos conseguir los rangos exactos de corriente que puede suministrar en el datasheet---, lo que produce \emph{achatamiento} y \emph{recorte} del esfuerzo de control y, en consecuencia, discrepancias de forma en las respuestas. Además, la referencia $\mathrm{VDDA}/2$ y las líneas de alimentación presentan \emph{desacople insuficiente}, favoreciendo derivas de nivel que se manifiestan como offset y ligeras variaciones de ganancia. Estas condiciones explican parte del desajuste restante aun tras corregir el offset en posprocesado.

\subsection*{Planta real vs.\ modelo nominal}
La planta implementada difiere levemente del modelo continuo asumido por \emph{tolerancias} de pasivos, ordenamiento de etapas e impedancia de entrada vista por la fuente del esfuerzo. Aun así, las respuestas medidas se ubican mayormente \textbf{dentro de la banda de tolerancias} obtenida por análisis estadístico, lo que respalda la \emph{validez del modelo} para propósitos de diseño y la \emph{robustez} del procedimiento seguido.



\subsection*{Acciones recomendadas}
\begin{itemize}
	\item Añadir un \textbf{buffer operacional} rail–to–rail y de baja impedancia a la salida del DAC para eliminar el error de carga y evitar recortes del esfuerzo.
	\item \textbf{Biaspassear} la referencia $\mathrm{VDDA}/2$ y las líneas de alimentación, mejorando estabilidad de nivel y rechazo de ruido.
\end{itemize}

\subsection*{Conclusión general}
En conjunto, los resultados muestran que el \textbf{método de diseño es robusto}: aun bajo carga del DAC, offsets y dispersión de componentes, las respuestas experimentales se mantienen \emph{razonablemente cercanas} a las simuladas, especialmente con el compensador  con integrador + adelanto, que satisface los objetivos de seguimiento y estabilidad con un compromiso adecuado entre rapidez y exactitud.
