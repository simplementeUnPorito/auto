\subsection{Modificación de la Dinámica de la Planta}
\subsubsection{Diseñar un compensador que logre un margen de fase de $60^\circ$} 

Para el diseño, en primer lugar se transforma la planta del plano $\mathcal{S}$ al plano $\mathcal{Z}$ y, posteriormente, al plano $\mathcal{W}$. La primera transformación se realiza empleando la relación presentada en la ecuación~\ref{eq:RelacionGen}.  

Luego, para pasar del plano $\mathcal{Z}$ al plano $\mathcal{W}$, se utiliza la relación de Tustin, expresada en la ecuación~\ref{eq:planoZ}, tal como se explicó en el ítem anterior (\ref{sec:disc}).

\textbf{Diseño de compensadores – análisis preliminar}  

En primer lugar, se analizan los márgenes de fase y de ganancia de la planta sin compensar, pudiendo ver las curvas de Bode en las figuras \ref{fig:bodeMatlabSinComp} y \ref{fig:bodeAMano}.


Se observa que el margen inicial de ganancia es de \SI{31.4}{dB}, lo que permite incrementar el valor de la ganancia $K$.  

\paragraph{Primer intento – Compensador Lead}  

Por criterio de diseño, se selecciona un margen de ganancia aceptable, correspondiente a valores superiores a \SI{10}{dB}. A partir de esta condición se calcula la ganancia:
\[
|G(\omega)| = 20 \log_{10}(K) = 31.4 - 10 
\;\;\Rightarrow\;\; K = 10^{21.4/20}.
\]
\[
K = 11.149.
\]

Con este valor de ganancia se obtiene el siguiente diagrama de Bode:

\insertarfigura{./img/C1_Lead/margenesConK.png}{Diagrama de Bode de la planta con la nueva ganancia.}{fig:bodenuevak}{1}

En la figura~\ref{fig:bodenuevak} se aprecia el desplazamiento producido por la nueva ganancia $K = 11.149$, con los siguientes resultados:
\begin{itemize}
	\item \textbf{Margen de ganancia:} $GM = \SI{10.00}{dB}$
	\item \textbf{Margen de fase:} $PM = \ang{41.13}$
	\item \textbf{Frecuencia de cruce de ganancia:} $\omega_{cg} = 1.2 \times 10^3$
	\item \textbf{Frecuencia de cruce de fase:} $\omega_{cp} = 523$
\end{itemize}

Se observa que el margen de fase obtenido es $PM = \ang{41.13}$, por lo que falta aproximadamente:
\[
\Delta PM = \ang{60} - \ang{41.13} = \ang{18.87} \ (\text{tomamos } \phi \approx \ang{20}).
\]

Conociendo la frecuencia de cruce actual $\omega_{cp,\text{now}}$ y la fase adicional requerida ($\phi \approx \ang{20}$), se procede al diseño del compensador Lead mediante:
\[
\alpha = \frac{1 - \sin(\phi)}{1 + \sin(\phi)}, 
\qquad 
T = \frac{1}{\omega_{cp,\text{now}} \sqrt{\alpha}} ,
\]
\[
C_0(s) = \frac{1 + T s}{1 + \alpha T s}, 
\qquad 
g = \frac{1}{\left| G(j\omega_c) \, C_0(j\omega_c) \right|}, 
\qquad
C(s) = g \, C_0(s).
\]

\insertarfigura{./img/C1_Lead/margenConC1.png}{Diagrama de Bode con el primer compensador diseñado.}{fig:bodeConC1}{1}

Los nuevos resultados son:
\begin{itemize}
	\item \textbf{Margen de ganancia:} $GM = \SI{10.83}{dB}$
	\item \textbf{Margen de fase:} $PM = \ang{61.12}$
	\item \textbf{Frecuencia de cruce de ganancia:} $\omega_{cg} = 1.57 \times 10^3$
	\item \textbf{Frecuencia de cruce de fase:} $\omega_{cp} = 523$
\end{itemize}

\textbf{Datos del compensador.} Formas equivalentes:

\textit{Explícita (redondeada)}:
\[
C_1(z) = \frac{1.185\,z - 0.7374}{z - 0.3743}.
\]

\textit{Explícita (todos los decimales)}:
\[
C(z) = \frac{1.18473440049107\, z - 0.737363460561365}{z - 0.374311732736374}.
\]

\textit{Cero–polo–ganancia (ZPK, redondeada)}:
\[
C_1(z) = 1.1847 \cdot \frac{z - 0.6224}{z - 0.3743}.
\]

\textit{Cero–polo–ganancia (ZPK, completa)}:
\[
C(z) = 1.184734400491065 \cdot \frac{z - 0.622387144541200}{z - 0.374311732736374}.
\]

Período de muestreo: 
\[
T_s = \SI{1,2452}{ms}.
\]

\insertarfigura{./img/C1_Lead/bodeC1.png}{Diagrama de Bode del primer compensador diseñado.}{fig:bodeC1}{1}
\insertarfigura{./img/C1_Lead/esfuerzoC1.png}{Esfuerzo del primer compensador diseñado.}{fig:esfuerzoC1}{1}
\insertarfigura{./img/C1_Lead/stepC1.png}{Respuesta al escalón del primer compensador diseñado.}{fig:stepC1}{1}

El resultado de este diseño no es óptimo. Con $K = 1.18$, el sistema (tipo $0$) presenta error en estado estacionario ante escalón:
\[
e = \frac{1}{1+K_p} = 0.458 \;\Rightarrow\; \SI{45.8}{\percent}.
\]
Además, en la figura~\ref{fig:esfuerzoC1} se observa saturación del esfuerzo de control en las transiciones de estado.

\paragraph{Segundo intento – Compensador Proporcional}  

A partir del Bode de la planta sin compensar (figura~\ref{fig:bodeMatlabSinComp}), se determina la frecuencia donde el margen de fase alcanza \ang{60}. 
\insertarfigura{./img/C1_K/analisisK.png}{Análisis de la ganancia de la planta.}{fig:AnalK}{1}

En la figura~\ref{fig:AnalK}, la frecuencia correspondiente es aproximadamente \SI{344}{rad/s} y el margen de ganancia allí es \SI{16.9}{dB}. Forzando el cruce:
\[
20\log_{10}(K) = MG_{\text{inicial}} - MG_{\text{deseado}} = 16.9
\;\;\Rightarrow\;\; K \approx 7.
\]

Con esta ganancia:
\begin{itemize}
	\item \textbf{GM:} \SI{14.47}{dB}
	\item \textbf{PM:} \ang{60.28}
	\item \textbf{$\omega_{cg}$}$ = 1.2 \times 10^3$, \quad \textbf{$\omega_{cp}$}$ = 344$
\end{itemize}

\insertarfigura{./img/C1_K/esfuerzoC1.png}{Esfuerzo de control con el compensador proporcional.}{fig:esfuerzoC1K}{1}
\insertarfigura{./img/C1_K/stepC1.png}{Respuesta al escalón con el compensador proporcional.}{fig:stepC1K}{1}

El error en estado estacionario es:
\[
e \approx \frac{1}{1+K_p} \approx 0.125 \;\Rightarrow\; \SI{12.5}{\percent}.
\]
Aunque mejora el error, la respuesta presenta un \emph{overshoot} mayor y esfuerzos de control bruscos (figura~\ref{fig:esfuerzoC1K}), principalmente en las transiciones de estado, provocando saturación del VDAC del \textsc{PSoC} aún para escalones pequeños. Este diseño tampoco cumple las espectativas esperadas para un buen controlador.

\subsubsection{El compensador debe permitir el seguimiento de una entrada rampa con un error en estado estacionario igual a $1/K_v$}

Para cumplir este requerimiento, se diseña un compensador con un polo en el origen (puesto que la planta es de tipo $0$ y necesitamos que sea tipo $1$), de modo que el error frente a rampa sea finito.  

Los márgenes iniciales obtenidos se muestran en la figura~\ref{fig:margIniC2}.  

\insertarfigura{./img/C2/margenInicio.png}{Diagrama de Bode con margen de ganancia inicial con un polo al origen.}{fig:margIniC2}{1}

Como criterio de diseño, se fija un margen de ganancia deseado de \SI{10}{dB}; se determina entonces la ganancia $K$ correspondiente.

\insertarfigura{./img/C2/margenPosK.png}{Diagrama de Bode con el margen de ganancia ajustado.}{fig:margenPosK}{1}

Los valores obtenidos (figura~\ref{fig:margenPosK}) son:
\begin{itemize}
	\item \textbf{Ganancia:} $K = 158.932$
	\item \textbf{GM:} \SI{10}{dB}
	\item \textbf{PM:} \ang{22.56}
	\item $\omega_{cg} = 160$, \quad $\omega_{cp} = 85.7$
\end{itemize}

Dado que el margen de fase $PM = \ang{22.56}$ aún se encuentra lejos del valor deseado de \ang{60}, se agrega un compensador de adelanto (Lead) para aportar aproximadamente \ang{37.44} adicionales.

\insertarfigura{./img/C2/margenConCompensador.png}{Diagrama de Bode con el compensador (etapa Lead).}{fig:margenConCompensador2}{1}

\paragraph{Datos del compensador $C_2$}  

\textit{Forma explícita:}
\[
C_2(z) = \frac{0.1856\,z^2 + 0.009526\,z - 0.1761}{z^2 - 1.805\,z + 0.805}, 
\qquad
T_s = \SI{0.0012452}{s}.
\]

\textit{Forma cero–polo–ganancia (redondeada):}
\[
C_2(z) = 0.18562 \cdot \frac{(z+1)(z-0.9487)}{(z-1)(z-0.805)}.
\]

Parámetros identificados:
\begin{itemize}
	\item \textbf{Ceros:} $Z = \{-1.0,\; 0.948680356607820\}$
	\item \textbf{Polos:} $P = \{1.0,\; 0.805044751405714\}$
	\item \textbf{Ganancia:} $K = 0.185620357009816$
	\item \textbf{Período de muestreo:} $T_s = \SI{1.2452}{ms}$
\end{itemize}

\insertarfigura{./img/C2/stepC2.png}{Respuesta al escalón con el compensador (etapa Lead).}{fig:stepC2}{1}
\insertarfigura{./img/C2/rampaC2.png}{Seguimiento a la rampa con el compensador (etapa Lead).}{fig:rampaC2}{1}
\insertarfigura{./img/C2/esfuerzoC2.png}{Esfuerzo de control con el compensador (etapa Lead).}{fig:esfuerzoC2}{1}

El escalón (figura~\ref{fig:stepC2}) muestra una buena respuesta, con $\zeta \approx 0.63$ y un tiempo de subida $t_r = \SI{14}{ms}$. En la figura~\ref{fig:esfuerzoC2}, el esfuerzo máximo alcanza \SI{2.6}{V}, muy por debajo de la saturación del \textsc{PSoC}, a diferencia de diseños anteriores que llegaban a \SI{4}{V}.  

Para la entrada rampa, tras un transitorio breve, el error se estabiliza y queda determinado por:
\[
K_v = \lim_{z \to 1} \left( \frac{z-1}{z} \cdot \frac{C_2(z)\, G_d(z)}{T_s} \right),
\qquad
e = \frac{1}{K_v}.
\]

Resolviendo $K_v$ para $T_s = \SI{1.24}{ms}$:
\[
K_v = \lim_{z \to 1} \left( \frac{3.2192 \,(z + 0.7419)(z + 1)(z - 0.9487)}
{z \,(z - 0.9333)(z - 0.805)(z - 0.436)} \right).
\]

Evaluando el límite, se obtiene $K_v = 78.429$. Por lo tanto, el error es:
\[
e = \frac{1}{K_v} = 0.01275 \;\Rightarrow\; \SI{12.75}{\percent}.
\]

Finalmente, observando la figura~\ref{fig:rampaC2}, se confirma que se cumplen de manera satisfactoria los objetivos planteados para esta etapa del laboratorio: el sistema logra un seguimiento estable de la rampa, con un error aceptable y consistente con el valor calculado.
