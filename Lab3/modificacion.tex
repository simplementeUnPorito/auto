\subsection{Modificación de la Dinámica de la Planta}
\subsubsection{Diseñar un compensador que logre un margen de fase de $60^\circ$} 

Para el diseño, en primer lugar se debe transformar la planta del plano $\mathcal{S}$ al plano $\mathcal{Z}$, y posteriormente al plano $\mathcal{\omega}$. Para la primera transformación se emplea la relación dada en la ecuación~\ref{eq:RelacionGen}.  

A continuación, para pasar del plano $\mathcal{Z}$ al plano $\mathcal{W}$, se utiliza la relación de Tustin, expresada en la ecuación~\ref{eq:planoZ}, tal como se explicó en el ítem anterior (\ref{sec:disc}).

\textbf{Diseño de compensadores, parte compartida:}

En primer lugar, se analizan los márgenes de fase y de ganancia de la planta sin compensar.

\insertarfigura{./img/C1_Lead/margenesPlanta.png}{Diagrama de Bode de la planta sin compensación.}{fig:bodeMatlabSinComp}{1}

Se observa que el margen inicial de ganancia es de $31.4\,\text{dB}$, por lo que es posible incrementar el valor de la ganancia $K$.

\paragraph{Primer intento - Compensador Lead}
Por decisi\'on de los diseñadores, se eligue un margen de ganancia aceptable, el cual es aquel que se encuentra por encima de los $10dB$, por lo que calculamos la ganancia:
\[
	|G(w)|=20Log(K) = 31.4-10 \Rightarrow K = 10^{21.4/20}
\]
La nueva ganancia calculada:
\[
	K = 11.149
\]

Con la nueva ganancia se obtiene el siguiente gr\'afico:
\insertarfigura{./img/C1_Lead/margenesConK}{Diagrama de Bode de la planta sin compensación con la nueva ganancia.}{fig:bodenuevak}{1}

Analizando la figura \ref{fig:bodenuevak} podemos observar el desplazamiento que supone con la nueva ganancia $K = 11.7057$, se observan los valores:
\begin{itemize}
	\item \textbf{Margen de Ganancia:} $GM = 10.00 dB$
	\item \textbf{Margen de Fase:} $PM = 61.12$
	\item \textbf{Frecuencia del cero:} $$
\end{itemize}

\subsubsection{El compensador debe permitir el seguimiento de una entrada rampa con un error en estado estable igual a $1/K_v$.}
