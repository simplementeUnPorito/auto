\subsection{Modificación de la Dinámica de la Planta}
\subsubsection{Diseñar un compensador que logre un margen de fase de $60^\circ$} 

Para el diseño, en primer lugar se transforma la planta del plano $\mathcal{S}$ al plano $\mathcal{Z}$, y posteriormente al plano $\mathcal{W}$. La primera transformación se realiza empleando la relación presentada en la ecuación~\ref{eq:RelacionGen}.  

Luego, para pasar del plano $\mathcal{Z}$ al plano $\mathcal{W}$, se utiliza la relación de Tustin, expresada en la ecuación~\ref{eq:planoZ}, tal como se explicó en el ítem anterior (\ref{sec:disc}).

\textbf{Diseño de compensadores – análisis preliminar:}  

En primer lugar, se analizan los márgenes de fase y de ganancia de la planta sin compensar.

\insertarfigura{./img/C1_Lead/margenesPlanta.png}{Diagrama de Bode de la planta sin compensación.}{fig:bodeMatlabSinComp}{1}

Se observa que el margen inicial de ganancia es de \SI{31.4}{dB}, lo que permite incrementar el valor de la ganancia $K$.  

\paragraph{Primer intento – Compensador Lead}  

Por criterio de diseño, se selecciona un margen de ganancia aceptable, correspondiente a valores superiores a \SI{10}{dB}. A partir de esta condición se calcula la ganancia:
\[
|G(\omega)| = 20 \log(K) = 31.4 - 10 
\;\;\Rightarrow\;\; K = 10^{21.4/20}.
\]

La nueva ganancia resulta:
\[
K = 11.149.
\]

Con este valor de ganancia se obtiene el siguiente diagrama de Bode:

\insertarfigura{./img/C1_Lead/margenesConK}{Diagrama de Bode de la planta con la nueva ganancia.}{fig:bodenuevak}{1}

En la figura~\ref{fig:bodenuevak} se aprecia el desplazamiento producido por la nueva ganancia $K = 11.7057$, con los siguientes resultados:
\begin{itemize}
	\item \textbf{Margen de ganancia:} $GM = \SI{10.00}{dB}$
	\item \textbf{Margen de fase:} $PM = 41.13^\circ$
	\item \textbf{Frecuencia de cruce de ganancia:} $\omega_{cg} = 1.2 \times 10^3$
	\item \textbf{Frecuencia de cruce de fase:} $\omega_{cp} = 523$
\end{itemize}

Se observa que el margen de fase obtenido es $PM = 41.13^\circ$, lo que representa una diferencia respecto al valor solicitado de $60^\circ$:
\[
\Delta PM = 60^\circ - 41.13^\circ = 18.87^\circ \approx 20^\circ.
\]

Denominamos a este valor $\phi$. Como aún faltan aproximadamente $20^\circ$ para alcanzar el margen de fase deseado, se decide emplear un compensador de adelanto (Lead).  

Conociendo la frecuencia de cruce actual $\omega_{cp,\text{now}}$ y la fase adicional requerida ($\phi \approx 20^\circ$), se procede al diseño del compensador, implementando las siguientes ecuaciones en \textsc{Matlab}:
\[
\alpha = \frac{1 - \sin(\phi)}{1 + \sin(\phi)}, 
\qquad 
T = \frac{1}{\omega_{cp,\text{now}} \sqrt{\alpha}} 
\]
\[
C_0 = \frac{1 + T s}{1 + \alpha T s}, 
\qquad 
g = \frac{1}{\left| G(j\omega_c) C_0(j\omega_c) \right|}
\]
\[
C = g \cdot C_0
\]

\insertarfigura{./img/C1_Lead/margenConC1}{Diagrama de Bode con el primer compensador diseñado.}{fig:bodeConC1}{1}

Los nuevos resultados obtenidos con este compensador son los siguientes:
\begin{itemize}
	\item \textbf{Margen de ganancia:} $GM = \SI{10.83}{dB}$
	\item \textbf{Margen de fase:} $PM = 61.12^\circ$
	\item \textbf{Frecuencia de cruce de ganancia:} $\omega_{cg} = 1.57 \times 10^3$
	\item \textbf{Frecuencia de cruce de fase:} $\omega_{cp} = 523$
\end{itemize}

\textbf{Datos del compensador:} El compensador obtenido puede expresarse de distintas formas.  

\textbf{Forma explícita con todos los decimales}  
\[
C(z) = \frac{1.18473440049107 \, z - 0.737363460561365}{z - 0.374311732736374}
\]

\textbf{Forma explícita con decimales redondeados}  
\[
C_1(z) = \frac{1.185 \, z - 0.7374}{z - 0.3743}
\]

El período de muestreo asociado es:
\[
T_s = 0.001245213061220 \; \text{s}
\]

\textbf{Forma cero-polo-ganancia (ZPK)}

\textbf{Con redondeo:}  
\[
C_1(z) = 1.1847 \cdot \frac{(z - 0.6224)}{(z - 0.3743)}
\]

\textbf{Con todos los decimales:}  
\[
C(z) = 1.184734400491065 \cdot \frac{(z - 0.622387144541200)}{(z - 0.374311732736374)}
\]

\noindent
donde los parámetros identificados son:
\begin{itemize}
	\item Cero: $Z = 0.622387144541200$
	\item Polo: $P = 0.374311732736374$
	\item Ganancia: $K = 1.184734400491065$
	\item Sample time: $t_s = 0.0012452 s$
\end{itemize}

\insertarfigura{./img/C1_Lead/bodeC1}{Diagrama de Bode del primer compensador diseñado.}{fig:bodeC1}{1}

\insertarfigura{./img/C1_Lead/esfuerzoC1}{Esfuerzo del primer compensador diseñado.}{fig:esfuerzoC1}{1}

\insertarfigura{./img/C1_Lead/stepC1}{El step del primer compensador diseñado.}{fig:stepC1}{1}

El resultado obtenido para este diseño no es el más óptimo. Con una ganancia $K = 1.18$, el sistema presenta un error significativo debido a que es de tipo $0$. Al aplicar una entrada escalón, el error en estado estacionario calculado es:
\[
e = \frac{1}{1+K_p} = 45.8\%.
\]

Este valor resulta demasiado elevado. Además, al observar la figura~\ref{fig:esfuerzoC1}, se aprecia que el esfuerzo de control presenta saturación tanto al inicio como al final de la respuesta.  

\paragraph{Segundo intento: Compensador Proporcional}  

A partir del análisis del diagrama de Bode de la planta sin compensar (figura~\ref{fig:bodeMatlabSinComp}), se busca la frecuencia en la que el margen de fase alcanza el valor deseado de $60^\circ$.

\insertarfigura{./img/C1_K/analisisK}{Análisis de la ganancia de la planta.}{fig:AnalK}{1}

Como se observa en la figura~\ref{fig:AnalK}, la frecuencia correspondiente a un margen de fase de $60^\circ$ es de aproximadamente \SI{344}{Hz}, el margen de ganancia correspondiente para esta frecuencia es de $16.9\textbf{dB}$. Vemos que si aumentamos la ganancia a dicho valor, podemos forzar el cruce de tal manera que tengamos $60^\circ$:
\[
	20Log(K) = (MG_inicial - MG_deseado)
\]
siendo $MG_inicial - MG_deseado = 16.9$, tenemos entonces que la nueva ganancia es $K \approx 7$, con dicha ganancia tenemos los nuevos valores:
\begin{itemize}
	\item Margen de Ganancia = $GM = 14.47$
	\item Margen de Fase = $PM = 60.28^\circ$
	\item Cero
\end{itemize}
\insertarfigura{./img/C1_K/esfuerzoC1}{Esfuerzo con el Compensador Proporcional.}{fig:AnalK}{1}

\subsubsection{El compensador debe permitir el seguimiento de una entrada rampa con un error en estado estable igual a $1/K_v$.}


