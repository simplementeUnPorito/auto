\subsection{Modificación de la Dinámica de la Planta}
\subsubsection{Diseñar un compensador que logre un margen de fase de $60^\circ$} 

Para el diseño, en primer lugar se transforma la planta del plano $\mathcal{S}$ al plano $\mathcal{Z}$, y posteriormente al plano $\mathcal{W}$. La primera transformación se realiza empleando la relación presentada en la ecuación~\ref{eq:RelacionGen}.  

Luego, para pasar del plano $\mathcal{Z}$ al plano $\mathcal{W}$, se utiliza la relación de Tustin, expresada en la ecuación~\ref{eq:planoZ}, tal como se explicó en el ítem anterior (\ref{sec:disc}).

\textbf{Diseño de compensadores – análisis preliminar:}  

En primer lugar, se analizan los márgenes de fase y de ganancia de la planta sin compensar.

\insertarfigura{./img/C1_Lead/margenesPlanta.png}{Diagrama de Bode de la planta sin compensación.}{fig:bodeMatlabSinComp}{1}

Se observa que el margen inicial de ganancia es de $31.4\,\text{dB}$, lo que permite incrementar el valor de la ganancia $K$.  

\paragraph{Primer intento – Compensador Lead}  

Por criterio de diseño, se selecciona un margen de ganancia aceptable, correspondiente a valores superiores a $10\,\text{dB}$. A partir de esta condición se calcula la ganancia:
\[
|G(\omega)| = 20 \log(K) = 31.4 - 10 
\;\;\Rightarrow\;\; K = 10^{21.4/20}.
\]

La nueva ganancia resulta:
\[
K = 11.149.
\]

Con este valor de ganancia se obtiene el siguiente diagrama de Bode:

\insertarfigura{./img/C1_Lead/margenesConK}{Diagrama de Bode de la planta con la nueva ganancia.}{fig:bodenuevak}{1}

En la figura~\ref{fig:bodenuevak} se aprecia el desplazamiento producido por la nueva ganancia $K = 11.7057$, con los siguientes resultados:
\begin{itemize}
	\item \textbf{Margen de ganancia:} $GM = 10.00\,\text{dB}$
	\item \textbf{Margen de fase:} $PM = 41.13^\circ$
	\item \textbf{Frecuencia de cruce de ganancia:} $\omega_{cg} = 1.2 \times 10^3$
	\item \textbf{Frecuencia de cruce de fase:} $\omega_{cp} = 523$
\end{itemize}

Se observa que el margen de fase obtenido es $PM = 41.13^\circ$, lo que representa una diferencia respecto al valor solicitado de $60^\circ$:
\[
\Delta PM = 60^\circ - 41.13^\circ = 18.87^\circ \approx 20^\circ.
\]

Denominamos a este valor $\phi$. Como aún faltan aproximadamente $20^\circ$ para alcanzar el margen de fase deseado, se decide emplear un compensador de adelanto (Lead).  

Conociendo la frecuencia de cruce actual $\omega_{cp_{now}}$ y la fase adicional requerida ($\phi \approx 20^\circ$), se procede al diseño del compensador mediante la siguiente función implementada en \textsc{Matlab}:

\begin{lstlisting}[style=matlabstyle,caption={Primer intento: Compensador 1.},label={matlab:comp1}]
alpha = (1 - sin(phi)) / (1 + sin(phi));
T = 1 / (wcp_now * sqrt(alpha));
C0 = (1 + T*s) / (1 + alpha*T*s);
g = 1 / abs(freqresp(C0, 1j*wc));     % |C(jwc)|=1 //normalizamos el compensador para que no atenue
C = g * C0;
\end{lstlisting}


\subsubsection{El compensador debe permitir el seguimiento de una entrada rampa con un error en estado estable igual a $1/K_v$.}
